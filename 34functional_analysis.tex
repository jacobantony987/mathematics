%Text Books : \cite{kreyszig}
%Module 1:
%Examples, Completeness proofs, Completion of Metric Spaces, Vector Space, Normed Space, Banach space, Further Properties of Normed Spaces, Finite Dimensional Normed spaces and Subspaces, Compactness and Finite Dimension
%(Chapter 1 - Sections 1.5, 1.6; Chapter 2 - Sections 2.1 to 2.5)
%Module 2:
%Linear Operators, Bounded and Continuous Linear Operators, Linear Functionals, Linear Operators and Functionals on Finite dimensional spaces, Normed spaces of operators, Dual space
%(Chapter 2 - Section 2.6 to 2.10)
%Module 3:
%Inner Product Space, Hilbert space, Further properties of Inner Product Space, Orthogonal Complements and Direct Sums, Orthonormal sets and sequences, Series related to Orthonormal sequences and sets, Total Orthonormal sets and sequences, Representation of Functionals on Hilbert Spaces
%(Chapter 3 - Sections 3.1 to 3.6, 3.8)
%Module 4:
%Hilbert-Adjoint Operator, Self-Adjoint, Unitary and Normal Operators, Zorn's lemma, Hahn- Banach theorem, Hahn-Banach theorem for Complex Vector Spaces and Normed Spaces, Adjoint Operators
%(Chapter 3 - Sections 3.9, 3.10; Chapter 4 - Sections 4.1 to 4.3, 4.5)

%Module 1 - \cite{kreyszig} 1
%Module 2 - \cite{kreyszig} 2
%Module 3 - \cite{kreyszig} 3
%Module 4 - \cite{kreyszig} 3, 4

%\chapter 1
\section{Metric Spaces}
The following are a few metric spaces and their distance functions,
\begin{enumerate}
	\item $\mathbb{R}, d(x,y) = |x -y|$
	\item $\mathbb{R}^2, d(x,y) = \left( |\xi_1-\eta_1|^2 + |\xi_2 - \eta_2|^2 \right)^\frac{1}{2} $ where $x = (\xi_1,\xi_2)$
	\item $\mathbb{R}^3, d(x,y) = \left( |\xi_1-\eta_1|^2 + |\xi_2 - \eta_2|^2 + |\xi_3-\eta_3|^2 \right)^\frac{1}{2} $ where $x = (\xi_1,\xi_2,\xi_3)$
	\item Euclidean space, $\displaystyle \mathbb{R}^n, d(x,y) = \left( \sum_{j=1}^n |\xi_j - \eta_j|^2 \right)^\frac{1}{2}$ where $x = (\xi_1,\xi_2,\dots,\xi_n)$
	\item $\mathbb{C}, d(x,y) = |x - y|$ where $x = a+ib$ and $|x| = (a^2+b^2)^\frac{1}{2}$
	\item Unitary space, $\displaystyle \mathbb{C}^n, d(x,y) =  \left( \sum_{j=1}^n |\xi_j - \eta_j|^2 \right)^\frac{1}{2}$ where  $x = (\xi_1,\xi_2)$ and $\xi_j \in \mathbb{C}$.
	\item $\displaystyle C[a,b], d(x,y) = \max_{t \in [a,b]} \left\{ |x(t)-y(t)| \right\}$ where $x$ is a continuous, complex-valued function defined on closed interval $[a,b] \subset \mathbb{R}$.
	\item $\displaystyle B(A), d(x,y) = \sup_{t \in A} \left\{ |x(t)-y(t)| \right\}$ where $x$ is a bounded, complex-valued function defined on $A \subset \mathbb{R}$
	\item $\displaystyle l^p, d(x,y) = \left( \sum_{j = 1}^\infty |\xi_j - \eta_j|^p \right)^\frac{1}{p}$ where $x = \sequence{\xi_j}$ and $\displaystyle \sum_{j=1}^\infty |\xi_j|^p < \infty$. That is, set of all sequences such that the $p$th power series is convergent.
	\item $\displaystyle l^\infty, d(x,y) = \sup_{j \in \mathbb{N}} \{ |\xi_j - \eta_j| \} $ where $x = \sequence{\xi_j}$ and $|\xi_j| \le c$. That is, $l^\infty$ is the space of all bounded sequences of complex numbers.
	\item $\displaystyle c, d(x,y) = \sup_{j \in \mathbb{N}} \{ |\xi_j - \eta_j| \}$ where $\sequence{\xi_j} = \xi_1,\xi_2,\dots$ is the space of all convergent sequences of complex numbers.
	\item $\displaystyle s, d(x,y) = \sum_{j = 1}^\infty \frac{1}{2^j} \frac{|\xi_j - \eta_j|}{1+|\xi_j-\eta_j|}$ where $x = \sequence{\xi_j} = \xi_1,\xi_2,\dots$ is the space of all sequence of complex numbers.
\end{enumerate}
Clearly, $l^p \subset l^\infty \subset s$.
But, $B(A),C[a,b]$ are non-comparable since there exists discontinuous, bounded functions and continuous, unbounded functions.

\begin{definition}[conjugate exponents]
	Real numbers $p,q$ are conjugates, if $p>1,q>1$ and $\frac{1}{p}+\frac{1}{q} = 1$.
\end{definition}
Note : if $p,q$ are conjugates, then $p+q=pq$ and $(p-1)(q-1) = 1$.
Also if $u = t^{p-1}$, then $t = u^{q-1}$.

\begin{theorem}[H\"older\footnote{O.H\"older} Inequality for sums]
	Let $p,q$ be conjugates and $(\xi_j), (\eta_j)$ be two complex sequences. $(\xi_j) \in l^p$ and $(\eta_j) \in l^q$.
	Then
	\begin{equation}
		\sum_{j=1}^\infty |\xi_j\eta_j| \le \left( \sum_{j=1}^\infty |\xi_j|^p \right)^\frac{1}{p} \quad \left( \sum_{j=1}^\infty |\eta_j|^q \right)^\frac{1}{q}
	\end{equation}
\end{theorem}
\begin{corollary}[Cauchy-Schwarz Inequality]
	When $p=2$, we have its conjugate $q=2$.
	Then H\"older inequality reduces to the following
	\begin{equation}
		\sum_{j=1}^\infty |\xi_j\eta_j| \le \left( \sum_{j=1}^\infty |\xi_j|^2 \right)^\frac{1}{2} \quad \left( \sum_{j=1}^\infty |\eta_j|^2 \right)^\frac{1}{2}
	\end{equation}

\end{corollary}
\begin{theorem}[Minkowski\footnote{H. Minkowski} Inequality]
	Let $p,q$ be conjugates and $(\xi_j), (\eta_j)$ be two complex sequences in $l^p$.
	Then
	\begin{equation}
		\left( \sum_{j=1}^\infty |\xi_j + \eta_j|^p \right)^\frac{1}{p} \le \left( \sum_{j=1}^\infty |\xi_j|^p \right)^\frac{1}{p} + \left( \sum_{j=1}^\infty |\eta_j|^q \right)^\frac{1}{q}
	\end{equation}
\end{theorem}
\subsection{A few more Concepts}
\begin{description}
	\item[product of metric spaces]
		Let $(X_1,d_1),(X_2,d_2)$ be two metric spaces.
		Let $x,y \in X_1 \times X_2$ where $x = (\xi_1,\xi_2 )$ and $y = (\eta_1,\eta_2)$.
		Then, $X_1 \times X_2$ is a metric space with following metrics,
		\begin{enumerate}
			\item $\displaystyle d(x,y) = d_1\left(\xi_1,\eta_1\right) + d_2\left(\xi_2,\eta_2\right)$
			\item $\displaystyle d(x,y) = \left( d_1\left( \xi_1,\eta_1 \right)^2 + d_2\left(\xi_2,\eta_2\right)^2 \right)^\frac{1}{2}$
			\item $\displaystyle d(x,y) = \max \{ d_1 \left(\xi_1,\eta_1\right) , d_2\left(\xi_2,\eta_2\right) \}$
		\end{enumerate}
	\item[separable] A space is separable if it has a countable, dense subset.\\
		Examples : $l^\infty$ is not separable. But, $l^p$ spaces are separable $(1 \le p <\infty)$. $\mathbb{R},\mathbb{C}$ are separable. $B[a,b]$ not seprable.
\end{description}
\subsection{Exercises}
\begin{enumerate}
	\item Find a sequence $\sequence{\xi_j}$ which converges to $0$ but does not belong to any $l^p$ space. $(1 \le p < \infty)$.
	\item Find a sequence $\sequence{\xi_j}$ which belong to $l^p$ space with $p>1$, but not in $l^1$.
\end{enumerate}
\setcounter{subsection}{4}
\subsection{Completeness}
Convergent sequences in a metric space are Cauchy sequences. But, Cauchy sequences are not necessarily convergent.
\begin{definition}[complete]
	A metric space is complete if every Cauchy sequence in it is convergent.
\end{definition}
\begin{theorem}
	$\mathbb{R}$ is complete.
\end{theorem}
\begin{proof}
	Let $\sequence{\xi_n}$ be a Cauchy sequence in $\mathbb{R}$.
	Let $M_0 > 0$.
	Then, there exists $N \in \mathbb{N}$ such that $\forall n,m > N,\ d(x_n,x_m) < M_0$.
	Let $M = \max \{M_0,M_1,\dots,M_N \}$ where $\forall n \le N,\ |\xi_j| < M_j$.
	Then $\sequence{\xi_n}$ is bounded by $M$.\\

	By Bolzano Weierstrass\dag\footnote{
		Every sequence has a monotone subsequence.
		And by monotone converges theorem, every monotone bounded sequence has a convergent subsequence.} 
	theorem, every bounded sequence in $\mathbb{R}^n$ has a convergent subsequence.
	Let $x \in \mathbb{R}$ be the limit of a convergent subsequence $\sequence{\xi_{n_k}}$ of $\sequence{\xi_n}$.
	Then $\xi_n \to x$ since $\sequence{\xi_n}$ is a Cauchy sequence.
\end{proof}

\begin{theorem}
	$\mathbb{C}$ is complete.
\end{theorem}
\begin{proof}
	Let $\sequence{\xi_n}$ be a Cauchy sequence in $\mathbb{C}$.
	Then the real, imaginary parts of $\xi_n$ are a Cauchy sequences in $\mathbb{R}$.
	We know that $\mathbb{R}$ is complete.
	Let $\Re(\xi_n) \to a$ and $\Im(\xi_n) \to b$.
	Then $\xi_n \to a+ib$ since $\sequence{\xi_n}$ is a Cauchy sequence.
\end{proof}

\begin{theorem}
	Finite dimensional Euclidean space, $\mathbb{R}^n$ is complete.
\end{theorem}
\begin{proof}
	Let $\sequence{x_k}$ be Cauchy sequence in $\mathbb{R}^n$.
	Let $\varepsilon > 0$.
	There exists $N \in \mathbb{N}$ such that $\forall m,r > N,\ d(x_m,x_r) < \varepsilon$.
	Let $x_m = (\xi_{1,m},\xi_{2,m},\dots,\xi_{n,m})$ and $x_r = (\xi_{1,r},\xi_{2,r},\dots,\xi_{n,r})$.
	\begin{align*}
		d(x_m,x_r) = \left( \sum_{j=1}^n |\xi_{j,m} - \xi_{j,r}|^2 \right)^\frac{1}{2} & <  \varepsilon\\
		\implies \sum_{j=1}^n |\xi_{j,m} - \xi_{j,r}|^2 & <  \varepsilon^2 \\
		\implies |\xi_{j,m} - \xi_{j,r}| & < \varepsilon^2,\ j = 1,2,\dots,n
	\end{align*}
	Therefore, $\sequence{\xi_{j,k}}$'s are Cauchy sequences in $\mathbb{R}$ for $j = 1,2,\dots,n$.
	Since $\mathbb{R}$ is convergent, $\xi_{j,k} \to \xi_j$ for each $j$.
	Let $x = (\xi_1,\xi_2,\dots,\xi_n)$.
	Let $\varepsilon > 0$.
	Then there exists $N_j \in \mathbb{N}$ such that $\forall m > N,\ |\xi_{j,m} - \xi_j|  < \frac{\varepsilon}{n} $.
	Let $N = \max \{ N_1,N_2,\dots,N_n\}$.
	Then $\forall m > N$,
	\[ d(x_m,x) = \left( \sum_{j=1}^n | \xi_{j,m} - \xi_j|^2 \right)^\frac{1}{2} = \left( \sum_{j=1}^n \frac{\varepsilon^2}{n^2} \right)^\frac{1}{2} = \frac{\varepsilon}{\sqrt{n}} < \varepsilon \]
	Therefore, $\sequence{x_n}$ converges to $x$ in $\mathbb{R}^n$.
\end{proof}

\begin{theorem}
	Finite dimensional Unitary space, $\mathbb{C}^n$ is complete.
\end{theorem}
\begin{proof}
	Same proof as above.
\end{proof}

\begin{theorem}
	The complex sequence space of all bounded sequences, $l^\infty$ is complete.
\end{theorem}
\begin{proof}
	Let $\sequence{x_k}$ be a Cauchy sequence in $l^\infty$ where $x_k = \xi_{k,1},\ \xi_{k,2},\ \dots$ are bounded sequences in $\mathbb{C}$ for each $k$.\\
	\textbf{Step 1 : Construct $x$}\\
	Let $\varepsilon > 0$.
	Then there exists $N \in \mathbb{N}$ such that $\forall m,n > N$,
	\[ d(x_m,x_n) = \sup_{k \in \mathbb{N}} \left\{ \left|\xi_{m,k} - \xi_{n,k} \right| \right\} < \varepsilon \]
	From the metric, we have $\sequence{\xi_{j,k}}$'s are Cauchy sequences in $\mathbb{C}$ for each $k$.
	\begin{figure}
		\[ \begin{matrix}
			x_1 = & \xi_{1,1} & \xi_{1,2} & \xi_{1,3} & \dots \\
			x_2 = & \xi_{2,1} & \xi_{2,2} & \xi_{2,3} & \dots \\
			x_3 = & \xi_{3,1} & \xi_{3,2} & \xi_{3,3} & \dots \\
			\vdots & \vdots & \vdots & \vdots & \vdots \\
			x = & \xi_1 & \xi_2 & \xi_3 & \dots 
		\end{matrix} \]
		\caption{Construction of limit in Sequence spaces}
	\end{figure}
	Thus, $\xi_{j,k} \to \xi_k \in \mathbb{C}$.
	Let $x = \xi_1,\ \xi_2,\ \dots $.
	Now we need to prove that $x_k \to x$ and $x \in l^\infty$.\\
	\textbf{Step 2 : $x \in l^\infty$}\\
	We have, $x_j \in l^\infty \implies |\xi_{m,j}| < c_j, \ \forall m \in \mathbb{N} $.
	Therefore,
	\[ |\xi_j| \le |\xi_j-\xi_{m,j}| + |\xi_{m,j}| \le \varepsilon + c_j, \quad \forall j \in \mathbb{N} \]
	Thus, $x \in l^\infty$.\\
	\textbf{Step 3 : $x_n \to x$}\\
	Since $|\xi_{m,j}-\xi_j| < \varepsilon$ for $m \in \mathbb{N}$,
	\[ d(x_m,x) = \sup_{j \in \mathbb{N}} |\xi_{m,j} - \xi_j| \le \varepsilon \]
\end{proof}

\begin{theorem}
	The complex sequence space of all convergent sequences, $c$ is complete
\end{theorem}
\begin{proof}
	Since every convergent sequence is bounded, $c \subset l^\infty$.
	And the sequence space $c$ is complete if $c$ is a closed subset of the complete space $l^\infty$.
	Therefore, it is sufficient to show that $c = \overline{c}$\\

	Let $x = \sequence{\xi_k} \in \overline{c}$.	
	Then there exists a convergent sequence $\sequence{x_k} = \sequence{\xi_{k,j}}$ in $c$ converging to $x$.
	That is, $\xi_{k,j} \to \xi_j$.\\

	Let $\varepsilon > 0$.
	Then there exists $N \in \mathbb{N}$ such that $\forall n \ge N$ and $\forall j,k \in \mathbb{N}$, we have $\displaystyle d(x_N,x) = \sup_{r \in \mathbb{N}} \{ |\xi_{n,r} - \xi_r| \} < \frac{\varepsilon}{3}$.
	Therefore,
	\begin{equation}
		|\xi_{N,j} - \xi_j| < \frac{\varepsilon}{3} \text{ and } |\xi_{N,k} - \xi_k| < \frac{\varepsilon}{3} 
	\end{equation}
	Every convergent sequence in complete metric space $l^\infty$ is also a Cauchy sequence.
	Since $x_N = \sequence{\xi_{N,k}}$ is a Cauchy sequence, there exists $N_1 \in \mathbb{N}$ and $\forall j,k \ge N_1$,
	\begin{equation}
		|\xi_{N,j} - \xi_{N,k}| < \frac{\varepsilon}{3}
	\end{equation}
	Therefore,
	\[ |\xi_j - \xi_k| \le |\xi_j - \xi_{N,j}| + |\xi_{N,j}-\xi_{N,k}| + |\xi_{N,k} - \xi_k| \le \frac{\varepsilon}{3} + \frac{\varepsilon}{3} + \frac{\varepsilon}{3} = \varepsilon \]
	Clearly, $x = \sequence{\xi_k}$ is a Cauchy sequence in $l^\infty$.
	And since $l^\infty$ is complete, every Cauchy sequence in $l^\infty$ is convergent.
	Therefore, $x$ is a convergent sequence and $x \in c$.
	Thus, $c = \overline{c}$ and the induced\dag\footnote{
		Let $(X,d)$ be a complete metric space.
		And $Y$ is a closed subset of $X$.
		Then the induced metric space $(Y,d_{|_Y})$ is complete.}
	metric space $c$ is complete.
\end{proof}

\begin{theorem}
	For any $p \ge 1$, the sequence space $l^p$ is complete.
\end{theorem}
\begin{proof}
	Let $\sequence{x_k}$ be a Cauchy sequence $l^p$ where $x_k = \sequence{\xi_{k,j}}$ and
	\[ \sum_{j = 1}^\infty \left| \xi_{k,j} \right|^p < \infty,\ \forall k \in \mathbb{N} \]
	Since $\sequence{x_k}$ is a Cauchy sequence, for $\varepsilon > 0$ there exists $N \in \mathbb{N}$ such that
	\begin{equation}
		\label{equ:lpcauchy}
		d(x_m,x_n) = \left( \sum_{j=1}^\infty |\xi_{m,j} - \xi_{n,j}|^p \right) ^\frac{1}{p} < \varepsilon,\quad \forall m,n > N \text{ since } x_k \in l^p
	\end{equation}
	Thus, $\forall m,n > N,\ |\xi_{m,j} - \xi_{n,j}| < \varepsilon$ for each $j$.

	\textbf{Step 1 : Construction of $x$}\\
	Define $x = \sequence{\xi_j}$ where $\xi_j$ is an accumulation point of $\sequence{\xi_{k,j}}$.\\% Then $\xi_{k,j} \to \xi_j$ OR $x_k \to x$. \\

	\textbf{Step 2 : $x \in l^p$}\\
	Let $j > N$.
	Then, $\sequence{\xi_{k,j}}$ is a Cauchy sequence of complex numbers.
	Then, $\xi_{k,j} \to \xi_j$.
	Applying limit $n \to \infty$ on equation \ref{equ:lpcauchy}, we get
	\begin{align*}
		\left( \sum_{j=1}^\infty |\xi_{m,j} - \xi_j|^p \right) ^\frac{1}{p} & < \varepsilon,\quad \forall m > N \\
		\implies \sum_{j=1}^\infty |\xi_{m,j} - \xi_j|^p  & < \varepsilon^p
	\end{align*}
	Let $m > N$. Then $x_m - x \in l^p$. By Minkowski inequality, we have
	\begin{equation}
		\sum_{j=1}^\infty |\xi_j|^p  \le \sum_{j=1}^\infty |\xi_{m,j}-\xi_j|^p + \sum_{j=1}^\infty |\xi_{m,j}|^p  < \infty
	\end{equation}
	Thus, $x \in l^p$. \\
	\textbf{Step 3: $x_m \to x$}\\
	Since $\displaystyle \sum_{j=1}^\infty |\xi_{m,j} - \xi_j|^p < \varepsilon^p$, we have $\xi_{k,j} \to \xi_j$.
	Therefore, $x_m \to x$.
\end{proof}

\begin{theorem}
	The function space of all continuous functions defined on closed interval $[a,b]$, $C[a,b]$ is complete.
\end{theorem}
\begin{proof}
	Let $\sequence{x_k}$ be a Cauchy sequence in $C[a,b]$.\\
	Let $\varepsilon > 0$.
	Then there exists $N \in \mathbb{N}$ such that $m,n > N$, 
	\[ d(x_m,x_n) = \max_{t \in [a,b]} \left\{ x_m(t) - x_n(t) \right\} < \varepsilon \]
	Thus, for each $t \in [a,b]$, sequence $\sequence{x_k(t)}$'s are Cauchy sequences.
	We know that $\mathbb{C}$ is complete.
	Thus every Cauchy sequence in $\mathbb{C}$ is convergent.\\

	Define $x : [a,b] \to \mathbb{C}$ defined by $x(t) = $ the limit of the sequence $\sequence{x_k(t)}$.
	As $n \to \infty$, $d(x_m,x_n) \to d(x_m,x)$ and for any $m > N$, $d(x_m,x) < \varepsilon$ uniformly.
	It is evident from the construction that, the convergence $x_k(t) \to x(t)$ is uniform.\\

	Since the function $x_k$'s are continuous and the convergence is uniform, the limit function $x$ is also continuous.
	Therefore $x \in C[a,b]$.
\end{proof}

\begin{definition}[uniform metric]
	Let $(X,d)$ be a metric space in which every convergence $x_k \to x$ is uniform.
	Then the metric $d$ is a uniform metric.
\end{definition}
For example, the metric of $C[a,b]$ is a uniform metric.

\subsubsection{A few examples of incomplete metric spaces}
The following metric spaces are not complete,
\begin{enumerate}
	\item The space of rational numbers, $\mathbb{Q}$ with usual metric $d(x,y) = |x-y|$ is not complete since the rational approximations of $\pi$ is sequence in $\mathbb{Q}$ which doesn't converge in $\mathbb{Q}$.
		\[ 3,\ 3.1,\ 3.14,\ 3.141,\ \dots \to \pi \notin \mathbb{Q} \]
	\item The space of polynomial functions with metric $d(x,y) = \sup \{ x(t) - y(t)|$ is not complete since taylor approximations of $\sin x$ is a sequence of polynomial funcitons which doesn't converge to a polynomial function.
		\[ x,\ x+\frac{-x^3}{3!},\ x+\frac{-x^3}{3!}+\frac{x^5}{5!},\ \dots \to \sin x \notin p \]
	\item The space of continuous function on unit interval $[0,1]$ with a different\dag\footnote{Area under the graph of difference function is well-defined for integrable functions.} metric 
		\[ d(x,y) = \int_0^1 \left| x(t)-y(t) \right| \ dt \]
		is not complete since $\sequence{x_k}$ where $x_k : [0,1] \to \mathbb{R}$ is defined by
		\[ x_k(t) = \begin{cases} 0 & t \in \left[0,\frac{1}{2}\right) \\
			(t-\frac{1}{2})k & t \in \left[\frac{1}{2},\frac{1}{2}+\frac{1}{k}\right) \\
		1 & t \in \left[\frac{1}{2}+\frac{1}{k},1\right] \end{cases} \]
		doesn't converge to a continuous polynomial function.
\end{enumerate}
\subsection{Completion of Metric Space}
Let $Y$ be an incomplete metric space.
Completion of $Y$ is a construction of a complete metric space $X$ in which $Y$ is dense.\\

For example, $\mathbb{Q}$ is not complete. However, $\mathbb{R}$ is a complete metric space in which $\mathbb{Q}$ is complete. Therefore, $\mathbb{R}$ is a completion of $\mathbb{Q}$.

\begin{definition}[Isometry]
	Isometric functions are distance preserving functions.
	And two metric spaces are isometric if there exists a bijective isometry between them.
\end{definition}

For example, function $f : X \to Y$ is an isometry if
\[ \forall x,y \in X,\quad \hat{d}(f(x),f(y)) = d(x,y) \]
where $d, \hat{d}$ are metrics in $X,Y$ respectively.
And if  $f$ is a bijection, then $X,Y$ are isometric space.\\

\begin{commentary}
	Two metric spaces are isometric is another way of saying that the spaces are identical (same) from metric point of view.
\end{commentary}

\begin{theorem}[completion]
	Let $(X,d)$ be a metric space.
	Then, there exists a complete metric space, $(\hat{X},\hat{d})$ such that it has a dense subspace $W$ which is isometric with $X$.
	And $\hat{X}$ is unique upto isomerties.
\end{theorem}
\begin{commentary}
	In other words, for any metric space $X$ there exists a unique complete metric space, $\hat{X}$ in which $X$ is dense.
\end{commentary}
\begin{proof}
	\textbf{Step 1 : Construction of $(\hat{X},\hat{d})$}\\
	Let $(X,d)$ be a metric space.
	Let $C$ be the set of all Cauchy sequence in $X$.
	Define relation $\sequence{x_k} \sim \sequence{y_k}$ if and only if $\displaystyle \lim_{k \to \infty} d(x_k,y_k) = 0$.
	This is an equivalence relation.(proof not required)

	\begin{commentary}
		{\footnotesize
		\begin{enumerate}
			\item Reflexive - $\displaystyle \hat{x} \sim \hat{x}$ since $\lim_{k \to \infty} d(x_k,x_k) = 0$. 
			\item Symmetric -  $\displaystyle \hat{x} \sim \hat{y} \implies \lim_{k \to \infty} d(x_k,y_k) = \lim_{k \to \infty} d(y_k,x_k) \implies \hat{y} \sim \hat{x}$.
		\item Transitive - Suppose, $\hat{x} \sim \hat{y}$ and $\hat{y} \sim \hat{z}$.\\
		$\displaystyle \hat{x} \sim \hat{y} \implies \hat{d}(\hat{x},\hat{y}) = \lim_{k \to \infty} d(x_k,y_k) = 0$, $\displaystyle \hat{y} \sim \hat{z} \implies \hat{d}(\hat{y},\hat{z}) = \lim_{k \to \infty} d(y_k,z_k) = 0$.\\
		Then $\displaystyle \hat{d}(\hat{x},\hat{z}) = \lim_{k \to \infty} d(x_n,z_n) \le \lim_{k \to \infty} d(x_n,y_n) + d(y_n,z_n) = \hat{d}(\hat{x},\hat{y}) + \hat{d}(\hat{y},\hat{z}) = 0$.\\
		Therefore $\hat{x} \sim \hat{z}$.
		\end{enumerate}
		}
	\end{commentary}

	Let $\hat{X}$ be the set of all equivalent classes in $C$.
	Define $\hat{d} : \hat{X} \times \hat{X} \to \mathbb{R}$ given by  $\displaystyle \hat{d}(\hat{x},\hat{y}) = \lim_{k \to \infty} d(x_k,y_k)$ where the Cauchy sequences $\sequence{x_k} \in \hat{x}$ and $\sequence{y_k} \in \hat{y}$.
	Then $\hat{d}$ is metric in $\hat{X}$.
	\begin{enumerate}
		\item $\hat{d}$ is well-defined\\
			Suppose $\sequence{x_k},\sequence{x_k'} \in \hat{x}$ and $\sequence{y_k},\sequence{y_k'} \in \hat{y}$.
			Then $\sequence{x_k} \sim \sequence{x_k'}$ and $\sequence{y_k} \sim \sequence{y_k'}$.
			In other words, $\displaystyle \lim_{k \to \infty} d(x_k,x_k') = 0$ and $\displaystyle \lim_{k \to \infty} d(y_k,y_k') = 0$.\\

			By triangular inequality, we have
			\[ d(x_k,y_k) \le d(x_k,x_k') + d(x_k',y_k') + d(y_k',y_k) \]
			\[ \implies d(x_k,y_k) - d(x_k',y_k') \le d(x_k,x_k') + d(y_k,y_k') \]
			Similarly,
			\[ d(x_k',y_k') \le d(x_k',x_k) + d(x_k,y_k) + d(y_k,y_k') \]
			\[ \implies d(x_k',y_k') - d(x_k,y_k) \le d(x_k,x_k') + d(y_k,y_k') \]
			Therefore, 
			\[ | d(x_k,y_k) - d(x_k',y_k')| \le d(x_k,x_k') + d(y_k,y_k')  \]
			Apply the limit $k \to \infty$ on either sides, we get
			\[ \lim_{k\to \infty} |d(x_k,y_k)-d(x_k',y_k')| \le \lim_{k \to \infty} d(x_k,x_k') + \lim_{k \to \infty} d(y_k,y_k') = 0 \]
			Thus, $\hat{d}(\hat{x},\hat{y})$ depends only on the equivalent class $\hat{x},\hat{y}$ to which $x_k,y_k$ belongs and is independent of the representative from these equivalent classes.
			Therefore, $\hat{d} : \hat{X} \times \hat{X} \to \mathbb{R}$ is well-defined.
		\item $\hat{d}(\hat{x},\hat{y}) = 0 \iff \hat{x} = \hat{y}$
			\[ \hat{d}(\hat{x},\hat{y}) = 0 \iff \forall \sequence{x_k} \in \hat{x}, \forall \sequence{y_k} \in \hat{y},\ \lim_{k \to \infty} d(x_k,y_k) = 0 \iff \sequence{x_k} \sim \sequence{y_k} \]
		\item $\hat{d}(\hat{x},\hat{y}) = \hat{d}(\hat{y},\hat{x})$ is trivial since $d(x_k,y_k) = d(y_k,x_k)$.
		\item $\hat{d}(\hat{x},\hat{y}) \le \hat{d}(\hat{x},\hat{z}) + \hat{d}(\hat{z},\hat{y})$\\

			Let $\sequence{z_k}$ be a Cauchy sequence in $X$.
			Then $\sequence{z_k} \in \hat{z}$ for some $\hat{z} \in \hat{X}$,
			\[ d(x_k,y_k) \le d(x_k,z_k) + d(z_k,y_k) \]
			Applying limit $k \to \infty$ on either sides, we get
			\[ \hat{d}(\hat{x},\hat{y}) = \lim_{k \to \infty} d(x_k,y_k) \le \lim_{k \to \infty} d(x_k,z_k) + \lim_{k \to \infty} d(z_k,y_k)  = \hat{d}(\hat{x},\hat{z}) + \hat{d}(\hat{z},\hat{y}) \]
	\end{enumerate}

	\textbf{Step 2: Construction of Isometry $T : X \to W,\ W \subset \hat{X}$.}\\
	Let $b \in X$.
	Then $b,b,\dots$ is a Cauchy sequence in $\hat{X}$.
	Let $\hat{b} \in \hat{X}$ be the equivalent class of Cauchy sequences containing the constant sequence $\sequence{b}$.
	Define $T : X \to \hat{X}$ given by $T(b) = \hat{b}$.\\
	Let $a,b \in X$, then
	\begin{align*}
		Ta=Tb & \implies \sequence{a} \sim \sequence{b} \\
		& \implies \lim_{k \to \infty} d(a,b) = 0\\
		& \implies d(a,b) = 0 \implies a =b 
	\end{align*}
	Thus $T : X \to T(X)$ is a bijection.
	Also $T$ is an isometry since 
	\[ \hat{d}(\hat{a},\hat{b}) = \hat{d}(Ta,Tb) = \lim_{k \to \infty} d(a,b) = d(a,b) \]

	\textbf{Step 3: $W = T(X)$ is dense in $\hat{X}$}\\
	Let $\hat{x} \in \hat{X}$, then $\sequence{x_k} \in \hat{x}$.
	Then, $\sequence{x_k}$ is a Cauchy sequence.\\
	Let $\varepsilon > 0$.
	Then there exists $N \in \mathbb{N}$ such that $\forall m,n \ge N,\ d(x_n,x_m) < \frac{\varepsilon}{2}$.\\
	Thus,
	\[ \forall \varepsilon > 0,\ \exists N \in \mathbb{N} \text{ such that } d(x_n,x_N) < \frac{\varepsilon}{2} \]
	Clearly, $x_N \in X$ and the constant sequence $\sequence{x_N} \in \hat{x_N} \in T(X) = W$.\\
	Thus,
	\[ \hat{d}(\hat{x},\hat{x_N}) = \lim_{k \to \infty} d(x_k,x_N) \le \frac{\varepsilon}{2} < \varepsilon \]
	For every $\varepsilon > 0$, there exists $\hat{x}_N \in W$ such that $\hat{d}(\hat{x},\hat{x}_N) < \varepsilon$.
	Therefore, $W$ is dense in $\hat{X}$.\\

	\textbf{Step 4: $\hat{X}$ is complete}\\
	Let $\sequence{\hat{x}_n}$ be any Cauchy sequence in $\hat{X}$.
	Since $W$ is dense in $\hat{X}$, for each $\hat{x}_n$ there exists $\hat{z}_n \in W$ such that $\hat{d}(\hat{x}_n,\hat{z}_n) < \frac{1}{n}$.
	Then, $\sequence{\hat{z}_n}$ is a Cauchy sequence since,
	\[ \hat{d}(\hat{z}_m,\hat{z}_n) \le \hat{d}(\hat{z}_m,\hat{x}_m) + \hat{d}(\hat{x}_m,\hat{x}_n)+\hat{d}(\hat{x}_n,\hat{z}_n) \le \frac{1}{m} + \hat{d}(\hat{x}_m,\hat{x}_n) + \frac{1}{n} \]

	We have, $T : X \to W$ defined by $T(z_n) = \hat{z}_n$ is a bijective, isometry.
	Thus, sequence $\sequence{z_n}$ is a Cauchy sequence in $X$, since $\hat{d}(\hat{z}_n,\hat{z}_m) = d(z_n,z_m)$.
	Let $\hat{x} \in \hat{X}$ be the equivalent class of Cauchy sequences containing $\sequence{z_n}$.
	Then,
	\[ \hat{d}(\hat{x}_n,\hat{x}) \le \hat{d}(\hat{x}_n,\hat{z}_n) + \hat{d}(\hat{z}_n,\hat{x}) = \frac{1}{n} + \hat{d}(\hat{z}_n,\hat{x}) \]
	Apply limit $n \to \infty$ on either sides, we get
	\[ \lim_{n \to \infty} \hat{d}(\hat{x}_n,\hat{x}) \le \lim_{n \to \infty} \frac{1}{n} + \lim_{n \to \infty} \hat{d}(\hat{z}_n,\hat{x}) = 0 \]
	Thus, $\hat{x}_n \to \hat{x}$ as $n \to \infty$.
	Therefore $\hat{X}$ is complete, since every Cauchy sequence in $\hat{X}$ converges.\\

	\textbf{Step 5: Uniqueness of $\hat{X}$}\\
	Suppose $(\tilde{X},\tilde{d})$ is a complete metric space with dense subset $\tilde{W}$ which is isometric with $X$.
	Let $\tilde{x},\tilde{y} \in \tilde{X}$.
	Since $\tilde{W}$ is dense in $\tilde{X}$, there exists sequences $\sequence{\tilde{x}_n},\sequence{\tilde{y}_n}$ in $\tilde{W}$ such that $\tilde{x}_n \to \tilde{x}$ and $\tilde{y}_n \to \tilde{y}$.\\
	We have,
	\[ \tilde{d}(\tilde{x},\tilde{y}) \le \tilde{d}(\tilde{x},\tilde{x}_n) + \tilde{d}(\tilde{x}_n,\tilde{y}_n) + \tilde{d}(\tilde{y}_n,\tilde{y}) \]
	\[ \tilde{d}(\tilde{x}_n,\tilde{y}_n) \le \tilde{d}(\tilde{x}_n,\tilde{x}) + \tilde{d}(\tilde{x},\tilde{y}) + \tilde{d}(\tilde{y},\tilde{y}_n) \]
	Thus,
	\[ | \tilde{d}(\tilde{x}_n,\tilde{y}_n) - \tilde{d}(\tilde{x},\tilde{y}) | \le \tilde{d}(\tilde{x},\tilde{x}_n) + \tilde{d}(\tilde{y},\tilde{y}_n) \]
	Applying limit $n \to \infty$ on either sides, we get
	\[ \lim_{n \to \infty} | \tilde{d}(\tilde{x}_n,\tilde{y}_n) - \tilde{d}(\tilde{x},\tilde{y}) | \le \lim_{n \to \infty} \tilde{d}(\tilde{x},\tilde{x}_n) + \lim_{n \to \infty} \tilde{d}(\tilde{y},\tilde{y}_n) = 0 \]
	\[ \implies \tilde{d}(\tilde{x},\tilde{y}) = \lim_{n \to \infty} \tilde{d}(\tilde{x}_n,\tilde{y}_n) \]
	We have, $\tilde{W}$ is isometric\dag\footnote{
		Suppose $\tilde{W}$ is isometric with $X$.
		Then there exists an isometry $S : \tilde{W} \to X$ such that $\tilde{d}(\tilde{x}_n,\tilde{y}_n)) = d(S(\tilde{x}_n),S(\tilde{y}_n))$.
		Since $T : X \to W$ is also an isometry, $T^{-1} \circ S : \tilde{W} \to W$ is an isometry where $\tilde{d}(\tilde{x}_n,\tilde{y}_n) = \hat{d}(T^{-1}\circ S(\tilde{x}_n), T^{-1}\circ S(\tilde{y}_n)).$}
		with $W$.
	Also $\tilde{W},W$ are dense in $\tilde{X},\hat{X}$ respectively.
	Therefore, $\hat{X}$ and $\tilde{X}$ are isometric.
	In other words, the completion of $X$ is unique except for isometries.
\end{proof}

%\chapter 2
\section{Banach Spaces}
\begin{definition}[vector space]
	A set of vectors $X$, a field of scalars $K$ together with vector addition $+ : X \times X \to X$ and scalar multiplication $\times : K \times X \to X$ is a vector(linear) space if it satisfies
	\begin{enumerate}
		\item $x+y = y+x$
		\item $x+(y+z) = (x+y)+z$
		\item $\exists 0 \in X \text{ such that } x+0=x$
		\item $\forall x \in X,\ \exists -x \in X \text{ such that } x+(-x) = 0$
		\item $\alpha (\beta x) = (\alpha\beta)x$
		\item $1x = x$
		\item $\alpha(x+y) = \alpha x + \alpha y$
		\item $(\alpha+\beta)x = \alpha x + \beta x$
	\end{enumerate}
\end{definition}

Note : $0x = 0$, $\alpha 0 = 0$ and $(-1)x = -x$.\\

For example, $\mathbb{R}^n, \mathbb{C}^n, C[a,b], l^p, l^\infty$ are vector spaces.

\subsection{A few definitions}
Unlike other texts\dag\footnote{
	Usually, we define \textbf{space generated by a set} as the intersection of all spaces containing that set.}.
	And the \textbf{subspace $Y$ generated a subset $M$} of a vector space $X$ is the set of all linear combinations of vectors in $M$.
	That is,
	\[ Y = \text{span } M \]

A set $M$ is \textbf{linear independent} if there exists a non-zero $r$-tuple $(\alpha_1,\alpha_2,\dots,\alpha_r)$ such that $\alpha_1 x_1 + \alpha_2 x_2 + \dots + \alpha_r x_r = 0$ where $x_1,x_2,\dots,x_r \in M$.\\

A vector space $X$ is \textbf{finite dimensional} if there exists an integer $n$ which is the maximum cardinality for any linearly independent subset of $X$, say \textbf{dimension} of $X$, $\text{dim } X$.\\

Let $X$ be a finite dimenstional vector space of dimension $n$.
Then \textbf{basis} of $X$ is a linearly independent subset of cardinality $n$.\\
Note : Let $B$ be a basis of vector space $X$. Then $X = \text{span }B$.
And every element $x \in X$ has a unique representation $x = \alpha_1 x_1 + \alpha_2 x_2 + \dots + \alpha_n x_n$ where $B = \{ x_1,x_2,\dots,x_n \}$.\\

Let $X$ be an infinite dimensional vector space.
Let $B$ be a linearly independent subset of $X$ which span $X$, then $B$ is a \textbf{ Hamel basis} for $X$.\\

By Zorn's lemma, every vector space has a basis.
And for any vector space, all the bases are of same cardinality (\textbf{dimension}).\\

Let $X$ be a finite dimensional vector space of dimension $n$.
Then any proper subspace $Y$ of $X$ has dimension less than $n$.

\subsection{Banach Space}
\begin{definition}[norm]
	A real-valued function $\| \cdot \|$ on a vector space $X$, $\| \cdot \| : X \to \mathbb{R}$ is a norm if it satisfies
	\begin{enumerate}
		\item $\| x \| \ge 0,\quad \forall x \in X$
		\item $\| x \| = 0 \iff x = 0$
		\item $\| \alpha x \| = |\alpha| \| x \|,\quad \forall x \in X,\quad \forall \alpha \in K$
		\item $\| x + y \| \le \| x \| + \| y \|,\quad \forall x,y \in X$
	\end{enumerate}
\end{definition}

\textbf{Note} : Let $\| \cdot \|$ be a norm on a vector space $X$.
Then $d(x,y) = \| x -y\|$ is a metric induced by the norm.
\begin{proof}
	Let $d(x,y) = \| x-y \|$.
	\begin{enumerate}
		\item $d(x,y) = \| x-y \| \ge 0$ since $x-y \in X,\ \| x-y \| \ge 0$
		\item $d(x,y) = \| x-y \| = 0 \iff x-y = 0 \iff x = y$
		\item $d(x,y) = \| x-y \| = \| -1(y-x) \| = |-1|\|y-x\| = d(y,x)$
		\item $d(x,y) = \| x -z +z-y\| \le \|x-z\| + \| z-y\| = d(x,z) + d(z,y)$
	\end{enumerate}
\end{proof}

\begin{description}
	\item[normed space] is a vector space with a norm defined on it.
	\item[Bananch space] is a complete, normed space.
\end{description}

\textbf{Note} : Norm is continuous.
\begin{proof}
	Let $X$ be a normed space.
	We have,
	\[ \lim_{h \to 0} \| x+hy \| \le \| x \| + \lim_{h \to 0} |h| \| y \| = \| x \| \]
	\[ \| x \| \le \lim_{h \to 0} \|x+hy\| + \lim_{h \to 0}|-h| \|y\| = \lim_{h \to 0} \| x+hy \| \]
	Therefore, $\displaystyle \lim_{h \to 0} \|x +hy \| = \| x \|$.
	And the norm is continuous.
\end{proof}

\begin{challenge}
	Characterise metric spaces, which are normed space ?\\
\end{challenge}
\begin{commentary}
	By defining $\|x\| = d(x,0)$, we may construct a norm on various metric spaces so that the induced metric is the same.
	We observe the following conditions are necessary and sufficient for this function to be a norm.
\begin{proof}
	Let $(X,d)$ be a metric space.
	Let $d(x+z,y+z) = d(x,y)$ and $d(\alpha x,\alpha y) = |\alpha|d(x,y)$ for any $x,y,z \in X$ and for any $\alpha \in K$.
	Then,
	\begin{enumerate}
		\item $ \| x \| = d(x,0) \ge 0 $
		\item $ \| x \| = d(x,0) = 0 \iff x = 0 $
		\item $ \| \alpha x \| = d(\alpha x,0) = |\alpha|d(x,0) $
		\item $ \| x+y \| \le d(x+y,0) \le d(x+y,y)+d(y,0) = d(x,0) + d(y,0) = \|x\| + \|y\| $
	\end{enumerate}
\end{proof}
\end{commentary}
\begin{lemma}[translational invariance]
	Let $d$ be a metric induced by a norm on a normed space $X$.
	Then
	\begin{enumerate}
		\item $d(x+z,y+z) = d(x,y), \quad \forall x,y,z \in X$
		\item $d(\alpha x,\alpha y) = |\alpha| d(x,y),\quad \forall x,y \in X, \forall \alpha \in K$
	\end{enumerate}
\end{lemma}
\begin{proof}
	Let $(X,\| \cdot \|)$ be a normed space.
	Define $d(x,y) = \| x-y \|$.
	Then,
	\begin{enumerate}
		\item $ d(x+z,y+z) = \| x+z-(y+z) \| = \| x-y \| = d(x,y)$
		\item $ d(\alpha x, \alpha y) = \| \alpha x - \alpha y \| = \| \alpha (x-y) \| = |\alpha| \| x-y \| = |\alpha|d(x,y)$
	\end{enumerate}
\end{proof}
\subsubsection{Normed spaces}
	The following normed spaces can be easily constructed using the above construction.
\begin{enumerate}
	\item $\mathbb{R}, \|x\| = |x|$
	\item $\mathbb{R}^2, \| x \| = \left( |\xi_1|^2 + |\xi_2|^2 \right)^\frac{1}{2} $ where $x = (\xi_1,\xi_2)$
	\item $\mathbb{R}^3, \| x \|= \left( |\xi_1|^2 + |\xi_2|^2 + |\xi_3|^2 \right)^\frac{1}{2} $ where $x = (\xi_1,\xi_2,\xi_3)$
	\item Euclidean space, $\displaystyle \mathbb{R}^n, \| x \| = \left( \sum_{j=1}^n |\xi_j|^2 \right)^\frac{1}{2}$ where $x = (\xi_1,\xi_2,\dots,\xi_n)$
	\item $\mathbb{C}, \| x \| = |x|$ where $x = a+ib$ and $|x| = (a^2+b^2)^\frac{1}{2}$
	\item Unitary space, $\displaystyle \mathbb{C}^n, \| x \| =  \left( \sum_{j=1}^n |\xi_j|^2 \right)^\frac{1}{2}$ where  $x = (\xi_1,\xi_2)$ and $\xi_j \in \mathbb{C}$.
	\item $\displaystyle C[a,b], \| x \| = \max_{t \in [a,b]} \left\{ |x(t)| \right\}$ where $x$ is a continuous, complex-valued function defined on closed interval $[a,b] \subset \mathbb{R}$.
	\item $\displaystyle B(A), \| x \| = \sup_{t \in A} \left\{ |x(t)| \right\}$ where $x$ is a bounded, complex-valued function defined on $A \subset \mathbb{R}$
	\item $\displaystyle l^p, \| x \| = \left( \sum_{j = 1}^\infty |\xi_j|^p \right)^\frac{1}{p}$ where $x = \sequence{\xi_j}$ and $\displaystyle \sum_{j=1}^\infty |\xi_j|^p < \infty$. That is, set of all sequences such that the $p$th power series is convergent.
	\item $\displaystyle l^\infty, \| x \| = \sup_{j \in \mathbb{N}} \{ |\xi_j| \} $ where $x = \sequence{\xi_j}$ and $|\xi_j| \le c$. That is, $l^\infty$ is the space of all bounded sequences of complex numbers.
	\item $\displaystyle c, \| x \| = \sup_{j \in \mathbb{N}} \{ |\xi_j| \}$ where $\sequence{\xi_j} = \xi_1,\xi_2,\dots$ is the space of all convergent sequences of complex numbers.
\end{enumerate}

\subsubsection{Test for norm which can be induced from a metric}
	There exists metric which cannot be induced by any norm.
	Suppose $(X,d)$ is a metric space which doesn't feature translational invariance.
	Suppose there exists a norm on $X$ which induces $d$.
	Then $d$ must have translational invariance which is a contradition.\\

	For example, metric space of all sequence of complex numbers with metric
	\[ d(x,y) = \sum_{j = 1}^\infty \frac{1}{2^j} \frac{|\xi_j-\eta_j|}{1+|\xi_j-\eta_j|} \]
	where $x = \sequence{\xi_j} = \xi_1,\xi_2,\dots$ is not translationally invariant since,

	\[ d(\alpha x, \alpha y) = \sum_{j = 1}^\infty \frac{1}{2^j} \frac{|\alpha\xi_j-\alpha\eta_j|}{1+|\alpha\xi_j - \alpha\eta_j|} = \sum_{j=1}^\infty \frac{1}{2^j} \frac{|\xi_j - \eta_j|}{\left|\frac{1}{\alpha}\right|+|\xi_j-\eta_j|} \ne |\alpha| d(x,y) \]

	Therefore, above metric cannot be induced from any norm defined on the sequence space $s$.
	
\subsubsection{Completion of incomplete normed space}
	Consider the set of all continuous functions defined on closed interval $[a,b]$ together with the norm $\| \cdot \| : C[a,b] \to \mathbb{R}$ defined by
	\[ \|x\| = \left( \int_a^b |x(t)|^2\ dt \right)^\frac{1}{2} \]
	is an incomplete normed space.
	The completion of this space is $L^2[a,b]$, the collection of all Lebesgue measurable function $x$ defined on closed interval $[a,b]$ such that $|x|^2$ is Lebesgue integrable .\\

	Consider the collection of continuous functions defined on $[a,b]$ with norm,
	\[ \|x\| = \left( \int_a^b |x(t)|^p\ dt \right)^\frac{1}{p} \]
	We have its completion $L^p[a,b]$, the collection of all Lebesgue measurable function $x$ defined on closed $[a,b]$ such that $|x|^p$ are Lebesgue integrable.

\subsection{Further properties of normed spaces}
\begin{enumerate}
	\item Let $X$ be a Banach space.
		And $Y$ is a subset of $X$.
		The subspace $Y$ is Banach if and only if $Y$ is closed in $X$.
	\item Every normed space $X$ has a Banach space $\hat{X}$ in which $X$ is dense.
\end{enumerate}
\begin{theorem}
	Let $X$ be a Banach space.
	And $Y$ is a subset of $X$.
	The subspace $Y$ is complete if and only if $Y$ is closed in $X$.
\end{theorem}
\begin{proof}
	Let $(X,\|\cdot\|)$ be a Banach space.
	Then $(X,d)$ is a metric space with induced metric $d : X \times X \to \mathbb{R}$ defined by $d(x,y) = \| x-y \|$.
	Let $Y \subset X$.
	Clearly, $Y$ is a normed space with induced norm.
	And if $Y$ is closed then every Cauchy sequence in $Y$ does converge to a point in $Y$.
	Therefore, $Y$ is a Banach space.\\

	Suppose $Y$ is a Banach space with induced norm.
	Since $Y$ is complete, every convergent sequence in $Y$ converges to a point in $Y$ itself.
	Therefore, $Y$ is a closed subset of $X$.
\end{proof}

\subsubsection{Convergence in normed spaces}
We have the notion of convergence defined for metric spaces.
Now, we may extend that notion to normed spaces.
\begin{definition}[convergence]
	A sequence $\sequence{x_n}$ is a normed $X$ converges to $x \in X$ if
	\[ \lim_{n \to \infty} \| x_n-x \| = 0 \]
\end{definition}

A series in a normed space $(X,\|\cdot\|)$ is convergent if the sequence of partial sums converges.
If it converges, the limit of the sequence of partial sums is the \textbf{sum} of the series.
And series is \textbf{absolutely convergent} if the series of norms converges.
\subsubsection{Schauder basis}
\begin{definition}[Schauder basis]
	A sequence $\sequence{e_n}$ of normed space $(X,\|\cdot\|)$ is a Schauder basis of $X$ if for any $x  \in X$, there exists a unique sequence of scalars $\sequence{\alpha_n}$ such that the series $\displaystyle \sum_{n = 1}^\infty \alpha_n e_n$ converges to $x$.
	\[ \lim_{n \to \infty} \left\| x - \sum_{k = 1}^n \alpha_k e_k \right\| = 0 \]
\end{definition}

\textbf{Note} : Every normed space with a Schauder basis is separable.
But there exists separable, Banach spaces without a Schauder basis.

\subsubsection{Completion of normed spaces}
\begin{theorem}[completion]
	Let $(X,\|\cdot\|)$ be a normed space.
	Then there exists a Banach space $\hat{X}$ with a dense subset $W$ and an isometry $A$ from $W$ onto $X$.
	And $\hat{X}$ is unique except for isometries.
\end{theorem}
\begin{proof}
	Let $(X,\|\cdot\|)$ be a normed space.
	Then $(X,d)$ where $d : X \times X \to \mathbb{R}$ defined by $d(x,y) = \| x-y \|$ is a metric space.
	By the completion of metric spaces, there exists a unique, complete metric space $\hat{X}$ with a dense subset $W$ and an isometry from $W$ onto $X$.\\

	\textbf{Step 1 : $\hat{X}$ is a vector space}\\
	Let $\hat{x},\hat{y} \in \hat{X}$.
	Let sequence $\sequence{x_n}$ be some Cauchy sequence in $\hat{x}$ and sequence $\sequence{y_n}$ be some Cauchy sequence in $\hat{y}$.
	\begin{enumerate}
		\item The vector set, $\hat{X}$ is the equivalent classes of Cauchy sequence in $X$.
		\item The scalar field is real/complex field $K$ of numbers.
		\item Vector addition $+ : \hat{X} \times \hat{X} \to \hat{X}$ is defined by $\hat{x} + \hat{y} = \hat{z}$.
			Let $\sequence{x_n} \in \hat{x}$, $\sequence{y_n} \in \hat{y}$.
			Then $\sequence{x_n+y_n} = \sequence{z_n} \in \hat{z}$.\\

			Suppose $\hat{x},\hat{y} \in \hat{X}$.
			We need prove that vector addition is well-defined as well as closed.
			In other words, $\hat{x}+\hat{y}$ is independent of the choice of $\sequence{x_n} \in \hat{x}$, $\sequence{y_n} \in \hat{y}$ and is an equivalent class of Cauchy sequences.\\

			\textbf{Step 1.3a : Vector Sum is well-defined}\\
			Let $\sequence{x_n},\sequence{x_n'} \in \hat{x}$ and $\sequence{y_n},\sequence{y_n'} \in \hat{y}$.
			Then, $\sequence{x_n} \sim \sequence{x_n'}$ and $\sequence{y_n} \sim \sequence{y_n'}$.
			By triangular inequality,
			\[ \| x_n + y_n - (x_n' + y_n') \| \le \| x_n - x_n'\| + \| y_n - y_n' \| \]
			Applying limit $n \to \infty $ on either sides we get,
			\[ \lim_{n \to \infty} \| x_n + y_n - (x_n'+y_n') \| \le \lim_{n \to \infty} \| x_n - x_n' \| + \lim_{n \to \infty} \| y_n - y_n' \| = 0 \]
			Clearly, $\sequence{x_n+y_n} \sim \sequence{x_n'+y_n'}$ and the vector sum $\hat{x}+\hat{y}$ is consistent, independent of the choice of the sequences.\\

			\textbf{Step 1.3b: Vector addition is closed}\\
			Let $\hat{x},\hat{y} \in \hat{X}$.
			Then every sequence $\sequence{x_n} \in \hat{x}$ and every sequence $\sequence{y_n} \in \hat{y}$ are Cauchy sequences.
			Then,
			\[ \forall \varepsilon > 0,\ \exists N_1 \in \mathbb{N} \text{ such that } \forall n,m \ge N_1,\ \| x_n - x_m \| < \frac{\varepsilon}{2} \]
			\[ \forall \varepsilon > 0,\ \exists N_2 \in \mathbb{N} \text{ such that } \forall n,m \ge N_2,\ \| y_n - y_m \| < \frac{\varepsilon}{2} \]
			We need to prove that $\hat{x}+\hat{y}$ is also an equivalent class of Cauchy sequences.
			Let $N = \max\{ N_1,N_2 \}$.
			Then $\forall \varepsilon > 0,\ \forall n,m \ge N$ we have,
			\[ \| x_n+y_n - (x_m+y_m) \| \le \| x_n - x_m \| + \| y_n - y_m \| < \frac{\varepsilon}{2} + \frac{\varepsilon}{2} = \varepsilon \]
		\item Scalar multiplication $\times : K \times \hat{X} \to \hat{X}$ defined by $\alpha \hat{x}$ is the collection of Cauchy sequences equivalent to $\sequence{\alpha x_n}$ where $\sequence{x_n} \in \hat{x}$ and $\alpha \in K$.\\

			We need to prove that scalar product is well-defined and is closed.\\
			Let $\hat{x} \in \hat{X}$ and $\alpha \in K$.\\

			\textbf{Step 1.4a : Scalar product is well-defined}\\
			Let $\sequence{x_n},\sequence{x_n'} \in \hat{x}$.
			Then $\sequence{x_n} \sim \sequence{x_n'}$.
			And both sequences are Cauchy sequences.
			\[ \lim_{n \to \infty} \| \alpha x_n - \alpha x_n' \| = |\alpha| \lim_{n \to \infty} \| x_n - x_n' \| = 0 \]
			Therefore, $\alpha\hat{x}$ is the equivalent class of sequences which are equivalent to sequence $\sequence{\alpha x_n}$ where sequence $\sequence{x_n} \in \hat{x}$.
			Clearly, $\alpha\hat{x}$ is independent of the choice of sequence $\sequence{x_n} \in \hat{x}$.\\

			\textbf{Step 1.4b: Scalar product is closed}\\
			We have, $\sequence{x_n}$ is a Cauchy sequence.
			Then, 
			\[ \forall \varepsilon > 0,\ \exists N \in \mathbb{N} \text{ such that } \forall n,m \ge N,\ \| x_n - x_m \| < \frac{\varepsilon}{|\alpha|} \]
			Clearly, $\forall n,m > N$
			\[ \| \alpha x_n - \alpha x_m \| = |\alpha|\ \| x_n - x_m \| < \varepsilon \]
			Therefore, $\alpha\hat{x}$ is an equivalent class of Cauchy sequences.
	\end{enumerate}
	Therefore, $\hat{X}$ is vector space.\\

	\textbf{Step 2 : Construction of norm}\\

	We introduce a norm on $\hat{X}$ such that the metric $\hat{d}$ coincides with the induced metric.\\
	
	Define $\| \cdot \| : W \to \mathbb{R}$ as $\| \hat{x} \| = \hat{d}(\hat{x},\hat{0})$ for every $\hat{x} \in W$.
	We know that $A : W \to X$ is an isometry.
	Therefore, $\hat{d}(\hat{x},\hat{y}) = d(x,y)$ for every $\hat{x},\hat{y} \in W$ where $A(\hat{x}) = x$ and the constant sequence $\sequence{x} \in \hat{x}$.
	Clearly, for each $\hat{x} \in W,\ \| \hat{x} \| = \hat{d}(\hat{x},\hat{0}) = d(x,0) = \| x \|$ where $x \in X$.\\

	Let $\hat{x} \in \hat{X}$. 
	We know that, $W$ is dense in $\hat{X}$, there exists a Cauchy sequence $\sequence{\hat{x}_n}$ in $W$ such that $\hat{d}(\hat{x}_n,\hat{x}) < \frac{1}{n}$.
	\begin{commentary}
	Thus, there exists a sequence $\sequence{\hat{x}_n}$ in $W$ such that $\hat{x}_n \to \hat{x}$.
	\end{commentary}
	Now we may extend the norm function to $\hat{X}$ 
	\begin{equation}
		\| \cdot \| : \hat{X} \to \mathbb{R} \text{ defined by } \| \hat{x} \| = \lim_{n \to \infty} \| \hat{x}_n \|
	\end{equation} 
	where $\sequence{\hat{x}_n} \in W$ and $\displaystyle \lim_{n \to \infty} \hat{d}(\hat{x}_n,\hat{x}) = 0$.\\

	%And there exists $x_n \in X$ such that the constant sequence $\sequence{x_n} \in \hat{x}_n$.
	%Also $\| \hat{x}_n \| = \| x_n \|$.

	Now, we show that $\| \cdot \|$ defined on $\hat{X}$ is a norm
	\begin{enumerate}
		\item We have, $\hat{d}$ is metric on $\hat{X}$. Thus, $\hat{d}(\hat{x},\hat{y}) \ge 0,\ \forall \hat{x},\hat{y} \in \hat{X}$.
			\[ \| \hat{x} \| = \lim_{n \to \infty} \| \hat{x}_n \| = \lim_{n \to \infty} \hat{d}(\hat{x}_n,\hat{0}) \ge 0 \]
		\item Suppose $\hat{x} = \hat{0}$.
			Then $\hat{x} \in W$ and $\|\hat{x}\| = \hat{d}(\hat{x},\hat{0}) = 0$.\\
			Suppose $\|\hat{x}\| = 0$.
			Then,
			\[ \| \hat{x} \| = \lim_{n \to \infty} \| \hat{x}_n \| = \lim_{n \to \infty} \hat{d}(\hat{x}_n,\hat{0}) = \lim_{n \to \infty} d(x_n,0) = 0 \]
			Clearly, sequence $\sequence{x_n}$ is a Cauchy sequence in $X$ converging to $0 \in X$.
			Therefore, $\sequence{x_n} \in \hat{0}$ and $\hat{x}_n \to \hat{0}$.
			But, we know that $\hat{x}_n \to \hat{x}$.
			Since $\hat{X}$ is a metric space, $\hat{x} = \hat{0}$.
		\item We have,
			\[ \| \alpha \hat{x} \| = \lim_{n \to \infty} \| \alpha \hat{x}_n \| = |\alpha|\lim_{n \to \infty} \| \hat{x}_n \| = |\alpha|\|\hat{x}\| \]
		\item Suppose $\hat{y} \in \hat{X}$.
			Since $W$ is dense in $\hat{X}$, there exists a Cauchy sequence $\sequence{\hat{y}_n}$ in $W$ such that $\hat{y}_n \to \hat{y}$.
			Thus,
			\[ \| \hat{x} + \hat{y} \| = \lim_{n \to \infty} \| \hat{x}_n + \hat{y}_n \| \le \lim_{n \to \infty} \| \hat{x}_n \| + \lim_{n \to \infty} \| \hat{y}_n \| = \| \hat{x} \| + \| \hat{y} \| \]
	\end{enumerate}
	Thus, $(\hat{X},\|\cdot\|)$ is complete, normed space in which $W$ is dense.\\

	\textbf{Step 3 : $\hat{X}$ is unique}\\
	Suppose $\tilde{X}$ is another Banach space with dense subset $\tilde{W}$ which has a isometry $\tilde{A}$ from $\tilde{W}$ onto $X$.
	Then, the norm $\|\cdot\|$ defined on $\tilde{W}$,
	\[ \| \tilde{x} \| = \tilde{d}(\tilde{x},\tilde{0}) = d(x,0),\ \forall \tilde{x} \in \tilde{W} \]
	And we may extend this norm to $\tilde{X}$,
	\[ \| \tilde{x} \| = \lim_{n \to \infty} \| \tilde{x}_n \| \]
	where sequence $\sequence{\tilde{x}_n}$ is a Cauchy sequence in $\tilde{W}$ and $\tilde{x}_n \to \tilde{x}$.\\

	Clearly, $\tilde{W}$ is isometric with $W$ and the topological closure of $\tilde{W}$ is isometric with the topological closure of $W$.
	That is, $\tilde{X}$ is isometric with $\hat{X}$.
\end{proof}

\subsubsection{Exercise}
\begin{definition}[seminorm]
	A real-valued function $p$ on a vector space $X$, $p : X \to \mathbb{R}$ is a seminorm if it satisfies
	\begin{enumerate}
		\item $p(x) \ge 0,\quad \forall x \in X$
		\item $x = 0 \implies p(x) = 0$
		\item $p(\alpha x) = |\alpha| p(x),\quad \forall x \in X,\quad \forall \alpha \in K$
		\item $p(x+y) \le p(x)+p(y),\quad \forall x,y \in X$
	\end{enumerate}
\end{definition}

\begin{lemma}
	Let $(X,\| \cdot \|)$ be a normed space.
	Then, the vector addition $+(x,y) = x+y$ and scalar multiplication $\times(\alpha,x)= \alpha x$ are continuous with respect to norm.
\end{lemma}
\begin{proof}
	Let $x,y \in X$ and $\alpha \in K$.
	Let sequences $\sequence{x_n},\sequence{y_n}$ converges to $x,y$ respectively.
	Let sequence $\sequence{\alpha_n}$ converges to $\alpha \in K$.
	Then, 
	\[ \lim_{n \to \infty} \|x_n +y_n - (x+y) \| \le \lim_{n \to \infty} \| x_n - x \| + \lim_{n \to \infty} \| y_n - y \| = 0 \]
	And,
	\begin{align*}
		\lim_{n \to \infty} \| \alpha_n x_n - \alpha x \| & = \lim_{n \to \infty} \| \alpha_n x_n - \alpha x_n + \alpha x_n - \alpha x \| \\
		& \le \lim_{n \to \infty} \| (\alpha_n-\alpha)x_n \| + \lim_{n \to \infty} \| \alpha(x_n-x) \| \\
		& = \lim_{n \to \infty}|\alpha_n-\alpha| \lim_{n \to \infty} \| x_n \| + |\alpha|\lim_{n \to \infty} \| x_n - x \| \\
		& = 0
	\end{align*}
\end{proof}

\begin{lemma}
	Let $(X,\| \cdot \|)$ be a normed space.
	Let sequence $\sequence{x_n}$ converges to $x \in X$ and sequences $\sequence{y_n}$ converges to $y \in X$.
	Then, sequence $\sequence{x_n+y_n}$ converges to $x+y$.
	Let sequence $\sequence{\alpha_n}$ converges to $\alpha$.
	Then sequence $\sequence{\alpha_n x_n}$ converges to $\alpha x$.
\end{lemma}
\begin{proof}
	Suppose $x_n \to x$, $y_n \to y$ and $\alpha_n \to \alpha$.
	We have, vector addition and scalar multiplication are continuous with repsect to norm.
	\[ \lim_{n \to \infty} \| x_n+y_n - (x+y) \| = 0 \text{ and } \lim_{n \to \infty} \| \alpha_n x_n - \alpha x \| = 0 \]
	Therefore, $x_n+y_n \to x+y$ and $\alpha_n x_n \to \alpha x$.
\end{proof}

\begin{lemma}[separable]
	Let $(X,\|\cdot\|)$ be normed space.
	Suppose $X$ has a Schauder basis.
	Then, $X$ is separable.
\end{lemma}
\begin{proof}
\begin{important}
	Seminar - Akshaya Peter\\
	refer : Exercise 2.3.10\\
	hint : Let $B = \{ e_1,e_2, \dots \}$ be a Schauder basis of normed space $(X,\|\cdot\|)$. Then $B$ is a dense subset of $X$.
\end{important}
\end{proof}

\begin{lemma}[quotient space]
	Let $(X,\| \cdot \|)$ be a normed space.
	Let $Y$ be a closed subset of $X$.
	Then $(X/Y,\|\cdot\|_0)$ is a normed space with norm defined by
	\[ \|\hat{x}\|_0 = \inf_{x \in \hat{x}} \| x \|, \quad \text{ for any coset } \hat{x} \in X/Y \]
\end{lemma}
\begin{proof}
\begin{important}
	Seminar - Amala Mathew\\
	refer : Exercise 2.3.14\\
	hint : You can skip algebraic details. N1,N2 are trivial. N3 - use $\alpha \hat{x} = \alpha x + Y$ where $\hat{x} = x+Y$.
\end{important}
\end{proof}
\begin{lemma}[product of normed spaces]
	Let $(X_1,\|\cdot\|_1)$, $(X_2,\|\cdot\|_2)$ be normed spaces.
	Let $X = X_1 \times X_2$.
	Then $(X,\|\cdot\|)$ is a normed space with norm defined by,
	\[ \| x \| = \max \left\{ \|x_1\|_1,\|x_2\|_2 \right\} \text{ where } x = (x_1,x_2) \]
\end{lemma}
\begin{proof}
\begin{important}
	Seminar - Rinkum Susan Punnoose\\
	refer : Exercise 2.3.15\\
	hint : Let $x = (x_1,x_2)$. WLOG suppose $\|x_1\|_1 \ge \|x_2\|_2$. Then $\| x \| = \| x_1 \|_1$ and the rest is obvious.
\end{important}
\end{proof}

\subsection{Finite dimensional normed spaces and subspaces}
\begin{lemma}
	Let $(X,\|\cdot\|)$ be a normed space.
	Let $\{ x_1,x_2,\dots,x_n \}$ be any linearly independent subset of $X$.
	Then there exists a real-number $c > 0$ such that for any subset $\{ \alpha_1,\alpha_2,\dots,\alpha_n \}$ from field $K$
	\begin{equation}
		\left\| \sum_{j=1}^n \alpha_j x_j \right\| \ge c \sum_{j=1}^n |\alpha_j| 
	\end{equation}
\end{lemma}
\begin{proof}
	Let $\{ x_1,x_2,\dots,x_n \}$ be a linearly independent subset of a normed space $X$.
	\begin{align*}
		\left\| \sum_{j=1}^n \alpha_j x_j \right\|  & \ge c \sum_{j=1}^n |\alpha_j| 
		\intertext{Let $\displaystyle s = \sum_{j=1}^n |\alpha_j|$. Suppose $s \ne 0$. Consider $\beta_j = \alpha_j/s$. Then,}
		\left\| \sum_{j=1}^n \beta_j x_j \right\|  & \ge c \text{ where } \sum_{j=1}^n |\beta_j| = 1 
	\end{align*}
	It is sufficient to prove that for any real-number $c>0$, there does not exists an $n$-tuple of scalars violating the above inequality.\\

	On the contrary, suppose that for every positive real number $c \in \mathbb{R}$ there exists a subset $\{ \beta_1,\beta_2,\dots,\beta_n \}$ such that $\| \beta_1 x_1 + \beta_2 x_2 + \dots + \beta_n x_n \| < c$.
	Then, for each $m \in \mathbb{N}$, there exist a subset of $n$ scalars $\{ \beta_{m,1}, \beta_{m,2},\dots,\beta_{m,n} \}$ such that
	\[ \| \beta_{m,1} x_1 + \beta_{m,2} x_2 + \dots + \beta_{m,n} x_n \| < \frac{1}{m} \]
	Define, $y_m = \beta_{m,1} x_1 + \beta_{m,2} x_2 + \dots + \beta_{m,n} x_n$.
	Then, $\|y_m\| \to 0$ as $m \to \infty$.\\

	Now, every $\beta_{j,k}$ is bounded since for each $m$, $|\beta_{m,1}| + |\beta_{m,2}| + \dots + |\beta_{m,n}| = 1$.
	The sequence $\sequence{\beta_{m,1}}$ of first terms of the sequence $\sequence{y_m}$ is a bounded sequence.
	By Bolzano-Weierstrass theroem, sequence $\sequence{\beta_{m,1}}$ has a convergent subsequence, say $\beta_{m,1} \to 1$ as $m \to \infty$.
	Let sequence $\sequence{y_{1,m}}$ be subsequence of sequence $\sequence{y_m}$ such that the sequence of first terms is convergent.\\

	Again, sequence $\sequence{\beta_{m,2}}$ of second terms of the sequence $\sequence{y_{1,m}}$ is a bounded sequence.
	And $\sequence{y_{1,m}}$ has a subsequence $\sequence{y_{2,m}}$ such that both the sequence of first terms $\sequence{\beta_{m,1}}$ and $\sequence{\beta_{m,2}}$ are convergent, say $\beta_{2,m} \to \beta_2$ as $m \to \infty$.\\

	Continuing like this $n$ times, we get sequence $\sequence{y_{n,m}}$ such that sequence corresponding each term is convergent.
	That is, $\beta_{m,j} \to \beta_j$ for $j=1,2,\dots,n$.\\

	Define $y = \beta_1 x_1 + \beta_2 x_2 + \dots + \beta_n x_n$.
	Clearly, the sequence $\sequence{y_{n,m}}$ is a convergent subsequence of $\sequence{y_m}$.
	And $y_{n,m} \to y$ as $m \to \infty$.
	Suppose $\sequence{y_{n,m}}$ is convergent, then $y_{n,m} \to y$ as the limit is unique.
	By the continuity of norm, we have $\| y_m \| \to \| y \|$.\\

	Since $\| y \| = 0$, we have $y = \beta_1 x_1 + \beta_2 x_2 + \dots + \beta_n x_n = 0$.
	We have $\{ x_1, x_2, \dots, x_n \}$ is linearly independent.
	Then, $\beta_j = 0$ for each $j$.
	And, for each $m$, sequence $\sequence{|\beta_{m,n} |}$ converges to $0$ and $|\beta_{m,1}|+|\beta_{m,2}|+\dots+|\beta_{m,n}| = 1$.
	This is a contradition.
	Therefore, there exists such a positive real-number $c$ satisfying the lemma.
\end{proof}
\begin{theorem}
	Every finite dimensional subspace of a normed space is complete.
\end{theorem}
\begin{proof}
	Let $Y$ be a finite dimensional subspace of a normed space $X$.
	Let $dim\ Y = n$ and $B = \{ y_1, y_2, \dots, y_n \}$ be a basis for $Y$.
	Let sequence $\sequence{y_m}$ be a Cauchy sequence in $Y$.
	Then,
	\[ y_m = \alpha_{m,1} y_1 + \alpha_{m,2} y_2 + \dots + \alpha_{m,n} y_n \]
	Since $\sequence{y_m}$ is a Cauchy sequence, we have $N \in \mathbb{N}$ such that $\forall s,t > N$
	\[ \| y_s - y_t \| = \left\| \sum_{j=1}^n \alpha_{s,j} y_j - \sum_{j=1}^n \alpha_{t,j} y_j \right\| < \varepsilon \]
	Since basis of $Y$ is a linearly independent subset of $X$, applying lemma we get
	\[ c \sum_{j=1}^n \left|\alpha_{s,j} - \alpha_{t,j} \right| \le \left \| \sum_{j=1}^n (\alpha_{s,j}-\alpha_{t,j}) y_j \right\| < \varepsilon \]
	Therefore, for each $j \in \mathbb{N}$, sequence $\sequence{\alpha_{m,j}}$ are Cauchy sequences in $K$.
	Suppose $\alpha_{m,j} \to \alpha_j$.
	Define $y = \alpha_1 y_1 + \alpha_2 y_2 + \alpha_n y_n$.
	Then, $y_m \to y$ as $m \to \infty$.
	Therefore, $Y$ is complete.
\end{proof}

\begin{theorem}
	Every finite dimensional subspace $Y$ of a normed space $X$ is closed in $X$.
\end{theorem}
\begin{proof}
	Let $Y$ be a finite dimensional subspace of a normed space $X$.
	Then $Y$ is complete.
	Let $\hat{X}$ be the completion of $X$.
	Now, $Y$ is complete subspace of a Banach space $\hat{X}$.
	Then, $Y$ is closed $\hat{X}$.
	And $Y \subset X \subset \hat{X}$.
	Therefore, $Y$ is closed in $X$.
\end{proof}

\subsubsection{Equivalent Norm}
	\par Consider normed space $(X,\|\cdot\|)$.
	Then $(X,d)$ is a metric space with induced metric $d(x,y) = \| x-y \|$.
	Since every metric space is a topological space, there exists a unique topology $(X,\mathcal{T})$ with respect to its norm.
	Let $\|\cdot\|_0,\|\cdot\|_1$ be two different norms defined a vector space $X$.
	Then, the norms are equivalent if their topologies coincide.\\

	Alternatively, we have the following inequality to characterise the equivalent of norms,

\begin{definition}
	Let $X$ be a vector space.
	Let $\|\cdot\|_0$, $\|\cdot\|_1$ be two normed defined on $X$.
	These norms are equivalent if there exists positive, real numbers $a,b$ such that for any $x \in X$,
	\begin{equation}
		a\|x\|_0 \le \|x\|_1 \le b\|x\|_0
	\end{equation}
\end{definition}
\begin{commentary}
	For a topologist, it will interesting to prove that the topologies coincide if and only if the above inequality holds.
\end{commentary}

\begin{theorem}
	Every norm on any finite dimensional normed spaces are equivalent.
\end{theorem}
\begin{commentary}
	Let $(X,\|\cdot\|)$ be a finite dimensional normed space.
	Then there exists a unique norm on $X$ upto equivalence.
	That is, the topology induced by norm is independent of the norm for finite dimensional normed spaces.	
\end{commentary}
\begin{proof}
	Let $X$ be a vector space with $dim\ X = n$ and basis $B = \{ x_1, x_2, \dots ,x_n \}$.
	Let $\|\cdot\|_1$, $\|\cdot\|_2$ be two norms defined on $X$.
	Let $x \in X$.
	Then,
	\[ x = \sum_{j=1}^n \alpha_j x_j \]
	We know that, basis is a linearly independent subset of $X$ itself.
	By the lemma for finite dimensional, normed spaces, there exists a positive, real number $c$ such that 
	\[ \| x \|_1 = \left\| \sum_{j=1}^n \alpha_j x_j \right\|_1 \ge c \sum_{j=1}^n |\alpha_j | \]
	By triangular inequality, we have
	\[ \| x \|_1 = \left\| \sum_{j=1}^n \alpha_j x_j \right\| \le \sum_{j=1}^n \| \alpha_j x_j \| = \sum_{j=1}^n |\alpha_j| \|x_j\| \le k \sum_{j=1}^n |\alpha_j| \]
	where $k = \max\{\|x_j\|\}$.\\

	Let $a = \frac{c}{k}$.
	Then $a\| x \|_1 \le \|x\|_2$.
	Similarly, we can obtain a positive, real number $b$ such that $b\| x \|_2 \le \| x \|_1$.
	Therefore, the norms are equivalent.
\end{proof}

\subsection{Compactness and Finite dimension}
\begin{definition}[compact]
	Let $X$ be a metric space.
	The space $X$ is \textbf{compact} if every sequences in $X$ has a convergent subsequence.
	A subset $M$ is a \textbf{compact subset} if every sequence in $M$ has a convergent subsequence.
\end{definition}
\begin{lemma}[compact]
	Compact subsets of metric spaces are closed and bounded.
\end{lemma}
\begin{proof}
\end{proof}

\begin{theorem}
	In a finite dimensional metric space, a subset $M$ is compact if and only if $M$ is closed and bounded.
\end{theorem}
\begin{proof}
\end{proof}

\begin{lemma}[Riesz]
	Let $Y,Z$ be subspaces of a normed space $X$.
	Suppose $Y$ is closed and is a proper subset of $Z$.
	Then for any real number $\theta \in (0,1)$, there exists $z \in Z$ such that
	\[ \| z \| = 1 \text{ and } \| z-y \| \ge \theta,\ \forall y \in Y \]
\end{lemma}
\begin{proof}
\end{proof}

\begin{theorem}[compact set characterisation of finiteness]
	Let $X$ be normed space in which closed unit ball is compact.
	Then $X$ is a finite dimensional normed space.
\end{theorem}
\begin{proof}
\end{proof}

\begin{theorem}
	Let $X,Y$ be metric spaces.
	Let $T : X \to Y$ be a continuous function.
	Then, the image of compact subset of $X$ under $T$ is a compact subset of $Y$.
\end{theorem}
\begin{proof}
\end{proof}

\begin{corollary}
\end{corollary}
\begin{proof}
\end{proof}

%Module 2
\subsection{Linear Operators}
\begin{definition}[linear operator]
	Let $X,Y$ be vector spaces with the same field $K$.
	A function $T : X \to Y$ is a linear operator if it satisfies
	\begin{enumerate}
		\item $T(x+y) = Tx+Ty,\ \forall x,y \in X$
		\item $T(\alpha x) = \alpha Tx,\ \forall x \in X,\forall \alpha \in K$
	\end{enumerate}
\end{definition}
\begin{commentary}
	Usually, we address a linear, function from a vector space into another as linear transformation.
	And linear operators are functions from a vector space into itself.
	Kreyszig uses linear operator for both.
\end{commentary}
\begin{enumerate}
	\item $T(\alpha x + \beta y) = T(\alpha x) + T(\beta y) = \alpha Tx + \beta Ty$
	\item $T0 = T(0x) = 0Tx = 0$
	\item $T : X \to T(X)$ is a homomorphism of vector(linear) spaces
\end{enumerate}
\subsubsection{Examples of linear operators}
\begin{enumerate}
	\item Identity operator, $I : X \to X,\ Ix = x, \forall x \in X $
		\[ I(\alpha x + \beta y) = \alpha Ix + \beta Iy = \alpha x + \beta y \]
	\item Zero operator, $O : X \to X,\ Ox = 0, \forall x \in X$
		\[ 0(\alpha x + \beta y) = \alpha 0(x) + \beta 0(y) = 0 \] 
	\item Differentiation operator, $T : X \to X,\ Tx(t) = x'(t), \forall t \in [a,b]$ where $X$ is the set of all polynomials over $[a,b]$.
		\[ T(\alpha x + \beta y) = \alpha T(x) + \beta T(y) = \alpha x' +\beta y'  \] 
	\item Integration operator, $T : X \to X$, $\displaystyle Tx(t) = \int_a^b k(t,\tau) x(\tau) d\tau$ where $X = C[a,b]$.
		\[ T(\alpha x + \beta y) = \alpha \left( \int_a^b k(t,\tau) x(\tau) d\tau \right) +\beta \left( \int_a^b k(t,\tau) y(\tau) d\tau \right)  \] 
	\item Multiplication by parameter, $T : X \to X$, $Tx(t) = tx(t)$ where $X = C[a,b]$.
		\[ T(\alpha x + \beta y) = \alpha T(x) + \beta T(y) = \alpha tx +\beta ty \] 
	\item $T : \mathbb{R}^3 \to \mathbb{R}^3$, $T(x) = x \times a$ where $a \in \mathbb{R}^3$.
		\[ T(\alpha x + \beta y) = \alpha T(x) + \beta T(y) = \alpha (x \times a) +\beta (y \times a)  \] 
	\item $T : \mathbb{R}^3 \to \mathbb{R}$, $T(x) = x \cdot a$ where $a \in \mathbb{R}^3$.
		\[ T(\alpha x + \beta y) = \alpha T(x) + \beta T(y) = \alpha (x \cdot a) +\beta (y \cdot a)  \] 
	\item $T : \mathbb{R}^n \to \mathbb{R}^r$, $T(x) = Ax$ where $A \in \mathbb{R}^{n \times r}$ is an $n \times r$ matrix of real-numbers.
		\[ T(\alpha x + \beta y) = \alpha T(x) + \beta T(y) = \alpha Ax +\beta Ay \] 
\end{enumerate}
\subsection{Bounded and Continuous Linear Operators}
\subsection{Linear Functionals}
\subsection{Linear Operators and Functionals on finite dimensional spaces}
\subsection{Normed spaces of Operators, Dual space}
%Module 3
%\chapter 3
\section{Hilbert Spaces}
\subsection{Inner Product space, Hilbert space}
\subsection{Further properties of inner product spaces}
\subsection{Orthogonal complements and direct sums}
\subsection{Orthonormal sets and sequences}
\subsection{Series related to Orthonormal sequences and sets}
\subsection{Total Orthonormal sets and sequences}
%\subsection{Legendre, Hermite and Laguerre Polynomials}
\setcounter{subsection}{7}
\subsection{Representation of Functionals on Hilbert spaces}
% Module 4
\subsection{Hilbert-Adjoint Operator}
\subsection{Self-Adjoint, Unitary and Normal Operators}

%\chapter 4
\section{Fundamental Theorems for Banach Spaces}
\subsection{Zorn's Lemma}
\subsection{Hahn-Banach Theorem}
\subsection{Hahn-Banach Theorem for complex vector spaces and normed spaces}
%\subsection{Application to Bounded Linear Functionals on $C[a,b]$}
\setcounter{subsection}{4}
\subsection{Adjoint Operator}
%\subsection{Reflexive Spaces}
%\subsection{Category Theory, Uniform boundedness Theorem}
