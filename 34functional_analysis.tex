%Text Books : \cite{kreyszig}
%Module 1:
%Examples, Completeness proofs, Completion of Metric Spaces, Vector Space, Normed Space, Banach space, Further Properties of Normed Spaces, Finite Dimensional Normed spaces and Subspaces, Compactness and Finite Dimension
%(Chapter 1 - Sections 1.5, 1.6; Chapter 2 - Sections 2.1 to 2.5)
%Module 2:
%Linear Operators, Bounded and Continuous Linear Operators, Linear Functionals, Linear Operators and Functionals on Finite dimensional spaces, Normed spaces of operators, Dual space
%(Chapter 2 - Section 2.6 to 2.10)
%Module 3:
%Inner Product Space, Hilbert space, Further properties of Inner Product Space, Orthogonal Complements and Direct Sums, Orthonormal sets and sequences, Series related to Orthonormal sequences and sets, Total Orthonormal sets and sequences, Representation of Functionals on Hilbert Spaces
%(Chapter 3 - Sections 3.1 to 3.6, 3.8)
%Module 4:
%Hilbert-Adjoint Operator, Self-Adjoint, Unitary and Normal Operators, Zorn's lemma, Hahn- Banach theorem, Hahn-Banach theorem for Complex Vector Spaces and Normed Spaces, Adjoint Operators
%(Chapter 3 - Sections 3.9, 3.10; Chapter 4 - Sections 4.1 to 4.3, 4.5)

%Module 1 - \cite{kreyszig} 1
%Module 2 - \cite{kreyszig} 2
%Module 3 - \cite{kreyszig} 3
%Module 4 - \cite{kreyszig} 3, 4

%\chapter 1
\section{Metric Spaces}
The following are a few metric spaces and their distance functions,
\begin{enumerate}
	\item $\mathbb{R}, d(x,y) = |x -y|$
	\item $\mathbb{R}^2, d(x,y) = \left( |\xi_1-\eta_1|^2 + |\xi_2 - \eta_2|^2 \right)^\frac{1}{2} $ where $x = (\xi_1,\xi_2)$
	\item $\mathbb{R}^3, d(x,y) = \left( |\xi_1-\eta_1|^2 + |\xi_2 - \eta_2|^2 + |\xi_3-\eta_3|^2 \right)^\frac{1}{2} $ where $x = (\xi_1,\xi_2,\xi_3)$
	\item Euclidean space, $\displaystyle \mathbb{R}^n, d(x,y) = \left( \sum_{j=1}^n |\xi_j - \eta_j|^2 \right)^\frac{1}{2}$ where $x = (\xi_1,\xi_2,\dots,\xi_n)$
	\item $\mathbb{C}, d(x,y) = |x - y|$ where $x = a+ib$ and $|x| = (a^2+b^2)^\frac{1}{2}$
	\item Unitary space, $\displaystyle \mathbb{C}^n, d(x,y) =  \left( \sum_{j=1}^n |\xi_j - \eta_j|^2 \right)^\frac{1}{2}$ where  $x = (\xi_1,\xi_2)$ and $\xi_j \in \mathbb{C}$.
	\item $\displaystyle C[a,b], d(x,y) = \max_{t \in [a,b]} \left\{ |x(t)-y(t)| \right\}$ where $x$ is a continuous, complex-valued function defined on closed interval $[a,b] \subset \mathbb{R}$.
	\item $\displaystyle B(A), d(x,y) = \sup_{t \in A} \left\{ |x(t)-y(t)| \right\}$ where $x$ is a bounded, complex-valued function defined on $A \subset \mathbb{R}$
	\item $\displaystyle l^p, d(x,y) = \left( \sum_{j = 1}^\infty |\xi_j - \eta_j|^p \right)^\frac{1}{p}$ where $x = \sequence{\xi_j}$ and $\displaystyle \sum_{j=1}^\infty |\xi_j|^p < \infty$. That is, set of all sequences such that the $p$th power series is convergent.
	\item $\displaystyle l^\infty, d(x,y) = \sup_{j \in \mathbb{N}} \{ |\xi_j - \eta_j| \} $ where $x = \sequence{\xi_j}$ and $|\xi_j| \le c$. That is, $l^\infty$ is the space of all bounded sequences of complex numbers.
	\item $\displaystyle c, d(x,y) = \sup_{j \in \mathbb{N}} \{ |\xi_j - \eta_j| \}$ where $\sequence{\xi_j} = \xi_1,\xi_2,\dots$ is the space of all convergent sequences of complex numbers.
	\item $\displaystyle s, d(x,y) = \sum_{j = 1}^\infty \frac{1}{2^j} \frac{|\xi_j - \eta_j|}{1+|\xi_j-\eta_j|}$ where $x = \sequence{\xi_j} = \xi_1,\xi_2,\dots$ is the space of all sequence of complex numbers.
\end{enumerate}
Clearly, $l^p \subset l^\infty \subset s$.
But, $B(A),C[a,b]$ are non-comparable since there exists discontinuous, bounded functions and continuous, unbounded functions.

\begin{definition}[conjugate exponents]
	Real numbers $p,q$ are conjugates, if $p>1,q>1$ and $\frac{1}{p}+\frac{1}{q} = 1$.
\end{definition}
Note : if $p,q$ are conjugates, then $p+q=pq$ and $(p-1)(q-1) = 1$.
Also if $u = t^{p-1}$, then $t = u^{q-1}$.

\begin{theorem}[H\"older\footnote{O.H\"older} Inequality for sums]
	Let $p,q$ be conjugates and $(\xi_j), (\eta_j)$ be two complex sequences. $(\xi_j) \in l^p$ and $(\eta_j) \in l^q$.
	Then
	\begin{equation}
		\sum_{j=1}^\infty |\xi_j\eta_j| \le \left( \sum_{j=1}^\infty |\xi_j|^p \right)^\frac{1}{p} \quad \left( \sum_{j=1}^\infty |\eta_j|^q \right)^\frac{1}{q}
	\end{equation}
\end{theorem}
\begin{corollary}[Cauchy-Schwarz Inequality]
	When $p=2$, we have its conjugate $q=2$.
	Then H\"older inequality reduces to the following
	\begin{equation}
		\sum_{j=1}^\infty |\xi_j\eta_j| \le \left( \sum_{j=1}^\infty |\xi_j|^2 \right)^\frac{1}{2} \quad \left( \sum_{j=1}^\infty |\eta_j|^2 \right)^\frac{1}{2}
	\end{equation}

\end{corollary}
\begin{theorem}[Minkowski\footnote{H. Minkowski} Inequality]
	Let $p,q$ be conjugates and $(\xi_j), (\eta_j)$ be two complex sequences in $l^p$.
	Then
	\begin{equation}
		\left( \sum_{j=1}^\infty |\xi_j + \eta_j|^p \right)^\frac{1}{p} \le \left( \sum_{j=1}^\infty |\xi_j|^p \right)^\frac{1}{p} + \left( \sum_{j=1}^\infty |\eta_j|^q \right)^\frac{1}{q}
	\end{equation}
\end{theorem}
\subsection{A few more Concepts}
\begin{description}
	\item[product of metric spaces]
		Let $(X_1,d_1),(X_2,d_2)$ be two metric spaces.
		Let $x,y \in X_1 \times X_2$ where $x = (\xi_1,\xi_2 )$ and $y = (\eta_1,\eta_2)$.
		Then, $X_1 \times X_2$ is a metric space with following metrics,
		\begin{enumerate}
			\item $\displaystyle d(x,y) = d_1\left(\xi_1,\eta_1\right) + d_2\left(\xi_2,\eta_2\right)$
			\item $\displaystyle d(x,y) = \left( d_1\left( \xi_1,\eta_1 \right)^2 + d_2\left(\xi_2,\eta_2\right)^2 \right)^\frac{1}{2}$
			\item $\displaystyle d(x,y) = \max \{ d_1 \left(\xi_1,\eta_1\right) , d_2\left(\xi_2,\eta_2\right) \}$
		\end{enumerate}
	\item[separable] A space is separable if it has a countable, dense subset.\\
		Examples : $l^\infty$ is not separable. But, $l^p$ spaces are separable $(1 \le p <\infty)$. $\mathbb{R},\mathbb{C}$ are separable. $B[a,b]$ not seprable.
\end{description}
\subsection{Exercises}
\begin{enumerate}
	\item Find a sequence $\sequence{\xi_j}$ which converges to $0$ but does not belong to any $l^p$ space. $(1 \le p < \infty)$.
	\item Find a sequence $\sequence{\xi_j}$ which belong to $l^p$ space with $p>1$ but not in $l^1$.
\end{enumerate}
\setcounter{subsection}{4}
\subsection{Completeness}
Convergent sequences in a metric space are Cauchy sequences. But, Cauchy sequences are not necessarily convergent.
\begin{definition}[complete]
	A metric space is complete if every Cauchy sequence in it is convergent.
\end{definition}
\begin{theorem}
	$\mathbb{R}$ is complete.
\end{theorem}
\begin{proof}
	Let $\sequence{\xi_n}$ be a Cauchy sequence in $\mathbb{R}$.
	Let $M_0 > 0$.
	Then, there exists $N \in \mathbb{N}$ such that $\forall n,m > N,\ d(x_n,x_m) < M_0$.
	Let $M = \max \{M_0,M_1,\dots,M_N \}$ where $\forall n \le N,\ |\xi_j| < M_j$.
	Then $\sequence{\xi_n}$ is bounded by $M$.\\

	By Bolzano Weierstrass\dag\footnote{
		Every sequence has a monotone subsequence.
		And by monotone converges theorem, every monotone bounded sequence has a convergent subsequence.} 
	theorem, every bounded sequence in $\mathbb{R}^n$ has a convergent subsequence.
	Let $x \in \mathbb{R}$ be the limit of a convergent subsequence $\sequence{\xi_{n_k}}$ of $\sequence{\xi_n}$.
	Then $\xi_n \to x$ since $\sequence{\xi_n}$ is a Cauchy sequence.
\end{proof}

\begin{theorem}
	$\mathbb{C}$ is complete.
\end{theorem}
\begin{proof}
	Let $\sequence{\xi_n}$ be a Cauchy sequence in $\mathbb{C}$.
	Then the real, imaginary parts of $\xi_n$ are a Cauchy sequences in $\mathbb{R}$.
	We know that $\mathbb{R}$ is complete.
	Let $\Re(\xi_n) \to a$ and $\Im(\xi_n) \to b$.
	Then $\xi_n \to a+ib$ since $\sequence{\xi_n}$ is a Cauchy sequence.
\end{proof}

\begin{theorem}
	Finite dimensional Euclidean space, $\mathbb{R}^n$ is complete.
\end{theorem}
\begin{proof}
	Let $\sequence{x_k}$ be Cauchy sequence in $\mathbb{R}^n$.
	Let $\varepsilon > 0$.
	There exists $N \in \mathbb{N}$ such that $\forall m,r > N,\ d(x_m,x_r) < \varepsilon$.
	Let $x_m = (\xi_{1,m},\xi_{2,m},\dots,\xi_{n,m})$ and $x_r = (\xi_{1,r},\xi_{2,r},\dots,\xi_{n,r})$.
	\begin{align*}
		d(x_m,x_r) = \left( \sum_{j=1}^n |\xi_{j,m} - \xi_{j,r}|^2 \right)^\frac{1}{2} & <  \varepsilon\\
		\implies \sum_{j=1}^n |\xi_{j,m} - \xi_{j,r}|^2 & <  \varepsilon^2 \\
		\implies |\xi_{j,m} - \xi_{j,r}| & < \varepsilon^2,\ j = 1,2,\dots,n
	\end{align*}
	Therefore, $\sequence{\xi_{j,k}}$'s are Cauchy sequences in $\mathbb{R}$ for $j = 1,2,\dots,n$.
	Since $\mathbb{R}$ is convergent, $\xi_{j,k} \to \xi_j$ for each $j$.
	Let $x = (\xi_1,\xi_2,\dots,\xi_n)$.
	Let $\varepsilon > 0$.
	Then there exists $N_j \in \mathbb{N}$ such that $\forall m > N,\ |\xi_{j,m} - \xi_j|  < \frac{\varepsilon}{n} $.
	Let $N = \max \{ N_1,N_2,\dots,N_n\}$.
	Then $\forall m > N$,
	$$ d(x_m,x) = \left( \sum_{j=1}^n | \xi_{j,m} - \xi_j|^2 \right)^\frac{1}{2} = \left( \sum_{j=1}^n \frac{\varepsilon^2}{n^2} \right)^\frac{1}{2} = \frac{\varepsilon}{\sqrt{n}} < \varepsilon $$
	Therefore, $\sequence{x_n}$ converges to $x$ in $\mathbb{R}^n$.
\end{proof}

\begin{theorem}
	Finite dimensional Unitary space, $\mathbb{C}^n$ is complete.
\end{theorem}
\begin{proof}
	Same proof as above.
\end{proof}

\begin{theorem}
	The complex sequence space of all bounded sequences, $l^\infty$ is complete.
\end{theorem}
\begin{proof}
	Let $\sequence{x_k}$ be a Cauchy sequence in $l^\infty$ where $x_k = \xi_{k,1},\ \xi_{k,2},\ \dots$ are bounded sequences in $\mathbb{C}$ for each $k$.\\
	\textbf{Step 1 : Construct $x$}\\
	Let $\varepsilon > 0$.
	Then there exists $N \in \mathbb{N}$ such that $\forall m,n > N$,
	$$d(x_m,x_n) = \sup_{k \in \mathbb{N}} \left\{ \left|\xi_{m,k} - \xi_{n,k} \right| \right\} < \varepsilon $$
	From the metric, we have $\sequence{\xi_{j,k}}$'s are Cauchy sequences in $\mathbb{C}$ for each $k$.
	\begin{figure}
		$$\begin{matrix}
			x_1 = & \xi_{1,1} & \xi_{1,2} & \xi_{1,3} & \dots \\
			x_2 = & \xi_{2,1} & \xi_{2,2} & \xi_{2,3} & \dots \\
			x_3 = & \xi_{3,1} & \xi_{3,2} & \xi_{3,3} & \dots \\
			\vdots & \vdots & \vdots & \vdots & \vdots \\
			x = & \xi_1 & \xi_2 & \xi_3 & \dots 
		\end{matrix}$$
		\caption{Construction of limit in Sequence spaces}
	\end{figure}
	Thus, $\xi_{j,k} \to \xi_k \in \mathbb{C}$.
	Let $x = \xi_1,\ \xi_2,\ \dots $.
	Now we need to prove that $x_k \to x$ and $x \in l^\infty$.\\
	\textbf{Step 2 : $x \in l^\infty$}\\
	We have, $x_j \in l^\infty \implies |\xi_{m,j}| < c_j, \ \forall m \in \mathbb{N} $.
	Therefore,
	$$ |\xi_j| \le |\xi_j-\xi_{m,j}| + |\xi_{m,j}| \le \varepsilon + c_j, \quad \forall j \in \mathbb{N} $$
	Thus, $x \in l^\infty$.\\
	\textbf{Step 3 : $x_n \to x$}\\
	Since $|\xi_{m,j}-\xi_j| < \varepsilon$ for $m \in \mathbb{N}$,
	$$ d(x_m,x) = \sup_{j \in \mathbb{N}} |\xi_{m,j} - \xi_j| \le \varepsilon $$
\end{proof}

\begin{theorem}
	The complex sequence space of all convergent sequences, $c$ is complete
\end{theorem}
\begin{proof}
	Since every convergent sequence is bounded, $c \subset l^\infty$.
	And the sequence space $c$ is complete if $c$ is a closed subset of the complete space $l^\infty$.
	Therefore, it is sufficient to show that $c = \overline{c}$\\

	Let $x = \sequence{\xi_k} \in \overline{c}$.	
	Then there exists a convergent sequence $\sequence{x_k} = \sequence{\xi_{k,j}}$ in $c$ converging to $x$.
	That is, $\xi_{k,j} \to \xi_j$.\\

	Let $\varepsilon > 0$.
	Then there exists $N \in \mathbb{N}$ such that $\forall n \ge N$ and $\forall j,k \in \mathbb{N}$, we have $\displaystyle d(x_N,x) = \sup_{r \in \mathbb{N}} \{ |\xi_{n,r} - \xi_r| \} < \frac{\varepsilon}{3}$.
	Therefore,
	\begin{equation}
		|\xi_{N,j} - \xi_j| < \frac{\varepsilon}{3} \text{ and } |\xi_{N,k} - \xi_k| < \frac{\varepsilon}{3} 
	\end{equation}
	Every convergent sequence in complete metric space $l^\infty$ is also a Cauchy sequence.
	Since $x_N = \sequence{\xi_{N,k}}$ is a Cauchy sequence, there exists $N_1 \in \mathbb{N}$ and $\forall j,k \ge N_1$,
	\begin{equation}
		|\xi_{N,j} - \xi_{N,k}| < \frac{\varepsilon}{3}
	\end{equation}
	Therefore,
	$$|\xi_j - \xi_k| \le |\xi_j - \xi_{N,j}| + |\xi_{N,j}-\xi_{N,k}| + |\xi_{N,k} - \xi_k| \le \frac{\varepsilon}{3} + \frac{\varepsilon}{3} + \frac{\varepsilon}{3} = \varepsilon $$
	Clearly, $x = \sequence{\xi_k}$ is a Cauchy sequence in $l^\infty$.
	And since $l^\infty$ is complete, every Cauchy sequence in $l^\infty$ is convergent.
	Therefore, $x$ is a convergent sequence and $x \in c$.
	Thus, $c = \overline{c}$ and the induced\dag\footnote{
		Let $(X,d)$ be a complete metric space.
		And $Y$ is a closed subset of $X$.
		Then the induced metric space $(Y,d_{|_Y})$ is complete.}
	metric space $c$ is complete.
\end{proof}

\begin{theorem}
	For any $p \ge 1$, the sequence space $l^p$ is complete.
\end{theorem}
\begin{proof}
	Let $\sequence{x_k}$ be a Cauchy sequence $l^p$ where $x_k = \sequence{\xi_{k,j}}$ and
	$$ \sum_{j = 1}^\infty \left| \xi_{k,j} \right|^p < \infty,\ \forall k \in \mathbb{N} $$
	Since $\sequence{x_k}$ is a Cauchy sequence, for $\varepsilon > 0$ there exists $N \in \mathbb{N}$ such that
	\begin{equation}
		\label{equ:lpcauchy}
		d(x_m,x_n) = \left( \sum_{j=1}^\infty |\xi_{m,j} - \xi_{n,j}|^p \right) ^\frac{1}{p} < \varepsilon,\quad \forall m,n > N \text{ since } x_k \in l^p
	\end{equation}
	Thus, $\forall m,n > N,\ |\xi_{m,j} - \xi_{n,j}| < \varepsilon$ for each $j$.

	\textbf{Step 1 : Construction of $x$}\\
	Define $x = \sequence{\xi_j}$ where $\xi_j$ is an accumulation point of $\sequence{\xi_{k,j}}$.\\% Then $\xi_{k,j} \to \xi_j$ OR $x_k \to x$. \\

	\textbf{Step 2 : $x \in l^p$}\\
	Let $j > N$.
	Then, $\sequence{\xi_{k,j}}$ is a Cauchy sequence of complex numbers.
	Then, $\xi_{k,j} \to \xi_j$.
	Applying limit $n \to \infty$ on equation \ref{equ:lpcauchy}, we get
	\begin{align*}
		\left( \sum_{j=1}^\infty |\xi_{m,j} - \xi_j|^p \right) ^\frac{1}{p} & < \varepsilon,\quad \forall m > N \\
		\implies \sum_{j=1}^\infty |\xi_{m,j} - \xi_j|^p  & < \varepsilon^p
	\end{align*}
	Let $m > N$. Then $x_m - x \in l^p$. By Minkowski inequality, we have
	\begin{equation}
		\sum_{j=1}^\infty |\xi_j|^p  \le \sum_{j=1}^\infty |\xi_{m,j}-\xi_j|^p + \sum_{j=1}^\infty |\xi_{m,j}|^p  < \infty
	\end{equation}
	Thus, $x \in l^p$. \\
	\textbf{Step 3: $x_m \to x$}\\
	Since $\displaystyle \sum_{j=1}^\infty |\xi_{m,j} - \xi_j|^p < \varepsilon^p$, we have $\xi_{k,j} \to \xi_j$.
	Therefore, $x_m \to x$.
\end{proof}

\begin{theorem}
	The function space of all continuous functions defined on closed interval $[a,b]$, $C[a,b]$ is complete.
\end{theorem}
\begin{proof}
	Let $\sequence{x_k}$ be a Cauchy sequence in $C[a,b]$.\\
	Let $\varepsilon > 0$.
	Then there exists $N \in \mathbb{N}$ such that $m,n > N$, 
	$$ d(x_m,x_n) = \max_{t \in [a,b]} \left\{ x_m(t) - x_n(t) \right\} < \varepsilon $$
	Thus, for each $t \in [a,b]$, sequence $\sequence{x_k(t)}$'s are Cauchy sequences.
	We know that $\mathbb{C}$ is complete.
	Thus every Cauchy sequence in $\mathbb{C}$ is convergent.\\

	Define $x : [a,b] \to \mathbb{C}$ defined by $x(t) = $ the limit of the sequence $\sequence{x_k(t)}$.
	As $n \to \infty$, $d(x_m,x_n) \to d(x_m,x)$ and for any $m > N$, $d(x_m,x) < \varepsilon$ uniformly.
	It is evident from the construction that, the convergence $x_k(t) \to x(t)$ is uniform.\\

	Since the function $x_k$'s are continuous and the convergence is uniform, the limit function $x$ is also continuous.
	Therefore $x \in C[a,b]$.
\end{proof}

\begin{definition}[uniform metric]
	Let $(X,d)$ be a metric space in which every convergence $x_k \to x$ is uniform.
	Then the metric $d$ is a uniform metric.
\end{definition}
For example, the metric of $C[a,b]$ is a uniform metric.

\subsubsection{A few examples of incomplete metric spaces}
The following metric spaces are not complete,
\begin{enumerate}
	\item The space of rational numbers, $\mathbb{Q}$ with usual metric $d(x,y) = |x-y|$ is not complete since the rational approximations of $\pi$ is sequence in $\mathbb{Q}$ which doesn't converge in $\mathbb{Q}$.
		$$ 3,\ 3.1,\ 3.14,\ 3.141,\ \dots \to \pi \notin \mathbb{Q}$$
	\item The space of polynomial functions with metric $d(x,y) = \sup \{ x(t) - y(t)|$ is not complete since taylor approximations of $\sin x$ is a sequence of polynomial funcitons which doesn't converge to a polynomial function.
		$$ x,\ x+\frac{-x^3}{3!},\ x+\frac{-x^3}{3!}+\frac{x^5}{5!},\ \dots \to \sin x \notin p $$
	\item The space of continuous function on unit interval $[0,1]$ with a different\dag\footnote{Area under the graph of difference function is well-defined for integrable functions.} metric 
		$$ d(x,y) = \int_0^1 \left| x(t)-y(t) \right| \ dt $$
		is not complete since $\sequence{x_k}$ where $x_k : [0,1] \to \mathbb{R}$ is defined by
		$$x_k(t) = \begin{cases} 0 & t \in \left[0,\frac{1}{2}\right) \\
			(t-\frac{1}{2})k & t \in \left[\frac{1}{2},\frac{1}{2}+\frac{1}{k}\right) \\
		1 & t \in \left[\frac{1}{2}+\frac{1}{k},1\right] \end{cases} $$
		doesn't converge to a continuous polynomial function.
\end{enumerate}
\subsection{Completion of Metric Space}
Let $Y$ be an incomplete metric space.
Completion of $Y$ is a construction of a complete metric space $X$ in which $Y$ is dense.\\

For example, $\mathbb{Q}$ is not complete. However, $\mathbb{R}$ is a complete metric space in which $\mathbb{Q}$ is complete. Therefore, $\mathbb{R}$ is a completion of $\mathbb{Q}$.

\begin{definition}[Isometry]
	Isometric functions are distance preserving functions.
	And two metric spaces are isometric if there exists a bijective isometry between them.
\end{definition}

For example, function $f : X \to Y$ is an isometry if
$$ \forall x,y \in X,\quad \hat{d}(f(x),f(y)) = d(x,y)$$
where $d, \hat{d}$ are metrics in $X,Y$ respectively.
And if  $f$ is a bijection, then $X,Y$ are isometric space.\\

\begin{commentary}
	Two metric spaces are isometric is another way of saying that the spaces are identical (same) from metric point of view.
\end{commentary}

\begin{theorem}[completion]
	Let $(X,d)$ be a metric space.
	Then, there exists a complete metric space, $(\hat{X},\hat{d})$ such that it has a dense subspace $W$ which is isometric with $X$.
	And $\hat{X}$ is unique upto isomerties.
\end{theorem}
\begin{commentary}
	In other words, for any metric space $X$ there exists a unique complete metric space, $\hat{X}$ in which $X$ is dense.
\end{commentary}
\begin{proof}
	\textbf{Step 1 : Construction of $(\hat{X},\hat{d})$}\\
	Let $(X,d)$ be a metric space.
	Let $C$ be the set of all Cauchy sequence in $X$.
	Define relation $\sequence{x_k} \sim \sequence{y_k}$ if and only if $\displaystyle \lim_{k \to \infty} d(x_k,y_k) = 0$.
	This is an equivalence relation.(proof not required)

	\begin{commentary}
		{\footnotesize
		\begin{enumerate}
			\item Reflexive - $\displaystyle \hat{x} \sim \hat{x}$ since $\lim_{k \to \infty} d(x_k,x_k) = 0$. 
			\item Symmetric -  $\displaystyle \hat{x} \sim \hat{y} \implies \lim_{k \to \infty} d(x_k,y_k) = \lim_{k \to \infty} d(y_k,x_k) \implies \hat{y} \sim \hat{x}$.
		\item Transitive - Suppose, $\hat{x} \sim \hat{y}$ and $\hat{y} \sim \hat{z}$.\\
		$\displaystyle \hat{x} \sim \hat{y} \implies \hat{d}(\hat{x},\hat{y}) = \lim_{k \to \infty} d(x_k,y_k) = 0$, $\displaystyle \hat{y} \sim \hat{z} \implies \hat{d}(\hat{y},\hat{z}) = \lim_{k \to \infty} d(y_k,z_k) = 0$.\\
		Then $\displaystyle \hat{d}(\hat{x},\hat{z}) = \lim_{k \to \infty} d(x_n,z_n) \le \lim_{k \to \infty} d(x_n,y_n) + d(y_n,z_n) = \hat{d}(\hat{x},\hat{y}) + \hat{d}(\hat{y},\hat{z}) = 0$.\\
		Therefore $\hat{x} \sim \hat{z}$.
		\end{enumerate}
		}
	\end{commentary}

	Let $\hat{X}$ be the set of all equivalent classes in $C$.
	Define $\hat{d} : \hat{X} \times \hat{X} \to \mathbb{R}$ given by  $\displaystyle \hat{d}(\hat{x},\hat{y}) = \lim_{k \to \infty} d(x_k,y_k)$ where the Cauchy sequences $x_k \in \hat{x}$ and $y_k \in \hat{y}$.
	Then $\hat{d}$ is metric in $\hat{X}$.
	\begin{enumerate}
		\item $\hat{d}$ is well-defined\\
			Suppose $x_k,x_k' \in \hat{x}$ and $y_k,y_k' \in \hat{y}$.
			Then $x_k \sim x_k'$ and $y_k \sim y_k'$.
			In other words, $\displaystyle \lim_{k \to \infty} d(x_k,x_k') = 0$ and $\displaystyle \lim_{k \to \infty} d(y_k,y_k') = 0$.\\

			By triangular inequality, we have
			$$ d(x_k,y_k) \le d(x_k,x_k') + d(x_k',y_k') + d(y_k',y_k) $$
			$$ \implies d(x_k,y_k) - d(x_k',y_k') \le d(x_k,x_k') + d(y_k,y_k')$$
			Similarly,
			$$ d(x_k',y_k') - d(x_k,y_k) \le d(x_k,x_k') + d(y_k,y_k')$$
			Therefore, 
			$$ | d(x_k,y_k) - d(x_k',y_k')| \le d(x_k,x_k') + d(y_k,y_k')$$
			Apply the limit $k \to \infty$ on either sides, we get
			$$ \lim_{k\to \infty} |d(x_k,y_k)-d(x_k',y_k')| \le \lim_{k \to \infty} d(x_k,x_k') + \lim_{k \to \infty} d(y_k,y_k') = 0 $$
			Thus, $\hat{d}(x_k,y_k)$ depends only on the equivalent class $\hat{x},\hat{y}$ to which $x_k,y_k$ belongs and is independent of the representative from these equivalent classes.
			Therefore, $\hat{d} : \hat{X} \times \hat{X} \to \mathbb{R}$ is well-defined.
		\item $\hat{d}(\hat{x},\hat{y}) = 0 \iff \hat{x} = \hat{y}$
			$$ \hat{d}(\hat{x},\hat{y}) = 0 \iff \forall x_k \in \hat{x}, \forall y_k \in \hat{y},\ \lim_{k \to \infty} d(x_k,y_k) = 0 \iff x_k \sim y_k $$
		\item $\hat{d}(\hat{x},\hat{y}) = \hat{d}(\hat{y},\hat{x})$ is trivial since $d(x_k,y_k) = d(y_k,x_k)$.
		\item $\hat{d}(\hat{x},\hat{y}) \le \hat{d}(\hat{x},\hat{z}) + \hat{d}(\hat{z},\hat{y})$
			$$ d(x_k,y_k) \le d(x_k,z_k) + d(z_k,y_k) $$
			Applying limit $k \to \infty$ on either sides, we get
			$$ \hat{d}(\hat{x},\hat{y}) = \lim_{k \to \infty} d(x_k,y_k) \le \lim_{k \to \infty} d(x_k,z_k) + \lim_{k \to \infty} d(z_k,y_k)  = \hat{d}(\hat{x},\hat{z}) + \hat{d}(\hat{z},\hat{y})$$
	\end{enumerate}
	\textbf{Step 2: Construction of Isometry $T : X \to W,\ W \subset \hat{X}$.}\\
	---yet to update---
	\textbf{Step 3: $\hat{X}$ is complete}\\
	\textbf{Step 4: Uniqueness of $\hat{X}$}\\
\end{proof}
%\chapter 2
\section{Banach Spaces}
%\chapter 3
\section{Hilbert Spaces}
%\chapter 4
\section{Fundamental Theorems for Banach Spaces}
