%Text Books : \cite{kreyszig}
%Module 1:
%Examples, Completeness proofs, Completion of Metric Spaces, Vector Space, Normed Space, Banach space, Further Properties of Normed Spaces, Finite Dimensional Normed spaces and Subspaces, Compactness and Finite Dimension
%(Chapter 1 - Sections 1.5, 1.6; Chapter 2 - Sections 2.1 to 2.5)
%Module 2:
%Linear Operators, Bounded and Continuous Linear Operators, Linear Functionals, Linear Operators and Functionals on Finite dimensional spaces, Normed spaces of operators, Dual space
%(Chapter 2 - Section 2.6 to 2.10)
%Module 3:
%Inner Product Space, Hilbert space, Further properties of Inner Product Space, Orthogonal Complements and Direct Sums, Orthonormal sets and sequences, Series related to Orthonormal sequences and sets, Total Orthonormal sets and sequences, Representation of Functionals on Hilbert Spaces
%(Chapter 3 - Sections 3.1 to 3.6, 3.8)
%Module 4:
%Hilbert-Adjoint Operator, Self-Adjoint, Unitary and Normal Operators, Zorn's lemma, Hahn- Banach theorem, Hahn-Banach theorem for Complex Vector Spaces and Normed Spaces, Adjoint Operators
%(Chapter 3 - Sections 3.9, 3.10; Chapter 4 - Sections 4.1 to 4.3, 4.5)

%Module 1 - \cite{kreyszig} 1
%Module 2 - \cite{kreyszig} 2
%Module 3 - \cite{kreyszig} 3
%Module 4 - \cite{kreyszig} 3, 4

%\chapter 1
\section{Metric Spaces}
The following are a few metric spaces and their distance functions,
\begin{enumerate}
	\item $\mathbb{R}, d(x,y) = |x -y|$
	\item $\mathbb{R}^2, d(x,y) = \left( |\xi_1-\eta_1|^2 + |\xi_2 - \eta_2|^2 \right)^\frac{1}{2} $ where $x = (\xi_1,\xi_2)$
	\item $\mathbb{R}^3, d(x,y) = \left( |\xi_1-\eta_1|^2 + |\xi_2 - \eta_2|^2 + |\xi_3-\eta_3|^2 \right)^\frac{1}{2} $ where $x = (\xi_1,\xi_2,\xi_3)$
	\item Euclidean space, $\displaystyle \mathbb{R}^n, d(x,y) = \left( \sum_{j=1}^n |\xi_j - \eta_j|^2 \right)^\frac{1}{2}$ where $x = (\xi_1,\xi_2,\dots,\xi_n)$
	\item $\mathbb{C}, d(x,y) = |x - y|$ where $x = a+ib$ and $|x| = (a^2+b^2)^\frac{1}{2}$
	\item Unitary space, $\displaystyle \mathbb{C}^n, d(x,y) =  \left( \sum_{j=1}^n |\xi_j - \eta_j|^2 \right)^\frac{1}{2}$ where  $x = (\xi_1,\xi_2)$ and $\xi_j \in \mathbb{C}$.
	\item $\displaystyle C[a,b], d(x,y) = \max_{t \in [a,b]} \left\{ |x(t)-y(t)| \right\}$ where $x$ is a continuous, complex-valued function defined on closed interval $[a,b] \subset \mathbb{R}$.
	\item $\displaystyle B(A), d(x,y) = \sup_{t \in A} \left\{ |x(t)-y(t)| \right\}$ where $x$ is a bounded, complex-valued function defined on $A \subset \mathbb{R}$
	\item $\displaystyle l^p, d(x,y) = \left( \sum_{j = 1}^\infty |\xi_j - \eta_j|^p \right)^\frac{1}{p}$ where $x = (\xi_j)$ and $\displaystyle \sum_{j=1}^\infty |\xi_j|^p < \infty$. That is, set of all sequences such that the $p$th power series is convergent.
	\item $\displaystyle l^\infty, d(x,y) = \sup_{j \in \mathbb{N}} \{ |\xi_j - \eta_j| \} $ where $x = (\xi_j)$ and $|\xi_j| \le c$. That is, $l^\infty$ is the space of all bounded sequences of complex numbers.
	\item $\displaystyle c, d(x,y) = \sup_{j \in \mathbb{N}} \{ |\xi_j - \eta_j| \}$ where $(\xi_j) = \xi_1,\xi_2,\dots$ is the space of all convergent sequences of complex numbers.
	\item $\displaystyle s, d(x,y) = \sum_{j = 1}^\infty \frac{1}{2^j} \frac{|\xi_j - \eta_j|}{1+|\xi_j-\eta_j|}$ where $x = (\xi_j) = \xi_1,\xi_2,\dots$ is the space of all sequence of complex numbers.
\end{enumerate}
Clearly, $l^p \subset l^\infty \subset s$.
But, $B(A),C[a,b]$ are non-comparable since there exists discontinuous, bounded functions and continuous, unbounded functions.

\begin{definition}[conjugate exponents]
	Real numbers $p,q$ are conjugates, if $p>1,q>1$ and $\frac{1}{p}+\frac{1}{q} = 1$.
\end{definition}
Note : if $p,q$ are conjugates, then $p+q=pq$ and $(p-1)(q-1) = 1$.
Also if $u = t^{p-1}$, then $t = u^{q-1}$.

\begin{theorem}[H\"older\footnote{O.H\"older} Inequality for sums]
	Let $p,q$ be conjugates and $(\xi_j), (\eta_j)$ be two complex sequences. $(\xi_j) \in l^p$ and $(\eta_j) \in l^q$.
	Then
	\begin{equation}
		\sum_{j=1}^\infty |\xi_j\eta_j| \le \left( \sum_{j=1}^\infty |\xi_j|^p \right)^\frac{1}{p} \quad \left( \sum_{j=1}^\infty |\eta_j|^q \right)^\frac{1}{q}
	\end{equation}
\end{theorem}
\begin{corollary}[Cauchy-Schwarz Inequality]
	When $p=2$, we have its conjugate $q=2$.
	Then H\"older inequality reduces to the following
	\begin{equation}
		\sum_{j=1}^\infty |\xi_j\eta_j| \le \left( \sum_{j=1}^\infty |\xi_j|^2 \right)^\frac{1}{2} \quad \left( \sum_{j=1}^\infty |\eta_j|^2 \right)^\frac{1}{2}
	\end{equation}

\end{corollary}
\begin{theorem}[Minkowski\footnote{H. Minkowski} Inequality]
	Let $p,q$ be conjugates and $(\xi_j), (\eta_j)$ be two complex sequences in $l^p$.
	Then
	\begin{equation}
		\left( \sum_{j=1}^\infty |\xi_j + \eta_j|^p \right)^\frac{1}{p} \le \left( \sum_{j=1}^\infty |\xi_j|^p \right)^\frac{1}{p} + \left( \sum_{j=1}^\infty |\eta_j|^q \right)^\frac{1}{q}
	\end{equation}
\end{theorem}
\subsection{A few more Concepts}
\begin{description}
	\item[product of metric spaces]
		Let $(X_1,d_1),(X_2,d_2)$ be two metric spaces.
		Then, $X_1 \times X_2$ is a metric space with following metrics,
		\begin{enumerate}
			\item $d(x,y) = d_1(x_1,y_1) + d_2(x_2,y_2)$ where $x = (x_1,x_2 ) \in X_1 \times X_2$
			\item $\displaystyle d(x,y) = \left( d_1(x_1,y_1)^2 + d_2(x_2,y_2)^2 \right)^\frac{1}{2}$ where $x = (x_1,x_2 ) \in X_1 \times X_2$
			\item $\displaystyle d(x,y) = \max \{ d_1(x_1,y_1) , d_2(x_2,y_2) \}$ where $x = (x_1,x_2 ) \in X_1 \times X_2$
		\end{enumerate}
	\item[separable] A space is separable if it has a countable, dense subset.\\
		Note : $l^\infty$ is not separable. But, $l^p$ spaces are separable $(1 \le p <\infty)$. $\mathbb{R},\mathbb{C}$ are separable. $B[a,b]$ not seprable.
	\item[complete] A space is complete if every Cauchy sequence in it converges.\\
		Note : $\mathbb{R},\mathbb{C},\mathbb{R}^n,\mathbb{C}^n$ are complete.
		Sequence spaces $c,l^p,l^\infty$ are complete.
		Function space $C[a,b]$ is complete.
		Every convergent sequence in a metric space is Cauchy.
\end{description}
\subsection{Exercises}
\begin{enumerate}
	\item Find a sequence $(\xi_j)$ which converges to $0$ but does not belong to any $l^p$ space. $(1 \le p < \infty)$.
	\item Find a sequence $(\xi_j)$ which belong to $l^p$ space with $p>1$ but not in $l^1$.
\end{enumerate}
%\chapter 2
\section{Banach Spaces}
%\chapter 3
\section{Hilbert Spaces}
%\chapter 4
\section{Fundamental Theorems for Banach Spaces}
