%Text Books : \cite{royden}
%Module 1: Lebesgue Measure
%Introduction, Lebesgue outer measure, The $\sigma$-algebra of Lebesgue measurable sets, Outer and inner approximation of Lebesgue measurable sets , Countable additivity, continuity and Borel-Cantelli Lemma, Non measureable sets, The Cantor set and Cantor Lebesgue function
%Chapter 2; Sections 2.1 to 2.7 (25 Hours)
%Module 2: Lebesgue Measurable Functions and Lebesgue Integration
%Sums, Products and Compositions, Sequential Pointwise Limits and Simple Approximation, The Riemann Integral, The Lebesgue Integral of a bounded measurable function over a set of finite measure, The Lebesgue Integral of a measurable non-negative function, The General Lebesgue Integral.
%Chapter 3; Sections 3.1 to 3.2, Chapter 4; Sections 4.1 to 4.4(25 Hours) 
%Module 3: General Measure Space and Measureable Functions \& Signed Measures
%Measures and Measurable Sets
%The Hanh and Jordan decompositions, The Caratheodory Measure induced by an outer measure, Measureable functions
%Chapter 17; Sections 17.1 to 17.3, Chapter 18; Section 18.1 upto corollory 7 (20 Hours)
%Module 4: Integration over General Measure Space and Product Measure
%Integration of non-negative meaurable functions, Integration of General Measurable functions, The Radon Nikodym Theorem
%The Theorems of Fubini and Tonelli
%Chapter 18; Sections 18.2 to 18.4, Chapter 20; Section 20.1 (20 Hours)

%Need to work on this
%Module 1 - \cite{royden} 2
%Module 2 - \cite{royden} 3, 4
%Module 3 - \cite{royden} 17, 18
%Module 4 - \cite{royden} 18, 20
%Missing - \cite{royden} 1, 5, 6, 7, 8, 9, 10, 11, 12, 13, 14, 15, 16, 19, 21?

\section{Lebesgue Measure}
\subsection{Introduction}
	Lebesgue(\texttt{Lay-Bay-Gu}) Measure allows us to perform an integration, \textbf{Lebesgue integration} which is different from the usual (Riemann) integration. The family of functions that are Lebesgue integrable is quite large compared to those which are Riemann integrable. And Lebesgue integration has nicer properties compared to Riemann integration.

\begin{definition}[$\sigma$ algebra]
	Let $X$ be a set. A family $\mathcal{A}$ of subset of $X$ is a $\sigma$-algebra (\texttt{sigma algebra}) if
	\begin{enumerate}
		\item $\mathcal{A}$ contains the empty set, $\phi \in \mathcal{A}$
		\item $\mathcal{A}$ contains the complement of each subset in it.
			$$E \in \mathcal{A} \iff X-E \in \mathcal{A}$$
		\item $\mathcal{A}$ is closed under countable unions
			$$\bigcup_{k = 1}^\infty E_k \in \mathcal{A},\text{ where } E_k \in \mathcal{A},\text{ for } k=1,2,\cdots$$
	\end{enumerate}
\end{definition}
\textbf{Note} : By de Morgan's Laws, every $\sigma$-algebra $\mathcal{A}$ is closed under countable intersections as well.
$$X-\bigcup_{k = 1}^\infty (X-E_k) = \bigcap_{k=1}^\infty X-(X-E_k) = \bigcap_{k=1}^\infty E_k \in \mathcal{A}$$

\subsection{Measure of a Set}
\begin{definition}[Length of an interval]
	Length is a real-valued set function. Let $I$ be a bounded interval say $[a,b)$. Then its length $l(I)=b-a$ is the difference between endpoints.
	If an interval $I$ is unbounded say $(a,\infty)$. Then its length, $l(I) = \infty$.
\end{definition}
\begin{definition}[Lebesgue Measure]
	The real-valued set function Lebesgue Measure $m$ is an extension of the length function $l$ and it has the following properties,
	\begin{enumerate}
		\item Measure of an interval is its length ($m$ is an extension of $l$)
		\item Measure is translation invariant 
	$$m(x+E) = m(E),\ \forall E \subset X,\ \forall x \in \mathbb{R}$$
		\item Measure is additive over countable disjoint unions 
	 $$m\left( \bigcup_{k=1}^\infty E_k \right) = \sum_{k=1}^\infty m(E_k), \text{ where } E_k \text{ are disjoint}$$
	\end{enumerate}
\end{definition}

\begin{definition}[Lebesgue Measurable Sets]
	The collection of all sets that are Lebesgue measurable.
\end{definition}

\textbf{Remark} : Lebesgue Measure has monotonicity. That is, $A \subseteq B \implies m(A) \le m(B)$

\textbf{Remark} In order to construct Lebesgue Measure, we start with another measure - (Lebesgue) Outer Measure by relaxing the third axiom. Then we restrict the family of set to Lebesgue Measurable Sets so that the third axiom of Lebesgue Measure is true.

\begin{definition}[Lebesgue Outer Measure]
	Lebesgue Outer measure $m^\ast$ is a set function with the following properties
	\begin{enumerate}
		\item Outer Measure of an interval is its length ($m^\ast$ is an extension of $l$)
		\item Outer Measure is translation invariant
	$$m^\ast(x+E) = m^\ast(E),\ \forall E \subset X,\ \forall x \in \mathbb{R}$$
		\item Outer Measure is countably subadditive
	 $$m^\ast \left( \bigcup_{k=1}^\infty E_k \right) \le \sum_{k=1}^\infty m^\ast (E_k) $$
	\end{enumerate}
\end{definition}

\begin{definition}[Counting Measure]
	The set function which maps sets to their cardinality.
\end{definition}

\subsection{$\sigma$-algebra of Lebesgue Measurable Sets}
