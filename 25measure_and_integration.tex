%Text Books : \cite{royden}
%Module 1: Lebesgue Measure
%Introduction, Lebesgue outer measure, The $\sigma$-algebra of Lebesgue measurable sets, Outer and inner approximation of Lebesgue measurable sets , Countable additivity, continuity and Borel-Cantelli Lemma, Non measureable sets, The Cantor set and Cantor Lebesgue function
%Chapter 2; Sections 2.1 to 2.7 (25 Hours)
%Module 2: Lebesgue Measurable Functions and Lebesgue Integration
%Sums, Products and Compositions, Sequential Pointwise Limits and Simple Approximation, The Riemann Integral, The Lebesgue Integral of a bounded measurable function over a set of finite measure, The Lebesgue Integral of a measurable non-negative function, The General Lebesgue Integral.
%Chapter 3; Sections 3.1 to 3.2, Chapter 4; Sections 4.1 to 4.4(25 Hours) 
%Module 3: General Measure Space and Measureable Functions \& Signed Measures
%Measures and Measurable Sets
%The Hanh and Jordan decompositions, The Caratheodory Measure induced by an outer measure, Measureable functions
%Chapter 17; Sections 17.1 to 17.3, Chapter 18; Section 18.1 upto corollory 7 (20 Hours)
%Module 4: Integration over General Measure Space and Product Measure
%Integration of non-negative meaurable functions, Integration of General Measurable functions, The Radon Nikodym Theorem
%The Theorems of Fubini and Tonelli
%Chapter 18; Sections 18.2 to 18.4, Chapter 20; Section 20.1 (20 Hours)

%Need to work on this
%Module 1 - \cite{royden} 2
%Module 2 - \cite{royden} 3, 4
%Module 3 - \cite{royden} 17, 18
%Module 4 - \cite{royden} 18, 20
%Missing - \cite{royden} 1, 5, 6, 7, 8, 9, 10, 11, 12, 13, 14, 15, 16, 19, 21?

\section{Lebesgue Measure}
\begin{description}
	\item[set function] A function which maps sets into (extended) real numbers.
	\item[$\sigma$-algebra] A family $\mathcal{A}$ of subsets of a nonempty set $X$ such that
	\begin{enumerate}
		\item $\mathcal{A}$ contains the empty set, 
		\item $\mathcal{A}$ contains complement of each of its members and
		\item $\mathcal{A}$ is closed under countable unions.
	\end{enumerate}
	
	From these 3 axioms, we can deduce the following,
	\begin{enumerate}
		\setcounter{enumi}{3}
		\item $\mathcal{A}$ is closed under countable intersections (by de Morgan's laws).
			$$\left( \bigcup_{k = 1}^\infty E_k^c \right)^c = \bigcap_{k=1}^\infty E_k \in \mathcal{A}$$
		\item $E,F \in \mathcal{A} \implies E-F \in \mathcal{A}$ since $E-F = E \cap F^c$
	\end{enumerate}
\end{description}

\begin{definition}[Length of an interval]
	Length is a real valued set function. Let $I$ be a bounded interval say $[a,b)$. Then its length $l(I)=b-a$ is the difference between endpoints.
	If an interval $I$ is unbounded say $(a,\infty)$, then its length, $l(I) = \infty$.
\end{definition}


\subsection{Exercise}
\subsubsection{Techniques in Measure Theory}
	Let $\mathcal{A}$ be a $\sigma$-algebra. Let Lebesgue Measure $m : \mathcal{A} \to [0,\infty]$ be countably additive over disjoint collection of sets in $\mathcal{A}$.
\begin{itemize}
	\item Lebesgue Measure $m$ has monotonicity.\\
		$A \subseteq B \implies B = A \cup (B-A)$ is a disjoint union \\
		$\implies m(B) = m(A) + m(B-A) \ge m(A)$
	\item If exists $E \in \mathcal{A}$ such that $m(E) < \infty$, then $m(\phi) = 0$\\
		Suppose $m(\phi) = c$ and $m(E) = k$ where $k < \infty$. If $c \ne 0$, then $m(E \cup \phi) = m(E) + m(\phi) = c+k > k = m(E)$ is a contradiction.
	\item $m\left(\cup_{k=1}^\infty E_k\right) \le \sum_{k=1}^\infty m(E_k)$\\
		Define $\{ F_k : k \in \mathbb{N} \}$ by $F_k = E_k - \cup_{j = 1}^{k-1} E_j$\\
		Then $F_1 = E_1$, $F_2 = E_2 - E_1$, $F_3 = E_3 - (E_1 \cup E_2)$, \dots\\
		Also $F_k \in \mathcal{A}$ and $F_k \subseteq E_k,\ \forall k \in \mathbb{N}$. Thus $m(F_k) \le m(E_k),\ \forall k$
		However, $\cup_{k=1}^\infty E_k = \cup_{k=1}^\infty F_k$\\
		$\implies m\left( \cup_{k=1}^\infty E_k \right) = m \left( \cup_{k=1}^\infty F_k \right) = \sum_{k=1}^\infty m(F_k) \le \sum_{k=1}^\infty m(E_k)$
\end{itemize}

\subsubsection{Counting Measure}
	The counting measure $c : \mathcal{A} \to [0,\infty]$ is a set function which maps sets to their cardinality. For example, if $E = \{2,3,4\}$, then $c(E) = 3$.
\begin{itemize}
	\item The counting measure is \textbf{translation invariant} since translation never increases the cardinality of the set.\\
		For example, $5+E = \{7,8,9\}$. And $m(5+E) = 3 = m(E)$.
	\item The counting measure is \textbf{countably additive} over disjoint collections since the cardinality of disjoint union of two sets is the sum of their cardinalities.
	\item However, counting measure of (non-degenerate) intervals are $\infty$ which is \textbf{not the same as their length} for bounded intervals.
\end{itemize}

\subsection{Lebesgue Outer Measure}
\begin{description}
	\item[$G_\delta$] A set which is countable intersection of open subsets.
	\item[$F_\sigma$] A set which is countable union of closed subsets.
\end{description}

\subsubsection{Set-theoretic Construction of Lebesgue Measure}
\begin{enumerate}
	\item Construct Lebesgue Outer Measure $m^\ast$ (with Axiom 3 relaxed) \\
		ie, Obtain the underlying relation of the set function
	\item Restrict $m^\ast$ to the $\sigma$-algebra of our interest \\
		ie, Choose a domain so that set function is well defined.
\end{enumerate}
\begin{definition}[Lebesgue Outer Measure]
	Let $A \subset \mathbb{R}$. Let $\mathcal{C} = \{ I_k : k \in \mathbb{N}\}$ be an open cover of $A$ such that $I_k$ are non-empty, bounded, open intervals. Consider the sum of length of intervals for such covers of $A$. (Lebesgue) Outer Measure $m^\ast(A)$ is the infimum of all such sums.
	$$ m^\ast(A) = \inf \left\{ \sum_{k=1}^\infty l(I_k) : A \subset \bigcup_{k=1}^\infty I_k \right\} $$
\end{definition}
\subsection{Properties of Lebesgue Outer Measure}
\begin{enumerate}
	\item Outer Measure of the empty set is zero\\
		Let $\epsilon > 0$. Then $\mathcal{C}_\epsilon = \{ (0,\frac{\epsilon}{2^n}) : n \in \mathbb{N} \}$ is an open cover of $\phi$ containing nonempty, bounded, open intervals. Clearly, sum of length of intervals in $C_\epsilon = \epsilon$. Suppose $m^\ast(\phi) = \delta$ and $\delta > 0$. There exists $\epsilon$ such that $0 < \epsilon < \delta$. The sum of intervals of $\mathcal{C}_\epsilon$ is less than $\delta$, which is a contradiction by the definition of Outer Measure.
	\item Outer Measure is monotone\\
		Suppose $A \subset B$. Then every cover of $B$ is also an  cover of $A$. Let $\mathcal{U}$ be the set of all open covers of $A$ with nonempty, bounded intervals and $\mathcal{V}$ be the set of all such open covers of $A$. Clearly, $\mathcal{V} \subset \mathcal{U}$. We know that, if $A \subset B$, then $\inf{B} \le \inf{A}$. Therefore, $m^\ast(A) \le m^\ast(B)$.
	\item Outer Measure of Countable Sets is zero
	\item Outer Measure of an Interval is its length
	\item Outer Measure is translation invariant
	\item Outer Measure is countably subadditive
\end{enumerate}

\subsubsection{Outer Measure of Countable Sets}
\subsubsection{Outer Measure of Intervals}
\subsubsection{Outer Measure is translation invariant}
\subsubsection{Outer Measure is countably subadditive}

\subsection{Exercise}


\subsection{$\sigma$-algebra of Lebesgue Measurable Sets}


