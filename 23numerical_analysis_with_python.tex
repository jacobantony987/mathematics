%Text books : \cite{brigs}, \cite{saha}, \cite{kiusalaas}
%Module 1:
%Defining Symbols and Symbolic Operations, Working with Expressions, Solving Equations and Plotting Using SymPy, problems on factor finder, summing a series and solving single variable inequalities
%Chapter 4 of \cite{saha} 
%Module 2:
%Finding the limit of functions, finding the derivative of functions, higher-order derivatives and finding the maxima and minima and finding the integrals of functions are to be done. in the section programming challenges, the following problems - verify the continuity of a function at a point, area between two curves and finding the length of a curve
%Chapter 7 of \cite{saha} 
%Module 3:
%Interpolation and Curve Fitting - Polynomial Interpolation - Lagrange's Method, Newton's Method and Limitations of Polynomial Interpolation, Roots of Equations - Method of Bisection and Newton-Raphson Method.
%Sections 3.1, 3.2, 4.1, 4.3, 4.5 of \cite{kiusalaas}
%Module 4:
%Gauss Elimination Method (excluding Multiple Sets of Equations), Doolittle's Decomposition Method only from LU Decomposition Methods, Numerical Integration, Newton-Cotes Formulas, Trapezoidal rule, Simpson's rule and Simpson's 3/8 rule.
%Sections 2.2, 2.3, 6.1, 6.2 of \cite{kiusalaas}

%Programs
%Online Program, Explanation, Algorithm, Function Syntax, Terminology
\part{ME010203 Numerical Analysis with Python}
\chapter{Expressions}
\chapter{Calculus}
\chapter{Interpolation \& Curve Fitting}
\begin{definition}
	Given $(n+1)$ data points $(x_k, y_k),\ 0 \le k \le n$, the problem of estimating $y(x)$ using a function $y : \mathbb{R} \to \mathbb{R}$ that satisfy the data points is the interpolation problem. ie, $y(x_k) = y_k,\ 0 \le k \le n$.
\end{definition}
\begin{definition}
	Given $(n+1)$ data points $(x_k,y_k),\ 0 \le k \le n$, the problem of estimating $y(x)$ using a function $y : \mathbb{R} \to \mathbb{R}$ that is sufficiently close to the data points is the curve-fitting problem. ie, Given $\epsilon > 0,\ |y(x_k)-y_k| < \epsilon,\ 0 \le k \le n$.
\end{definition}
\begin{remark}
	The data could be from scientific experiments or computations on mathematical models. The interpolation problem assumes that the data is accurate. But, curve-fitting problem assumes that there are some errors involved which are sufficiently small.
\end{remark}
\begin{definition}
	Given $(n+1)$ data points $(x_k,y_k),\ 0 \le k \le n$, the problem of estimating $y(x)$ using a polynomial function of degree $n$ that satisfy the data points is the polynomial interpolation problem.
\end{definition}
\begin{remark}
	Polynomial interpolation is the simplest of all interpolations.$\star$
\end{remark}

\section{Polynomial Interpolation}
There exists a unique polynomial of degree $n$ that satisfy $(n+1)$ distinct data points. There are a few methods to find this polynomial : 
	\begin{enumerate*}
		\item Lagrange's method
		\item Newton's method
		\item Neville's method
	\end{enumerate*}
\subsection{Lagrange's Method}
\begin{equation}
	P_n(x) = \sum_{i=0}^n y_i l_i(x),\text{ where } \l_i(x) = \prod_{j = 0,j \ne i}^n \frac{x-x_i}{x_j-x_i}
\end{equation}
\chapter{Matrix Operations}

