%Text Books : \cite{joshi}, \cite{munkres}
%Module 1: Separation axioms
%Compactness and Separation axioms , The Urysohn Characterisation of normality, Tietze Characterisation of normality.
%( Chapter 7: Sections 2; 2.1 to 2.10 Section 3; 3.1 to 3.6 – Proof of Lemma 3.4 excluded Section 4; 4.1 to 4.7 of \cite{joshi}) (20hours)
%Module 2: 
%Products and Co-products - Cartesian products of families of sets – The product topology -Productive properties.
%( Chapter 8 : Section 1; 1.1 to 1.9 Section 2; 2.1 to 2.8 , Section 3 – 3.1 to 3.6 of \cite{joshi}) (25hours)
%Module 3: 
%Embedding and Metrisation - Evaluation functions into products, Embedding lemma and Tychonoff Embedding, The Urysohn Metrisation Theorem.
%Variation of compactness
%( Chapter 9: Section 1; 1.1 1.5, Section 2; 2.1 to 2.5 Section, 3; 3.1 to 3.4 of \cite{joshi})
%(Chapter 11: Sections 1.1 to 1.11 of \cite{joshi}) (25 hours)
%Module 4:
%Definition and convergence of nets, Homotopy of paths.
%(Chapter 10: Section 1 of \cite{joshi})
%(Chapter 9 : Section 1 of \cite{munkres}) (20hours)

%Need to plan this topic with activities

\part{ME010202 Advanced Topology}
%Module 1 - \cite{joshi} 7
%Module 2 - \cite{joshi} 8
%Module 3 - \cite{joshi} 9
%Module 4 - \cite{joshi} 10, \cite{munkres} 9

%Warning : The results are reordered for presentation
%Obviously, forward references will occur.

\setcounter{chapter}{6}
\chapter{Separation Axioms}
\section{Compactness and Separation Axioms}
\begin{proposition}
	Let $X$ be a $T_2$ space, $x \in X$ and $F$ is a compact subset of $X$ not containing $x$. Then there exist opensets $U,\ V$ such that $x \in U$, $F \subset V$ \& $U \cap V = \phi$.
\end{proposition}
%\begin{proof}
%\end{proof}

\begin{corollary}
	A compact subset in a $T_2$ space is closed.
\end{corollary}
%\begin{proof}
%\end{proof}

\begin{corollary}
	Every map from a compact space into a $T_2$ space is closed. And its range is a quotient space of the domain.
\end{corollary}
%\begin{proof}
%\end{proof}

\begin{corollary}
	A continuous bijection from a compact space onto a $T_2$ sspace is a homeomorphism.
\end{corollary}
%\begin{proof}
%\end{proof}

\begin{corollary}
	Every continuous, one-to-one function from a compact space into a $T_2$ space is an embedding.
\end{corollary}
%\begin{proof}
%\end{proof}

\begin{theorem}
	Every compact $T_2$ space is a $T_3$ space.
\end{theorem}
%\begin{proof}
%\end{proof}

\begin{proposition}
	Let $X$ be a regular space, $C$ a closed subset of $X$ and $F$ a compact subset of $X$, such that $C \cap F = \phi$. Then there exist open sets $U,\ V$ such that $C \subset U$, $F \subset V$ and $U \cap V = \phi$.
\end{proposition}
%\begin{proof}
%\end{proof}

\begin{theorem}
	Every regular, Lindeloff space is normal.
\end{theorem}
%\begin{proof}
%\end{proof}

\begin{corollary}
	Every regular, second countable space is normal.
\end{corollary}
%\begin{proof}
%\end{proof}

\begin{corollary}
	Every compact $T_2$ space is $T_4$.
\end{corollary}
%\begin{proof}
%\end{proof}

%\begin{remark} Exercises 7.2
%	\begin{enumerate}
%		\item Prove that unit circle $S^1$ is compact.
%		\item For any map \(f : S^1 \to \mathbb{R}\) prove that there exists a point \( x_0 \in S^1 \) such that \( f(x_0) = f(-x_0) \).
%		\item Let $A,\ B$ be closed subsets of $S^1$ such that \( S^1 = A \cup B \). Prove that at least one of $A$ and $B$ contains a pair of mutually antipodal points.
%		\item Let $X$ be any infinite set with a distinguished element $\ast$. Let $\mathcal{T}$ be the topology on $X$ consisting of the empty set and all subsets of $X$ containing $\ast$. Prove that $X$ has a compact subset whose closure is not compact.
%		\item Prove that the closure of a compact subset of a regular space is compact.
%		\item Prove that the real line with the semi-open interval topology is normal.
%			---continue page 176---
%	\end{enumerate}
%\end{remark}

\section{The Urysohn Characterisation of Normality}
\begin{proposition}
	Let $A\ B$ be subsets of a space $X$ and suppose there exists a continuous function $f:X \to [0,1]$, such that $f(x)=0,\ \forall x \in A$ and $f(x)=1,\ \forall x \in B$. Then there exists disjoint open sets $U,\ V$ such that $A \subset U$ and $B \subset V$.
\end{proposition}
%\begin{proof}
%\end{proof}

\begin{corollary}
	If $X$ has the property that for any disjoint closed subsets $A,\ B$ of $X$, there exists a continuous function $f : X \to [0,1]$ such that $f(x)=0,\ \forall x \in A$ and $f(x)=1,\ \forall x \in B$, then $X$ is normal.
\end{corollary}
%\begin{proof}
%\end{proof}

\begin{theorem}
	A topological space $X$ is normal iff it has the property that for every mutually disjoint, closed subsets $A,\ B$ of $X$, there exists a continuous function \( f : X \to [0,1] \) such that \( f(x) = 0 \) for all $x \in A$ and \( f(x) = 1 \) for all \( x \in B \)
\end{theorem}
%\begin{proof}
%\end{proof}

\begin{lemma}
	Let \( f : X \to [0,1] \) be continuous. For each \( t \in \mathbb{R} \) let \( F_t  \{ x \in X : f(x) < t \} \). Then the indexed family \( \{F_t : t \in \mathbb{R} \} \) has the following properties
	\begin{enumerate}
		\item \( F_t \) is an open subset of \( X \) for each \( t \in \mathbb{R} \)
		\item \( F_t = \phi \) for \( t < 0 \)
		\item \( F_t = X \) for \( t > 1 \)
		\item For any \( s,\ t \in \mathbb{R},\ s < t \implies \overline{F_s} \subset F_t \).
	\end{enumerate}
	Moreover, for each \( x \in X,\ f(x) = \inf \{t \in \mathbb{Q} : x \in F_t \} \).
\end{lemma}
%\begin{proof}
%\end{proof}

\begin{lemma}
	Let \( X \) be a topological space and suppose \( \{ F_t : t \in \mathbb{Q} \} \) is a family of sets in \( X \) such that 
	\begin{enumerate}
		\item \( F_t \) is open in \( X \) for each \( t \in \mathbb{Q} \)
		\item \( F_t = \phi \) for \( t \in \mathbb{Q},\ t < 0 \)
		\item \( F_t = X \) for \( t \in \mathbb{Q},\ t > 1 \)
		\item \( \overline{F_s} \subset F_t \) for \( s,\ t \in \mathbb{Q},\ s < t \)
	\end{enumerate}
	For \( x \in X \), let \( f(x) = \inf \{ t \in \mathbb{Q} : x \in F_t \} \). Then \( f \) is a continuous real-valued function on \( X \) and it takes values in the unit interval \( [0,1] \).
\end{lemma}
%\begin{proof}
%\end{proof}

\begin{corollary}
	All \( T_4 \) spaces are completely regular and hence Tychonoff.
\end{corollary}
\begin{proof}
	Let \( x \in X \) and \( D \) be closed subset not containing \( x \). We have \( X \) is a \( T_4 \) space. Therefore \( X \) is \( T_1 \) as well as normal.  Now the singleton set, \( \{ x \} \) is closed, since \( X \) is a \( T_1 \) space. And by Urysohn's lemma for disjoint, closed subsets \( \{ x \},\ D \) there exists a continuous, real-valued function \( f : X \to [0,1] \) such that \( f(x) = 0 \) and \( f(y) = 1 \) for all \( y \in D \). Therefore the space \( X \) is completely regular and hence Tychonoff.
\end{proof}

\begin{remark}[Urysohn function]
	The function whose existence is asserted by Urysohn's lemma is called a Urysohn function
\end{remark}

\section{Tietze Characterisation of Normality}
\begin{proposition}
	Let \( A \) be a subset of a space \( X \) and let \( f : A \to \mathbb{R} \) be continuous. Then any two extensions of \( f \) to \( X \) agree on \( \overline{A} \). In other words, if at all an extension of \( f \) exists its values on \( \overline{A} \) are uniquely determined by values of \( f \) on \( A \).
\end{proposition}
%\begin{proof}
%\end{proof}

\begin{proposition}
	Suppose a topological space \( X \) has the property that for every closed subset \( A \) of \( X \), every continuous real valued function on \( A \) has a continuous extension to \( X \). Then \( X \) is normal.
\end{proposition}
%\begin{proof}
%\end{proof}

\begin{definition}[Pointwise Convergence]
	Let \( X \) be a topological space and \( (Y,d) \) a metric space. Then a sequence of functions \( \{ f_n \} \) from \( X \) to \( Y \) converges pointwise to \( f \) if for every \( x \in X \) the sequence \( \{ f_n(x) \} \) converges to \( f(x) \) in \( Y \).\\
	
	In other words, given a very small value, \( \epsilon > 0 \), there exists some \( \delta > 0 \) such that for every \( x \in X \) there exists \( N_x \in \mathbb{N} \). This \( N_x \) may be different for different values of \( x \) and for every \( n > N_x \), \( d(f(x),f_n(x)) < \delta \).
\end{definition}

\begin{definition}[Uniform Convergence]
	Let \( X \) be a topological space and \( (Y,d) \) a metric space. Then a sequence of functions \( \{ f_n \} \) from \( X \) to \( Y \) converges uniformly to \( f \) if given a small \( \epsilon > 0 \), there exists \( \delta > 0 \) such that there exists \( N \in \mathbb{N} \). This $N$ is independent of the value of \( x \) and for every \( n > N \), \( d(f(x),f_n(x)) < \delta \).
\end{definition}

\begin{proposition}
	Let \( X,\ (Y,d),\ \{ f_n \} \text{ and } f \) be as above and suppose \( \{ f_n \} \) converges to \( f \) uniformly. If each \( f_n \) is continuous, then \( f \) is continuous.
\end{proposition}

\begin{definition}[Uniform Convergence of Series]
	Let \( X \) be a topological space and \( (Y,d) \) be a metric space. Then a series of function \( \sum^\infty_{n = 1} f_n \) converges uniformly to \( f \) if the sequence of partial sums converges uniformly to \( f \).\\

	In other words, let \( g_m = \sum^m_{n = 1} f_n \). Then \( \sum^\infty_{n = 1} f_n \) converges to \( f \) uniformly if the sequence of partial sums \( \{ g_m \} \) converges to \( f \) uniformly.
\end{definition}

\begin{proposition}
	Let \( \sum^\infty_{n = 1} M_n \) be a convergent series of non-negative real numbers. Suppose \( \{ f_n \} \) is a sequence of real valued functions on a space \( X \) such that for each \( x \in X \) and \( n \in \mathbb{N},\ |f_n(x)| \le M_n \). Then the series \( \sum^\infty_{n = 1} f_n \) converges uniformly to a real valued function on \( X \).
\end{proposition}
---continue page 185---
\chapter{Products and Coproducts}
\section{Cartesian Products of Families of Sets}
\section{The Product Topology}
\section{Productive Properties}
%\section{Countably Productive Properties*}

\chapter{Embedding and Metrisation}
\section{Evaluation Functions into Products}
\section{Embedding Lemma and Tychonoff Embedding}
\section{The Urysohn Metrisation Theorem}

\chapter{Nets and Filters}
\section{Definition and Convergence of Nets}
\begin{definition}[Directed Set]\cite[10.1.1]{joshi}\\
	A directed set \(D\) is a pair \( (D,\ge) \) where \( D \) is a nonempty set and  \( ge \) is a binary relation on \( D \) such that
	\begin{enumerate}
		\item The relation `follows'( \( \ge \) ) is transitive. ie,  \( m \ge n,\ n \ge p \implies m \ge p \)
		\item The relation `follows'( \( \ge \) ) is reflexive. ie, For every \( m \in D,\ m \ge m \)
		\item For any \( m,n \in D \), there exists \( p \in D \) such that \( p \ge m \) and \( p \ge n \).
	\end{enumerate}
\end{definition}

\begin{description}
	\item[sequence in a set \( X \)] is a function \( f \) from the set of all integers into \( X \).
\end{description}

\begin{definition}[Net]\cite[10.1.2]{joshi}\\
	A net in a set \( X \) is a function \( S \) from a directed set \( D \) into the set \( X \).
\end{definition}

\begin{remark}
	The set \( \mathbb{N} \) together with the relation `less than or equal to'( \( \le \) ) is a directed set. Clearly, the relation `less than or equal to' is reflexive and trasitive. And the third condition is true iff every finite subset \( E \) of \( D \) has an element \( p \in E \) such that \( p \) follows each element of \( E \). This is a weaker notion compared to the well ordering principle\footnote{Well-ordering principle : Every subset of \( \mathbb{N} \) has a least element in it.}of the set of all integers. Thus \( \mathbb{N} \) is a directed set and every sequence in \( X \) is also a net in \( X \). 
\end{remark}

\begin{remark}[Significance of Net]
	A net on a set is a generalisation of `a sequence on a set' obtained by simplifying the domain of the sequence into a directed set. The notion directed set is derived by assuming a few properties of \( \mathbb{N} \).\\

	The convergence of sequence is not strong enough to characterise topologies as the limit of convergent sequences are unique for both Hausdorff and Co-countable spaces. The notion of Net allows us to differentiate between Hausdorff spaces from Co-countable spaces in terms of convergence of nets. The limit of a convergent net on a topological space is unique iff it is a Hausdorff space. ie, We have removed a few restrictions, so that we will have some convergent nets (which are obviously not sequences) with multiple limit points for Co-countable spaces.
\end{remark}

\begin{remark}
	Examples of Directed Sets
	\begin{enumerate}
		\item Let \( X \) be a topological space and \( x \in X \). Then the neighbourhood system \( \mathcal{N}_x \) is a directed set with the binary relation \( \subset \) (subset/inclusion).
			\begin{enumerate}
				\item Let \( U,\ V,\ W \) be any three neighbourhoods of \( x \in X \) such that \( U \subset V \) and \( V \subset W \). Then, clearly \( U \subset W \).\\ Therefore, \( U \ge V,\ V \ge W \implies U \ge W \).
				\item Let \( U \) be any neighbourhood of \( x \in X \), then \( U \subset U \).\\ Therefore, \( U \ge U \).
				\item Let \( U,V \) be any two neighbourhoods of \( x \in X \), then there exists their intersection \( W = U\cap V \), which is a neighbourhood of \( x \). Clearly \( W \subset U \) and \( W \subset V \).\\ Therefore \( \forall U,V \in \mathcal{N}_x, \exists W \in \mathcal{N}_x \) such that \( W \ge U \) and \( W \ge V \).
			\end{enumerate}
		\item Let \( \mathcal{P} \) be the set of all partitions on closed unit interval \( [0,1] \). A partition \( P \in \mathcal{P} \) is a refinement of \( Q \in \mathcal{P} \) if every subinterval in \( P \) is contained in some subinterval of \( Q \). Then \( \mathcal{P} \) with the binary relation refinement is a directed set.\\

			For example, let \( P = \{ 0,\ 0.3,\ 0.7,\ 1 \} \). Then the subintervals in \( P \) are \( [0,0.3] \), \( [0.3,0.7] \) and \( [0.7,1] \). Let \( Q = \{ 0,\ 0.3,\ 0.5,\ 1 \} \) and \( R = \{ 0,\ 0.3,\ 0.5,\ 0.7,\ 1 \} \). Then \( R \) is a refinement of \( P \), but \( Q \) is not a refinement of \( P \) since there is a subinterval \( [0.5,1] \) in \( Q \) which is not properly contained in any subinterval of \( P \). However, \( R \) is a refinement of \( Q \) as well.
			\begin{enumerate}
				\item Suppose \(P,\ Q,\ R \)  are three partitions of \( [0,1] \) such that \( P \) is a refinement of \( Q \) and \( Q \) is a refinement of \( R \), then clearly \( P \) is a refinement of \( R \) since each subinterval of \( P \) is contained some subinterval of \( Q \), which is contained in some subinterval of \( R \).\\
					Therefore, \( P \ge Q,\ Q \ge R \implies P \ge R \)
				\item Suppose \( P \) is a partition of \( [0,1] \). Then trivialy, \( P \) is a refinement of itself since every subinterval of \( P \) is contained in the same subinterval of \( P \).\\
					Therefore, \( \forall P \in \mathcal{P},\ P \ge P \)
				\item Suppose \( P,\ Q \) be any two partition of \( [0,1] \). Then \( R = P \cup Q \) is a refinement of both the partitions.\\
					Therefore \( \forall P,Q \in \mathcal{P},\ \exists R \in \mathcal{P} \) such that \( R \ge P \) and \( R \ge Q \)
			\end{enumerate}
	\end{enumerate}
\end{remark}

\begin{remark}
	Examples of Nets
	\begin{enumerate}
		\item Let \( X \) be a topological space and \( x \in X \). Let \( \mathcal{N}_x \) be the set of all neighbourhoods of \( x \). Let \( D = (\mathcal{N}_x,X) \) be the directed set given by \( (N,y) \in (\mathcal{N}_x,X) \) if \( N \in \mathcal{N}_x \) and \( y \in N \) and \( (N,y) \ge (M,z) \) if \( N \subset M \). Then the function \( S : (\mathcal{N}_x,X) \to X \) given by \( S(N,y) = y \) is a net on \( X \).\\

			For example, let $X = \{ a,b,c,d \}$ and $\mathcal{T} = \{ \{a\},\ \{a,b\},\ \{a,b,c\},\ \{a,b,c,d\} \}$. Also let $S : (\mathcal{N}_b,X) \to X$ defined by $S(N,y) = y$. Suppose $C = \{a,b,c\}$. Then $C \in \mathcal{N}_b$. ie, $C$ is a neighbourhood of $b$. Then $S(C,c) = c$.
		\item Riemann Net - Let $D = (\mathcal{P},\xi)$ where $\mathcal{P}$ is the set of all partitions on $[0,1]$ and $\xi$ is a finite sequence in $[0,1]$ such that consecutive terms belongs to consecutive subintervals of the partition. The set $(\mathcal{P},\xi)$ is directed set with $\ge$ given by $(P,\eta) \ge (Q,\psi)$ iff $P$ is a refinement of $Q$.\\
			
			For example, let $P \in \mathcal{P}$ is given by $P = \{\ 0,\ 0.3,\ 0.7,\ 1\ \}$ and $\eta = \{\ 0.2,\ 0.6,\ 0.9\ \}$. Then $(P,\eta) \in (\mathcal{P},\xi)$.\\

			Let $f : \mathbb{R} \to \mathbb{R}$ be any function, then the function,
			$$S : (\mathcal{P},\xi) \to \mathbb{R} \text{ defined by } S(P,\eta) = \sum_{j=1}^k f(\eta_k)(a_k-a_{k-1})$$
			where $P = \{ a_0,\ a_1,\cdots,\ a_k \ \}$ is the Riemann Net with respect to the real function $f$.\\

			For example, let $f(x) = x^2$ and $P,\eta$ are same as above example, then $S(P,\eta) = 0.2^2(0.3-0) + 0.6^2(0.7-0.3) + 0.9(1-0.7) = 3.99$
	\end{enumerate}
\end{remark}
	
\begin{definition}[Convergence of a Net]\cite[10.1.3]{joshi}\\
	A net $S:D \to X$ converges to a point $x \in X$ if for any nbd $U$ of $x$, there exists $m \in D$ such that $n \in D,\ n \ge m \implies S(n) \in U$. And $x$ is a limit of the net $S$.
\end{definition}

\begin{remark}
	The choice of $m$ depends on the choice of neighbourhood $U$.
	$$S: D \to X,\ S \to x \iff \left( \forall U \in \mathcal{N}_x,\ \exists m_U \in D,\text{ such that } n \ge m_U \implies S(n) \in U \right)$$
\end{remark}

\begin{theorem}[Net characterisation of Hausdorff space]\cite[10.1.4]{joshi}\\
	A topological space is Hausdorff iff limits of all nets in it are unique.
\end{theorem}
\begin{proof}
	Let $X$ be a  Hausdorff space. Suppose $S : D \to X$ is net on $X$ such that $S$ converges to two distinct points $x,y \in X$.
	Since $X$ is a Hausdorff space and $x \ne y$, there exists open sets $U,V$ such that $x \in U,\ y \in V, U \cap V = \phi$.\\

	The net $S$ converges to $x \in X$, therefore $\exists m_x \in D$ such that $n \ge m_x \implies S(n) \in U$
	And, the net $S$ converges to $y \in X$, therefore $\exists m_y \in D$ such that $n \ge m_y \implies S(n) \in V$.\\

	Since $D$ is a directed set and $m_x, m_y \in D$, there exists $p \in D$ such that $p \ge m_x$ and $p \ge m_y$. Now, $n \ge p \implies n \ge m_x,\ n \ge m_y$, since $\ge$ is transitive. (ie, $n \ge p,\ p \ge m_x \implies n \ge m_x$, and $n \ge p,\ p \ge m_y \implies n \ge m_y$).\\

	We have $n \ge p \implies n \ge m_x$ and $n \ge m_x \implies S(n) \in U$. Therefore, $n \ge p \implies S(n) \in U$. Similarly, $n \ge p \implies n \ge m_y \implies S(n) \in V$. Therefore $S(n) \in U \cap V$ which is a contradiction, since $U \cap V = \phi$. Therefore, if a net $S$ converges to two points $x,y$, then $x = y$. That is, if a net $S$ in a Hausdorff space $X$ is convergent then its limit is unique.\\

	Conversely, suppose that $X$ is a topological space and every convergent net in $X$ has a unique limit. Suppose $X$ is not a Haudorff space. Then there exists at least two distinct points $x,y \in X$ such that every neighbourhood of $x$ intersects with every neighbourhood of $y$. Now consider the set $D = \mathcal{N}_x \times \mathcal{N}_y$ and relation $\ge$ on $D$ such that $(U_1,V_1) \ge (U_2,V_2)$ if $U_1 \subset U_2$ and $V_1 \subset V_2$.\\

	By the axiom of choice, a function $S : D \to X$ such that $S(U,V) \in U\cap V$ is well defined, since every nbd of $x$ intersects every nbd of $y$. Thus, $S$ is a net in $X$. We claim that $S$ converges to both $x$ and $y$.\\

	Let $U$ be a nbd of $x$. Then $S(U',V') \in U' \cap V'$. We have $(U,X) \in D$ such that $(U',V') \ge (U,X) \implies U' \subset U$. Then, $S(U',V') \in U' \cap V' \subset U \cap X = U$. Thus, for any nbd $U$ containing $x$, we have $(U,X) \in D$ such that $(U',V') \ge (U,X) \implies S(U',V') \in U$. Therefore, $S$ converges to $x \in X$.\\
	
	Similarly, Let $V$ be a nbd of $y$. Then for any nbd $V$ containing $y$, we have $(X,V) \in D$ such that $(U',V') \ge (X,V) \implies S(U',V') \in V$, since $S(U',V') \in U' \cap V' \subset X \cap V = V$. Therefore, $S$ converges to $y \in X$ as well, where $x \ne y$. This is a contradition to the assumption that every convergent net in $X$ has a unique limit. Therefore, for any two points $x,y \in X$, there should be some nbd of $x$ that doesn't intersect some nbd of $y$. Therefore, $X$ is a Hausdorff space.
\end{proof}

\begin{definition}[Eventual Subset]\cite[10.1.5]{joshi}\\
	A subset $E$ of a directed set $D$ is an eventual subset of $D$ if there exists $m \in D$ such that $n \ge m \implies n \in E$.
\end{definition}

\begin{remark}
	Let $E$ be an eventual subset of $D$ such that $n \ge m \implies n \in E$. Then $p \in E \not\!\!\!\implies p \ge m$. ie, Subset $E$ may contain elements that doesn't follow the above $m$.
\end{remark}

\begin{remark}\cite[10.1.6]{joshi}\\
	Let $E$ be an eventual subset of $D$, then $E$ is a directed set.
	\begin{enumerate}
		\item $m,n,p \in E,\ m \ge n,\ n \ge p \implies m,n,p \in D,\ m \ge n,\ n \ge p \implies m \ge p$
		\item $m \in E \implies m \in D \implies m \ge m$
		\item $m,n \in E \implies m,n \in D \implies \exists p \in D$ such that $p \ge m$ and $\ \ge n$.\\
			Since $E$ is eventual, $\exists m' \in D$ such that $n' \ge m' \implies n' \in E$.\\
			Also, $p,m' \in D$, $\exists p' \in D$ such that $p' \ge p$ and $p' \ge m'$.(Condition 3)\\
			Since, $E$ is eventual subset of $D$ with respect to $m'$, $p' \ge m' \implies p' \in E$.\\
			And since $p' \ge p,\ p \ge m \implies p' \ge m$ and $p' \ge p,\ p \ge n \implies p' \ge n$.\\
			Therefore $\forall m,n \in E,\ \exists p' \in E$ such that $p' \ge m$ and $p' \ge n$.
	\end{enumerate}
\end{remark}

\begin{definition}[Net eventually in $A$]\cite[10.1.5]{joshi}\\
	Let $S : D \to X$ be a net in a topological space $X$. Then $S$ is eventually in subset $A$ of $X$ if $S^{-1}(A)$ is an eventual subset of $D$.
\end{definition}

\begin{remark}
	Let $S : D \to X$ be a net in $X$. Then $S$ converges to $x \in X$ if $S$ is eventually in each nbd $U$ of $x$.
\end{remark}

\begin{definition}[Cofinal subset]\cite[10.1.7]{joshi}\\
	A subset $F$ of a directed $D$ is a cofinal subset of $D$ if for any $m \in D$, there exists $n \in F$ such that $n \ge m$.
\end{definition}

\begin{remark}
	Let $X$ be a topological space and $x \in X$. Let $\mathcal{N}_x$ be the set of all neighbourhood of $x$ and $\mathcal{L}$ be a local base of $X$ at $x$. We have, $(\mathcal{N}_x,\ge)$ is a directed set where $\forall U,V \in \mathcal{N}_x,\ U \ge V \iff U \subset V$, then $\mathcal{L}$ is cofinal in $\mathcal{N}_x$.
\end{remark}

\begin{remark}\cite[10.1.8]{joshi}\\
	Let $F$ be a cofinal subset of $D$, then $F$ is a directed set.
	\begin{enumerate}
		\item $m,n,p \in F,\ m \ge n,\ n \ge p \implies m,n,p \in D,\ m \ge n,\ n \ge p \implies m \ge p$
		\item $m \in F \implies m \in D \implies m \ge m$
		\item $m,n \in F \implies m,n \in D \implies \exists p \in D$ such that $p \ge m$ and $\ \ge n$.\\
			Since $E$ is cofinal, $p \in D \implies \exists p' \in F$ such that $p' \ge p$.\\
			And since $p' \ge p,\ p \ge m \implies p' \ge m$ and $p' \ge p,\ p \ge n \implies p' \ge n$.\\
			Therefore $\forall m,n \in F,\ \exists p' \in F$ such that $p' \ge m$ and $p' \ge n$.
	\end{enumerate}
\end{remark}

\begin{definition}[Net frequently in $A$]\cite[10.1.7]{joshi}\\
	Let $S : D \to X$ be a net in a topological space $X$. Then $S$ is frequently in subset $B$ of $X$ if $S^{-1}(B)$ is a cofinal subset of $D$.
\end{definition}

\begin{proposition}\cite[10.1.6]{joshi}\\
	Let $S : D \to X$ be a net in a topological space $X$. Let $E$ be an eventual subset of $D$. Then, $S$ converges to $x$ iff $S_{/_E}$ converges to $x$.\cite[10.1.6]{joshi}
\end{proposition}
\begin{proof}
	Let $S : D \to X$ be a net in $X$, $E$ be an eventual subset of $D$, and $x \in X$. Then, $S_{/_E} : E \to X$ is defined by $n \in E \implies S_{/_E}(n) = S(n)$\\
	
	Suppose $S$ converges to $x$. Let $U$ be a nbd of $x$, then $S$ is eventually in $U$. ie, $S^{-1}(U)$ is an eventual subset of $D$. Then $\exists m \in D$ such that $n \ge m \implies n \in S^{-1}(U) \implies S(n) \in U$. Since set $E$ is eventual subset of $D$, $\exists m' \in D$ such that $n \ge m' \implies n \in E$.\\
	
	Since $E$ is a directed set, $S_{/_E} : E \to X$ is a net in $X$. And $m,m' \in D \implies \exists p \in D$ such that $p \ge m$ and $p \ge m'$. We have, $p \ge m' \implies p \in E$. And $n \ge' p \implies n \ge p,\ p \ge m \implies n \ge m \implies S(n) \in U \implies S_{/_E}(n) \in U$. Therefore, $n \ge' p \implies S_{/_E}(n) \in U$. Since $U$ is arbitrary, $S_{/_E}$ converges to $x$.\\

	Conversely, suppose that $S_{/_E}$ converges to $x$. Let $U$ be a nbd of $x$, then $S_{/_E}$ is eventually in $U$. ie, $S_{/_E}^{-1}(U)$ is an eventual subset of $D$. ie, $\exists m \in D$ such that $n \ge m \implies n \in S_{/_E}^{-1}(U) \implies S_{/_E}(n) \in U \implies S(n) \in U$. Therefore, $n \ge m \implies S(n) \in U$. Since, $U$ is arbitrary, $S$ converges to every nbd of $x$. ie, $S$ converges to $x$.
\end{proof}

\begin{proposition}\cite[10.1.8]{joshi}\\
	Let $S : D \to X$ be a net in a topological space $X$. Let $F$ be a cofinal subset of $D$. If $S$ converges to $x$, then $S_{/_F}$ converges to $x$.
\end{proposition}
\begin{proof}
	Let $S : D \to X$ be a net in $X$ and $S$ converges to $x \in X$. Also let $F$ be a cofinal subset of $D$. Then $S_{/_F}$ is also a net in $X$, since $(F,\ge')$ is a directed set where $\forall m,n \in F,\ m \ge n \implies m \ge' n$.\\

	Since $S$ converges to $x$, for any nbd $U$ of $x$, $\exists m \in D$, such that $n \ge m \implies S(n) \in U$. Since $F$ is cofinal, $\exists p \in F$ such that $p \ge m$. Thus $n \ge' p \implies n \ge p,\ p \ge m \implies n \ge m \implies S(n) \in U \implies S_{/_F}(n) \in U$. Therefore, $\exists p \in F$ such that $n \ge' p \implies S_{/_F}(n) \in U$. Since $U$ is arbitrary, $S_{/_F}$ is eventually in every nbd of $x$. ie, $S_{/_F}$ converges to $x$.
\end{proof}

\begin{remark}
	But converse of the above is not true. $S_{/_F}$ converges to $x$ does not imply that $S$ converges to $x$, since cofinal subset $F$ not necessarily contain every element following a particular $m$.
\end{remark}

\begin{definition}[Cluster point]\cite[10.1.9]{joshi}\\
	Let $S : D \to X$ be a net in a topological space $X$. Then $x \in X$ is a cluster point of $S$, if $S$ is frequently in each nbd $U$ of $x$ in $X$.
\end{definition}

\begin{proposition}\cite[10.1.10]{joshi}\\
	Let $S : D \to X$ be a net in a topological space $X$. Then $x \in X$ is a cluster points of $X$, if $S_{/_F}$ converges to $x$ for some cofinal subset $F$ of $D$.
\end{proposition}
\begin{proof}
	Let $S : D \to X$ be a net in $X$ and $(F,\ge')$ be a cofinal subset of $(D,\ge)$. Then $S_{/_F}$ is also a net in $X$. Suppose $S_{/_F}$ converges to $x \in X$. Let $U$ be a nbd of $x$, then $\exists m \in F$ such that $n \ge' m \implies S_{/_F}(n) \in U$.\\

	Let $m' \in D$. Then $\exists p' \in F$ such that $p' \ge m'$, since $F$ is a cofinal subset of $D$. We have, $m,p' \in F$, then $\exists p \in F$ such that $p \ge' m$ and $p \ge' p'$. Since $F \subset D$, we have $p,m \in F \implies p,m \in D$ and $p \ge' m \implies p \ge m$.\\
	
	Also $p \ge' m \implies S_{/_F}(p) \in U \implies S(p) \in U$. Therefore, $\forall m' \in D,\ \exists p \in D$ such that $p \ge m'$ and $S(p) \in U$. Since $U,m'$ are arbitrary, $S$ is frequently in every nbd of $x$. ie, $x$ is a cluster point of $S$.
\end{proof}

\begin{definition}[Subnet]\cite[10.1.11]{joshi}\\
	Let $S : D \to X$ be a net in a topological space $X$. Then a net $T : E \to X$ in $X$, is a subnet of $S$ if there exists a function $N : E \to D$ such that $S \circ N = T$ and $\forall m \in D,\ \exists p \in E$ such that $n \ge' p \implies N(n) \ge m$.
\end{definition}

\begin{remark}
	A net $T : E \to X$ is a subnet of $S : D \to X$ if $\exists N : E \to D$ such that $S \circ N = T$ and $S$ is frequently in $T(E)$.\\
	A net $T : E \to X$ is a subnet of $S : D \to X$. If $T$ eventually in $A$ subset of $X$, then $S$ is frequently in $A$.
\end{remark}

\begin{proposition}\cite[10.1.12]{joshi}\\
	Let $S : D \to X$ be a net in a topological space $X$. Then $x \in X$ is a cluster point of $S$ iff there exists a subnet of $S$ which converges to $x$.
\end{proposition}
\begin{synopsis}
	Let $(D,\ge)$, $(E,\ge')$ be two directed sets.\\

	If $T$ converges to $x$, then $T$ is eventually in each nbd $U$ of $x$. And since $T$ is a subnet of $S$, there exists $N : E \to D$ such that $N(E)$ is a cofinal subset of $D$. Therefore, $S$ is frequently in each nbd $U$ of $x$. Thus, $x$ is a cluster point of $S$.
\end{synopsis}
\begin{proof}
	Let $S : D \to X$ be a net in $X$. Suppose there exists a subnet $T : E \to X$ that converges to $x \in X$. By the definition of subnet, we have $\exists N : E \to D$ such that $S \circ N = T$ and $S$ is frequently in $T(E)$.\\

	We have, $T$ convergers to $x$, thus for any neighbourhood $U$ of $x$, there exists $m' \in E$ such that $n' \ge' m' \implies T(n') \in U$.\\

	Also we have, $T$ is a subnet of $S$. Then $\exists N : E \to D$ such that
	$\forall m \in D,\ \exists p' \in E$ such that $n' \ge' p' \implies N(n') \ge m$.\\

	Now, for any $m \in D$, we have $m',p' \in E$. Since $E$ is a directed set, there exists $n' \in E$ such that $n' \ge' m'$ and $n' \ge' p'$.\\

	Then, $n' \ge m' \implies T(n') \in U$ and $n' \ge' p' \implies N(n') \ge m$.\\
	
	Thus for any $m \in D$, there exists $N(n') = n \in D$ such that $S(n) = S(N(n')) = T(n') \in U$.\\

	Thus $S$ is frequently in any neighbourhood $U$ of $x$. Therefore, $x$ is a cluster point of $S$.\\

	Conversely, suppose that $x$ is a cluster point of $S$. We have to construct a directed set $(E,\ge')$ and a function $N : E \to D$ such that $T$ is a subnet of $S$ and $T$ converges to $x$.\\

	Consider $E = D \times \mathcal{N}(x)$ and define $\ge'$ by $(n,U) \ge' (m,V)$ if $n \ge m$ and $U \subset V$. Trivially, $(n,U) \ge' (m,V) \ge' (p,W) \implies (n,U) \ge' (p,W)$ and $(n,U) \ge' (n,U)$. Also, for any $(n,U),(m,V) \in E$, we have $n,m \in D$ and $U,V \in \mathcal{N}(x)$. Since $D$ is a directed set, $\exists p \in D$ such that $p \ge n$ and $p \ge m$. And $U \cap V \in \mathcal{N}(x)$ such that $U \cap V \subset U$ and $U \cap V \subset V$. Thus $\exists (p,U\cap V) \in E$ such that $(p,U\cap V) \ge' (n,U)$ and $(p,U\cap V) \ge' (m,V)$. Therefore, $(E,\ge')$ is a directed set.\\

	Define $N : E \to D$ by $N(n,U) = n$. Again for any $(m,V) \in E$, there exists $m \in D$ such that $(n,U) \ge' (m,U)$ implies there exists $n \in D$ such that  $N(n,U) = n$ and $n \ge m$. Now, we have $T : E \to X$ defined by $T(n,U) = S(N(n,U)) = S(n)$. Therefore, $T$ is a subset of $S$.
-- to be continued page 234--
\end{proof}

\begin{remark}
	A proof that doesn't work : If $x$ is a cluster point of a net $S$ in $X$, then $S$ is frequently in some cofinal subset of $D$. Thus, there exists a cofinal subset $E \subset D$ which is a direct set with $\ge$ restricted to $E$. Then $N : E \to D$ defined by $N(n)=n$ gives a subset $T : E \to X$ of $S$. However, this subnet need not converge to $x$. The strongest statement, we can make on $T$ is that `$x$ is a cluster point of $T$'.\\

	$N : D \times \mathcal{N}(x) \to D,\ N(n) = n$ is completely independent of $U$.
\end{remark}

%\section{Topology and Convergence of Nets*}
%\section{Filters and Their Convergence*}
%\section{Ultrafilters and Compactness*}

\chapter{Compactness}
\section{Variations of Compactness}
	In this chapter, we have two other notions of compactness - countable compactness and sequential compactness.\footnote{For $\mathbb{R}$, Compactness \& Sequentially compactness are equivalent to the completeness axiom.}

	\begin{description}
		\item[Compact] A topological space is compact iff every open cover of it has a finite subcover. (\cite[6.1.1]{joshi}) [Heine-Borel]
		\item[Countably Compact] A topological space is countably compact iff every countable, open cover of it has a finite subcover. \cite[11.1.1]{joshi}
		\item[Sequentially Compact] A topological space is sequentially compact iff every sequence in it has a convergent subsequence. \cite[11.1.8]{joshi} [Bolzano-Weierstrass]
	\end{description}

	Countable compactness is a weaker notion compared to compactness.\footnote{Every compact space is countably compact.} However, sequentially compact and compact are not necessarily comparable.\footnote{$\mathcal{T}_1, \mathcal{T}_2$ are non-comparable, if $\mathcal{T}_1 \not\subset \mathcal{T}_2$ and $\mathcal{T}_2 \not\subset \mathcal{T}_1$.\cite[4.2.1]{joshi}}.\\

	We have seen earlier that compactness has the following properties
\begin{enumerate*}
	\item compactness is weakly hereditary.\cite[6.1.10]{joshi}
	\item compactness is preserved under continuous functions.\cite[6.1.8]{joshi}
	\item every continuous real functions on compact space is bounded and attains its extrema.\cite[6.1.6]{joshi}
	\item every continuous real function on a compact, metric space is uniformly continuous by Lebesgue covering lemma.\cite[6.1.7]{joshi}
\end{enumerate*}\\

	Countably compact spaces, Sequentially compact spaces have all the four properites listed above.

\subsection{Countable compactness}
\subsubsection{Weakly hereditary property}
	A subspace $(A,\mathcal{T}_{/_A})$ being countably compact doesn't imply that $(X,\mathcal{T})$ is countably compact. However, if $(X,\mathcal{T})$ is a countably compact space and \textsf{$A$ is a closed subset of $X$}, then $(A,\mathcal{T}_{/_A})$ is also a countably compact space. In other words, countably compactness is weakly hereditary.

\begin{theorem}
	Countable compactness is weakly hereditary.\cite[11.1.3]{joshi}
\end{theorem}
\begin{synopsis}
	Let $A$ be a closed subset of countably compact space, $X$. If $A$ has a countable open cover $\mathcal{U}$, then we can obtain a respective countable, open cover for $X$ by attaching $X-A$ to the extensions of members of $\mathcal{U}$ to $X$. This cover has a finite subcover. Then restricting them to $A$, we get a finite subcover of $\mathcal{U}$.
\end{synopsis}
\begin{proof}
	Suppose $X$ is a countably compact space. And $A$ is a closed subset of $X$. We need to show that $A$ is countably compact. Without loss of generality,\footnote{Suppose $A$ is not a proper subset of $X$. Then $X = A$ and $A$ is countably compact.} assume that $A$ is a proper subset of $X$. Then $X-A$ is a non-empty, open subset of $X$.\\ 

	Let $\mathcal{U}$ be a countable open cover of $A$. Then $\mathcal{U} = \{ U_1, U_2, \cdots \}$ where each element $U_k \in \mathcal{U}$ is an open subset of $A$. Since $A$ is a subspace of $X$, every open set $U_k$ in $A$ is of the form $G \cap A$ for some open set $G$ in $X$. Therefore, there exists open sets $V(U_k)$ for each $U_k$ such that $A \cap V(U_k) = U_k$.\footnote{Relative topology,$\mathcal{T}_{/_A} = \{ G \cap A : G \in \mathcal{T} \}$}\\

	Define $\mathcal{V} = \{ X-A, V(U_1), V(U_2), \cdots \}$. Clearly, $\mathcal{V}$ is a countable open cover\footnote{$X-A$ is open in $X$. If $y \not\in A$, then $y \in X-A$. If $y \in A$, then $y \in U_k$ for some $k$.} of $X$. We have $X$ is countably compact, thus $\mathcal{V}$ has a finite subcover, say $\mathcal{V}'$. Without loss of generality assume\footnote{Otherwise, you will have to consider two cases: $X-A \in \mathcal{V}'$ and $X-A \not\in \mathcal{V}'$} that $X-A \in \mathcal{V}'$. Suppose $X-A \not\in \mathcal{V}'$, then we can define another finite subcover $\mathcal{V}' \cup \{X-A\}$. Thus $\mathcal{V}' = \{ X-A,\ V(U_{n_1}),\ V(U_{n_2}),\cdots,\ V(U_{n_k})\}$.\\

	Then the corresponding subcover $\mathcal{U}'=\{U_{n_1},U_{n_2},\cdots,U_{n_k}\}$ is a finite subcover of $\mathcal{U}$. Since countable open cover $\mathcal{U}$ and closed subset $A$ are arbitrary, every closed subset of $X$ with relative topology is countably compact. Therefore, countable compactness is weakly hereditary.
\end{proof}

\begin{remark}
	Proof depends on the following,
	\begin{enumerate}
		\item There is an extension map, $\psi : P(A) \to P(X)$ that preserve open sets (and closed sets). This $\psi$ is an open map which not a true inverse of the restriction, $r : P(X) \to P(A)$, defined by $r(G) = G \cap A$ for every subset $G$ of $X$.
		\item Also we rely on the subset $A$ being closed. Suppose $X$ have many countable open covers, but $X$ has only uncountable open covers corresponding to a particular countable open cover of $A$. In such a case, $X$ being countably compact is insufficient for $A$ to be countably compact.
	\end{enumerate}
\end{remark}

\subsubsection{The behaviour of countinous functions}
	We will now study the nature of continuous functions defined on countably compact spaces. Suppose $X,Y$ are topological space and function $f : X \to Y$ is continuous. If $X$ is countably compact, then $f(X)$ is also countably compact. Continuous images of countably compact spaces are countably compact. In other words, countable compactness is preserved under continuous functions.\footnote{A topological property is preserved under continuous functions if whenever a space has that property so does every continuous image of it.\cite[6.1.9]{joshi}}
	
\begin{theorem}
	Countable compactness is preserved under continuous functions.\cite[11.1.2]{joshi}
\end{theorem}
\begin{synopsis}
	Let $X$ be countably compact and $f:X\to Y$ be continuous. Suppose $\mathcal{U}$ is a countable cover of $f(X)$, then $X$ has a countable cover $\mathcal{V}$ obtained by taking inverse images. Since $X$ is countably compact, $\mathcal{V}$ has a finite subcover $\mathcal{V}'$. Now taking images of members of $\mathcal{V}'$, we get a finite subcover $\mathcal{U}'$ of $f(X)$.
\end{synopsis}
\begin{proof}
	Suppose $X$ is a countably compact space, $Y$ is a topological space and $f:X \to Y$ is a continuous function. Let $\mathcal{U} = \{ U_1, U_2,\cdots \}$ be a countable cover of $f(X)$ by set open in $f(X)$. We have to show that $\mathcal{U}$ has a finite subcover.\\

	Define $\mathcal{V} = \{ f^{-1}(U_1), f^{-1}(U_2), \cdots \}$. Then $\mathcal{V}$ is a countable open cover of $X$, since $f^{-1}(U_k)$ are open subsets of $X$ and,

\begin{align*}
	\bigcup_{k = 1}^\infty U_k = f(X) \implies & f^{-1}\left(\bigcup_{k=1}^\infty U_k\right) = X\\
	\implies & \bigcup_{k = 1}^\infty f^{-1}(U_k) = X
\end{align*}

	We have, $\mathcal{V}$ is a countable open cover of $X$, which is a countably compact space. Therefore $\mathcal{V}$ has a finite subcover $\mathcal{V}' = \{ f^{-1}(U_{n_1}),\ f^{-1}(U_{n_2}),\cdots,\ f^{-1}(U_{n_k}) \}$.

\begin{align*}
	\bigcup_{j=1}^k f^{-1}(U_{n_j}) = X \implies & f^{-1}\left(\bigcup_{j=1}^k U_{n_j}\right) = X\\
	\implies & \bigcup_{j=1}^k U_{n_j} = f(X)
\end{align*}

	Clearly $\mathcal{U}' = \{ U_{n_1},U_{n_2},\cdots,U_{n_k}\}$ is a finite subcover of $\mathcal{U}$. Thus every countable open cover of $f(X)$ by sets open in $f(X)$ has a finite subcover. Therefore, continuous images of countably compact spaces are countably compact.
\end{proof}

\begin{remark}
	\begin{enumerate}
		\item For a continuous function, $f : X \to Y$ the inverse images of open sets are open in $X$. The relation $f^{-1} \subset f(X) \times X$ is not a function. However, we may consider a function, $\psi : P(Y) \to P(X)$ such that $\psi(U) = f^{-1}(U)$ for any subset $U$ of $Y$. This $\psi$ is an open map which maps open subsets of $Y$ to open subsets of $X$.
	\end{enumerate}
\end{remark}

\begin{theorem}
	Every continuous, real-valued function on a countably compact, metric space is bounded and attains its extrema.\cite[11.1.7]{joshi}
\end{theorem}
\begin{synopsis}
	Let $X$ be a countably compact space and function $f : X \to \mathbb{R}$ be continuous. Then $f(X) \subset \mathbb{R}$ is countably compact. Real line $\mathbb{R}$ is metrisable\footnote{\cite[4.2 Example 4]{joshi}, $\mathbb{R}$ with usual metric $d:R \to R,\ d(x,y) = |x-y|$}. Then $f(X)$ is countably compact, metric space. Therefore $f(X)$ compact.\footnote{\cite[11.1.11]{joshi} On metric spaces, countable compactness $\implies$ compactness.}. The subset $f(X)$ of $\mathbb{R}$ is bounded and closed, since every compact subset of $\mathbb{R}$ is bounded and closed. Thus $f(X)$ contains its supremum and infimum. Therefore, $f$ is bounded and attains its extrema.
\end{synopsis}
\begin{proof}
	Let $X$ be a countably compact space and $f : X \to \mathbb{R}$ be continuous, real-valued function on the countably compact space, $X$. We have to show that $f$ is bounded and attains its extrema.\\
	
	Since countable compactness is preserved under continuous functions, $f(X)$ is countably compact subset of $\mathbb{R}$. Since, $f(X)$ is a subset of the metric space, $\mathbb{R}$ and metrisability is hereditary, $f(X)$ is again metrisable. (suppose) We have, every countably compact, metric space is compact. Then $f(X)$ is a compact subset of $\mathbb{R}$.\\

	Since every compact subset of $\mathbb{R}$ is bounded and closed, $f(X)$ is bounded and closed. Since every closed subset of $\mathbb{R}$ contains supremem and infimum, $f(X)$ contains its extrema. Therefore, every continuous, real-valued function on a countably compact space is bounded and attains its extrema.\\

	We have assumed that every countably compact, metric space is compact. This result will be proved in the last section of this chapter.
\end{proof}

\begin{remark}
	Since countably compact, metric spaces are compact. The above theorem can be used to prove that continuous, real-valued functions on a compact, metric space attains its extrema.
\end{remark}

Due to the Lebesgue covering lemma, next result is quite simple.$^\star$

\begin{theorem}
	Every continuous, real-valued function on a countably compact, metric space is uniformly continuous.
\end{theorem}
%\begin{proof}
%Let $f : X \to Y$ be a continuous function on countaby compact, metric space $X$. Then $f(X)$ is a countably compact, metric space and thus a compact, metric space. Let $\mathcal{U}$ be an open cover of $f(X)$. Then by Lebesgue covering lemma, there exists a positive real number $r$ such that for any $y \in f(X)$, the open ball $B(y,r) \subset U$ for each $U \in \mathcal{U}$. Then $\mathcal{V} = \{ f^{-1}(U) : U \in \mathcal{U} \}$ is an open cover of $X$ and for each $x \in X$, $f^{-1}\left( B(f(x),r) \right) \subset f^{-1}(U)$.
%\end{proof}

\begin{proposition}
	Let $X$ be a first countable, Hausdorff space. Then every countably compact subset $A$ of $X$ is closed.\cite[Exercises 11.1.7]{joshi}
\end{proposition}
%\begin{synopsis}
%\end{synopsis}
%\begin{proof}
%\end{proof}

%Things to ponder
%\paragraph{Questions}
%\begin{enumerate}
%	\item Countable compactness, sequential compactness are absolute\footnote{Property of the subset, that depends only on the relativised topology} properties?
%	\item Prove that $(X,\mathcal{T})$ is countably compact iff $(A,\mathcal{T}_{/_A})$ is countably compact for every closed subsets $A \subset X$ ?
%	\item Every countable open cover of $f(X)$ can be extended to a countable open cover of $Y$ ?
%\end{enumerate}

\subsection{Sequential Compactness}
\subsubsection{Weakly hereditary property}
\begin{theorem}
	Sequential compactness is weakly hereditary.\cite[Exercises 11.1.3]{joshi}
\end{theorem}
%\begin{synopsis}
%\end{synopsis}
%\begin{proof}
%\end{proof}

\subsubsection{The behaviour of countinous functions}
\begin{theorem}
	Sequential compactness is preserved under continuous functions.\cite[Exercises 11.1.4]{joshi}
\end{theorem}
\begin{synopsis}
	Let $X$ be sequentially compact and function $f : X \to Y$ be continuous. Then any sequence, $\{y_k\}$ in $f(X)$ has a sequence, $\{x_k\}$ in $X$ such that $f(x_k) = y_k$. Sequence $\{x_k\}$ has a subsequence $\{x_{n_k}\}$ converging to $x$, then sequence $\{ f(x_n)\}$ in $f(X)$ has the subsequence $\{f(x_{n_k})\}$ converging to $f(x)$.
\end{synopsis}
\begin{proof}
	Let $X$ be a sequentially compact space, function $f: X \to Y$ be continuous and $\{y_n\}$ be a sequence in $f(X)$ subset of $Y$. Construct a sequence $\{x_n\}$ such that $f(x_k) = y_k,\ \forall k$.\\

	Every sequence in $X$ has a convergent subsequence. Thus $\{x_n\}$ has a subsequence $\{x_{n_k}\}$ converging to $x \in X$. The image of this subsequence $\{f(x_{n_k})\}$ is a subsequence of $\{y_k\}$. We claim that, $\{f(x_{n_k})\}$ converges to $f(x) \in f(X)$.\\

	Let $U$ be an open set containing $f(x)$, then $f^{-1}(U)$ is an open set containing $x$. Since $\{x_{n_k}\}$ converges to $x$. There exists an integer $n$ such that for every $k \ge n$, $x_k \in f^{-1}(U)$. Clearly, for each $k \ge n$, $f(x_k) \in U$. Since $U$ is arbitrary, $\{ f(x_{n_k})\}$ converges to $f(x)$. Therefore, the image of any sequentially compact space is sequentially compact. In other words, sequentially compactness is preserved under continuous functions.
\end{proof}

\begin{remark}
	\begin{enumerate}
		\item Given a sequence $\{y_n\}$ in $f(X)$, there is a sequence of subsets $\{U_n\}$ in $P(Y)$ such that $U_n = f^{-1}(y_n)$. Since each $U_n$ is non-empty, we can construct a sequence $\{x_n\}$ in $X$ using a choice function. The convergent subsequence of $\{y_n\}$ depends on the selection of this choice function.
	\end{enumerate}
\end{remark}

Given every sequentially compact, metric space is countably compact. We may assert the properties of countably compact, metric spaces on sequentially compact, metric spaces.

\begin{theorem}
	Every continuous, real-valued function on a sequentially compact, metric space is bounded and attains its extrema.
\end{theorem}

\begin{theorem}
	Every continuous, real-valued function on a sequentially compact, metric space is uniformly continuous.\cite[Exercises 11.1.6]{joshi}
\end{theorem}

\subsection{Countable Compactness on $T_1$ spaces}
	In this section, we are going to see four different characterisations of countable compactness in $T_1$ spaces.	The first two characterisations doesn't have anything to do with the $T_1$ axiom.

\begin{description}
	\item[$T_1$ Space] A topological space $X$ satisfy $T_1$ axiom if for any two distinct points $x,y \in X$, there exists an open set $U \subset X$ containing $x$ but not $y$.\cite[7.1.2]{joshi}
	\item[countable compactness] A topological space is countably compact if every countable open cover has a finite subcover.\cite[11.1.1]{joshi}
	\item[finite intersection property] A family $\mathcal{F}$ of subsets of $X$ has finite intersection property(f.i.p.) if every finite subfamily of $\mathcal{F}$ has a non-empty intersection.\cite[10.2.6]{joshi}
	\item[accumulation point] A point $x \in X$ is accumulation point of a subset $A \subset X$ if every open set containing $x$ has atleast one point of $A$ other than $x$.\cite[5.2.7]{joshi}
	\item[limit point] A point $x \in X$ is a limit point of a sequence $< x_k >$ in $X$ if for every open set $U$ containing $x$, there exists an integer $N \in \mathbb{N}$ such that $x_k \in U$ for every $k \ge N$.\cite[4.1.7]{joshi}
	\item[cluster point] A point $x \in X$ is a cluster point of a sequence $< x_k >$ in $X$ if for any neighbourhood $V$ of $x$, the sequence $< x_k >$ assumes a point in $V$ infinitely many times.\footnote{$x$ is a cluster point of $< x_k >$ if for every integer N, there exists $k > N$ such that $x_k \in V$.In other words, $< x_k >$ is frequently in $V$.\cite[10.1.9]{joshi}}
\end{description}

\subsubsection{Countable compactness in $T_1$ spaces}
\begin{theorem}
	In a $T_1$ space $X$, following statements are equivalent,
\begin{enumerate}
	\item $X$ is countably compact
	\item Every countably family of closed subsets of $X$ with finite intersection property have non-empty intersection.
	\item Every infinite subset $A \subset X$ has an accumulation point.\footnote{Every infinite subset of $\mathbb{R}$ has a limit point is equivalent to the completeness axiom.}
	\item Every sequence $< x_k >$ in $X$ has a cluster point.
	\item Every infinite open cover of $X$ has a proper subcover.[Arens-Dugundji]
\end{enumerate}
\end{theorem}

\begin{proof}
	$1 \implies 2$\\
	Suppose $X$ is countably compact. Let $\mathcal{C} = \{C_1,C_2,\cdots\}$ be a countable family of closed subsets of $X$ with empty intersection. Define $\mathcal{U} = \{X-C_1, X-C_2, \cdots \}$ is a family of open subsets of $X$. By de Morgan's law, \footnote{Complement of Intersection = Union of complements, $X - (C \cap D) = (X-C) \cup (X-D)$, }

	$$\bigcap_{k = 1}^\infty C_k = \phi, \text{ then } X =  X - \left(\bigcap_{k = 1}^\infty C_k\right) = \bigcup_{k = 1}^\infty (X-C_k)$$

	We have $\mathcal{U}$ is a countable cover of $X$ and $X$ is countably compact space. Thus $\mathcal{U}$ has a finite subcover $\mathcal{U}' = \{X-C_{n_1},X-C_{n_2},\cdots,X-C_{n_k}\}$.

	$$ \mathcal{U}' \text{ is a cover of } X, \text{ then } X = \bigcup_{j = 1}^k \left( X-C_{n_j} \right)$$

	$$X - \bigcup_{j = 1}^k \left( X-C_{n_j} \right) = \bigcap_{j = 1}^k \left( X - \left( X - C_{n_j} \right) \right) = \bigcap_{j = 1}^k C_{n_j} = \phi $$

	Now $\mathcal{C}' = \{ C_{n_1},C_{n_2},\cdots,C_{n_k}\}$ has empty intersection. This is a contradiction to the finite intersection property of $\mathcal{C}$. Thus $\mathcal{C}$ has non-empty intersection. Therefore, every countably family of closed subsets of $X$ have non-empy intersection.\\

	$2 \implies 1$\\

	Let $\mathcal{U}=\{ U_1, U_2, \cdots \}$ be a countable cover of $X$. Then $\mathcal{C} = \{ X-U_1,X-U_2,\cdots \}$ is a countable family of closed subsets of $X$.\\

	Let $\mathcal{U}'= \{ U_{n_1},U_{n_2},\cdots,U_{n_k}\}$ be any finite subfamily of $\mathcal{U}$. Suppose $X$ is not countably compact, then $\mathcal{U}$ doesn't have a finite subcover. Therefore, $\mathcal{U}'$ is not a cover of $X$. And $\mathcal{C}$ is a family of closed sets with finite intersection property.\\

	Therefore by assumption, the countable family of closed sets $\mathcal{C}$ has a non-empty intersection.

	$$ \bigcap_{k=1}^\infty C_k \ne \phi, \text{ then } \bigcap_{k=1}^\infty C_k = \bigcap_{k=1}^\infty \left( X - U_k \right) = X - \left( \bigcup_{k=1}^\infty U_k \right) \ne \phi $$

	Then $\mathcal{U}$ is not a cover of $X$ as well. This is a contradiction, therefore $X$ is countably compact.\\

	$1 \implies 3$\\
	Suppose $X$ is countably compact. Let $A$ be an infinite subset of $X$. Suppose $A$ doesn't have an accumulation point.\\

	Let $B$ be a countably infinite subset of $A$. Then $B$ also doesn't have any accumulation point. Therefore, the derived set $B'$ is empty. Thus $B$ is a closed subset of $X$. Since countable compactness is weakly hereditary, subspace $B$ is again countably compact.\\

	For each point $b \in B$, there is an open set $V_b$ such that $V_b \cap B = \{ b\}$, since $b \in B$ is not an accumulation point. Thus $\mathcal{U} = \{ V_b \cap B : b \in B \}$ is a countable open cover of $B$. Clearly, $\mathcal{U}$ doesn't have any finite subcover.\\

	This is a contradiction to $B$ being countably compact. Therefore, $A$ has an accumulation point.
\end{proof}

\subsection{Variations of Compactness on Metric Spaces}
	In this document, we will see that from metric space point of view these two notions were equivalent to the compactness and were used alternatively. For example : in functional analysis (semester 3), you will find definitions like `a normed space is compact iff every sequence in it has a convergent subsequence', which is clearly sequential compactness for a topologist.

\begin{description}
	\item[Lindeloff] A topological space is Lindeloff iff every open cover has a countable subcover.
	\item[First countable] A topological space is first countable iff every point in it has a countable local base.
	\item[Second countable] A topological space is second countable iff it has a countable base.
	\item[Base] A family of subsets $\mathcal{B}$ of $X$ is a base of a topological space if every open set can be expressed as union of some members of $\mathcal{B}$
	\item[Base Characterisation] A family of subsets $\mathcal{B}$ of $X$ is a base of a topological space iff for every $x \in X$, and for every neighbourhood $U$ of $x$, there is a member $B \in \mathcal{B}$ such that $ x \in B \subset U$.
	\item[Local Base] A family of subsets $\mathcal{L}$ of $X$ is a local base at point $x \in X$ if for every neighbourhood $U$ of $x$, there is a member $L \in \mathcal{L}$ such that $x \in L \subset U$.
\end{description}

\subsubsection{Equivalence}
	We are going to see when these three notions: compactness, countable compactness and sequentially compactness are equivalent.

\begin{theorem}
	Countably compact, metric spaces are second countable.
\end{theorem}
\begin{synopsis}
	For every positive real number $r$,   there exists a non-empty maximal subsets $A_r$ with every pair of points atleast $r$ distance apart. $A_r$ are finite. The union of maximal subsets $A_\frac{1}{n}$ for each natural number $n$ is a countable, dense subset $D$ of $X$. Thus countably compact, metric spaces are separable. The family $\mathcal{B}$ of all open balls with center at $d \in D$ and rational radius is a countable, base for $X$. Thus countably compact, metric spaces are second countable.
\end{synopsis}
\begin{proof}
	Let $(X;d)$ be a countably compact,, metric space. For each positive real number $r \in \mathbb{R},\ r > 0$ construct a family of subsets $A_r \subset X$ such that it is a maximal set of points which are atleast $r$ distances apart.\\

	Then $A_r$ is finite for every positive real number $r$. Suppose $A_r$ is infinite for some real number $r > 0$, then $A_r$ has a accumulation point, say $x$ by the Characterisation of countable compactness of $X$.\\

	Then every neighbourhood of $x$ must intersect $A_r$ at infinitely many points, since every metric space is a $T_1$ space. Consider $B(x,\frac{r}{2})$. Since any two points of $B(x,\frac{r}{2})$ are less than $r$ distances apart, the intersection $B(x,\frac{r}{2}) \cap A_r$ can have atmost one point in it. Thus for every positive real number $r$, $A_r$ is finite.\\

	Define $D = \cup_{n = 1}^\infty A_\frac{1}{n}$. We claim that $D$ is a countable, dense subset of $X$.\\

	Let $x \in X$ and $B(x,r)$ be an open ball containing $x$, then there exists integer $n \in \mathbb{N}$ such that $\frac{1}{n} < r$.\footnote{By archimedean property of integers, we have $\forall r \in \mathbb{R},\ r>0,\ \exists n \in \mathbb{N}$ such that $nr > 1$.}\\

	Then $B(x,r) \cap D \ne \phi$, since $B(x,r) \cap A_\frac{1}{n} \ne \phi$. Suppose $B(x,r) \cap A_\frac{1}{n} = \phi$, then $A_\frac{1}{n}$ is not maximal. Since, $x$ is atleast $r > \frac{1}{n}$ distance apart from each points of $A_\frac{1}{n}$. Therefore, $D$ intersects with every open set and thus dense in $X$.\\

	We have have a countable, dense subset $D$ of $X$. Therefore, $X$ is separable. Now define $\mathcal{B} = \{ B(x,r) : r \in \mathbb{Q},\ x \in D \}$. Clearly, $\mathcal{B}$ is a countable base for $X$. By the construction of $\mathcal{B}$, $X$ is second countable.\footnote{Every separable, metric space is second countable.}
\end{proof}

\subsubsection{Countable Compactness, Lindeloff $\iff$ Compactness}
\begin{theorem}
	A topological space $X$ is compact iff it is countably compact, Lindeloff space.
\end{theorem}
\begin{proof}
	Let $X$ be a compact space. Let $\mathcal{U}$ be a  countable open cover of $X$, then $\mathcal{U}$ has a finite subcover $\mathcal{U}'$. Therefore, every compact space is countably compact.\footnote{Countable compactness is a weaker notion than compactness.}\\

	Conversely, suppose $X$ is a countably compact, Lindeloff space.  Since $X$ is Lindeloff, every open cover $\mathcal{U}$ has a countable subcover $\mathcal{U}'$. Since $X$ countably compact, every countable open cover $\mathcal{U}'$ has a finite subcover $\mathcal{U}''$. Thus every open cover $\mathcal{U}$ has a finite subcover $\mathcal{U}''$. Therefore every countably compact, Lindeloff space is compact.
\end{proof}

\subsubsection{Countable Compactness, First Countable $\implies$ Seq. Compactness}
\begin{theorem}
	Every countably compact, first countable space is Sequentially compact.
\end{theorem}
\begin{proof}
	Let $X$ be a countably compact, first countable space. Let $\{x_n\}$ be a sequence in $X$. By, equivalent conditions\footnote{\cite[11.1]{joshi} Conditions 1,2, and 4 are equivalent. $2 \implies 4$ without $T_1$ axiom is out of scope.} of countably compact spaces, every sequence in countably compact space $X$ has a cluster point, say $x$. We have, $X$ is first countable. Therefore, $X$ has a countable local base $\mathcal{L}$ at $x \in X$. How to construct a subsequence of $\{x_n\}$ converging to $x$ ?$\star$\footnote{\cite[Exercises 10.1.11]{joshi}}
\end{proof}

\begin{remark}
	Every sequentially compact space is countably compact.$\star$
\end{remark}

\begin{theorem}
	In a second countable space, all the three forms of compactness are equivalent.\cite[11.1.10]{joshi}
\end{theorem}
\begin{proof}
	Every second countable space is both first countable and Lindeloff. Every countably compact, Lindeloff space is countably compact. Therefore every countably compact, second countable space compact. Again, every countably compact, first countable space is sequentially compact. Therefore every countably compact, second countable space is sequentially compact. Conversely, every compact space is countably compact and every sequentially compact space is countably compact.\footnote{Countable compactness is a weaker notion than sequential compactness as well.}
\end{proof}

\begin{theorem}
	In a metric space, all the three forms of compactness are equivalent.\cite[11.1.11]{joshi}
\end{theorem}
\begin{proof}
	In a metric space each form of compactness implies second countability. And in second countable spaces, they are all equivalent.
%	Compact, metric spaces are second countable. Countably compact, metric spaces are second countable. Sequentially compact, metric spaces are second countable.$\star$
\end{proof}

%\section{The Alexander Sub-base Theorem*}
%\section{Local Compactness*}
%\section{Compactifications*}

%\chapter{Complete Metric Spaces*}
%\section{Complete Metrics}
%\section{Consequences of Completeness}
%\section{Some Applications}
%\section{Completions of a Metric}

%\chapter{Category Theory*}
%\section{Basic Definitions and Examples}
%\section{Functors and Natural Transformations}
%\section{Adjoint Functors}
%\section{Universal Objects and Categorical Notions}

%\chapter{Uniform Spaces*}
%\section{Uniformities and Basic Definitions}
%\section{Metrisation}
%\section{Completeness and Compactness}

%\chapter{Selected Topics*}
%\section{Function Spaces}
%\section{Paracompactness}
%\section{Use of Ordinal Numbers}
%\section{Topological Groups}
