%Text Books : \cite{ahlfors}
%Module 1
%The spherical representation of complex numbers, Riemann Sphere, Stereographic projection, Distance between the stereographic projections
%Elementary Theory of power series,Abel's Theorem on convergence of the power series, Hadamard's formula, Abel's limit Theorem
%Arcs and closed curves, Analytic functions in regions, Conformal mappings, Length and area, Linear transformations, The cross ratio, Symmetry, Oriented circles, Families of circles.
%Chapter – 1 Section 2.4 Chapter – 2 Sections 2.1 to 2.5,Chapter – 3 Sections 2.1, 2.2, 2.3, 2.4 and 3.1 to 3.4 (25 hours)
%Module 2
%Fundamental theorems on complex integration: line integrals, rectifiable arcs, line integrals as functions of arcs, Cauchy's theorem for a rectangle, Cauchy's theorem in a disk, Cauchy's integral formula: the index of a point with respect to a cloud curve, the integral formula.
%(Chapter 4 – Sections 1 , 2.1 and 2.2 )  (20 hours.)
%Module 3
%Higher derivatives. Differentiation under the sign of integration, Morera's Theorem, Liouville's Theorem, Fundamental Theorem, Cauchy's estimate
%Local properties of analytical functions: removable singularities, Taylor's theorem, zeroes and poles, Weirstrass Theorem on essential singularity, the local mapping, the maximum principle.Schwarz lemma
%Chapter-4 Sections 2.3, 3.1, 3.2, 3.3, and 3.4 (20 hours)
%Module 4
%The general form of Cauchy's theorem: chains and cycles, simple connectivity, homology, general statement of Cauchy's theorem, proof of Cauchy's theorem, locally exact differentiation, multiply connected regions
%Calculus of Residues: the residue theorem, the argument principle, evaluation of definite integrals.
%Chapter-4 Sections 4 and 5 (25 hours)

%Need to work on this
%Module 1 - \cite{ahlfors} 1, 2
%Module 2 - \cite{ahlfors} 4
%Module 3 - \cite{ahlfors} 4
%Module 4 - \cite{ahlfors} 4

%\chapter 1
{\Large Module 1 }
\section{Complex Numbers}
%\subsection{The Algebra of Complex Numbers}
%\subsubsection{Conjugation,Absolute Value}
%\begin{commentary}
%	Usually, Ahlfors uses greek alphabets $\alpha,\beta,\dots$ for real numbers and english alphabets $a,b,\dots$ for complex numbers.
%	And rarely, he writes $ z= x+iy$.
%\end{commentary}
%\subsubsection{Inequalities}
%\paragraph{Lagrange's Identity}(complex form)
%\begin{equation}
%	\left| \sum_{i=1}^n a_ib_i \right|^2 = \sum_{i=1}^n|a_i|^2\sum_{i=1}^n |b_i|^2 - \sum_{1 \le i < j \le n} |a_i\conj{b_j} - a_j\conj{b_i}|^2
%\end{equation}
%\paragraph{Triangle Inequality}
%\begin{equation}
%	\left| \sum_{i=1}^n a_i \right| \le \sum_{i=1}^n |a_i|
%\end{equation}
%\paragraph{Cauchy's Inequality}
%\%begin{equation}
%	\left| \sum_{i=1}^n a_ib_i \right|^2 \le \sum_{i=1}^n |a_i|^2 \sum_{i=1}^n |b_i|^2
%\end{equation}

%\subsection{The Geometric Representation of Complex Numbers}
%\subsubsection{Geometric Addition and Multiplcation}
%\begin{important}
%	Argument of $0$ is undefined.
%\end{important}
%\subsubsection{Binomial Theorem}
%\paragraph{de Moivre's formula}
%\begin{equation}
%	(\cos \phi+i\sin \phi)^n = \cos n\phi + i \sin n\phi
%\end{equation}
%\paragraph{nth root of $a = r(\cos \phi+i\sin \phi)$}
%\begin{equation}
%	z = \sqrt[n]{r} \left(\cos \left(\frac{\phi}{n}+k\frac{2\pi}{n}\right) + i\sin\left(\frac{\phi}{n} + k\frac{2\pi}{n}\right) \right)
%\end{equation}

%Let $\omega = \cos \frac{2\pi}{n} + i\sin \frac{2\pi}{n}$.
%Then, nth roots of unity are $1,\omega,\omega^2,\dots,\omega^{n-1}$.
%Then for any integer $h$ which is not a multiple of $n$, we have
%\begin{equation}
%	1+\omega^h + \omega^{2h} + \dots + \omega^{(n-1)h} = 0
%\end{equation}
%And,
%\begin{equation}
%	1-\omega^h + \omega^{2h} + \dots + (-1)^{n-1}\omega^{(n-1)h} = \begin{cases} 0 & n \text{ is odd } \\ 1+i\tan(\frac{\pi h}{n}) & n \text{ is even} \end{cases}
%\end{equation}
%\begin{commentary}
%	The proof for the even cases are bit complicated.
%	Refer \url{https://math.stackexchange.com/q/4362927} for the proof.
%\end{commentary}

%\subsubsection{Analytic Geometry}
%\paragraph{Circle} with center $a$ and radius $r$.
%\begin{equation}
%	|z-a| = r
%\end{equation}
%\[ (z-a)(\conj z - \conj a) = r^2 \]
%Also, $|z-a| < r$ and $|z-a| > r$ are inside and outside of the cirle.
%\paragraph{Straight Line} through $a,b$.
%\begin{equation}
%	z = a+bt
%\end{equation}
%Direction of the line is given by $arg(b)$.
%\[ L1 : z = a+bt,\qquad L2 : z = a'+b't \]
%Angle between $L1,L2$ is given by $arg (b/b')$.
%$L1, L2$ are the same line if $a'-a,b'$ are (real) multiples of $b$.
%$L1,L2$ are orthogonal if $b/b'$ is purely imaginary.\\

%Also, $\Im\frac{(z-a)}{b} < 0$ and $\Im\frac{(z-a)}{b} > 0$ are left and right half plane of the directed line from $a$ to $b$.

%\paragraph{Exercise}
%\begin{enumerate}
%	\item When does $az+b\conj z+c = 0$ represent a line ?
%		\begin{proof}[Answer]
%			Let $z = x+iy$.
%			\[ a(x+iy) + b(x-iy) + c = 0 \iff (a+b)x+(a-b)iy + c = 0 \]
%			\[ y = \frac{c}{(b-a)i}+\frac{(a+b)}{(b-a)i}x \]
%			Thus, $az+b\conj z+c = 0$ is a straight line if $a+b,c$ are (real) multiples of $(a-b)i$.
%		\end{proof}
%	\item Write the equation of an ellipse, hyperbola, parabola in complex form.
%		\begin{proof}[Answer]
%		Ellipse : $|z-a|+|z-b|=k$ \\
%		Parabola : $||z-a|-|z-b|| = k$ \\
%		Hyperbola : $ $
%		\end{proof}
%	\item Prove that
%		\begin{enumerate}
%			\item the diagonals of a parallelogram bisect each other.
%			\begin{proof}[Answer]
%			\end{proof}

%			\item the diagonals of a rhombus are orthogonal.
%			\begin{proof}[Answer]
%			\end{proof}
%		\end{enumerate}
%	\item Prove analytically that the midpoints of parallel chords to a circle lie on a diameter perpendicular to the chords.
%	\begin{proof}[Answer]
%	\end{proof}

%	\item Show that all circles circles that pass through $a$ and $1/\conj a$ intersect the circle $|z|=1$ at right angles.
%	\begin{proof}[Answer]
%	\end{proof}
%\end{enumerate}

\setcounter{subsection}{2}
\setcounter{subsubsection}{3}
\subsubsection{The Spherical Representation}
\begin{definition}[extended complex plane]
	Let $\mathbb{C}$ be the set of all complex numbers o f the form $z = x+iy$ where $x,y$ are real numbers.
	And,
	\[ z_1+z_2 = (x_1+iy_1) + (x_2+iy_2) = (x_1+x_2) + i (y_1+y_2) \]
	\[ z_1 z_2 = (x_1+iy_1) (x_2+iy_2) = x_1x_2 + ix_1y_2 + iy_1x_2 + i^2y_1y_2 = (x_1x_2-y_1y_2) + i(x_1y_2+x_2y1) \]
	We also have,
	\[ z_1/z_2 = \frac{x_1+iy_1}{x_2+iy_2} = \frac{(x_1+iy_1)(x_2-iy_2)}{(x_2+iy_2)(x_2-iy_2)} = \frac{x_1x_2+y_1y_2}{x_2^2+y_2^2}+i\frac{x_2y_1-x_1y_2}{x_2^2+y_2^2},\ (z_2 \ne 0) \]
	Define $\infty$ such that
	\[ z/0 = \infty,\ \forall z \in \mathbb{C},\ (z \ne 0)\]
	\[ z \pm \infty = \infty,\ \forall z \in \mathbb{C} \]
	\[ z \infty = \infty,\ \forall z \in \mathbb{C}\ (z \ne 0) \]
	\[ z / \infty = 0,\ \forall z \in \mathbb{C} \]
	\[ \infty/z = \infty,\ \forall z \in \mathbb{C}\ (z \ne 0) \]
	Then set $\mathbb{C}$ together with $\infty$ is the extended complex plane, $\mathbb{C}^\ast$.
\end{definition}

\begin{definition}[Riemann Sphere]
	The function which maps $z \to (x_1,x_2,x_3)$ such that
	\begin{equation}
		z = \frac{x_1+ix_2}{1-x_3}
		\label{eqn:projection}
	\end{equation}
	And maps $\infty \to (0,0,1)$.
	Now we have a bijection from extended complex plane into the unit sphere in $\mathbb{R}^3$.
	This spherical representation of extended complex plane is the Riemann sphere.
\end{definition}

\paragraph{Motivation} The significance of this projection is that on the unit sphere, point $(0,0,1)$ represents $\infty$, the point at infinity.

\paragraph{Riemann Sphere : Justification}
Let $S$ be the unit sphere in three dimensional space.
\begin{equation}
	 S : x_1^2 + x_2^2 + x_3^2 = 1 
	 \label{eqn:sphere}
\end{equation}
Consider the map given by (eqn.\ref{eqn:projection}).
Then,
	\[ |z|^2  = \frac{x_1^2+x_2^2}{(1-x_3)^2} = \frac{1-x_3^2}{(1-x_3)^2} = \frac{1+x_3}{1-x_3}\quad (\because equ. \ref{eqn:sphere}) \]

	\[ \frac{|z|^2-1}{|z|^2+1} = \frac{(1+x_3)-(1-x_3)}{(1+x_3)+(1-x_3)} = x_3  \]
	\[ 1-x_3 = \frac{(|z|^2+1) - (|z|^2-1)}{|z|^2+1} = \frac{2}{|z|^2+1} \]
Therefore,
\begin{align}
	 x_1 & = \frac{(z+\conj{z})(1-x_3)}{2} = \frac{z+\conj{z}}{|z|^2+1}\quad (\because z + \conj{z} = \frac{2x_1}{1-x_3}) \\
	 x_2 & = \frac{(z-\conj{z})(1-x_3)}{2i} = \frac{z-\conj{z}}{i(|z|^2+1)}\quad (\because z - \conj{z} = \frac{2ix_2}{1-x_3}) \\
	x_3 & = \frac{|z|^2-1}{|z|^2+1}
\end{align}

Clearly, the map $z \to \displaystyle \frac{x_1+ix_2}{1-x_3}$ is a one-one correspondence.
And when $x_3 = 0$, we have $|z| = 1$.
Consider the hemisphere, $x_3 < 0$.
Then $|z|<1$ since $|z|^2<1$.
Similarly, when $x_3 > 0$ we have $|z|>1$.

\paragraph{Riemann Sphere : Sterographic Projection}
Let $z = x+iy$.
Consider the map $z \to (x,y)$.
Then,
\[ x:y:-1 = \frac{x_1}{1-x_3} : \frac{x_2}{1-x_3} : -1 = x_1 : x_2 : x_3-1 \]
Thus we have,
\[ \begin{vmatrix} 0 & 0 & 1 \\ x & y & 0 \\ x_1 & x_2 & x_3 \end{vmatrix} = \begin{vmatrix} 0 & 0 & 1 \\ x & y & -1 \\ x_1 & x_2 & x_3-1\end{vmatrix} = 0 \]
Clearly, the points $(x,y,0)$, $(0,0,1)$ and $(x_1,x_2,x_3)$ lies on a straight line.\\

\begin{commentary}
\begin{definition}[collinear]
	Let $(x_1,x_2,x_3), (x'_1,x'_2,x'_3), (x''_1,x''_2,x''_3) \in \mathbb{R}^3$.
	These points are collinear if
	\[ \begin{vmatrix} x_1 & x_2 & x_3 \\ x'_1 & x'_2 & x'_3 \\ x''_1 & x''_2 & x''_3 \end{vmatrix} = 0\]
\end{definition}
\end{commentary}

If you consider the complex plane as the plane $x_3 = 0$ in the three dimensional space, then the straight lines through $(0,0,1)$ will geometrically give you the image and preimage under the bijection given by (eqn.\ref{eqn:projection}).
Therefore, this is a sterographic projection.\\

\begin{commentary}
	From $(0,0,1)$, you could draw a line which gives you the correspondence between the sphere and the complex plane.
	This projection is sterographic in the sense that you could see points on the unit sphere projected on the plane as well as the points on the plane projected on the unit sphere using the same straight line.
\end{commentary}

\paragraph{Circles on the Riemann Sphere}
\begin{remark}
	The sterographic projection $z = \frac{x_1+ix_2}{1-x_3}$ transforms straight lines in the complex plane into a circle on the unit sphere $S$ through $(0,0,1)$.
	And any circle on the unit sphere transforms into a circle or straight line in the complex plane.
\end{remark}
\begin{proof}
	Any circle on the unit sphere is an intersection of 
	the unit sphere $S : x_1^2+x_2^2+x_3^2=1$ and
	a plane $P : \alpha_1x_1+\alpha_2x_2+ \alpha_3x_3 = \alpha_0$.\\

	Without loss of generality, $\alpha_1^2+\alpha_2^2+\alpha_3^2 = 1$ and $0 \le \alpha_0 < 1$.
	Suppose $P : \alpha'_1 x_1 + \alpha'_2 x_2 + \alpha'_3 x_3 = \alpha'_0$.
	Suppose $\alpha'_0 > 0$.
	If $\alpha'_0 < 0$, then consider $P : -\alpha'_1 x_1 - \alpha'_2 x_2 - \alpha'_3 x_3 = -\alpha'_0$.
	And define
	\[ 
	\alpha_1 = \frac{\alpha'_1}{|\alpha'|} \qquad
	\alpha_2 = \frac{\alpha'_2}{|\alpha'|} \qquad
	\alpha_3 = \frac{\alpha'_3}{|\alpha'|} \qquad
	\alpha_0 = \frac{\alpha'_0}{|\alpha'|}\] 

	where $|\alpha'| = \sqrt{{\alpha'}_1^2 + {\alpha'}_2^2 + {\alpha'}_3^2}$.
	Thus, $\alpha_1^2 + \alpha_2^2+\alpha_3^3 = 1$ and $0 \le \alpha_0 < 1$.\\

\begin{commentary}
	For any circle on the unit sphere, we have a plane intersecting the sphere at a positive distance $\alpha_0$ from the origin which is parallel to the tangent plane at the point $(\alpha_1,\alpha_2,\alpha_3)$ on the unit sphere.
	Clearly, the plane won't touch the sphere if $\alpha_0 > 1$ and when $\alpha_0 = 1$ the plane is tangential to the sphere.
	Thus, $0 \le \alpha_0 < 1$.\\
\end{commentary}

	\noindent Substituting the values of $x_1,x_2,x_3$, we get
	\begin{align*}
		\alpha_1 \frac{z+\conj{z}}{|z|^2+1} + \alpha_2 \frac{z - \conj{z}}{i(|z|^2+1)} + \alpha_3 \frac{|z|^2-1}{|z|^2+1} & = \alpha_0 \\
		\alpha_1 (z+\conj{z}) - i\alpha_2 (z - \conj{z}) + \alpha_3 (|z|^2-1) & = \alpha_0 (|z|^2+1)
	\end{align*}
		We have, $z = x+iy$.
		Then $|z|^2 = x^2+y^2$, $z+\conj{z} = 2x$ and $z - \conj{z} = 2iy$.
	\begin{align*}
		2x\alpha_1 + 2y\alpha_2 + x^2\alpha_3 + y^2\alpha_3-\alpha_3 & = x^2\alpha_0 + y^2 \alpha_0+\alpha_0 \\
		x^2(\alpha_3-\alpha_0) + y^2(\alpha_3-\alpha_0) + 2x\alpha_1 + 2y\alpha_2-(\alpha_3+\alpha_0) &= 0
	\end{align*}
	
	Suppose $\alpha_3 \ne \alpha_0$.
	Then, we have 
	\begin{equation}
		x^2(\alpha_3-\alpha_0)+y^2(\alpha_3-\alpha_0) + 2x\alpha_1 + 2y\alpha_2 - (\alpha_3+\alpha_0) = 0
		\label{eqn:circle}
	\end{equation}
	Clearly, (eqn.\ref{eqn:circle}) represents a circle in the complex plane.\\

	\begin{commentary}
	\begin{definition}[circle]
		Circle with center $(h,k)$ and radius $r$ is given by
		\[ (x-h)^2+(y-k)^2 = r^2 \]
		\[ x^2+y^2-2xh-2ky+(h^2+k^2-r^2) = 0 \]
	\end{definition}
		\noindent In the above case, we have
		\[ x^2+y^2+2x\frac{\alpha_1}{\alpha_3-\alpha_0}+2y\frac{\alpha_2}{\alpha_3-\alpha_0}-\frac{\alpha_3+\alpha_0}{\alpha_3-\alpha_0} = 0 \]
		is a circle with center $\left(\frac{\alpha_1}{\alpha_0-\alpha_3},\frac{\alpha_2}{\alpha_0-\alpha_3}\right)$ and radius $\frac{\sqrt{1-\alpha_0^2}}{\alpha_0-\alpha_3}$.\\
	\end{commentary}

	Suppose $\alpha_3 = \alpha_0$.
	Then (eqn.\ref{eqn:circle}) becomes $x\alpha_1+ y\alpha_2 - \alpha_0 = 0$.
	Clearly, it represents a straight line in the complex plane.
\end{proof}
\begin{important}
	Any straight line transforms into a circle through $(0,0,1)$ on the Riemann sphere.
	Thus, straight lines are circles through the point at infinity.
\end{important}

\paragraph{Distance between two points on the Riemann Sphere}
Let $z,z' \in \mathbb{C}$.
Let $z \to (x_1,x_2,x_3)$ and $z' \to (x'_1,x'_2,x'_3)$.
We have,
\[ |z-z'|^2 = (z-z')(\conj{z}-\conj{z'})  = |z|^2+|{z'}|^2 - (z\conj{z'}+\conj{z}z') \]
And,
\[ (|z|^2+1)(|{z'}|^2+1)-(|z|^2-1)(|{z'}|^2-1) = 2(|z|^2+|{z'}|^2)  \]
Thus,
\begin{align*}
	z\conj{z'}+\conj{z}z'  
	& = |z|^2+|{z'}|^2 - |z-z'|^2 \\
	2(z\conj{z'}+\conj{z}z')  
	& = \left\{ (|z|^2+1)(|{z'}|^2+1) - (|z|^2-1)(|{z'}|^2-1)\right\} - 2|z-{z'}|^2
\end{align*}

\noindent Rearranging the terms, we get
\begin{equation}
	2z\conj{z'}+2\conj{z}z' + (|z|^2-1)(|{z'}|^2-1) = |z|^2+1)(|{z'}|^2+1) - 2|z-{z'}|^2
	\label{eqn:distSub}
\end{equation}

\noindent Let $d(z,z')$ denote the distance between their images on the sphere.
Then,
\begin{align*}
	\left( d(z,z') \right)^2
	& = (x_1-x'_1)^2+(x_2-x'_2)^2+(x_3-x'_3)^2 \\
	& = (x_1^2+x_2^2+x_3^2) + ({x'_1}^2+{x'_2}^2+{x'_3}^2) -2(x_1x'_1+x_2x'_2+x_3x'_3) \\
	& = 2 - 2(x_1x'_1+x_2x'_2+x_3x'_3)
\end{align*}

\noindent Substituting the values, we get
\begin{align*}
	x_1x'_1+x_2x'_2+x_3x'_3
	& = \frac{(z+\conj{z})(z'+\conj{z'}) + i^2(z-\conj{z})(z'-\conj{z'}) + (|z|^2-1)(|{z'}|^2-1) }{ (|z|^2+1)(|{z'}|^2+1) } \\
	& = \frac{2z\conj{z'}+2\conj{z}z'+(|z|^2-1)(|{z'}|^2-1)}{ (|z|^2+1)(|{z'}|^2+1) } \\
	& = \frac{(|z|^2+1)(|{z'}|^2+1)-2|{z-z'}|^2}{ (|z|^2+1)(|{z'}|^2+1) } \quad (\because eqn.\ref{eqn:distSub}) \\
	& = 1-\frac{2|{z-z'}|^2}{ (|z|^2+1)(|{z'}|^2+1) } \\
	\left( d(z,z') \right)^2
	& = \frac{4|{z-z'}|^2}{ (|z|^2+1)(|{z'}|^2+1) } 
\end{align*}

\noindent Therefore, we have
\begin{equation}
	d(z,z')= \frac{2|z-z'|}{\sqrt{(1+|z|^2)(1+|z'|^2)}}
	\label{eqn:distRiemann}
\end{equation}

Suppose $z' = \infty$.
Then $\frac{|z-z'|}{\sqrt{1+|z'|^2}} \to 1$ as $z' \to \infty$.
Therefore,
\begin{equation}
	d(z,\infty)= \frac{2}{\sqrt{1+|z|^2}}
	\label{eqn:distRiemannInfinity}
\end{equation}

\paragraph{Exercises}
\begin{enumerate}
	\item $z,z'$ are diametricall opposite points on the Riemann Sphere if and only if $z\conj{z'} = -1$.
	\begin{proof}
		Suppose $z\conj{z'} = -1$.
		Then $z' = \frac{-z}{|z|^2}$.
		\[ |z-z'| = \left| z + \frac{z}{|z|^2} \right| = \frac{|z|^2+1}{|z|} \]
		\[ 1+|z'|^2 = 1+ \frac{1}{|z|^2} = \frac{|z|^2+1}{|z|^2}  \]
		Then,
		\[ d(z,z') = \frac{2|z-z'|}{\sqrt{(1+|z|^2)(1+|{z'}|)^2}} = \frac{2\frac{|z|^2+1}{|z|}}{\sqrt{(1+|z|^2)\frac{|z|^2+1}{|z|^2}}} = 2 \]
		Clearly, the points are diametrically opposite on the unit sphere.\\

		Suppose the points are diametrically opposite on the Riemann sphere.
		Then
		\[ |{z-z'}|^2 = (1+|z|^2)(1+|{z'}|^2) \]
		\begin{align*}
			|{z-z'}| ^2
			& =  (z-z')(\conj{z}-\conj{z'}) \\
			& = |z|^2+|{z'}|^2 - z\conj{z'} - \conj{z}z'  \\
			& = 1+|z|^2+|{z'}|^2+|z|^2|{z'}|^2 \\
			0 & = 1+z\conj{z'}+\conj{z}z'+|z|^2|{z'}|^2 \\
			& = (1+z\conj{z'})(1+\conj{z}z')
		\end{align*}
		Then either $z\conj{z'}+1 = 0$ or $\conj{z}z'+1=0$.
		In either case, we have $z\conj{z'} = -1$ since $z\conj{z'} = \conj{z}z'$ when there are purely real.
	\end{proof}

	\item Find vertices of the cube inscribed in the Reimann sphere with edges parallel to the coordinate axes.
	\begin{proof}[Answer]
		Let $a$ be the length of the sides of the cube inscribed in the Riemann sphere.
		Then the main diagonal is of length $a\sqrt{3} = 2$, the diameter of the unit sphere.
		Thus, $a = 2/\sqrt{3}$.
		The vertices of the cube on the Riemann sphere are $(\pm a/2,\pm a/2,\pm a/2)$.
		The corresponding images are given by 
		\[ z = \frac{a}{(2 \pm a)}(\pm 1 \pm i) = \frac{\pm 1 \pm i}{\sqrt{3} \pm 1} \]
		Therefore, the vertices are $\frac{1+i}{\sqrt{3} + 1}$, $\frac{1-i}{\sqrt{3} + 1}$, $\frac{-1+i}{\sqrt{3} + 1}$, $\frac{-(1+i)}{\sqrt{3} + 1}$, $\frac{1+i}{\sqrt{3} - 1}$, $\frac{1-i}{\sqrt{3} - 1}$, $\frac{-1+i}{\sqrt{3} - 1}$, and $\frac{-(1+i)}{\sqrt{3} - 1}$.

	\end{proof}

	\item Find the vertices of the regular tetrahedron inscribed in Riemann Sphere.
	\begin{proof}[Answer]
		Let $a$ be the length of the sides of the regular tetrahedron inscribed in the Riemann sphere.
		Radius of circumsphere of regular tetrahedron is $\frac{\sqrt{6}}{4}a = 1$.
		Thus, $a = 4/\sqrt{6}$.\\

		One vertex at $(0,0,1)$ is the point at infinity.
		The other three points are at a distance of $4/\sqrt{6}$ from $(0,0,1)$.
		We have, 
		\[d(z,\infty) = \frac{2}{\sqrt{1+|z|^2}} = \frac{4}{\sqrt{6}} = \frac{2}{\sqrt{\frac{6}{4}}} \]
		Thus, $1+|z|^2 = 1+\frac{1}{4}$.
		Therefore, $|z| = \frac{1}{2}$.\\

		In its general position the projection of one of the base vertex on the regular tetrahedron is purely imaginary ($\frac{i}{2}$) and the other vertices makes $120^\circ$ with the imaginary axis.
		Therefore, the vertices are $\infty$, $\frac{i}{2}$, $\frac{1}{4}\left(-\sqrt{3}-i\right)$, and $\frac{1}{4}\left(\sqrt{3}-i\right)$
		\end{proof}

	\item Find the radius of the spherical image of the circle $|z-a|=r$.
	\begin{proof}[Answer]
		The spherical image is the intersection of the unit sphere $x_1^2+x_2^2+x_3^2=1$ and $\alpha_1x_1 + \alpha_2x_2 + \alpha_3x_3 = \alpha_0$, where $\alpha_1^2+\alpha_2^2+\alpha_3^2 =1$ and $0 \le \alpha_0 < 1$.
		The intersection is a circle with center $\left(\frac{\alpha_1}{\alpha_0-\alpha_3},\frac{\alpha_2}{\alpha_0-\alpha_3}\right)$ and radius $\frac{\sqrt{1-\alpha_0^2}}{\alpha_0-\alpha_3}$.
		Clearly, the radius of the spherical image lies on this plane of intersection which is at $\alpha_0$ distance from origin.\\

		Suppose $a = x+iy$.
		Let $k = \alpha_0 - \alpha_3$.
		Then,
		\[ \alpha_1 = kx,\quad \alpha_2 = ky,\quad \alpha_3 = \sqrt{1-k^2|a|^2} \]
		And,
		\[ \alpha_0^2 = (k+\alpha_3)^2 = k^2+\alpha_3^2+2k\alpha_3 = k^2+(1-k^2|a|^2)+2k(\alpha_0-k) \]
		\begin{equation}
			\alpha_0^2-2k\alpha_0+k^2+k^2|a|^2-1 = 0 
			\label{eqn:alpha0a}
		\end{equation}
		We have $r^2 = \frac{1-\alpha_0^2}{k^2}$.
		Thus,
		\begin{equation}
			\alpha_0^2 + k^2r^2 - 1 = 0 
			\label{eqn:alpha0b}
		\end{equation}

		From equations (\ref{eqn:alpha0a},\ref{eqn:alpha0b}), we have
		\[ 2k\alpha_0-k^2|a|^2+k^2r^2-k^2 = 0 \]
		\[ \alpha_0 = \frac{k(1+|a|^2-r^2)}{2} \]

		Thus,
		\[ \alpha_3^2 = 1-k^2|a|^2 = (\alpha_0-k)^2 = \frac{k^2(1+r^2-|a|^2)^2}{4} \]
		Therefore,
		\[ k = \frac{2}{\sqrt{(1+r^2-|a|^2)^2+4|a|^2}} \]
		\[ \alpha_0 = \frac{1+|a|^2-r^2}{\sqrt{(1+r^2-|a|^2)^2+4|a|^2}} = \frac{\lambda+2}{\sqrt{\lambda^2+4|a|^2}} \]
		where $\lambda = |a|^2-r^2-1$.
		Now, we have right triangle with sides $\alpha_0$, $R$ and $1$.
		Thus, the radius $R$ of the circle of intersection is given by 
		\[ R = \sqrt{1-\alpha_0^2} = \frac{2\sqrt{|a|^2-\lambda-1}}{\sqrt{\lambda^2+4|a|^2}} = \frac{2r}{\sqrt{(|a|^2-r^2-1)^2+4|a|^2}} \]
	\end{proof}
\begin{commentary}
	Exercise 4 is a bit complicated.
	For $a = 3+4i$ and $r = 2$, I have substituted the value of $k = 0.04\sqrt{5}$ to find the equation for the plane.
	I find the value of $|\alpha_0| = 0.44\sqrt{5} = 0.98 < 1$.
	And $R = 0.18$. \\

	There is an alternate solution for the radius of the spherical image of a circle at \url{https://math.stackexchange.com/q/190270}.
	Let me know if you have an opinion different from mine.\\

	Let $z = a+r$ and $z' = a-r$.
	Then $|z-z'| = 2r$.
	Let $R$ be the radius of the spherical image.
	We claim that the points $z,z'$ are diametrically opposite on the spherical image.
	Then,
	\[ R = \frac{d(z,z')}{2} = \frac{|z-z'|}{\sqrt{(1+|z|^2)(1+|{z'}|^2)}} = \frac{2r}{\sqrt{(1+|a+r|^2)(1+|a-r|^2)}} \]

	\textbf{However, I think that the claim doesn't stand when $a \ne 0$.}
	That is, the projections of diametrically opposite points of a circle need not be diametrically opposite on the spherical image of it.\\

	Suppose the claim is true.
	Consider, $z = a+ir$ and $z'=a-ir$ are diamterically opposite points of the circle.
	Then, by above relation the radius of the spherical image assumes a different value, which is not possible.
	\end{commentary}
\end{enumerate}

\section{Complex Functions}
\subsection{Analytic Function}
\begin{definition}[region]
	A region $\Omega$ is an open, connected subset of $\mathbb{C}$.
\end{definition}

\subsubsection{Limit and Continuity}
\begin{definition}[limit]
	Let $\Omega$ be a region.
	Let function $f : \Omega \to \mathcal{C}$.
	Function $f$ has limit $A$ as $x$ tends to $a$ if for every $\varepsilon > 0$, there exists $\delta > 0$ such that $|f(x)-A| < \varepsilon$ for any $x$ such that $|x-a| < \delta$ and $x \ne a$.
	\[ \text{ Notation: } f(x) \to A \text{ as } x \to a \]
\end{definition}

\paragraph{Properties of Limit}
Suppose $f(x) \to A$ and $g(x) \to B$ as $x \to a$.
\begin{enumerate}
	\item $(f+g)(x) \to A+B$ as $x \to a$
	\item $(fg)(x) \to AB$ as $x \to a$
	\item $(f/g)(x) \to A/B$ as $x \to a$ \dag\footnote{If $g$ is bounded away from zero.}
	\item $\overline{f(x)} \to \bar{A}$ as $x \to a$
\end{enumerate}

\begin{definition}[continuity]
	$f$ is continuous at $a$ if $f(x) \to f(a)$ as $x \to a$.
	And $f : \Omega \to \mathbb{C}$ is continuous if it is continuous at each point of $\Omega$.
\end{definition}

\paragraph{Properties of Continuous Functions}
Let $f,g$ be two continuous function.
\begin{enumerate}
	\item $f \pm g$, $fg$ and $f/g$ are continuous ($g \ne 0$).
	\item $Re{(f)}, Im{(f)}, |f|$ are continuous.
\end{enumerate}

\begin{definition}[derivative]
	The derivative of $f$ at $a$ is 
	\begin{equation}
		f'(a) = \lim_{x \to a} \frac{f(x)-f(a)}{x-a}
	\end{equation}
\end{definition}

\paragraph{Properties of Derivatives}
\begin{enumerate}
	\item $(f+g)'(a) = f'(a) + g'(a)$
	\item $(fg)'(a) = f'(a)g(a) + f(a)g'(a)$
	\item $(f/g)'(a) = \frac{g(a)f'(a)-f(a)g'(a)}{g^2(a)}$
	\item $(f \circ g)'(a) = f'(g(a)) g'(a)$
\end{enumerate}

\begin{important}
	A real function of complex variable either has the derivative zero or else the derivative does not exist.
	And complex function of real variable can be reduced to the real case.
	$$ f(x) = g(x)+ih(x) \qquad f'(x) = g'(x) +ih'(x) $$
\end{important}

\subsubsection{Analytic Functions}
\begin{definition}[analytic]
	An analytic function is a complex function of complex variable which has derivative wherever it is defined.
\end{definition}

\begin{definition}[Cauchy-Riemann]
	Cauchy-Riemann differential equations for a complex function $f : \Omega \to \mathbb{C},\ f(z) = u(z)+iv(z)$ are
\begin{equation}
	\textcolor{purple}{\frac{\partial f}{\partial x} = -i\frac{\partial f}{\partial y}}      \qquad \text{ OR }   \textcolor{blue}{\qquad \frac{\partial u}{\partial x} = \frac{\partial v}{\partial y} \qquad \text{ and } \qquad \frac{\partial u}{\partial y} = -\frac{\partial v}{\partial x}}
	\label{eqn:CR}
\end{equation}
	Then the pair $(u,v)$ is said to satify Cauchy-Riemann differential equations.
	And $f = u+iv$ is said to satisfy Cauchy-Riemann conditions for analyticity.
\end{definition}

\begin{definition}[harmonic]
	A real function $u : \Omega \to \mathbb{R}$ of complex variable is a harmonic function if $u(z)$ satisfies the Laplace equation for every $z \in \Omega$.
\begin{equation}
	\Delta u = \frac{\partial^2 u}{\partial x^2} + \frac{\partial^2 u}{\partial y^2} = 0
	\label{eqn:laplace}
\end{equation}
\end{definition}

\begin{definition}[conjugate harmonic]
	Two harmonic functions $u,v$ are conjugate harmonic if the pair $(u,v)$ satisfy Cauchy-Riemann differential equations.
\end{definition}

\paragraph{Properties of Analytic Functions}
Let $f,g$ be analytic functions.
\[ f'(z) = \lim_{h \to 0} \frac{f(z+h)-f(z)}{h} \]
\begin{enumerate}
	\item $f \pm g, fg, f/g$ are analytic functions ($g \ne 0$).

	\item Analytic functions are continuous.
	\begin{proof}
		Let $f : \Omega \to \mathbb{C}$ be an analytic function.
		Let $z \in \mathbb{C}$.
		Then for sufficietly small $h$, we have $z+h \in \Omega$.
		And,
		\begin{align*}
			f(z+h) - f(z) 
			& =  h\frac{f(z+h)-f(z)}{h}
		\end{align*}
		Thus, $f(z+h)-f(z) \to 0f'(z) = 0$ as $h \to 0$.
		In other words, $f(z+h) \to f(z)$ as $h \to 0$.
		Clearly, $f$ is continuous.
	\end{proof}

	\item Real and imaginary parts of analytic functions satisfies Cauchy-Riemann differential equations.
	\begin{proof}
		Let $f : \Omega \to \mathbb{C}$ be an analytic function.
		Let $z = x+iy \in \Omega$.
		Then $f'(z)$ exists.
		And the value of $f'(z)$ is independent of the path along which $z+h$ approaches $z$.\\

		Suppose $h$ is real, then $f(z+h) = f(x+h+iy)$. Thus,
		\begin{align}
			f'(z) 
			& = \lim_{h \to 0} \frac{f(x+h+iy)-f(x+iy)}{h} = \frac{\partial f}{\partial x} \\
			& = \lim_{h \to 0} \frac{u(x+h,y)-u(x,y)}{h} + \lim_{h \to 0} i\frac{v(x+h,y)-v(x,y)}{h} \nonumber \\
			& = \frac{\partial u}{\partial x} + i \frac{\partial v}{\partial x} 
		\end{align}
		Suppose $h$ is imaginary, then $f(z+h) = f(x+i(y+k))$.
		Thus,
		\begin{align}
			f'(z) 
			& = \lim_{k \to 0} \frac{f(x+i(y+k)) - f(x+iy)}{ik} = -i\frac{\partial f}{\partial y} \\
			& = \lim_{k \to 0} \frac{u(x,y+k)-u(x,y)}{ik} + \lim_{k \to 0} i\frac{v(x,y+k)-v(x,y)}{ik} \nonumber \\
			& = -i\frac{\partial u}{\partial y} + \frac{\partial v}{\partial y}
		\end{align}
		Comparing the equations, we get
		\begin{equation*}
			\frac{\partial f}{\partial x} = -i\frac{\partial f}{\partial y} \qquad \text{ OR } \qquad \frac{\partial u}{\partial x} = \frac{\partial v}{\partial y} \qquad \text{ and } \qquad \frac{\partial u}{\partial y} = -\frac{\partial v}{\partial x}
		\end{equation*}
		Therefore, real and imaginary parts of $f$ satisfies Cauchy-Riemann differential equations.	
		In other words, $f$ satisfying the Cauchy-Riemann conditions for analyticity.
	\end{proof}

	\item $|f'(z)|^2$ is the Jacobian of $u$ and $v$ with respect to $x$ and $y$.
	\begin{proof}
		Let $f = u+iv$ be an analytic function.
		Then,
	\begin{align*}
		f'(z) & = \frac{\partial u}{\partial x} + i \frac{\partial v}{\partial x} \\
		|f'(z)|^2 & = \left(\frac{\partial u}{\partial x} \right)^2 + \left(\frac{\partial v}{\partial x}\right)^2 \\
		& = \frac{\partial u}{\partial x} \frac{\partial v}{\partial y} - \frac{\partial u}{\partial y} \frac{\partial v}{\partial x} = \left|\begin{matrix} \frac{\partial u}{\partial x} & \frac{\partial u}{\partial y} \\ \frac{\partial v}{\partial x} & \frac{\partial v}{\partial y} \end{matrix} \right|
	\end{align*}
	\end{proof}

	\item Derivative of an analytic function is analytic. (\textcolor{blue}{proof : \ref{sect:higherDerivatives}})

	\item Analyticity implies existence of the partial dertivatives $\frac{\partial u}{\partial x}, \frac{\partial u}{\partial y}, \frac{\partial v}{\partial x}, \frac{\partial v}{\partial y}$ of real and imaginary parts.
	Since derivative of an analytic function is also an analytic function, real and imaginary parts of an analytic function has continuous partial derivatives of all orders.
	And mixed partial derivatives are equal since the second order partial derivatives are continuous.

	\item If $f = u +iv$ is analytic, then $u,v$ are harmonic functions.
	And $v$ is the unique conjugate harmonic function of $u$ upto an additive constant.
\end{enumerate}
\begin{theorem}[conjugate harmonic]
	Function $f = u+iv$ is an analytic function if and only if $u,v$ are conjugate harmonic functions.
\end{theorem}
\begin{proof}
	Suppose $v$ is the conjugate harmonic function of $u$.
	Then the four first order partial derivatives of $u,v$ exists, they are continuous and they satisfy Cauchy-Riemann differential equations.
	\begin{align*}
		u(x+h,y+k) - u(x,y) 
		& = u(x+h,y+k) - u(x,y+k) + u(x,y+k) - u(x,y) \\
		& = h\frac{u(x+h,y+k) - u(x,y+k)}{h} + k\frac{u(x,y+k) - u(x,y)}{k} 
		\intertext{By existence and continuity of the partial derivatives,}
		u(x+h,y+k) & - u(x,y) \to h\frac{\partial u}{\partial x} + k\frac{\partial u}{\partial y} \text{ as } h+ik \to 0
		\intertext{For sufficiently small values of $h,k$ we have,}
		u(x+h,y+k) - u(x,y) 
		& = h\frac{\partial u}{\partial x} + k\frac{\partial u}{\partial y} + \varepsilon_1 \text{ where } \frac{\varepsilon_1}{h+ik} \to 0 \text{ as } h+ik \to 0
		\intertext{Similarly,}
		v(x+h,y+k) - v(x,y) 
		& = h\frac{\partial v}{\partial x} + k\frac{\partial v}{\partial y} + \varepsilon_2 \text{ where } \frac{\varepsilon_2}{h+ik} \to 0 \text{ as } h+ik \to 0
		\intertext{Substituting the values,}
		f(z+h+ik) - f(z) 
		& = u(x+h,y+k)-u(x,y) + iv(x+h,y+k) - v(x,y)\\
		& = h\frac{\partial u}{\partial x}  + k\frac{\partial u}{\partial y} + \varepsilon_1 + ih\frac{\partial v}{\partial x} + ik\frac{\partial v}{\partial y} + i\varepsilon_2 \\
		& = \left(\frac{\partial u}{\partial x} + i\frac{\partial v}{\partial x} \right) (h+ik) + (\varepsilon_1 + i \varepsilon_2) \\
		\implies f'(z) & \to \frac{\partial u}{\partial x} + i \frac{\partial v}{\partial x} \text{ as } h+ik \to 0
	\end{align*}
	Thus, $f$ is analytic.\\

	Suppose $f = u+iv$ is an analytic function.
	We know that, 
	\[ f'(z) = \frac{\partial u}{\partial x} + i \frac{\partial v}{\partial x} = -i\frac{\partial u}{\partial y}+\frac{\partial v}{\partial y} \]
	Suppose $u,v$ are harmonic functions.
	Comparing real and imaginary parts, we have $v$ is the conjugate harmonic function of $u$ upto an additive constant.\\

	It remains to prove that $u,v$ are harmonic functions.
	We have, 
	\[\frac{\partial u}{\partial x} = \frac{\partial v}{\partial y}  \implies \frac{\partial^2 u}{\partial x^2} = \frac{\partial^2 v}{\partial x \partial y}\]
	\[\frac{\partial u}{\partial y} = -\frac{\partial v}{\partial x}  \implies \frac{\partial^2 u}{\partial y^2} = -\frac{\partial^2 v}{\partial y \partial x}\]
	Since the second order partial derivatives of $v$ are continuous, these mixed partial derivatives of $v$ are equal.
	\[ \Delta u = \frac{\partial^2 u}{\partial x^2} + \frac{\partial^2 u}{\partial y^2} = \frac{\partial^2 v}{\partial x \partial y}-\frac{\partial^2 v}{\partial y \partial x} = 0 \]
	Therefore, $u$ is harmonic and similarly $v$ is also harmonic.
\end{proof}

\begin{commentary}
\begin{remark}
	An analytic function can't be reduced to a function of two real variables.
\end{remark}
\begin{align*}
	f(z) 
	& = f(x+iy) \\
	& = f\left(\frac{z+\bar{z}}{2}-i\frac{z-\bar{z}}{2}\right) \\
	\implies \frac{\partial f}{\partial \bar{z}} 
	& = \frac{1}{2}\frac{\partial f}{\partial x} + \frac{i}{2}\frac{\partial f}{\partial y} = 0 \qquad \left(\because f \text{ is analytic, }\frac{\partial f}{\partial x} = -i\frac{\partial f}{\partial y}\right)
\end{align*}
Thus, $f$ is independent of $\bar{z}$.	
Therefore, an analytic function $f(x+iy)$ can't be truely represented by $f(x,y)$.
\end{commentary}

\paragraph{Computing $f$ from rational function $u(x,y)$}
\begin{equation}
	f(z) = 2u(z/2,z/2i)-u(0,0)+c
\end{equation}
\paragraph{Exercise}
\begin{enumerate}
	\item Composition of analytic functions is also analytic.
	\begin{proof}
	\begin{align*}
		(g \circ f)'(z)
		& = \lim_{h \to 0} \frac{(g \circ f)(z+h) - (g \circ f)(z)}{h} \quad ? 
	\end{align*}
	\end{proof}
	\item 
	\begin{enumerate}
		\item $z^2  = x^2-y^2 + 2xyi$
			\[ \frac{\partial f}{\partial x} = 2x+2iy = -i(2ix-2y) \qquad \frac{\partial f}{\partial y} = -2y+2ix \]
		\item $z^3 = x^3-3xy^2+i(3x^2y-y^3)$
			\[ \frac{\partial f}{\partial x} = 3x^2-3y^2+6ixy = -i(3ix^2-3iy^2-6xy) \qquad \frac{\partial f}{\partial y} = -6xy+3ix^2-3iy^2 \]
	\end{enumerate}
	\item $u = ax^3+bx^2y+cxy^2+dy^3$
	\begin{align*}
		\frac{\partial u}{\partial x} 
		&= 3ax^2+2bxy+cy^2 & \frac{\partial^2 u}{\partial x^2} = 6ax+2by \\
		\frac{\partial u}{\partial y}
		& = bx^2+2cxy+3dy^2  & \frac{\partial^2 u}{\partial y^2} = 2cx+6dy \\
		\Delta u & = 0 \implies c = -3a,\ d = -b/3 \\
		u & = ax^3+bx^2y-3axy^2-y^2/3 \\
		\frac{\partial v}{\partial y} & = 3ax^2 + 2bxy-3ay^2 \\
		\frac{\partial v}{\partial x} & = -bx^2 + 6axy+by^2 
	\end{align*}
	\item
	\item 
	\item
	\item
\end{enumerate}

\subsubsection{Polynomials}
\paragraph{Polynomials are analytic functions}
Polynomials are the \textit{simplest} analytic functions.
\begin{proof}
Constant functions are analytic with derivative $0$ and $z$ is the simplest non-constant analytic function and it has derivative $1$.
Since sum and product of analytic functions are analytic, the polynomial
\begin{equation}
	P(z) = a_0 + a_1 z + \dots + a_nz^n,\quad (a_n \ne 0)
\end{equation}
with degreen $n$ is analytic with derivative
\begin{equation}
	P'(z) = a_1 + 2a_2z + \dots + na_nz^{n-1}
\end{equation}
\end{proof}

By fundamental theorem of algebra(theorem \ref{thm:fundamental}), every polynomial $P(z)$ with degree $n > 0$ has atleast one root, say $\alpha_1$.
Thus, $P(z) = (z-\alpha_1)P_1(z)$ where $P_1(z)$ is a polynomial of degree $n-1$.
Repeating the process, we get $P(z) = a_n(z-\alpha_1)(z-\alpha_2)\dots(z-\alpha_n)$ where $\alpha_j$'s are not necessarily distinct.\\

\paragraph{Order of a zero}
These $\alpha_j$'s are the zeroes of $P(z)$.
If there are $h$ number of $\alpha_j$'s such that $\alpha = \alpha_j$. Then $\alpha$ is a \textbf{zero of order $h$}.\\

Also order of a zero $\alpha$ is the order of the first nonvanishing derivative of $P(z)$ at $\alpha$.
That is, the least positive integer $m$ such that the derivative \[ P^{(h)}(\alpha) = \left.\frac{d^n}{dz^n}P(z)\right|_{z=\alpha} \ne 0 \]
Thus, \textbf{simple zero} of $P(z)$ is an $\alpha$ such that $P(\alpha) = 0$ and $P^\prime(\alpha) \ne 0$.

\begin{theorem}[Lucas]
	If all zeroes of a polynomial $P(z)$ lies in a half plane, then all zeroes of the derivative $P^\prime(z)$ lie in the same half plane.
\end{theorem}
\begin{proof}
	Let $H$ be the half plane defined by
	\[ H = \left\{ z : Im{\left(\frac{z-a}{b}\right)}<0\right\} \]
	Suppose every zero $\alpha_k$ of $P(z)$ belong to $H$.
	Then,
	\[ Im\left(\frac{\alpha_k-a}{b}\right) < 0,\ \forall k \]
	Suppose $z$ belongs to the other half plane.
	Then, $P(z) \ne 0$ and
	\[ Im\left(\frac{z-a}{b}\right) > 0 \]
	Thus,
	\[ Im\left(\frac{z-\alpha_k}{b}\right) = Im\left(\frac{z-a}{b}-\frac{\alpha_k-a}{b}\right) = Im\left(\frac{z-a}{b}\right)-Im\left(\frac{\alpha_k-a}{b}\right) > 0 \]
	Since imaginary part of reciprocals have opposite sign, we have
	\[ Im\left(\frac{b}{z-\alpha_k}\right) < 0,\ \forall k \]

	We have,
	\begin{align*}
		\log P(z) 
			& = \log a_n + \log (z-\alpha_1) + \log (z-\alpha_2) + \dots + \log (z-\alpha_n) \\
		\implies \frac{P^\prime(z)}{P(z)} 
			& = \frac{1}{z-\alpha_1}+\frac{1}{z-\alpha_2}+\dots+\frac{1}{z-\alpha_n} = \sum_{k=1}^n \frac{1}{z-\alpha_k} \\
		\implies Im\left(\frac{bP^\prime(z)}{P(z)}\right)
			& = \sum_{k=1}^n Im\left(\frac{b}{z-\alpha_k}\right) < 0
\end{align*}
	 Therefore, $P^\prime(z) \ne 0$ for any $z \notin H$.
\end{proof}
\begin{commentary}
\begin{theorem}[Lucas]
	The smallest convex polygon containing the zeroes of $P(z)$ contains all the zeroes of $P^\prime(z)$. In other words, it contains all critical points of $P(z)$.
\end{theorem}
\begin{proof}
	Let $C$ be the smallest convex polygon containing all zeroes of $P(z)$.
	Without loss of generality, all the zeroes of $P(z)$ lies on the same half plane defined by each edge of this polygon.
	By above proof, we know that the critical points of $P(z)$ which are the zeroes of $P^\prime(z)$ belongs to the same half plane.
\end{proof}
\end{commentary}

\subsubsection{Rational Functions}
\begin{definition}
	A function of the form $R(z) = \frac{P(z)}{Q(z)}$ where $P(z),Q(z)$ are polynomials such that they don't have any common zeroes.
\end{definition}
The zeroes of $P(z)$ are the \textbf{zeroes} of $R(z)$ and the zeroes of $Q(z)$ are the \textbf{poles} of $R(z)$.
The \textbf{order of a pole} of $R(z)$ is the order of the respective zero of $Q(z)$.

\paragraph{Derivative}
The derivative of $R(z)$ exists only when $Q(z) \ne 0$.
\[ R^\prime(z) = \frac{P^\prime(z)Q(z) - P(z)Q^\prime(z)}{Q(z)^2} \]

\paragraph{Order of a rational function}
Rational functions have common number of zeroes and poles in the extended plane.
This number is the \textbf{order} of the rational function.\\

Let $n,m$ be the order of the polynomials $P(z),Q(z)$.
The order of the rational function $R(z) = \frac{P(z)}{Q(z)}$ is $\max\{m,n\}$.
If $n > m$, then $R(z)$ has a pole of order $n-m$ at $\infty$.
If $m < n$, then $R(z)$ has a zero of order $m-n$ at $\infty$.\\

If $R(z)$ is a rational polynomial of order $n$.
Then order or $R(z)-a$ is $n$.
Thus, $R(z)=a$ has $n$ zeroes.
\begin{definition}
	A linear transformation is a rational function of order $1$.
	\[ S(z) = \frac{\alpha z + \beta}{\gamma z + \delta} ,\ \alpha\delta - \beta\gamma \ne 0 \]
\end{definition}
\begin{description}
	\item[parallel translations] are linear trasformations of the form $z+a$.
	\item[inversion] is the linear transformation $1/z$.
\end{description}

\begin{definition}
	Fixed point of a linear transformation $S$ is a complex number $z$ in the extended plane for which $S(z) = z$.
\end{definition}
Fixed point of parallel translation is $\infty$ and that of inversion are $1,-1$.

\paragraph{Partial Fraction Representation}
\[ R(z) = \frac{P(z)}{Q(z)} = G(z) + H(z) \]
where $H(z)$ is finite at $\infty$ and $G(z)$ is a polynomial without constant term.
In other words, $H(z) = \frac{P_1(z)}{Q(z)}$ where $P_1(z)$ and $Q(z)$ are of the same degree.
Then, there exists a unique $G(z)$ such that $P(z) = G(z)Q(z) + P_1(z)$.
Clearly, $G(z)$ is \textbf{singular part} of $R(z)$ at $\infty$.\\

Suppose $R(z)$ has $q$ distinct, finite poles, say $\beta_1, \beta_2,\dots,\beta_q$.
Then,
\begin{equation}
	R(z) = G(z) + \sum_{j=1}^q G_j\left(\frac{1}{z-\beta_j}\right)
\end{equation}
where $G_j$ is the singular part of $R(z)$ at $\beta_j$.

\subsection{Elementary Theory of Power Series}
\subsubsection{Sequences}
\begin{definition}
	A sequence of complex numbers $\sequence{a_n}$ converges to $A$ if \begin{equation}
		\forall\varepsilon > 0, \exists N \in \mathbb{N} \text{ such that }\forall n > N,\ |a_n - A| < \varepsilon
	\end{equation}
\end{definition}

\begin{remark}
	Sequence diverges to infinity if $\displaystyle \lim_{n \to \infty} a_n = \infty$.
\end{remark}
\begin{equation}
	\forall \varepsilon > 0, \exists N \in \mathbb{N} \text{ such that } \forall n > N, \ |a_n| > \frac{1}{\varepsilon}
\end{equation}

\begin{definition}[Cauchy]
	A sequence $\sequence{a_n}$ is Cauchy/Fundamental if \begin{equation}
		\forall \varepsilon > 0, \exists N \in \mathbb{N} \text{ such that } \forall m,n > N, |a_m-a_n| < \varepsilon
\end{equation}
\end{definition}

\begin{theorem}
	A sequence $\sequence{a_n}$ is convergent if and only if Cauchy.
\end{theorem}
\begin{proof}
	Suppose $\sequence{a_n}$ converges to $A$.
	Then \[\forall \varepsilon > 0,\ \exists N \in \mathbb{N},\ \forall m,n > N,\ |a_n - A| < \varepsilon,\ |a_m - A| < \varepsilon \]
	Consider,
	\[ |a_m - a_n| = |a_m - A + A - a_n \le |a_m - A| + |a_n - A| < 2\varepsilon \]
	Therefore, the sequence $\sequence{a_n}$ is Cauchy.\\

	Suppose $\sequence{x_n}$ is Cauchy.
	Consider the sequence of real parts,
	\[\forall \varepsilon > 0,\ \exists N \in \mathbb{N},\ \forall m,n > N,\ |\alpha_m-\alpha_n| < \varepsilon \]
	where $x_n = \alpha_n + i \beta_n$.
	Define $a_n = \max\{\alpha_1, \alpha_2, \dots, \alpha_n\}$.
	Then the sequence $\sequence{a_n}$ is a bounded, monotone sequence.
	Thus, the $\sequence{a_n}$ has a limit, say $A$
	Therefore, $A = \overline{\lim \alpha_n}$.
	Similarly, let $a_n = \min\{\alpha_1,\alpha_2,\dots,\alpha_n\}$.
	Then, the $\sequence{a_n}$ has a limit, $a = \underline{\lim \alpha_n}$. \\
	
	Suppose $a \ne A$.
	Let $\varepsilon = \frac{A-a}{3}$.
	Then there exists $\alpha_n < A-\varepsilon$ and exists $\alpha_m > a + \varepsilon$.
	Therefore, \[A-a = A - \alpha_m + \alpha_m - \alpha_n + \alpha_n - a < 3\varepsilon \]
\end{proof}

\paragraph{Limit Superior and Limit Inferior}
\begin{definition}
	Let $\sequence{\alpha_n}_{n=1}^\infty$ be a real sequence.
	Define $a_n = \max\{\alpha_1,\alpha_2,\dots,\alpha_n\}$.
	?
\end{definition}
\paragraph{Properties of Limit Superior/Inferior}
\begin{enumerate}
	\item The hierarchy
	\[ \varliminf \alpha_n + \varliminf \beta_n \le \varliminf (\alpha_n+\beta_n) \le \varliminf \alpha_n + \varlimsup \beta_n \le \varlimsup (\alpha_n + \beta_n) \le \varlimsup \alpha_n + \varlimsup \beta_n \]
\end{enumerate}

\subsubsection{Series}
\begin{definition}
	A sequence $\sequence{b_n}$ is a \textbf{contraction} of sequence $\sequence{a_n}$ if $|b_m-b_n| \le |a_m - a_n|,\ \forall m,n \in \mathbb{N}$.
\end{definition}

\begin{definition}
	An infinite series is a formal infinite sum
	\[ \sum_{n = 1}^\infty a_n = a_1 + a_2 + \dots + a_n + \dots \]
\end{definition}
\begin{definition}
	Series $\sum a_n$ converges if and only if the sequence of partial sums $\sequence{s_n}$ where $s_n = \sum_{j=1}^n a_n$ converges.
	The limit of the sequence of partial sum is the \textbf{sum} of the series.
\end{definition}
\paragraph{Cauchy Condition}
The series converges if and only if for every $\varepsilon > 0$ there exists a natural number $n_0$ such that $|a_n + a_{n+1}+\dots+a_{n+p}| < \varepsilon$.\\

For $p=0$, we have $|a_n| < \varepsilon$.
In other words, $a_n \to 0$.\\

If a finite number of terms are omitted from a series, then the new series converges/diverges simultaneously with the first series.
And the sum of the original(first) series $S = s_n + R_n$ where $s_n$ is the sum of the omitted terms and $R_n$ is the sum of the second series.

\paragraph{Absolute Convergence}
Let $\sum a_n$ be a series.
The series converges absolutely if $\sum |a_n|$ converges.
Absolute convergence is stronger than convergence in the case of series.
In other words, absolute convergence implies convergence.

\subsubsection{Uniform Convergence}
\begin{definition}
	A sequence of functions $\sequence{f_n(x)}$ defined on $E$ is said to converge uniformly to a function $f(x)$ if
	\[ \forall \varepsilon > 0, \ \exists n_0 \in \mathbb{N} \text{ such that } \forall x \in E,\ \forall n > n_0, \ |f_n(x)-f(x)| < \varepsilon \]
\end{definition}

Convergence is pointwise if $n_0$ depends on $x \in E$.
\paragraph{Properties of Uniform Convergence}
\begin{enumerate}
	\item The limit function of a uniformly convergent sequence of continuous functions is continuous.
\end{enumerate}

\paragraph{Cauchy Condition}
A sequence of functions $\sequence{f_n(x)}$ is uniformly convergence if and only if
	\[ \forall \varepsilon > 0, \ \exists n_0 \in \mathbb{N} \text{ such that } \forall x \in E,\ \forall m,n > n_0, \ |f_m(x)-f_n(x)| < \varepsilon \]

\paragraph{Weierstrass M Test}
If $\sequence{f_n(x)}$ is a contraction of a convergent sequence $\sequence{a_n}$, then $\sequence{f_n(x)}$ is uniformly convergent.
\[ |f_n(x)-f_m(x)| \le |a_n - a_m|, \ \forall n,m \in \mathbb{N} \implies \sum f_n(x) = f(x) \]

Suppose $\sum a_n$ is convergent and $\sum f_n(x)$ is a minorant of $\sum Ma_n$ for some constant $M$.
\[ \exists n_0 \in \mathbb{N} \text{ such that } \forall n > n_0,\ |f_n(x)| \le M|a_n| \]
Then $\sum f_n(x) $ is uniformly convergent.

\subsubsection{Power Series}
\begin{definition}
	A power series is a series of the form
	\[ \sum a_n (z-z_0)^n = a_0 + a_1(z-z_0) + \dots + a_n(z-z_0)^n + \dots \]
	where $a_n,z_0 \in \mathbb{C}$.
\end{definition}

\begin{theorem}
	For every power series about origin there exists a number $R$ called \textbf{radius of convergence} such that $0 \le R \le \infty$ and
	\begin{enumerate}
		\item The series converges absolutely for every $z$ with $|z|<R$.
		If $0 \le \rho < R$, the convergence is uniform for $|z| \le \rho$

		\item If $|z|>R$, the terms of the series are unbounded, and the series is divergent.

		\item In $|z| <R$, the sum of the series is an analytic function. The derivative can be obtained by temwise differentiation.
			And the derived series has the same radius of convergence.
	\end{enumerate}
\end{theorem}
\begin{proof}
\end{proof}

The circle $|z| = R$ is the \textbf{circle of convergence}.
\paragraph{Hadamard's Formula}
\begin{equation}
	\frac{1}{R} = \varlimsup \sqrt[n]{|a_n|}
\end{equation}

\paragraph{Taylor-Maclaurin Development}
\begin{equation}
	f(z) = f(0) + f'(0)z + \frac{f^{\prime\prime}(0)}{2!}z^2 + \dots + \frac{f^{(n)}(0)}{n!}z^n + \dots 
\end{equation}

\subsubsection{Abel's Limit Theorem}
\begin{theorem}[Abel]
	If $\sum a_n$ converges, then $f(z) = \sum a_n$ tends to $f(1)$ as $z$ approaches $1$ in such a way that $|1-z|/(1-|z|)$ remains bounded.
\end{theorem}
\begin{proof}
\end{proof}

In other words, the series converges at $z=1$ if the approach is in Stolz angle.

\subsection{The exponential and trignometric funtions}
\subsubsection{The Exponential}
\begin{definition}
	The exponential function is the function $f(z)$ of power series form $\sum a_n z^n$ such that $f'(z) = f(z)$.
\end{definition}
\[ e^z = \sum_{n = 0}^\infty \frac{z^n}{n!} = 1 + z + \frac{z^2}{2!} + \dots + \frac{z^n}{n!} + \dots \]
\subsection{The Trignometric Functions}
All the trignometric functions are rational functions of $e^{iz}$.
\[ \cos z = \frac{e^{iz}+e^{-iz}}{2} = 1 - \frac{z^2}{2!} + \frac{z^4}{4!} + \dots \]
\[ \sin z = \frac{e^{iz}-e^{-iz}}{2i} = z - \frac{z^3}{3!} + \frac{z^5}{5!} + \dots \]
\[ \tan z = -i \frac{e^{iz}-e^{-iz}}{e^{iz}+e^{-iz}} \]

\subsubsection{The Periodicity}
The function $f(z)$ has period $c$ if $f(z+c) = f(z),\ \forall z$.
Then, period of $e^{z}$ is $c = 2\pi i$.
And period of $e^{iz}$ is $c = 2\pi$.
The smallest positive period of $e^{iz}$ is $2\pi$.
\subsubsection{The Logarithm}
The inverse function of the exponential function.
There exists only one elementary transcendental function, say $e^z$.

\section{Analytic Functions as Mappings}
\subsection{Elementary point set topology}
\subsubsection{Sets and Elements}
Point, Set, Intersection, Union, Complement, De Morgan laws
\subsubsection{Metric Spaces}
Metric, Open ball, Open set, Neighbourhood characterisation of open set, Closed set, Interior, Exterior, Boundary, Accumulation Point
\subsubsection{Connectedness}
Connected, Path connectedness, Connected components
\begin{definition}
	A region is a nonempty, connected, open set.
\end{definition}
In a locally connected, separable space every open set is a countable union of disjoint regions.
\subsubsection{Compactness}
complete, compact, bounded, Sequential compactness
\subsubsection{Continuous Functions}
compactness, connectedness, extrema, uniform continuity
\subsubsection{Topological Spaces}
Hausdorff
\subsection{Conformality}
\subsubsection{Arcs and Closed Curves}
\begin{definition}[Arc]
	A continuous function $\gamma$ defined on a closed, finite interval $[\alpha,\beta]$.	
	\begin{equation}
		z = \gamma(t), \text{ where } \gamma : [\alpha,\beta] \to \mathbb{C},\ \alpha \le t \le \beta,\text{ and } \gamma \text{ continuous }
	\end{equation}
\end{definition}

\begin{remark}
	Arc is a point set, with a succession of points.
\end{remark}

\begin{definition}[change of parameter]
	Let $\gamma : [\alpha,\beta] \to \mathbb{C}$ be an arc.
	Let $\varphi : [\alpha',\beta'] \to [\alpha,\beta]$ be a non-decreasing onto function.
	Then $\gamma \circ \varphi$ defines the same sucession of points as the arc $\gamma$.
\end{definition}

\begin{description}
	\item[differentiable] $z^\prime(t)$ exist and is continuous at all points.
	\item[regular] differentiable and $z^\prime(t) \ne 0$ at points on $\gamma$.
	\item[piecewise differentiable] differentiable except for finitely many points
	\item[piecewise regular] regular except for finitely many points.
	\item[opposite arc] set of points $z = z(-t),\ -b \le t \le -a$.
\end{description}
\subsubsection{Analytic Functions in Regions}
\subsubsection{Conformal Mapping}
\subsubsection{Length and Area}
\subsection{Linear Transformations}
\subsubsection{The Linear Group}
\subsubsection{The Cross Ratio}
\subsubsection{Symmetry}
\subsubsection{Oriented Circles}
\subsubsection{Family of Circles}

{\Large Module 2 }
\section{Complex Integration}
\subsection{Fundamental Theorems}
\subsubsection{Line Integrals}
\subsubsection{Rectifiable Arcs}
\subsubsection{Line Integrals as Functions of Arcs}
\subsubsection{Cauchy's Theorem for a Rectangle}
\subsubsection{Cauchy's Theorem in a Disk}
\subsection{Cauchy's Integral Formula}
\subsubsection{The Index of a Point wrt a Closed Curve}
\subsubsection{The Integral Formula}
\pagebreak

{\Large Module 3 }
\subsubsection{Higher Derivatives}
\label{sect:higherDerivatives}
\begin{lemma}
	Suppose $\phi(\zeta)$ is continuous on the closed curve arc $\gamma$. Then the function
	\begin{equation}
		F_n(z) = \int_\gamma \frac{\phi(\zeta)d\zeta}{(\zeta-z)^n}
	\end{equation}
	is analytic in each of the regions determined by the closed curve $\gamma$ and its derivative is $F_n^\prime(z) = nF_{n+1}(z)$.
\end{lemma}
\begin{theorem}[Morera]
	If $f(z)$ is defined and continuous in a region $\Omega$, and if $\int_\gamma f(z)\ dz = 0$ for all closed curves $\gamma$ in $\Omega$, then $f(z)$ is analytic in $\Omega$.
\end{theorem}
\begin{theorem}[Liouville]
	A function which is analytic and bounded in the whole plane must reduce to a constant.
\end{theorem}
\begin{theorem}[fundamental theorem]
	Every non-constant polynomial with complex coefficients has a zero.
	\label{thm:fundamental}
\end{theorem}
\subsection{Local Properties of Analytical Functions}
\subsubsection{Removable Singularities, Taylor's Theorem}
\subsubsection{Zeros and Poles}

\subsubsection{The Local Mapping}

\subsubsection{The Maximum Principle}
{\Large Module 4 }
\subsection{The General Form of Cauchy's Theorem}
\subsubsection{Chains and Cycles}
\begin{definition}[chain]
	A chain is a formal sum of arcs.
\end{definition}
For example, $\gamma = \gamma_1 + \gamma_2 + \dots + \gamma_n$ is a chain if $\gamma_1,\gamma_2,\dots,\gamma_n$ are arcs.
\paragraph{Properties of Chains}
\begin{enumerate}
	\item Different formal sums can represent the same chain.
	\item Two chains are identical if they yield the same line integral for all functions $f$.
		\[ \int_{\gamma_1+\gamma_2+\dots+\gamma_n} f\ dz = \int_{\gamma_1} f\ dz + \int_{\gamma_2} f\ dz + \dots + \int_{\gamma_n} f\ dz \]
	\item The following operations does not affect any chain
	\begin{enumerate}
		\item Permutation of two arcs. $\gamma_1+\gamma_2+\gamma_3 = \gamma_1+\gamma_3+\gamma_2$
		\item Subdivision of an arc. $\gamma_1+\gamma_2 = \gamma_1 + \gamma_{21} + \gamma_{22}$
		\item Fusion of subarcs into a single arc. $\gamma_1 + \gamma_{21}+\gamma_{22} = \gamma_1+\gamma_2$
		\item Reparametrisation of an arc. $\gamma(\tau(t)) = \gamma(t)$
		\item Cancellation of opposite arcs. $\gamma_1 + \gamma_2 - \gamma_2 = \gamma_1$
	\end{enumerate}
	\item Sum of two chains defined to preserve the additive property of the line integrals.
		\[ \gamma = \alpha_1 \gamma_1 + \alpha_2 \gamma_2 + \dots + \alpha_n \gamma_n \]
		where $\alpha_j$ are integers and $\gamma_j$ are arcs.
		That is, instead of $\gamma_1 + \gamma_1 = 2\gamma_1$.
		And $-\gamma_1-\gamma_1 = -2\gamma_1 = 2(-\gamma_1)$.
		Also, $\gamma_1 - \gamma_1 = 0 \gamma_1$.
		Terms with zero coefficients can be added/removed without altering the chain.
\end{enumerate}
\begin{definition}[cycle]
	A cycle is a formal sum of closed curves.
\end{definition}
\begin{definition}[index of a point]
	Index of a point with respect to a cycle $\gamma = \alpha_1\gamma_1+\alpha_2\gamma_2+\dots+\alpha_n\gamma_n$ is given by
	\begin{equation}
		n(\alpha_1\gamma_1+\alpha_2\gamma_2+\dots+\alpha_n\gamma_n,a) = \alpha_1n(\gamma_1,a) + \alpha_2n(\gamma_2,a)+\dots + \alpha_nn(\gamma_n,a)
	\end{equation}
\end{definition}

\subsubsection{Simple Connectivity}
\begin{definition}[simply connected]
	A region is simply connected if it complement with respect to the extended plane is connected.
\end{definition}

\begin{theorem}
	A region $\Omega$ is simply connected if and only if $n(\gamma,a) = 0$ fo all cycle $\gamma$ in $\Omega$ and all points $a$ which does not belong to $\Omega$.
\end{theorem}
\begin{proof}
\end{proof}

\subsubsection{Homology}
\begin{definition}
	A cycle $\gamma$ in an open set $\Omega$ is \textbf{homologous to zero with respect to $\Omega$} if $n(\gamma,a)=0$ for all points $a$ in the complement of $\Omega$.
	\[ \text{Notation : } \gamma \sim 0 \!\!\! \pmod{\Omega} \]
\end{definition}
\begin{remark}
	\[ \gamma_1 \sim \gamma_2 \!\!\! \pmod{\Omega} \iff \gamma_1-\gamma_2 \sim 0 \!\!\! \pmod{\Omega} \]
	\[ \Omega \subset \Omega',\ \gamma \sim 0 \!\!\! \pmod{\Omega} \implies \gamma \sim 0 \!\!\! \pmod{\Omega'} \]
\end{remark}

\subsubsection{The General Statement of Cauchy's Theorem}
\begin{theorem}
	If $f(z)$ is analytic in $\Omega$, then
	\[ \int_\gamma f(z)\ dz = 0 \]
	for every cycle $\gamma$ which is homologous to zero in $\Omega$.
\end{theorem}
\begin{corollary}
	If $f(z)$ is analytic in a simply connected region $\Omega$, then $ \int_\gamma f(z)\ dz = 0$ for all cycles $\gamma$ in $\Omega$.
\end{corollary}
\begin{proof}
	Every cycle $\gamma$ in a simply connected region $\Omega$ is homologous to zero.
	Thus by general Cauchy's theorem, $\int_\gamma f(z)\ dz = 0$.
\end{proof}

\begin{corollary}
	If $f(z)$ is analytic and $f \ne 0$ in a simply connected region $\Omega$ then it is possible to define a single valued analytic braches of $\log f(z)$ and $\sqrt[n]{f(z)}$ in $\Omega$.
\end{corollary}
\begin{proof}
\end{proof}

\subsubsection{Proof of Cauchy's Theorem}
\subsubsection{Locally Exact Differentials}
\begin{definition}
	A differential $pdx+qdy$ is exact if there exists an analytic function $U$ such that $\frac{\partial U}{\partial x} = p$ and $\frac{\partial U}{\partial y} = q$.
\end{definition}
\begin{commentary}
\begin{remark}
	If $Mdx+Ndy$ is exact differential then 
	\[ \frac{\partial M}{\partial y} = \frac{\partial^2 U}{\partial y \partial x} = \frac{\partial^2 U}{\partial x \partial y} = \frac{\partial N}{\partial x} \]
\end{remark}
\end{commentary}

\begin{definition}
	A differential $pdx+qdy$ is locally exact in $\Omega$ if it is exact in some neighbourhood of every point in $\Omega$.
\end{definition}
\begin{theorem}
	If $pdx+qdy$ is locally exact in $\Omega$, then
	\[ \int_\gamma pdx+qdy = 0 \]
	for every cycle $\gamma \sim 0 \!\!\! \pmod{\Omega}$.
\end{theorem}
\subsubsection{Multiply Connected Regions}
\begin{definition}
	A multiply connected region is a region which is not simply connected.
\end{definition}

\subsection{The Calculus of Residues}
\subsubsection{The Residue Theorem}
\begin{definition}
	The residue of $f(z)$ at an isolated singularity of the unique complex number $R$ which makes $f(z)-R/(z-a)$ the derivative of a single valued analytic function in the annulus $0<|z-a|<\delta$.
\end{definition}
\begin{theorem}
	Let $f(z)$ be analytic except for a isolated singularities $a_j$ in a region $\Omega$.
	Then,
	\begin{equation}
		\frac{1}{2\pi i} \int_\gamma f(z)\ dz = \sum_j n(\gamma,a_j) Res_{z=a_j} f(z) 
	\end{equation}
	for any cycle $\gamma \sim 0 \!\!\! \pmod{\Omega}$ and $\gamma$ doesn't pass through any of the points $a_j$.
\end{theorem}
\begin{proof}
\end{proof}

\begin{definition}
	A cycle $\gamma$ bounds the region $\Omega$ if and only if $n(\gamma,a)$ is defined and equal to $1$ for all points $a \in \Omega$ and either undefined or equal to zero for all points $a \notin \Omega$.
\end{definition}
\begin{commentary}
	In other words, $\gamma$ is the boundary of the region $\Omega$.
\end{commentary}

\subsubsection{The Argument Principle}
\begin{theorem}
	If $f(z)$ is memromorphic in $\Omega$ with the zeroes $a_j$ and the poles $b_j$, then
	\begin{equation}
		\frac{1}{2\pi i} \int_\gamma \frac{f^\prime(z)}{f(z)} dz = \sum_j n(\gamma,a_j) - \sum_k n(\gamma,b_k)
	\end{equation}
\end{theorem}
\begin{proof}
\end{proof}

\begin{corollary}[Rouch\'e]
	Let $\gamma \sim 0 \pmod{\Omega}$ such that $n(\gamma,z)$ is either $0$ or $1$ for any point $z \notin \gamma$.
	Suppose $f(z), g(z)$ are analytic in $\Omega$ and satisfies the inequality $|f(z)-g(z)| < |f(z)|$ on $\gamma$.
	Then $f(z)$ and $g(z)$ have the same number of zeroes enclosed by $\gamma$.
\end{corollary}
\begin{proof}
\end{proof}
\paragraph{Exercise}
\begin{enumerate}
	\item
	\item
	\item
\end{enumerate}
\subsubsection{Evaluation of Definite Integrals}
\paragraph{Type 1}
\begin{equation}
	\int_0^{2\pi} R(\cos \theta,\sin \theta) d\theta = -i\int_{|z|=1} \!\!\! R\left(\frac{1}{2}(z+\frac{1}{z}),\frac{1}{2i}(z-\frac{1}{z}) \right) \frac{dz}{z}
\end{equation}

\paragraph{Type 2}
\begin{equation}
	\int_{-\infty}^\infty R(x)\ dx = 2\pi i \sum_{y>0} Res\ R(z)
\end{equation}

\paragraph{Type 3}
\begin{equation}
	\int_{-\infty}^\infty R(x)e^{ix}\ dx = 2\pi i \sum_{y > 0} Res\ R(z)e^{iz}
\end{equation}
\begin{equation}
	\int_{-\infty}^\infty R(x) \sin x\ dx = Re \left( 2\pi i \sum_{y > 0} Res\ R(z)e^{iz} \right)
\end{equation}
\begin{equation}
	\int_{-\infty}^\infty R(x) \cos x\ dx = Im \left( 2\pi i \sum_{y > 0} Res\ R(z)e^{iz} \right)
\end{equation}

\paragraph{Type 4}
\begin{equation}
	\int_0^\infty x^\alpha R(x)\ dx	= (1-e^{2\pi i \alpha})\int_0^\infty z^{2\alpha+1} R(z^2)\ dz
\end{equation}

\paragraph{Type 5}
\begin{equation}
	\int_0^\pi \log \sin \theta\ d\theta = -\pi \log 2
\end{equation}

\paragraph{Exercise}
\begin{enumerate}
	\item Find the poles and residues of the following functions
	\begin{enumerate}
		\item 
			\[ \underset{z=-2}{Res}\ \frac{1}{z^2+5z+6} = \lim_{z \to -2} \frac{z+2}{(z+2)(z+3)} = 1  \]
			\[ \underset{z = -3}{Res}\ \frac{1}{z^2+5z+6} = \lim_{z \to -3} \frac{z+3}{(z+2)(z+3)} = -1  \]
		\item 
			\[ \underset{z=-1}{Res}\ \frac{1}{(z^2-1)^2} = \lim_{z \to -1} \frac{d}{dz}\frac{(z+1)^2}{(z+1)^2(z-1)^2} = \frac{1}{4} \]
			\[ \underset{z=1}{Res}\ \frac{1}{(z^2-1)^2} = \lim_{z \to 1} \frac{d}{dz}\frac{(z-1)^2}{(z+1)^2(z-1)^2} = -\frac{1}{4} \]
		\item 
			\[ \underset{z=2n\pi}{Res} \ \frac{1}{\sin z} = \lim_{z \to 2n\pi} \frac{z-2n\pi}{\sin z} = 1 \]
			\[ \underset{z=(2n+1)\pi}{Res} \ \frac{1}{\sin z} = \lim_{z \to (2n+1)\pi} \frac{z-2n\pi-\pi}{\sin z} = -1 \]
		\item
			\[ \underset{z=n\pi}{Res} \ \frac{\cos z}{\sin z} = \lim_{z \to n\pi} \frac{(z-n\pi)\cos z}{\sin z} = 1 \]
		\item 
			\[ \underset{z=n\pi}{Res} \ \frac{1}{\sin^2 z} = \lim_{z \to n\pi}\frac{d}{dz}\frac{(z-n\pi)^2}{\sin^2 z} = 0 \]
		\item
			\[ \underset{z=0}{Res} \ \frac{1}{z^m(1-z)^n} = \lim_{z \to 0} \frac{1}{m!} \frac{d^{m-1}}{dz^{m-1}}\frac{1}{(1-z)^n} =  \]
			\[ \underset{z=1}{Res} \ \frac{1}{z^m(1-z)^n} = \lim_{z \to 1} \frac{1}{n!} \frac{d^{n-1}}{dz^{n-1}}\frac{1}{z^m} =  \]
	\end{enumerate}
	\item 
	\item
	\begin{enumerate}
		\item hint : $\int_0^{2\pi} \frac{1}{a+\sin^2 x} = 4 \int_0^{\pi/2} \frac{1}{a+\sin^2 x}$.
			\[ R(\cos \theta,\sin \theta) = \frac{1}{a+\sin^2 x} \]
			\[ \int_0^{\pi/2} \frac{1}{a + \sin^2 x}\ dx = -i \int_{|z|=1} \frac{z}{z^4+(a-2)z^2+1} dz \]
		\item  hint : $\int_{-\infty}^\infty R(x)\ dx = 2\int_0^\infty R(x)\ dx$ if $R(-x) = R(x)$
			\[ R(x) = \frac{x^2}{x^4+5x^2+6} \]
			\[ \int_0^\infty \frac{x^2}{x^4+5x^2+6} = \pi i \sum_{y > 0} Res\ R(z) = \frac{(\sqrt{3}-\sqrt{2})\pi}{2} \]
			\[ \underset{z=\sqrt{2}i}{Res}\ R(z) = \lim_{z \to \sqrt{2}i} \frac{z^2(z-\sqrt{2}i)}{(z^2+2)(z^2+3)} = \frac{i}{\sqrt{2}} \]
			\[ \underset{z=\sqrt{3}i}{Res}\ R(z) = \lim_{z \to \sqrt{3}i} \frac{z^2(z-\sqrt{3}i)}{(z^2+2)(z^2+3)} = -\frac{\sqrt{3}i}{2} \]
		\item
			\[ R(x) = \frac{x^2-x+2}{x^4+10x^2+9} \]
			\[ \int_{-\infty}^\infty \frac{x^2-x+2}{x^4+10x^2+9} = 2\pi i \sum_{y > 0} Res\ R(z) = \frac{5\pi}{12}\]
			\[ \underset{z=i}{Res}\ R(z) = \lim_{z \to i} \frac{(z^2-z+2)(z-i)}{(z^2+1)(z^2+9)} = -\frac{1+i}{16} \]
			\[ \underset{z=3i}{Res}\ R(z) = \lim_{z \to 3i} \frac{(z^2-z+2)(z-3i)}{(z^2+1)(z^2+9)} = \frac{3-7i}{48} \]
		\item
			\[ R(x) = \frac{x^2}{(x^2+a^2)^3} \]
		\item
			\[ R(x)\cos x = \frac{1}{x^2+a^2} \cos x \]
		\item
			\[ R(x)\sin x = \frac{x}{x^2+a^2} \sin x \]
		\item
			\[ x^\alpha R(x) = x^{\frac{1}{3}} \frac{1}{1+x^2} \]
		\item 
		\item
	\end{enumerate}
\end{enumerate}
