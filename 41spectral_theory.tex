%Text Books : \cite{kreyszig}
%Module 1:
%Reflexive Spaces, Category theorem(statement only), Uniform Boundedness theorem ( applications excluded), Strong and Weak Convergence, Convergence of Sequences of Operators and Functionals, Open Mapping Theorem, Closed Linear Operators, Closed Graph Theorem
%(Chapter 4 - Sections 4.6 to 4.9, 4.12, 4.13) (20 Hours)
%Module 2:
%Banach Fixed point theorem, Spectral theory in Finite Dimensional Normed Spaces, Basic Concepts, Spectral Properties of Bounded Linear Operators, Further Properties of Resolvent and Spectrum, Use of Complex Analysis in Spectral Theory
%(Chapter5 – Section 5.1; Chapter 7 - Sections 7.1 to 7.5) (25 Hours)
%Module 3:
%Banach Algebras, Further Properties of Banach Algebras, Compact Linear Operators on Normed spaces, Further Properties of Compact Linear Operators, Spectral Properties of compact Linear Operators on Normed spaces, Further Spectral Properties of Compact Linear Operators 
%(Chapter 7 - Sections 7.6, 7.7; Chapter 8 - Sections 8.1 to 8.4) ( 25 Hours)
%Module 4:
%Spectral Properties of Bounded Self adjoint linear operators, Further Spectral Properties of Bounded Self Adjoint Linear Operators, Positive Operators, Projection Operators, Further Properties of Projections
%(Chapter 9 - Sections 9.1 to 9.3, 9.5, 9.6) (20 Hours)

%Module 1 - \cite{kreyszig} 4.6,7,8,9,12,13
%\chapter 1
\setcounter{section}{3}
{\Large Module 1}
\section{Normed and Banach Spaces}
\begin{commentary}
\begin{description}
	\item[Algebraically Reflexive]
		A vector space $X$ is algebraically reflexive if the canoical mapping $C : X \to X^{\ast\ast}$ defined by $C(x) = g_x \in X^{\ast\ast}$ is surjective where $X^\ast, X^{\ast\ast}$ are the first and second algebraic dual spaces of $X$ and $\forall f \in X^\ast,\ g_x(f) = f(x)$.\cite[\S2.8]{Kreyszig}\\

	\item Finite dimensional vector spaces are algebraically reflexive.\cite[\S2.9.3]{Kreyszig}
\end{description}
\end{commentary}

\setcounter{subsection}{5}
\subsection{Reflexive Spaces}
\begin{commentary}
\paragraph{\S4.6 Summary}
\begin{enumerate}
	\item reflexive $\implies$ completeness, However completeness $\nimplies$ reflexive
	\item finite dimensional $\implies$ reflexive
	\item Hilbert spaces are reflexive
	\item Separable space with non-separable dual space is non-reflexive
\end{enumerate}
\end{commentary}

\begin{lemma}
	For every fixed $x \in X$, the funcitonal $g_x : X^\prime \to X$ defined by $g_x(f) = f(x)$ is a bounded linear functional on $X'$ so that $g_x \in X^{\prime\prime}$, and has the norm
	\[ \|g_x\| = \|x\| \]
\end{lemma}
\begin{proof}
	From Hahn-Banach theorem for normed spaces, we know that for any $x \in X$
	\begin{equation}
		\|x\|  = \sup_{\substack{f \in X'\\ f \ne 0}} \frac{|f(x)|}{\|f\|}  =  \sup_{\substack{f \in X' \\ f \ne 0}} \frac{|g_x(f)|}{\|f\|} = \|g_x\| 
	\end{equation}
	Clearly, $g_x$ is bounded, $g_x \in X^{\prime\prime}$ and $\|g_x\| = \|x\|$.
\end{proof}

\begin{lemma}
	The Canonical mapping $C : X \to X^{\prime\prime}$ given by $C(x) = g_x$ such that $\forall f \in X^\prime,\ g_x(f) = f(x)$ is an isomorphism of $X$ onto $\mathscr{R}(C)$.
\end{lemma}
\begin{proof}
	\textbf{Step 1 : $C$ is linear}
	\begin{equation}
	g_{\alpha x + \beta y} = f(\alpha x + \beta y) = \alpha f(x) + \beta f(y) = \alpha g_x(f) + \beta g_y(f)
	\end{equation}
	\indent Therefore, $C$ is linear.\\

	\textbf{Step 2 : $C$ is injective}\\
	\indent We have, $C$ is linear.
	Thus, $g_{x-y} = g_x - g_y$.
	Then,
	\begin{equation}	
		\| C(x) - C(y) \| = \| g_x - g_y \| = \|g_{x-y}\| = \|x-y\|
	\end{equation}
	\indent Thus, $C$ is an isometry.
	If $C(x) = C(y)$, then $\|C(x) - C(y)\| = 0$.
	Thus, $\|x-y\| = 0$ implies $x = y$.
	Therefore, $C$ is injective.
	Therefore, $C$ is an isomorphism from $X$ onto $\mathscr{R}(C)$.
\end{proof}

\begin{definition}[Reflexive]
	A normed space $X$ is reflexive if the canonical mapping $C : X \to X^{\prime\prime}$ defined by $C(x) = g_x$ such that $\forall f \in X^\prime,\ g_x(f) = f(x)$ is surjective where $X^\prime, X^{\prime\prime}$ are the first and second dual spaces of $X$.
\end{definition}

Example, $\mathbb{R}^n$ is reflexive since ${\mathbb{R}^n}^{\prime\prime} = {\mathbb{R}^n}^\prime = \mathbb{R}^n$.

\begin{theorem}
	Reflexive $\implies$ Completeness
\end{theorem}
\begin{proof}
	Let $X$ be a reflexive normed space.
	Then $C : X \to X^{\prime\prime}$ is surjective and $X \simeq X^{\prime\prime}$.
	We know that, $X^{\prime\prime}$ is a dual space of $X^\prime$.
	Thus, $X^{\prime\prime}$ is complete since $B(X^\prime,K)$ is a Banach space.
	Therefore, $X$ is complete.
\end{proof}

\begin{theorem}
	Finite dimensional $\implies$ reflexive
\end{theorem}
\begin{proof}
	Let $X$ be a finite dimensional normed space.
	Then every linear operator $T : X \to K$ is bounded.
	Thus, $X^\prime = X^\ast$ and $dim\ X = dim\ X^\ast$.
	Thus, $X^{\prime\prime} = X^{\ast\ast}$.
	We know that, finite dimensional normed spaces are algebraically reflexive.
	Thus, $\mathscr{R}(C) = X^{\ast\ast} = X^{\prime\prime}$.
	Therefore, $X$ is reflexive.
\end{proof}

\begin{theorem}
	Every Hilbert space is reflexive.
\end{theorem}
\begin{proof}
	Let $H$ be a Hilbert space.
	Define $A : H^\prime \to H$, $A(f) = z$ such that $f(x) = \inner{x}{z}$.
	By Riesz representation theorem, $A$ is well-defined, bijective and conjugate linear.
	That is, $A(\alpha f) = \conj{\alpha}Af$.\\

	\textbf{Claim} : $H^\prime$ is a Hilbert space with inner product,
	\begin{equation}
		\inner{f_1}{f_2} = \inner{Af_2}{Af_1} = \inner{z_2}{z_1} 
	\end{equation}
	\begin{enumerate}[label=IP\arabic*]
		\item \begin{align*}
			\inner{f_1+f_2}{f_3} & = \inner{Af_3}{A(f_1+f_2)} = \inner{Af_3}{Af_1+Af_2} \\
			& = \inner{Af_3}{Af_1} + \inner{Af_3}{Af_2} = \inner{f_1}{f_3}+\inner{f_2}{f_3}
		\end{align*}
		\item $\inner{\alpha f_1}{f_2} = \inner{Af_2}{A(\alpha f_1)} = \inner{Af_2}{\conj{\alpha}Af_1} = \alpha \inner{Af_2}{Af_1}$
		\item $\inner{f_1}{f_2} = \inner{Af_2}{Af_1} = \conj{\inner{Af_1}{Af_2}} = \conj{\inner{f_2}{f_1}}$
		\item $\inner{f_1}{f_1} = \inner{Af_1}{Af_1}  \ge 0$ and \\
		$\inner{f_1}{f_1} = 0 \iff \inner{Af_1}{Af_1} = 0 \iff Af_1 = 0 \iff f_1 = 0$\\
	\end{enumerate}

	Similarly, let $g \in H^{\prime\prime}$ and $g(f) = \inner{f}{f_0}$ be its Riesz representation.
	Then,
	\[ g(f) = \inner{f}{f_0} = \inner{Af_0}{Af} \] 
	We have, $f(x) = \inner{x}{z}$.
	Let $Af_0 = x$.
	Then,
	\[ f(x) = \inner{x}{z} = \inner{Af_0}{Af} \]
	Thus, there exists $x \in H$ such that $g(f) = f(x)$.
	Therefore, $C : H \to H^{\prime\prime}$ is surjective.
	Thus, $\mathscr{R}(C) = H^{\prime\prime}$ and $H$ is reflexive.
\end{proof}

\begin{lemma}
	Let $Y$ be a proper, closed subspace of a normed space $X$.\\
	Let $x_0 \in X-Y$ and $\displaystyle \delta = \inf_{y \in Y} \| y-x_0\|$.
	Then, there exists $f \in X^\prime$ such that 
	\begin{equation}
		\|f\| = 1,\qquad f(y) = 0,\ \forall y \in Y,\qquad f(x_0) = \delta
	\end{equation}
\end{lemma}
\begin{proof}
	Let $Y$ be a proper, closed subspace of $X$.
	Let $x_0 \in X-Y$ and $\displaystyle \delta = d(x_0,Y) = \inf_{y \in Y} \|x_0-y\|$.
	Consider the subspace $Z$ spanned by $Y$ and $x_0$.
	Then, every element $z \in Z$ is of the form $z = y+\alpha x_0$ where $\alpha \in K$.\\

	Consider $f : Z \to K$ be defined by $f(y+\alpha x_0) = \alpha \delta$.
	Clearly, we have $f(y) = f(y+0x_0) = 0,\ \forall y \in Y$ and $f(x_0) = f(0+1x_0) = \delta$.\\

	\textbf{Step 1} : $f$ is linear.\\
	Let $z_1 = y_1 + \alpha_1 x_0$ and $z_2 = y_2 + \alpha_2 x_0$.
	Then, $f(z_1) = \alpha_1 \delta$ and $f(z_2) = \alpha_2 \delta$.
	\begin{align*}
	f(\beta_1 z_1 + \beta_2 z_2) 
		& = f(\beta_1 y+\beta_1 \alpha_1 x_0 + \beta_2 y + \beta_2 \alpha_2 x_0) \\
		& = f((\beta_1 + \beta_2)y + (\beta_1\alpha_1 + \beta_2\alpha_2)x_0) \\
		& = (\beta_1 \alpha_1 + \beta_2 \alpha_2 ) \delta
		 = \beta_1 \alpha_1 \delta + \beta_2 \alpha_2 \delta 
		 = \beta_1 f(z_1) + \beta_2 f(z_2)
	\end{align*}
	Thus, $f$ is linear.\\

	\textbf{Step 2} : $f$ is bounded.\\
	Suppose $z = y+\alpha x_0$.
	\[ |f(z)| 	
		 = |\alpha| \delta 
		 = |\alpha| \inf_{y' \in Y} \|y'-x_0\|  \]
	By the definition of infimum, we have
	$ \displaystyle \inf_{y^\prime \in Y} \|y^\prime-x_0\| \le \|y-x_0\|,\ \forall y \in Y$.
	Consider $-y/\alpha \in Y$. 
	\[ |f(z)| 	
		 \le |\alpha| \left\|\frac{-y}{\alpha}-x_0 \right\| 
		 \le \|-y - \alpha x_0\| 
		 \le \|z\| \]
	Thus, $f$ is a bounded and $\|f\| \le 1$.\\

	By the definition of infimum, there exists a sequence $\sequence{y_n}$ such that \\
	$z_n = y_n - x_0$ and $\|z_n\| = \|y_n - x_0\| \to \delta$ as $n \to \infty$.
	Clearly, $f(z_n) \to -\delta$ as $n \to \infty$.
	Then,
	\[ \|f\| = \sup_{z \in Z} \frac{f(z)}{\|z\|} \ge \frac{f(z_n)}{\|z_n\|} \]
	Thus, $\|f\| \ge 1$ as $n \to \infty$.
	Therefore, we have $\|f\| = 1$.	
	By Hahn-Banach theorem, we can extended $f$ to $X$ without increasing the norm.
\end{proof}

\begin{theorem}
	If $X^\prime$ is separable, then $X$ is separable.
\end{theorem}
\begin{proof}
	Suppose $X^\prime$ is separable.
	Consider the countable dense subset of the unit sphere, $\sequence{f_n}$.
	Clearly, $\displaystyle \|f_n\| = \sup_{\|x\|=1}|f_n(x)| = 1$.
	By the definition of supremum, there exists a sequence $\sequence{x_n}$ in $X$ such that $\|x_n\| = 1$ and $\|f_n(x_n)\| \ge \frac{1}{2}$.\\

	Suppose $X$ is not separable.
	Let $Y$ be the span of $\sequence{x_n}$.
	Then $Y$ is a separable subspace of the non-separable space $X$.
	Since $Y$ is closed, there exists bounded, linear functional $f \in X'$ such that $\|f\| = 1$, $f(y) = 0,\forall y \in Y$ and $f(x_n) = 0,\forall n$.
	\begin{align*}
	\frac{1}{2}
		& \le \|f_n(x_n)\| 
		 \le \|f_n(x_n) - f(x_n)\| \\
		& \le \|(f_n-f)(x_n)\| 
		 \le \|f_n-f\|\ \|x_n\| 
		 \le \|f_n-f\|
	\end{align*}
	Then, $\sequence{f_n}$ is not dense in unit sphere of $X'$ which is a contradiction.
	Therefore, $X$ is separable.
\end{proof}

\begin{remark}
	Let $X$ be separable space with a non-separable dual space $X^\prime$.
	Then $X$ is non-reflexive.
\end{remark}
\begin{proof}
	Let $X$ be a separable space with non-separable dual space $X^\prime$.
	Suppose $X$ is reflexive.
	Then $X \simeq X^{\prime\prime}$ and $X^{\prime\prime}$ is separable.
	However, $X^\prime$ is separable since $X^{\prime\prime}$ is separable, which is a contradiction.
	Therefore, $X$ is not reflexive.
\end{proof}
\begin{important}
	$l^1$ is non-reflexive.
\end{important}
\begin{proof}
	$l^1$ is separable and ${l^1}^\prime = l^\infty$ is non-separable.
	Thus, $l^1$ is non-reflexive.
\end{proof}

\paragraph{Problems}
\begin{enumerate}
	\item What the functionals $f,g_x$ if $X = \mathbb{R}^n$.
		\begin{proof}[Answer]
			Every linear functional on finite dimensional space is bounded.
			Let $x = (x_1,x_2,\dots,x_n)$.
			A linear functional $f : \mathbb{R}^n \to \mathbb{R}$ is of the form,
			\[ f(x) = a_0+a_1x_1+\dotsb+a_nx_n,\ a_k \in \mathbb{R} \]
			Let $F = \{ f : \mathbb{R}^n \to \mathbb{R} : f \text{ is linear } \}$.
			A linear functional $g_x : F \to \mathbb{R}$ is of the form
			\[ g_x(f) = a_0 + a_1f_1(x)+a_2f_2(x)+\dotsb+a_nf_n(x) \text{ where } f_j(e_i) = \delta_{ij},\  \]
		\end{proof}
	\item Prove lemma 4.6-7 for Hilbert space $X$
		\begin{proof}[Answer]
		\end{proof}
	\item If a normed space $X$ is reflexive, show that $X^\prime$ is reflexive
		\begin{proof}[Answer]
		\end{proof}
	\item Show that a Banach space $X$ is reflexive if and only if its dual space $X^\prime$ is reflexive.
		\begin{proof}[Answer]
		\end{proof}
	\item Let $Y$ be a proper, closed subspace of a normed space $X$.
		Let $x_0 \in X-Y$.
		Then there exists a bounded, linear functional $h$ on $X$ such that $\|h\| = \frac{1}{\delta}$, $h(y) = 0,\ \forall y \in Y$ and $h(x_0) = 1$.
		\begin{proof}[Answer]
		\end{proof}
	\item Show that different closed subspaces $Y_1,Y_2$ of a normed space $X$ have different annihilators.
		\begin{proof}[Answer]
		\end{proof}
	\item Let $Y$ be a closed subsapce of a normed space $X$ such that every $f \in X^\prime$ which is zero everywhere on $Y$ is zero everywhere on the whole space $X$.
		Show that then $Y = X$.
		\begin{proof}[Answer]
		\end{proof}
	\item Let $M$ be any subset of a normed space $X$.
		Show that an $x_0 \in X$ is an element of $A = \underline{span\ M}$ if and only if $f(x_0) = 0,\ \forall f \in X^\prime$ such that $f|_M = 0$.
		\begin{proof}[Answer]
		\end{proof}
	\item Show that a subset $M$ of a normed space $X$ is total in $X$ if and only if every $f \in X^\prime$ which is zero everywhere on $M$ is zero everywhere in $X$.
		\begin{proof}[Answer]
		\end{proof}
	\item Show that if a normed space $X$ has a linearly independent subset of $n$ elements, so does the dual space $X^\prime$.
		\begin{proof}[Answer]
		\end{proof}
\end{enumerate}
\pagebreak

\subsection{Category Theorem}
\begin{description}
	\item[rare] A subset $M$ in $X$ is rare if its closure has no interior points.
	\item[meager] A subset $M$ is meager in $X$ if $M$ is union of countable family of rare subsets of $X$.
	\item[non-meager] A subset $M$ is non-meager in $X$ if it is not meager in $X$.
\end{description}

\begin{theorem}[Blaire's Category Theorem]
	Every complete, non-empty metric space is nonmeager in itself.
\end{theorem}
\begin{proof}
	Let $X$ be a complete, non-empty metric space.
	Suppose $X$ is meager in itself.
	Then, there exists a countable family of rare sets $M_k$ such that 
	\[ X = \bigcup_{k=1}^\infty M_k \]

	We have, $M_1$ is rare in $X$ and $\bar{M_1} \ne X$.
	Thus, $\bar{M_1}^c \ne \phi$ and $\bar{M_1}^c$ is an open set.
	Let $p_1 \in \bar{M_1}^c$ such that 
	\[ p_1 \in B_1 = B(p_1,\varepsilon_1) \subset \bar{M_1}^c \text{ where }\varepsilon_1 < \frac{1}{2} \]

	We have, $M_2$ is rare in $X$.
	Then $\bar{M_2}^c \cap B(p_1,\frac{\varepsilon_1}{2}) \ne \phi$.
	Suppose $\bar{M_2}^c \cap B(p_1,\frac{\varepsilon_1}{2}) = \phi$.
	Then $B(p_1,\frac{\varepsilon_1}{2}) \subset \bar{M_2}$ which is a contradiction.\\

	Let $p_2 \in \bar{M_2}^c \cap B(p_1,\frac{\varepsilon_1}{2})$ such that 
	\[p_2 \in B_2 = B(p_2,\varepsilon_2) \subset \bar{M_2}^c \cap B(p_1,\frac{\varepsilon_1}{2}) \text{ where }\varepsilon_2 < \frac{\varepsilon_1}{2} \]
	Then, $\bar{M_3}^c \cap B(p_2,\frac{\varepsilon_2}{2}) \ne \phi$.
	Let $p_3 \in \bar{M_3}^c \cap B(p_2,\frac{\varepsilon_2}{2})$ such that 
	\[ p_3 \in B_3 = B(p_3,\varepsilon_3) \subset \bar{M_3}^c \cap B(p_2,\frac{\varepsilon_2}{2}) \text{ where }\varepsilon_3 <\frac{\varepsilon_2}{2} \]

	Continuing like this, we get a sequence $\sequence{p_n}$ such that 
	\[\forall k \in \mathbb{N},\ p_k \in B_k = B(p_k,\varepsilon_k) \subset \bar{M_k}^c \cap B(p_{k-1},\frac{\varepsilon_{k-1}}{2}) \text{ where }\varepsilon_k < \frac{\varepsilon_{k-1}}{2} \]
	We have, \[B_{k+1} \subset B(p_k,\frac{\varepsilon_k}{2}) \subset B_k,\qquad \varepsilon_k < \frac{1}{2^k} \]
	This sequence is Cauchy since the open ball $B_{k+1} \subset B_k \subset \dotsb \subset B_3 \subset B_2 \subset B_1$ and the radius of the open balls, $\varepsilon_k < 2^{-k}$ reduces at least by half in each successive step.\\

	Since $X$ is complete, the sequence $\sequence{p_n}$ converges to $p \in X$.
\end{proof}

\begin{theorem}[Uniform Boundedness]
	Let $X$ be a Banach space and $Y$ be a normed space.
	Let $\sequence{T_n}$ be sequence of bounded, linear operators $T_n : X \to Y$ such that $\sequence{\|T_n(x)\|}$ is bounded for every $x \in X$.
	Then the sequence of norms $\sequence{\|T_n\|}$ is bounded.
\end{theorem}
\begin{proof}
	Let $X$ be a Banach space.
	Let $X$
\end{proof}

\begin{remark}
	Space of polynomials $x(t) = \sum_j \alpha_j t^j$ with norm $\|x\| = \max_j |\alpha_j|$ is not complete.
\end{remark}
\begin{proof}
\end{proof}

\begin{remark}
	There exist real-valued continuous functions whose Fourier series diverge at a given point $t_0$.
\end{remark}
\begin{proof}
\end{proof}

\subsection{Strong and Weak Convergence}
\subsection{Convergence of Sequences of Operators and Functionals}
%\subsection{Application to Summability of Sequences}
%\subsection{Numerical Integration and Weak\ast Convergence}
\setcounter{subsection}{11}
\subsection{Open Mapping Theorem}
\subsection{Closed Linear Operators}
\pagebreak

%Module 2 - \cite{kreyszig} 5.1, 7.1,2,3,4,5
{\Large Module 2}
\section{Linear Operators}
\subsection{Bananch Fixed Point Theorem}
%\subsection{Application of Bananch Theorem to Linear Equations}
\setcounter{section}{6}
\section{Spectral Theory of Linear Operators in Normed Spaces}
\subsection{Spectral Theory in Finite Dimensional Normed Spaces}
\subsection{Basic Concepts}
\subsection{Spectral Properties of Bounded Linear Operators}
\subsection{Further Properties of Resolvant and Spectrum}
%Wednesday 13 July 2022 10:44:06 AM IST
\subsubsection{Resolvant Equation}
\begin{theorem}
	Let $X$ be a complex Banach space.
	Let $T \in B(X,X)$ and $\lambda,\mu \in \rho(T)$.
	Then
	\begin{enumerate}
		\item $R_\mu - R_\lambda = (\mu - \lambda)R_\mu R_\lambda$ 
		\item $R_\lambda$ commutes with any $S \in B(X,X)$ which commutes with $T$
		\item $R_\mu R_\lambda = R_\lambda R_\mu$
	\end{enumerate}
\end{theorem}
\begin{proof}
	\textbf{ Part 1 : } Hilbert Relation/Resolvant Equation
	\begin{align*}
		R_\mu - R_\lambda 
		& = R_\mu (T_\lambda R_\lambda) - (R_\mu T_\mu)R_\lambda \\
		& = R_\mu (T_\lambda R_\lambda - T_\mu R_\lambda)\\
		& = R_\mu (T_\lambda - T_\mu) R_\lambda \\
		& = R_\mu (T-\lambda I - T+\mu I) R_\lambda \\
		& = R_\mu (\mu - \lambda) R_\lambda
	\end{align*}
	\textbf{ Part 2 : } Commutativity Property 1\\
		Suppose $T$ commutes with $S$, that is $TS = ST$.
		Let $\lambda \in \rho(T)$.
		Then 
		\[ T_\lambda S = (T - \lambda I)S = TS - \lambda S = ST - \lambda S = S(T-\lambda I) = ST_\lambda \]
	\begin{align*}
		R_\lambda S 
		& = R_\lambda S (T_\lambda R_\lambda)\qquad \because T_\lambda R_\lambda = I\\
		& = R_\lambda T_\lambda S R_\lambda \qquad \because ST_\lambda = T_\lambda S\\
		& = S R_\lambda
	\end{align*}
	That is, $T$ commutes with $R_\lambda$.\\
	\textbf{ Part 3 : } Commutativity Property 2\\
	$T$ commutes with $T \implies T$ commutes with $R_\mu$ by part 2.\\
	$T$ commutes with $R_\mu \implies R_\mu$ commutes with $R_\lambda$ by part 2.
	\end{proof}
\subsubsection{Spectral Mapping Theorem - Polynomials}
\begin{theorem}
	Let $X$ be a complex Bananch space.
	Let $T \in B(X,X)$ and $p$ be a complex polynomial given by
	\[ p(\lambda) = \alpha_n \lambda^n + \alpha_{n-1} \lambda^{n-1} + \dotsb + \alpha_0,\ \alpha_j \in \mathbb{C},\ \alpha_n \ne 0 \]
	Then,
	\begin{equation}
		p(\sigma(T)) = \sigma(p(T)) 
	\end{equation}
\end{theorem}
\begin{proof}
	The result is trivial when $\sigma(T) = \phi$ or $n = 0$.
	Suppose $\sigma(T) \ne \phi$ and $n > 0$.\\

	\textbf{Part 1 :} $\sigma(p(T)) \subset p(\sigma(T))$ \\
	Let $\mu \in \sigma(p(T))$.
	Define $S = p(T)$.
	Then $\mu \in \sigma(S)$ and $S_\mu = p(T) - \mu I$.
	Define $s_\mu(\lambda) = p(\lambda)-\mu$.
	Being a complex polynomial, $s_\mu(\lambda)$ must factorise into linear factors.
	Thus,
	\[ s_\mu(\lambda) = \alpha_n (\lambda-\gamma_1)(\lambda-\gamma_2) \dotsm (\lambda-\gamma_n) \]
	And,
	\[ S_\mu = \alpha_n (T-\gamma_1 I)(T-\gamma_2 I) \dotsm (T-\gamma_n I) \]
	Suppose $\gamma_k \in \rho(T)$ for every $k$.
	Then $(T-\gamma_k I)$ has bounded inverse and consequently $S_\mu$ will have a bounded inverse,
	\[ S_\mu^{-1} = \frac{1}{\alpha_n} (T-\gamma_1 I)^{-1} (T-\gamma_2 I)^{-1} \dotsm (T-\gamma_n I)^{-1} \]
	This is a contradiction, since $\mu \in \sigma(S)$.
	Thus there exists some $\gamma_k \in \sigma(T)$.
	\[ s_\mu(\gamma_k) = p(\gamma_k) - \mu = 0 \]
	Thus, $\mu = p(\gamma_k)$ where $\gamma_k \in \sigma(T)$.
	Therefore, $\mu \in p(\sigma(T))$.\\

	\textbf{Part 2 :} $p(\sigma(T)) \subset \sigma(p(T))$\\
	Let $\kappa \in p(\beta)$ where $\beta \in \sigma(T)$.
	Now, there are two alternate cases.\\

	\textbf{Part 2A :} $(T-\beta I)$ has no inverse\\
	Suppose $(T-\beta I)$ has no inverse.
	Define $s_\kappa(\lambda) = p(\lambda)-\kappa = (\lambda-\kappa)g(\lambda)$.
	And $S_\kappa = (T-\beta I)g(T)$.
	Clearly, $g(T)$ commutes with $(T-\beta I)$.
	Thus, $S_\kappa = g(T)(T-\beta I)$.
	If $S_\kappa$ has an inverse, then
	\[ I = (T-\beta I)g(T)S_\kappa^{-1} = S_\kappa^{-1} g(T)(T-\beta I) \]
	Then, $(T-\beta I)$ has an inverse which is a contradiction.
	Therefore $S_\kappa$ does not have an inverse.
	Thus, $\kappa \in \sigma(p(T))$.\\

	\textbf{Part 2B :} $(T-\beta I)$ has an inverse\\
	Suppose $(T-\beta I)$ has an inverse.
	We have, $\mathscr{R}(T-\beta I) \ne X$.
	If not, the map is surjective and by bounded inverse theorem, $(T-\beta I)^{-1}$ is bounded.
	Then, $\beta$ is a regular point of $T$.
	That is, $\beta \notin \sigma(T)$.
	This is a contradiction.\\

	We have $S_\kappa = (T-\beta I)g(T)$.
	Thus $\mathscr{R}(S_\kappa) \ne X$.
	And $\kappa \in \sigma(p(T))$.
\end{proof}
\subsubsection{Linear Independence}
\begin{theorem}
	Let $T$ be a linear operators and $\lambda_1,\lambda_2,\dotsc$ be its eigenvalues.
	Then the eigenvectors $x_1,x_2,\dots,x_n$ corresponding to different eigenvalues $\lambda_1,\lambda_2,\dots,\lambda_n$ form a linearly independent set.
\end{theorem}
\begin{proof}
	Suppose $\{x_1,x_2,\dots,x_n\}$ is linearly dependent.
	Let $x_m$ be the first vector such that $\{ x_1,x_2,\dots,x_{m-1},x_m\}$ is linearly dependent.
	\[ x_m = \alpha_1 x_1 + \alpha_2 x_2 + \dotsb + \alpha_{m-1}x_{m-1} \]
	Then,
	\begin{align*}
		(T-\lambda_m I)x_m 
		= & \sum_{j=1}^{m-1} \alpha_j (T-\lambda_m I)x_j \\
		0 = & \sum_{j=1}^{m-1} \alpha_j (\lambda_j-\lambda_m)x_j
	\end{align*}
	This is a contradiction to the assumption that $\{ x_1,x_2,\dots,x_{m-1} \}$ is linearly independent.
	Thus, no such $m$ exists.
	Therefore, $\{x_1,x_2,\dots,x_n\}$ is linearly independent.
\end{proof}
\pagebreak

\subsection{Use of Complex Analysis in Spectral Theory}

%Module 3 - \cite{kreyszig} 7.6,7, 8.1,2,3,4
{\Large Module 3}
\subsection{Banach Algebra}
\subsection{Further Properties of Banach Algebra}

\section{Compact Linear Operators}
\subsection{Compact Linear Operators on Normed Spaces}
\subsection{Further Properties of Compact Linear Operators}
\subsection{Spectral Properties of Compact Linear Operators on Normed Spaces}
\subsection{Further Spectral Properties of Compact Linear Operators}
\pagebreak

%Module 4 - \cite{kreyszig} 9.1,2,3,5,6
{\Large Module 4}
\section{Bounded, Self-Adjoint Linear Operators}
\subsection{Spectral Properties of Bounded Self-Adjoint Linear Operators}
\textbf{Note} : A bounded, self-adjoint operator may not have eigenvalues.
\begin{theorem}
	Let $H$ be a complex, Hilbert space.
	Let $T : H \to H$ be a bounded, self-adjoint linear operator.
	Then
	\begin{enumerate}
		\item All the eigenvalues(if they exist) are real.
		\item Eigenvectors corresponding to different eigenvalues are orthogonal.
	\end{enumerate}
\end{theorem}
\begin{proof}
	\textbf{Part 1} : Eigenvalues are real.\\
	Let $\lambda$ be an eigenvalue of $T$ and $x \in H$ such that $Tx = \lambda x$, ($x \ne 0$).
	Then,
	\[ \lambda\inner{x}{x} = \inner{\lambda x}{x} = \inner{Tx}{x} = \inner{x}{Tx} = \inner{x}{\lambda x} = \conj{\lambda}\inner{x}{x} \]
	Since $x \ne 0$, we have $\inner{x}{x} \ne 0$ and $\lambda \in \mathbb{R}$.\\

	\textbf{Part 2} : Eigenvectors are orthogonal.\\
	Let $\lambda,\mu$ be eigenvalues of $T$ and $x,y$ be corresponding eigenvectors.
	That is, $Tx = \lambda x$ and $Ty = \mu y$.
	\[ \lambda \inner{x}{y} = \inner{\lambda x}{y} = \inner{Tx}{y} = \inner{x}{Ty} = \inner{x}{\mu y} = \conj{\mu} \inner{x}{y} = \mu \inner{x}{y} \]
	Since $\lambda \ne \mu$, we have $\inner{x}{y} = 0$.
	That is, $x \perp y$.
\end{proof}

\begin{theorem}
	Let $T : H \to H$ be a bounded, self-adjoint linear operator.
	Then $\lambda \in \rho(T)$ if and only if there exists $c > 0$ such that $\|T_\lambda x\| \ge c\|x\|$.
\end{theorem}
\begin{proof}
	Let $\lambda \in \rho(T)$.
	Then $R_\lambda = T_\lambda^{-1} : H \to H$ exists and is bounded.
	Since $R_\lambda \ne 0$, $\|R_\lambda\| = k > 0$.
	\[ \|x\| = \| R_\lambda T_\lambda x\| \le \|R_\lambda\| \|T_\lambda x\| = k\|T_\lambda x\| \]
	Thus,
	\begin{equation}
		\|T_\lambda x\| \ge c\|x\| \text{ where } c = \frac{1}{k} 
	\end{equation}

	\textbf{Converse Part}\\
	Suppose $T_\lambda x_1 = T_\lambda x_2$.
	Then,
	\[ 0 = \|T_\lambda x_1 - T_\lambda x_2\| = \| T_\lambda (x_1-x_2) \| \ge c\|x_1-x_2\| \]
	Since $c > 0$, we have $\|x_1 - x_2\| = 0$ and$x_1 = x_2$.
	Therefore, $T_\lambda : H \to T_\lambda(H)$ is bijective.\\

	Let $x_0 \perp \overline{T_\lambda(H)}$.
	Then $x_0 \perp T_\lambda(H)$.
	\[ 0 =  \inner{T_\lambda x}{x_0} = \inner{Tx}{x_0} - \inner{\lambda x}{x_0} = \inner{Tx}{x_0} - \lambda\inner{x}{x_0} \]
	\[ \inner{x}{\conj{\lambda} x_0} = \lambda \inner{x}{x_0} = \inner{Tx}{x_0} = \inner{x}{Tx_0} \]
	Thus, $Tx_0 = \conj{\lambda}x_0$.\\

	Suppose $x_0 \ne 0$.
	Then, $\conj{\lambda}$ is an eigenvalue of $T$.
	We know that, the eigenvalues of bounded, self-adjoint linear operators are real.
	Thus, $\lambda \in \mathbb{R}$.
	That is, $Tx_0 = \lambda x_0$.
	Then, $T_\lambda x_0 = Tx_0 - \lambda x_0 = 0$ which is a contradiction since $0 = \|T_\lambda x_0\| \ge c\|x_0\| > 0$.
	Therefore, $x_0 = 0$.\\

	By projection theorem, $H = Y \oplus Z$ where $Y = \overline{T_\lambda(H)},\ Z = Y^\perp = \{ 0 \}$.
	Thus, $\overline{T_\lambda(H)} = H$.\\

	Suppose $y \in \overline{T_\lambda(H)}$.
	Then there exists a sequence $\sequence{y_n}$ in $T_\lambda(H)$ such that $y_n \to y$.
	Since $y_n \in T_\lambda(H)$, there exists $x_n \in H$ such that $y_n = T_\lambda x_n$.
	The sequence $\sequence{x_n}$ is Cauchy since
	\[ \|x_n - x_m\| \le \frac{1}{c}\|T_\lambda(x_n-x_m)\| = \frac{1}{c}\|y_n-y_m\| \]
	We know that, $\sequence{y_n}$ is convergent and thus Cauchy.
	Thus, sequence $\sequence{x_n}$ is Cauchy in $H$.
	Since $H$ is complete, $x_n \to x \in H$.
	And $T$ is continuous.
	Therefore, $y_n = T_\lambda x_n \to T_\lambda x$.
	Since the limit is unique, $T_\lambda x = y$.
	And $y \in T_\lambda(H)$.
	Therefore, $T_\lambda(H)$ is closed.
\end{proof}

\begin{theorem}[Spectrum]
	Let $T : H \to H$ be a bounded, self-adjoint linear operator on a complex Hilbert space $H$.
	Then the spectrum $\sigma(T)$ of $T$ is real.
\end{theorem}
\begin{proof}
	Let $\lambda \in \sigma(T)$ where $\lambda = \alpha + i\beta$.
	\[ \inner{T_\lambda x}{x} = \inner{Tx}{x} - \lambda\inner{x}{x} \]
	We know that $\inner{Tx}{x} \in \mathbb{R}$ since $T$ is self-adjoint.
	Thus,
	\[ \overline{\inner{T_\lambda x }{x}} = \overline{\inner{Tx}{x}} - \bar{\lambda}\overline{\inner{x}{x}} = \inner{Tx}{x} - \bar{\lambda}\inner{x}{x} \]
	Then, the imaginary part of $\inner{T_\lambda x}{x}$ can be obtained as,
	\[ -2i\ Im \inner{T_\lambda x}{x} = \overline{\inner{T_\lambda x}{x}} - \inner{T_\lambda x}{x} = (\lambda-\bar{\lambda})\inner{x}{x} = 2i\beta \|x\|^2 \]
	The absolute value of imaginary part is always less than or equal to the absolute of the complex number.
	And by Schwarz inequality,
	\[ |\beta|\ \|x\|^2 = |Im\inner{T_\lambda x}{x}| \le |\inner{T_\lambda x}{x}| \le \|T_\lambda x\|\ \|x\| \]
	Then, $|\beta|\ \|x\| \le \|T_\lambda x\|$.
	Suppose $\beta \ne 0$.
	Then we have $|\beta| = c > 0$ such that $\inner{T_\lambda x}{x} \| \ge c\|x\|$.
	That is, $\lambda \in \rho(T)$ which is a contradiction.
	Therefore, $\beta = 0$ and $\lambda \in \mathbb{R}$.
\end{proof}

\subsubsection{Exercise}
\begin{enumerate}
	\item 
	\item
	\item $R_\lambda$ is bounded since
		\[ \|x\| = \|T_\lambda R_\lambda x\| \ge c\|R_\lambda x\| \]
	\item $T : H \to H$ defined by $Tx = \lambda_0 x$.
		Then $c \le |\lambda-\lambda_0|$.
	\item $\inner{W^\ast TWx}{x} = \inner{TWx}{Wx} = \inner{Wx}{T^\ast Wx} = \inner{Wx}{TWx} = \inner{x}{W^\ast TWx}$ 
	\item $T$ is bounded, but not self-adjoint.
		And $S:l^2 \to l^2$ is defined by $(\xi_1,\xi_2,\xi_3,\dotsc) \to (0,\xi_1,\xi_2,\xi_3,\dotsc)$.
	\item
	\item closure or exterior ? (I don't get it)
	\item Let $x,x' \in L^2[0,1]$ be real-valued functions which are Lebesgue square-integrable.
		Then $\bar{x}(t) = x(t)$ and $\bar{x}'(t) = \bar{x}'(t)$.
		\[ \|Tx(t)\| = \|y(t)\| = \|tx(t)\| \le \|x(t)\|\quad \because t \in [0,1] \]
		\[ \inner{Tx(t)}{x'(t)} = \int_0^1 tx(t)\bar{x}'(t)\ dt = \inner{\bar{x}(t)}{T\bar{x}'(t)} = \inner{x(t)}{Tx'(t)} \]
	\item
\end{enumerate}

\subsection{Further Spectral Properties of Bounded Self-Adjoint Linear Operators}
\subsection{Positive Operators}
%\subsection{Square Roots of a Positive Operator}
\setcounter{subsection}{4}
\subsection{Projection Operators}
\subsection{Further Properties of Projections}
%Missing - Chapter 6, 10
