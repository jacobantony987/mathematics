%Text Books : \cite{kreyszig}
%Module 1:
%Reflexive Spaces, Category theorem(statement only), Uniform Boundedness theorem ( applications excluded), Strong and Weak Convergence, Convergence of Sequences of Operators and Functionals, Open Mapping Theorem, Closed Linear Operators, Closed Graph Theorem
%(Chapter 4 - Sections 4.6 to 4.9, 4.12, 4.13) (20 Hours)
%Module 2:
%Banach Fixed point theorem, Spectral theory in Finite Dimensional Normed Spaces, Basic Concepts, Spectral Properties of Bounded Linear Operators, Further Properties of Resolvent and Spectrum, Use of Complex Analysis in Spectral Theory
%(Chapter5 – Section 5.1; Chapter 7 - Sections 7.1 to 7.5) (25 Hours)
%Module 3:
%Banach Algebras, Further Properties of Banach Algebras, Compact Linear Operators on Normed spaces, Further Properties of Compact Linear Operators, Spectral Properties of compact Linear Operators on Normed spaces, Further Spectral Properties of Compact Linear Operators 
%(Chapter 7 - Sections 7.6, 7.7; Chapter 8 - Sections 8.1 to 8.4) ( 25 Hours)
%Module 4:
%Spectral Properties of Bounded Self adjoint linear operators, Further Spectral Properties of Bounded Self Adjoint Linear Operators, Positive Operators, Projection Operators, Further Properties of Projections
%(Chapter 9 - Sections 9.1 to 9.3, 9.5, 9.6) (20 Hours)

%Module 1 - \cite{kreyszig} 4.6,7,8,9,12,13
%\chapter 1
\setcounter{section}{3}
{\Large Module 1}
\section{Normed and Banach Spaces}
\setcounter{subsection}{5}
\subsection{Reflexive Spaces}
\begin{commentary}
\begin{description}
	\item[Algebraically Reflexive]
		A vector space $X$ is algebraically reflexive if the canoical mapping $C : X \to X^{\ast\ast}$ defined by $C(x) = g_x \in X^{\ast\ast}$ is surjective where $X^\ast, X^{\ast\ast}$ are the first and second algebraic dual spaces of $X$ and $\forall f \in X^\ast,\ g_x(f) = f(x)$.\cite[\S2.8]{Kreyszig}\\

	\item Finite dimensional vector spaces are algebraically reflexive.\cite[\S2.9.3]{Kreyszig}
\end{description}
\end{commentary}

\begin{commentary}
\paragraph{Summary}
\begin{enumerate}
	\item Reflexive $\implies$ Completeness
	\item Completeness $\nimplies$ Reflexive
	\item Finite Dimensional $\implies$ Reflexive
	\item Hilbert Space $\implies$ Reflexive
	\item Separable space with non-separable dual space is non-reflexive
	\item $X^\prime$ separable $\implies X$ separable
\end{enumerate}
\end{commentary}

\begin{lemma}
	For every fixed $x \in X$, the funcitonal $g_x : X^\prime \to X$ defined by $g_x(f) = f(x)$ is a bounded linear functional on $X'$ so that $g_x \in X^{\prime\prime}$, and has the norm
	\[ \|g_x\| = \|x\| \]
\end{lemma}
\begin{proof}
	From Hahn-Banach theorem for normed spaces, we know that for any $x \in X$
	\begin{equation}
		\|x\|  = \sup_{\substack{f \in X'\\ f \ne 0}} \frac{|f(x)|}{\|f\|}  =  \sup_{\substack{f \in X' \\ f \ne 0}} \frac{|g_x(f)|}{\|f\|} = \|g_x\| 
	\end{equation}
	Clearly, $g_x$ is bounded, $g_x \in X^{\prime\prime}$ and $\|g_x\| = \|x\|$.
\end{proof}

\begin{lemma}
	The Canonical mapping $C : X \to X^{\prime\prime}$ given by $C(x) = g_x$ such that $\forall f \in X^\prime,\ g_x(f) = f(x)$ is an isomorphism of $X$ onto $\mathscr{R}(C)$.
\end{lemma}
\begin{proof}
	\textbf{Step 1 : $C$ is linear}
	\begin{equation}
	g_{\alpha x + \beta y} = f(\alpha x + \beta y) = \alpha f(x) + \beta f(y) = \alpha g_x(f) + \beta g_y(f)
	\end{equation}
	\indent Therefore, $C$ is linear.\\

	\textbf{Step 2 : $C$ is injective}\\
	\indent We have, $C$ is linear.
	Thus, $g_{x-y} = g_x - g_y$.
	Then,
	\begin{equation}	
		\| C(x) - C(y) \| = \| g_x - g_y \| = \|g_{x-y}\| = \|x-y\|
	\end{equation}
	\indent Thus, $C$ is an isometry.
	Therefore, $C$ is injective.\\

	Therefore, $C$ is an isomorphism from $X$ onto $\mathscr{R}(C)$.
\end{proof}

\begin{definition}[Reflexive]
	A normed space $X$ is reflexive if the canonical mapping $C : X \to X^{\prime\prime}$ defined by $C(x) = g_x$ such that $\forall f \in X^\prime,\ g_x(f) = f(x)$ is surjective where $X^\prime, X^{\prime\prime}$ are the first and second dual spaces of $X$.
\end{definition}

Example, $\mathbb{R}^n$ is reflexive since ${\mathbb{R}^n}^{\prime\prime} = {\mathbb{R}^n}^\prime = \mathbb{R}^n$.

\begin{theorem}
	Reflexive $\implies$ Completeness
\end{theorem}
\begin{proof}
	Let $X$ be a reflexive normed space.
	Then $C : X \to X^{\prime\prime}$ is surjective and $X \simeq X^{\prime\prime}$.
	We know that, $X^{\prime\prime}$ is a dual space of $X^\prime$.
	Thus, $X^{\prime\prime}$ is complete since $B(X^\prime,K)$ is a Banach space.
	Therefore, $X$ is complete.
\end{proof}

\begin{theorem}
	Finite dimensional $\implies$ reflexive
\end{theorem}
\begin{proof}
	Let $X$ be a finite dimensional normed space.
	Then every linear operator $T : X \to K$ is bounded.
	Thus, $X^\prime = X^\ast$ and $dim\ X = dim\ X^\ast$.
	Thus, $X^{\prime\prime} = X^{\ast\ast}$.
	We know that, finite dimensional normed spaces are algebraically reflexive.
	Thus, $\mathscr{R}(C) = X^{\ast\ast} = X^{\prime\prime}$.
	Therefore, $X$ is reflexive.
\end{proof}
\subsection{Category Theory}
\subsection{Strong and Weak Convergence}
\subsection{Convergence of Sequences of Operators and Functionals}
%\subsection{Application to Summability of Sequences}
%\subsection{Numerical Integration and Weak\ast Convergence}
\setcounter{subsection}{11}
\subsection{Open Mapping Theorem}
\subsection{Closed Linear Operators}
\pagebreak

%Module 2 - \cite{kreyszig} 5.1, 7.1,2,3,4,5
{\Large Module 2}
\section{Linear Operators}
\subsection{Bananch Fixed Point Theorem}
%\subsection{Application of Bananch Theorem to Linear Equations}
\setcounter{section}{6}
\section{Spectral Theory of Linear Operators in Normed Spaces}
\subsection{Spectral Theory in Finite Dimensional Normed Spaces}
\subsection{Basic Concepts}
\subsection{Spectral Properties of Bounded Linear Operators}
\subsection{Further Properties of Resolvant and Spectrum}
\subsection{Use of Complex Analysis in Spectral Theory}
\pagebreak

%Module 3 - \cite{kreyszig} 7.6,7, 8.1,2,3,4
{\Large Module 3}
\subsection{Banach Algebra}
\subsection{Further Properties of Banach Algebra}

\section{Compact Linear Operators}
\subsection{Compact Linear Operators on Normed Spaces}
\subsection{Further Properties of Compact Linear Operators}
\subsection{Spectral Properties of Compact Linear Operators on Normed Spaces}
\subsection{Further Spectral Properties of Compact Linear Operators}
\pagebreak

%Module 4 - \cite{kreyszig} 9.1,2,3,5,6
{\Large Module 4}
\section{Bounded, Self-Adjoint Linear Operators}
\subsection{Spectral Properties of Bounded Self-Adjoint Linear Operators}
\subsection{Further Spectral Properties of Bounded Self-Adjoint Linear Operators}
\subsection{Positive Operators}
%\subsection{Square Roots of a Positive Operator}
\setcounter{subsection}{4}
\subsection{Projection Operators}
\subsection{Further Properties of Projections}
%Missing - Chapter 6, 10
