%Text Books : \cite{kreyszig}
%Module 1:
%Reflexive Spaces, Category theorem(statement only), Uniform Boundedness theorem ( applications excluded), Strong and Weak Convergence, Convergence of Sequences of Operators and Functionals, Open Mapping Theorem, Closed Linear Operators, Closed Graph Theorem
%(Chapter 4 - Sections 4.6 to 4.9, 4.12, 4.13) (20 Hours)
%Module 2:
%Banach Fixed point theorem, Spectral theory in Finite Dimensional Normed Spaces, Basic Concepts, Spectral Properties of Bounded Linear Operators, Further Properties of Resolvent and Spectrum, Use of Complex Analysis in Spectral Theory
%(Chapter5 – Section 5.1; Chapter 7 - Sections 7.1 to 7.5) (25 Hours)
%Module 3:
%Banach Algebras, Further Properties of Banach Algebras, Compact Linear Operators on Normed spaces, Further Properties of Compact Linear Operators, Spectral Properties of compact Linear Operators on Normed spaces, Further Spectral Properties of Compact Linear Operators 
%(Chapter 7 - Sections 7.6, 7.7; Chapter 8 - Sections 8.1 to 8.4) ( 25 Hours)
%Module 4:
%Spectral Properties of Bounded Self adjoint linear operators, Further Spectral Properties of Bounded Self Adjoint Linear Operators, Positive Operators, Projection Operators, Further Properties of Projections
%(Chapter 9 - Sections 9.1 to 9.3, 9.5, 9.6) (20 Hours)

%Module 1 - \cite{kreyszig} 4.6,7,8,9,12,13
%\chapter 1
\setcounter{section}{3}
{\Large Module 1}
\section{Normed and Banach Spaces}
\setcounter{subsection}{5}
\subsection{Reflexive Spaces}
\begin{definition}[reflexive]
	A vector space $X$ is algebraically reflexive if the canoical mapping $C : X \to X^{\ast\ast}$ defined by $C(x) = g_x \in X^{\ast\ast}$ is surjective where $X^\ast, X^{\ast\ast}$ are the first and second algebraic dual spaces of $X$ and $\forall f \in X^\ast,\ g_x(f) = f(x)$.\\

	A normed space $X$ is reflexive if the canonical mapping $C : X \to X^{\prime\prime}$ defined by $C(x) = g_x \in X^{\prime\prime}$ is surjective where $X^\prime, X^{\prime\prime}$ are the first and second dual spaces of $X$ and $\forall f \in X^\prime,\ g_x(f) = f(x)$.
\end{definition}

%Module 2 - \cite{kreyszig} 5.1, 7.1,2,3,4,5
{\Large Module 2}
\section{Linear Operators}
%Module 3 - \cite{kreyszig} 7.6,7, 8.1,2,3,4
{\Large Module 3}
\section{Compact Linear Operators}
%Module 4 - \cite{kreyszig} 9.1,2,3,5,6
{\Large Module 4}
\section{Bounded, Self-Adjoint Linear Operators}
%Missing - 6, 10?
