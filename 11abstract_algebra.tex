%Text Books : \cite{fraleigh}
%Module 1
%Direct products and finitely generated Abelian groups, fundamental theorem, Applications
%Factor groups, Fundamental homomorphism theorem, normal subgroups and inner automorphisms.
%Group action on a set, Isotropy subgroups, Applications of G- sets to counting.
%(Part II – Sections 11, 14, 16 & 17) (25 hours)
%Module 2
%Isomorphism theorems, Sylow theorems , Applications of the Sylow theory.
%(Part VII Sections34, 36 & 37) (25 hours)
%Module 3
%Fermat’s and Euler Theorems, The field of quotients of an integral domain,
%Rings of polynomials, Factorisation of polynomials over a field.
%(Part IV – Sections 20, 21, 22 & 23) (20 hours)
%Module 4
%Non commutative examples, Homeomorphisms and factor rings, Prime and
%Maximal Ideals
%(Part V – Sections 24, 26 & 27) (20 hours)

%Module 1 - \cite{fraleigh} 11, 14, 16, 17
%Module 2 - \cite{fraleigh} 34, 36, 37
%Module 3 - \cite{fraleigh} 20, 21, 22, 23
%Module 4 - \cite{fraleigh} 24, 26, 27

%Advanced Abstract Algebra
%Module 1 - \cite{fraleigh} 29, 31, 32, 33
%Module 2 - \cite{fraleigh} 45, 46, 47
%Module 3 - \cite{fraleigh} 48, 49, 50
%Module 4 - \cite{fraleigh} 51, 53, 54, 55

%Missing - \cite{fraleigh} 1, 2, 3, 4, 5, 6, 7, 8, 9, 10,
%	12, 13, 15, 18, 19, 25, 28, 30, 35, 38, 39, 40, 41, 42, 43, 44


\section{Introduction to Abstract Algebra}
%Week 01 Day 01 \S1-10
%\begin{story}
%\paragraph{`Abstract'}
%\footnote{There are many algebras. But, we are interested in the abstract notion 'algebra'.}
%Consider the collection of of all students in a class.
%A person can't be repeated in this collection so you might think this collection is always a set.
%But in this collection, there may be two with name starting with an `S'.
%Thus the collection of first letter of their names can give you the impression that some element is repeated.\\
%
%Thus, though the collection of students is a reality, set is an abstract idea which depends on the details we are interested in.
%And if a particular collection doesn't qualify to be a `set', we can't use set-theoretic results on it.
%(which is, everything below)
%\end{story}

\begin{definition}Set-theoretical foundation of Abstract Algebra
	\begin{itemize}
		\item A \textbf{cartesian product}, $A \times B = \{ (a,b) \ : \ a \in A,\ b \in B \}$ %\S0.4
		\item A \textbf{relation} on a set $A$ is a subset of $A \times A$. %\S0.7
		\item A \textbf{function} $f$ from $A$ into $B$,
			$f : A \to B$ is a relation such that \textit{every element of $A$ is related to some unique element of $B$}
			(well defined). %\S0.10
	\end{itemize}
\end{definition}

\begin{definition} Functions :
	A \textbf{binary operation} on a set $A$ is a function $\ast : A \times A \to A$. %\S2.1
	%bi,tri,\dots are numerical prefixes with Latin origin.
	%di is a numerical prefix with Greek origin.
\end{definition}

``A binary operation on a set $A$ gives an algebra on $A$.'' %\S1.0
\begin{story}
\paragraph{Abstract Algebra}
	It is the study of algebraic structures.
	We are interested in a few algebraic structures :
	\begin{enumerate*}
		\item Group
		\item Ring
		\item Integral Domain
		\item Field
	\end{enumerate*}
\end{story}

\begin{definition}
	A \textbf{binary algebraic structure} $<\!G,\ast\!>$ is a set $G$ together with a binary operation $\ast$.
\end{definition}

\begin{definition}[Group]
	A set $G$, closed under a binary relation $\ast$ satisfying the following three axioms -G1, G2, \& G3 is a group.%\S4.1
\begin{enumerate}[label=G\arabic*]
	\item Associativity \\
		For any three elements $a,b,c \in G,\ a \ast (b \ast c) = (a \ast b) \ast c$.
	\item Identity element\\
		There exists a unique element $e \in G$ such that $a \ast e = a = e \ast a$ for any element $a \in G$.
	\item Inverse elements\\
		For any element $a \in G$ there exists a unique element $a^{-1}$ such that $a \ast a^{-1} = e = a^{-1} \ast a$.
\end{enumerate}
\end{definition}

\begin{definition}Group terminologies :
	\begin{itemize}
		\item A group is \textbf{abelian} if the binary operation is commutative. $a \ast b = b \ast a$ %\S4.3
		\item The \textbf{order} of a group $G,\ast$ is the number of elements in $G$.%\S5.3
	\end{itemize}
\end{definition}

\begin{definition}
	$\mathbb{Z}_n = \{ 0, 1, 2, \cdots, n-1 \}$
\end{definition}

\begin{remark}
	Consider $3,4 \in \mathbb{Z}_5$,
	$3 \ast 4 = 2 = 4 \ast 3$ since $7 \cong 2 \pmod 5$.
	$<\!\mathbb{Z}_5,+_5\!>$ is an abelian group of order 5.
\end{remark}

\begin{definition} Homomorphism \& Isomorphism
	\begin{itemize}
		\item A function $\phi : A \to B$ is a \textbf{homomorphism} if for any two elements $x,y \in A$, $\phi (xy) = \phi(x)\phi(y)$%\S3.7
		\item A function $\phi : A \to B$ is an \textbf{isomorphism} if $\phi$ is a bijective, homomorphism.
			If two binary structures  are isomorphic, then they have the same (algebraic) structure.%\S3.7
	\end{itemize}
\end{definition}

\begin{definition} Subgroup
	\begin{itemize}
		\item A subset $H$ of a group $<\! G,\ast\! >$ is a \textbf{subgroup} of $G$ if $H$ is group with the same binary operation $\ast$. %\S5.4
		\item $G$ is the \textbf{improper} subgroup of $G$ and every other subgroup is \textbf{proper}.%S5.5
		\item $\{e\}$ is the \textbf{trivial} subgroup of $G$ and every other subgroup is \textbf{non-trivial}.%\S5.5
		\item The \textbf{subgroup generated by} $g \in G$ is the subgroup $\{ g^n : n \in \mathbb{Z} \}$.%\S5.19
		\item The \textbf{order} of an element $g$ is order of the subgroup generated by $g$.%\S5.19
		\item An element $g \in G$ is a \textbf{generator} of $G$ if $g$ generates $G$.%\S5.19
		\item A group is \textbf{cyclic} if it has a generator.%\S5.19
	\end{itemize}
\end{definition}

\begin{remark}
	Cyclic Groups :
	\begin{itemize}
		\item Cyclic groups are abelian.%\S6.1
		\item Subgroups of cyclic groups are cyclic.%\S6.6
	\end{itemize}
\end{remark}

\subsection{Some Proof Techniques}
\paragraph{Equality of two Sets}
$A = B \iff A \subset B$ and $B \subset A$\\
If $x \in A \implies x \in B$, then $A \subset B$
\paragraph{Uniqueness}
	Suppose there are two elements that qualify our conditions.
	We show (using the conditions) that they are the same, that is, unique.

	For example, $3 + a = \pi$.
	Suppose $a = x,y$.
	Then $3 + x = 3 + y \implies x = y$, provided that the values of $a$ comes from a set in which left cancelation law can be applied.
	Then $a$ is unique.

	Remember : We usually don't care to show what this unique element is.
	It may be also be the case that there is no such element, that is, proof of uniqueness doesn't imply existence.

\paragraph{Existence}
	There are constructive and non-constructive proofs for existence problems.
	Suppose we want to prove that $a\ast b$ has an inverse element.
	We know that $a,b$ has inverse elements $a^{-1},b^{-1}$.
	From those elements, we construct an element $b^{-1} \ast a^{-1}$ which is an inverse of $a \ast b$ by construction.

	And we may also prove existence without actually giving an object.
	Suppose we want to prove that $x^y \in \mathbb{Q}$ for some irrational numbers $x$ and $y$.
	We know that $\sqrt{2} \notin \mathbb{Q}$.
	Then, $\sqrt{2}^{\sqrt{2}}$ is either rational or irrational.
	Suppose it is irrational, then $\left(\sqrt{2}^{\sqrt{2}}\right)^{\sqrt{2}} = 2$ is rational.
	Thus the proof is complete, but we are yet to know whether $\sqrt{2}^{\sqrt{2}}$ is an irrational or rational.

%\pagebreak

%Week 01 Day 02 \S11.1-11
\section{Direct Products and Finitely Generated Abelian Groups}
\begin{definition}[Cartesian product of sets]
	Let $S_1,\ S_2, \cdots,\ S_n$ be a sets.
	Their cartesian product,
	\begin{equation}
		S_1 \times S_2 \times \cdots \times S_n = \prod_{i = 1}^n S_n = \{ (a_1,a_2,\cdots,a_n) :  a_i \in S_i \} 
	\end{equation}
	For example, $ \{A,B,C\} \times \{ 1,2 \} = \{ (A,1),(A,2),(B,1),(B,2),(C,1),(C,2) \}$.
\end{definition}

\begin{question}
	How many elements in $\mathbb{Z}_3 \times \mathbb{Z}_{10} \times \mathbb{Z}_9$ ?
\end{question}

\begin{theorem}[Direct product of Groups]
	Let $G_1,\ G_2,\ \cdots,\ G_n$ be groups.
	Then their cartesian product is a group with the binary operation $\ast$,
	\begin{equation}
		(a_1,\ a_2,\ \cdots,\ a_n) \ast (b_1,\ b_2,\ \cdots,\ b_n) = (a_1 \ast b_1,\ a_2 \ast b_2,\ \cdots,\ a_n \ast b_n)
	\end{equation}
	where the binary operation in  $a_i \ast b_i$ is the binary operation of the group $G_i$.
\end{theorem}
\begin{proof}
	$\prod\limits_{i = 1}^n G_i$ is a group if it satisfies the group axioms.
	\begin{enumerate}[label=G\arabic*]
		\item Associativity
			\begin{align*}
				(a_1,\ a_2,\ & \cdots,\ a_n) \big( (b_1,\ b_2,\ \cdots,\ b_n)(c_1,\ c_2,\ \cdots,\ c_n) \big) \\
				= & (a_1,\ a_2,\ \cdots,\ a_n)(b_1c_1,\ b_2c_2,\ \cdots,\ b_nc_n) \\
				= & \big( a_1(b_1c_1),\ a_2(b_2c_2),\ \cdots,\ a_n(b_nc_n) \big) \\
				= & \big( (a_1 b_1)c_1,\ (a_2b_2)c_2,\ \cdots,\ (a_nb_n )c_n \big) \\
				= &  (a_1b_1,\ a_2b_2,\ \cdots,\ a_nb_n)(c_1,\ c_2,\ \cdots,\ c_n) \\
				= & \big( (a_1,\ a_2,\ \cdots,\ a_n)(b_1,\ b_2,\ \cdots,\ b_n) \big)(c_1,\ c_2,\ \cdots,\ c_n)
			\end{align*}
		\item Existence of a unique identity element in $\prod\limits_{i = 1}^n G_i$\\
			Let $e_i$ be the identity element in $G_i$.
			Then $(e_1,\ e_2,\ \cdots,\ e_n)$ is the identity element in $\prod\limits_{i = 1}^n G_i$.
			\begin{align*}
				(a_1,\ a_2,\ \cdots,\ a_n)&(e_1,\ e_2,\ \cdots, e_n) \\
				& = (a_1e_1,\ a_2e_2,\ \cdots,\ a_ne_n) \\
				& = (a_1,\ a_2,\ \cdots,\ a_n)
			\end{align*}
		\item Existence of unique inverse element for each element in $\prod\limits_{i = 1}^n G_i$\\
			Let $(a_1,\ a_2,\ \cdots, a_n)$ be in $\prod\limits_{i = 1}^n G_i$.
			Then it has the inverse element $( a_1^{-1},\ a_2^{-1},\ \cdots,\ a_n^{-1} )$ in $\prod\limits_{i = 1}^n G_i$.
			\begin{align*}
				(a_1,\ a_2,\ \cdots a_n)&(a_1^{-1},\ a_2^{-1},\ \cdots, a_n^{-1}) \\
				& = (a_1a_1^{-1},\ a_2a_2^{-1},\ \cdots,\ a_na_n^{-1}) \\
				& = (e_1,\ e_2,\ \cdots,\ e_n)
			\end{align*}
	\end{enumerate}
\end{proof}

\begin{remark}
	We usually write $ab$ instead of $a \ast b$ and relevent binary operations are used in different contexts.
	Student should be able to recongnise the difference from the context.
\end{remark}

\begin{remark}
	$\mathbb{Z}_n = \{ 0,1,\cdots,(n-1) \}$ is a group with $+_n$.
	(addition modulo $n$)

	For example, Consider $(1,2) \in \mathbb{Z}_2 \times \mathbb{Z}_3$.
	We have, $(1,2) + (1,2) = (0,1)$ since $1 +_2 1=0$ and $2 +_3 2 = 1$.
\end{remark}

\begin{definition}
	Suppose all the groups $G_i$ are abelian.
	Then $\prod\limits_{i = 1}^n G_i$ is the direct sum of the groups $G_i$.
	And is represented by $\oplus_{i = 1}^n G_i$.
\end{definition}

\begin{theorem}
	$\mathbb{Z}_m \times \mathbb{Z}_n$ is cyclic and is isomorphic to $\mathbb{Z}_{mn}$ if and only if $m$ and $n$ are relatively prime.
\end{theorem}
\begin{proof}
	Sufficient part :
	Consider the cyclic subgroup\footnote{$(1,1) \in Z_m \times Z_n$.
	The cyclic group generated by $(1,1)$ has all its elements in $Z_m \times Z_n$.
	And therefore, it is a subgroup of $Z_m \times Z_n$} $H$ generated by $(1,1) \in \mathbb{Z}_m \times \mathbb{Z}_n$.
	It is enough to prove that the order of this cyclic subgroup $H$ is $mn$.
	The order of $H$ is the smallest power of $(1,1)$ that gives the identity $(0,0)$.
	The first component gives $0$ for multiples of $m$.
	And the second component gives $0$ for multiples of $n$.
	Since $m,n$ are relatively prime, $mn$ is the smallest power of $(1,1)$ that will give $(0,0)$.
	Thus $\mathbb{Z}_m \times \mathbb{Z}_n = H$ and is cyclic.
	\begin{important} Every cyclic group of order $mn$ is isomorphic to $\mathbb{Z}_{mn}$. Therefore, $\mathbb{Z}_m \times \mathbb{Z}_n \approx \mathbb{Z}_{mn}$.\end{important}

	Necessary part :
	Suppose $\gcd(m,n) = d > 1$.
	Then $mn/d$ is the smallest integer divisible by both $m$ and $n$.
	Consider $(r,s) \in \mathbb{Z}_m \times \mathbb{Z}_n$.
	$r$ gives $0$ in $mn/d$ since it is a multiple of $m$.
	Similarly, $s$ gives $0$ in $mn/d$ since it is a multiple of $n$.
	Thus, $\frac{mn}{d} (r,s) = (0,0)$.
	And the cyclic group generated by any element of $\mathbb{Z}_m \times \mathbb{Z}_n$ is a proper subgroup.
	Therefore $\mathbb{Z}_m \times \mathbb{Z}_n$ has no generators and it is not cyclic.
\end{proof}

\begin{corollary}
	$\prod\limits_{i = 1}^n \mathbb{Z}_{m_i}$ is cyclic and is isomorphic to $Z_{m_1m_2\cdots m_n}$ if and only if any two of the numbers $m_i$ are relatively prime.
\end{corollary}

\begin{question}
	Prove : For any non-negative integer $n$, there exists a cyclic group of order $n$, which is unique upto isomorphism.
\end{question}

\begin{theorem}
	Let $(a_1,\ a_2,\ \cdots,\ a_n) \in \prod\limits_{i = 1}^n G_i$. And $a_i$ are of finite order $r_i$ in $G_i$. Then the order of $\prod\limits_{i = 1}^n G_i$ is the least common multiple of $r_i$s.
\end{theorem}
\begin{proof}
	Least common multiple of $r_i$s is the smallest positive integer $d$ which is a multiple of all $r_i$s.
	For each $i$, the $r_i$th multiple of $a_i$ gives $0$ (identity).
	Thus, the order of the cyclic subgroup generated by $(a_1,\ a_2,\ \cdots,\ a_n)$ is the least common multiple of all the $r_i$s.
\end{proof}
\begin{remark}
	Consider $(3,6,12,16) \in \mathbb{Z}_4 \times \mathbb{Z}_{12} \times \mathbb{Z}_{20} \times \mathbb{Z}_{24}$.
	Order of $3 \in \mathbb{Z}_4$ is ${4}/{\gcd(3,4)} = 4$ ie, $<3> = \{ 3,2,1,0 \}$\\
	Order of $6 \in \mathbb{Z}_{12}$ is ${12}/{\gcd(6,12)} = 2$ ie, $<6> = \{ 6,0 \}$\\
	Order of $12 \in \mathbb{Z}_{20}$ is ${20}/{\gcd(12,20)} = 5$ ie, $<12> = \{ 12,4,16,8,0 \}$\\
	Order of $16 \in \mathbb{Z}_{24}$ is ${24}/{\gcd(16,24)} = 3$ ie, $<16> = \{ 16,8,0 \}$\\
	Order of $(3,6,12,16)$ is $lcm(4,2,5,3) = 2^2\ 3\ 5 = 60$.
\end{remark}

\begin{remark}
	Define $\overline{G_i} = \{ (e_1,\ e_2,\ \cdots,\ e_{i-1},\ a_i,\ e_{i+1},\ \cdots,\ e_n) : a_i \in G_i\}$.
	Then $G_i \approx \overline{G_i}$.
	And $\prod\limits_{i=1}^n G_i$ is the internal direct product of $\overline{G_i}$s.

	For example, $\mathbb{Z}_2 \times \mathbb{Z}_3 \approx \left( \mathbb{Z}_2 \times \{ 0 \} \right) \otimes \left( \{ 0 \} \times \mathbb{Z}_3 \right)$
\end{remark}

\begin{question}
	Internal direct product form of $\mathbb{Z}_{12} \times \mathbb{Z}_{60} \times \mathbb{Z}_{24}$ ?
\end{question}
%\pagebreak

%Week 01 Day 03 \S11.11-17
\section{Fundamental Theorem}
\begin{definition}
	A group $G$ is \textbf{finitely generated} if $G$ has a finite subset that generates $G$.
%	For example, $\{ 1,\rho^2,\mu,\mu\rho^2 \}$ is a finitely generated subgroup of $D_8$.
\end{definition}

\begin{theorem}[fundamental theorem of finitely generated abelian groups]
	Every finitely generated abelian group $G$ is isomorphic to a direct product of cyclic groups in the form
	\begin{equation}
		\mathbb{Z}_{{p_1}^{r_1}} \times\mathbb{Z}_{{p_2}^{r_2}} \times \cdots \times \mathbb{Z}_{{p_n}^{r_n}} \times \mathbb{Z} \times \mathbb{Z} \times \cdots \times \mathbb{Z}
	\end{equation}
	where $p_i$ are primes, not necessarily distict and $r_i$ are positive integers.
	The direct product is unique, except for the possible rearrangement of the factors.
\end{theorem}
\begin{proof}
	---proof is omitted---
\end{proof}

For example, $ G = \mathbb{Z}_{20} \times \mathbb{Z} \times \mathbb{Z}_{15} \times \mathbb{Z} \approx \mathbb{Z}_{2^2} \times \mathbb{Z}_3 \times \mathbb{Z}_5 \times \mathbb{Z}_5 \times \mathbb{Z} \times \mathbb{Z}$.
In the above case, Betti number of $G$ is $2$ (number of $\mathbb{Z}$ factors).
For any finite abelian group, Betti number is $0$.

\begin{remark}[finite abelian groups]
	Every finite group is finitely generated.
	And thus we can enumerate finite abelian group of any order.
\end{remark}

\begin{remark}
	There are precisely $6$ different abelian groups of order $360 = 2^3 3^2 5$.
	\begin{enumerate}
		\item $\mathbb{Z}_2 \times \mathbb{Z}_2 \times \mathbb{Z}_2 \times \mathbb{Z}_3 \times \mathbb{Z}_3 \times \mathbb{Z}_5$ 
		\item $\mathbb{Z}_2 \times \mathbb{Z}_{2^2} \times \mathbb{Z}_3 \times \mathbb{Z}_3 \times \mathbb{Z}_5$
		\item $\mathbb{Z}_2 \times \mathbb{Z}_2 \times \mathbb{Z}_2 \times \mathbb{Z}_{3^2} \times \mathbb{Z}_5$
		\item $\mathbb{Z}_{2^3} \times \mathbb{Z}_3 \times \mathbb{Z}_3 \times \mathbb{Z}_5$
		\item $\mathbb{Z}_2 \times \mathbb{Z}_{2^2} \times \mathbb{Z}_{3^2} \times \mathbb{Z}_5$
		\item $\mathbb{Z}_{2^3} \times \mathbb{Z}_{3^2} \times \mathbb{Z}_5$
	\end{enumerate}
\end{remark}

\begin{question}
	Group of order 360 with at least an element of order 8
\end{question}

\begin{definition}
	A group $G$ is \textbf{decomposible} if it is isomorphic to a direct product of two proper, non-trivial subgroups.
	Otherwise $G$ is \textbf{indecomposible}.
\end{definition}

	For example, $\mathbb{Z}_{2^3} \times \mathbb{Z}_3 \times \mathbb{Z}_3 \times \mathbb{Z}_5 \approx \mathbb{Z}_{24} \times \mathbb{Z}_{15}$ is decomposible.

\begin{remark}
	$G = \mathbb{Z}_2 \times \mathbb{Z}_2 \times \mathbb{Z}_2 \times \mathbb{Z}_{3} \times \mathbb{Z}_3 \times \mathbb{Z}_5 \approx \mathbb{Z}_2 \times \mathbb{Z}_{6} \times \mathbb{Z}_{30}$ is also decomposible.
	Since $G \approx (\mathbb{Z}_2 \times \mathbb{Z}_{6}) \times \mathbb{Z}_{30}$.
\end{remark}

\begin{theorem}
	The finite indecomposible abelian groups ar exactly the cyclic groups with order a power of a prime.
\end{theorem}
\begin{proof}
	Necessary part: 
	Let $G$ be a finite, indecomposible abelian group.
	By fundamental theorem of finitely generated abelian groups, $G$ is isomorphic to a direct product of cyclic groups of prime power order.
	$$ G \approx \mathbb{Z}_{{p_1}^{r_1}} \times \mathbb{Z}_{{p_2}^{r_2}} \times \cdots \times \mathbb{Z}_{{p_n}^{r_n}}$$
	Thus for $G$ to be indecomposible the direct product should be a cyclic group of prime power order.
	$G \approx \mathbb{Z}_{{p_1}^{r_1}}$.

	Sufficient part :
	Let $p$ be a prime and $r$ a non-negative integer.
	Cyclic group of order $p^r$ is isomorphic to $\mathbb{Z}_{p^r}$.
	Since every cyclic groups are abelian, $\mathbb{Z}_{p^r}$ is an abelian group of finite order $p^r$.
	It is enough to prove that $\mathbb{Z}_{p^r}$ is indecomposible.

	A proper, non-trivial subgroup of $\mathbb{Z}_{p^r}$ is of the form $\mathbb{Z}_{p^i}$ where $0 < i < r$.
	Suppose $\mathbb{Z}_{p^r}$ is decomposible.
	Then $\exists i,j \in \mathbb{Z}^+$ such that $\mathbb{Z}_{p^r} \approx \mathbb{Z}_{p^i} \times \mathbb{Z}_{p^j}$ and $i+j=r$.
	Clearly, $p^i$ and $p^j$ are not relatively prime, thus $\mathbb{Z}_{p^r} \not\approx \mathbb{Z}_{p^i} \times \mathbb{Z}_{p^j}$.
	Therefore, cyclic groups of order prime power are indecomposible.
\end{proof}

\begin{theorem}
	If $m$ divides the order of a finite abelian group $G$, then $G$ has a subgroup of order $m$.
\end{theorem}
\begin{proof}
	Let $G \approx \mathbb{Z}_{{p_1}^{r_1}} \times \mathbb{Z}_{{p_2}^{r_2}} \times \cdots \times \mathbb{Z}_{{p_n}^{r_n}}$.
	Then $|G| = p_1^{r_1} p_2^{r_2} \cdots p_n^{r_n}$.
	Suppose $m$ divides $|G|$, then $m = p_1^{s_1} p_2^{s_2} \cdots p_n^{s_n}$ where $0 \le s_i \le r_i$.
	Define $H = \mathbb{Z}_{{p_1}^{s_1}} \times \mathbb{Z}_{{p_2}^{s_2}} \times \cdots \times \mathbb{Z}_{{p_n}^{s_n}}$.
	Then $H$ is subgroup of order $m$.
\end{proof}

\begin{theorem}
	If $m$ is square-free integer, then every abelian group of order $m$ is cyclic.
\end{theorem}
\begin{proof}
	Let $m$ be a square-free integer and $G$ be an abelian group of order $m$.
	By fundamental theorem of finitely generated abelian groups
	$$ G \approx \mathbb{Z}_{{p_1}^{r_1}} \times \mathbb{Z}_{{p_2}^{r_2}} \times \cdots \times \mathbb{Z}_{{p_n}^{r_n}}$$
	We have, $m$ is square-free.
	Thus $r_i = 1$ and $p_i$ are distinct.
	Therefore, $G \approx \mathbb{Z}_{p_1 p_2 \cdots p_n}$ is a cyclic group of order $m$.
\end{proof}
%\pagebreak

%Week 01 Day 04 \S11E
\section{Exercises \S11}

\begin{question}
	Enumerate subgroups of $\mathbb{Z}_2 \times \mathbb{Z}_2 \times \mathbb{Z}_4$
\end{question}

\subsection{Abelian Groups}
\begin{remark}
Direct product of abelian groups is abelian.%\S11Ex46
\end{remark}

\begin{question}
	Enumerate abelian groups of order 127008 
\end{question}

\begin{remark}
	Let $G$ be an abelian group.
	The subset $H$ of $G$ with identity element and all elements of order $n$ is subgroup of $G$ if and only if $n$ is a prime. %\S11Ex49
\end{remark}
\begin{proof}
	Suppose $a \in H$.
	Then every power of $a$ has order $n$.
	Suppose $n$ is not prime.
	Then $d$ divides $n$ and $a^d$ has order $n/d$.
\end{proof}

\subsection{Torsion Group and Torsion Coefficients}
\begin{remark}
If group $G$ is abelian, then its elements of finite order forms a subgroup. %\S11Ex39
(hint : $a,b \in G$ has finite order, then $ab$ has finite order. And $a \in G$ has finite order, then $a^{-1}$ has finite order)
\end{remark}
\begin{definition} 
	The \textbf{torsion group} of an abelian group $G$ is the subgroup of $G$ containing only those elements of finite order.%\S11Ex39\\
	An abelian group is \textbf{torsion free} if identity element is the only element of finite order.
	$G$ is Torsion free if Torsion group of $G$ is trivial, $\{ e \}$.%\S11Ex43
\end{definition}

\begin{definition}
	The integers $m_1,m_2,\cdots,m_n$ are torsion coefficients of $G$ such that $G \approx \mathbb{Z}_{m_1} \times \mathbb{Z}_{m_2} \times \cdots \times \mathbb{Z}_{m_n}$ where $m_i$ divides $m_{i+1}$.%\S11Ex44
\end{definition}

For example, $\mathbb{Z}_6 \times \mathbb{Z}_{12} \times \mathbb{Z}_{20}$ has torsion coefficients $2, 12, 60$

\begin{remark}[Algorithm to find torsion coefficients of a group] Suppose $G$ has a direct product form.%\S11Ex44c
	\begin{enumerate}[label=Step \arabic*]
		\item Find power of each prime in the direct product form
		\item List power of each prime
		\item Append 1s on left to make all lists to equal length
		\item Product of $i$th number on each list gives $m_i$
	\end{enumerate}
\end{remark}

For example, $G \approx \mathbb{Z}_6 \times \mathbb{Z}_{12} \times \mathbb{Z}_{20}$
\begin{enumerate}[label=Step \arabic*]
	\item $\mathbb{Z}_6 \times \mathbb{Z}_{12} \times \mathbb{Z}_{20} \approx \mathbb{Z}_2 \times \mathbb{Z}_3 \times \mathbb{Z}_{2^2} \times \mathbb{Z}_3 \times \mathbb{Z}_{2^2} \times \mathbb{Z}_5$
	\item (2,4,4), (3,3), (5)
	\item (2,4,4), (1,3,3), (1,1,5)
	\item (2,12,60)
\end{enumerate}

\begin{remark}
	Let $G = H \times K$.
	$g \in G \implies g = (h,k) \in H \times K$.%\S11Ex50\\
	$\implies H \text{is a subset of } G,\ H \times \{e\}$.
	Thus $h \in H \subset G \implies h = (h,e)$.
	Similarly $k = (e,k)$.
	$\implies hk = (h,e)(e,k) = (h,k) = (e,k)(h,e) = kh$
\end{remark}
%\pagebreak

%Week 02 Day 01 \S13
\section{Cosets and Homomorphism}

\begin{definition}
	A \textbf{permutation group} $S_n$ is the set of all permutations on the set $\{1,2,\cdots,n\}$.	
\end{definition}

\begin{remark}
	Consider, $(1\ 2\ 3)(4\ 5), (1\ 2)(3\ 4) \in S_5$.
	$$\begin{pmatrix} 1 & 2 & 3 \end{pmatrix} \begin{pmatrix} 4 & 5 \end{pmatrix} \ast \begin{pmatrix} 1 & 2 \end{pmatrix} \begin{pmatrix} 3 & 4 \end{pmatrix} =  \begin{pmatrix} 1 & 2 & 3 & 4 & 5 \\ 2 & 3 & 1 & 5 & 4 \\ 1 & 4 & 2 & 5 & 3 \end{pmatrix}  = \begin{pmatrix} 2 & 4 & 5 & 3 \end{pmatrix} $$
	$$\begin{pmatrix} 1 & 2 \end{pmatrix} \begin{pmatrix} 3 & 4 \end{pmatrix} \ast \begin{pmatrix} 1 & 2 & 3\end{pmatrix} \begin{pmatrix} 4 & 5 \end{pmatrix} =  \begin{pmatrix} 1 & 2 & 3 & 4 & 5 \\ 2 & 1 & 4 & 3 & 5 \\ 3 & 2 & 5 & 1 & 4 \end{pmatrix}  = \begin{pmatrix} 1 & 3 & 5 & 4 \end{pmatrix} $$
	Clearly, $S_5$ is a non-abelian group of order 120.
\end{remark}

\begin{definition}
	Kernel of a function $\phi : G \to G'$ is the inverse image of the identity element in $G'$. %\S13.13
\end{definition}

\begin{definition}Cosets and Normal Subgroup,
	\begin{itemize}
		\item A \textbf{left coset} $gH$ is the subset $\{ gh \in G : h \in H \}$ where $g \in G$ and  $H$ is a subgroup of $G$.%\S10.2
		\item A \textbf{right coset} $Hg$ is the subset $\{ hg \in G : h \in H \}$ where $g \in G$ and  $H$ is a subgroup of $G$.%\S10.2
		\item A subgroup $H$ of group $G$ is \textbf{normal} if $gH = Hg,\ \forall g \in G$.
		\item All subgroups of abelian groups are normal.
	\end{itemize}
\end{definition}

\begin{remark}
	For example, $H = \{1,\rho^2,\mu,\mu\rho^2\}$ is a normal subgroup of $D_4$.
	And $K = \{1,\mu\}$ is a subgroup of $D_4$ which is not normal.
	Note that, $\rho\mu \ne \mu\rho$.
	Clearly, $\rho K \ne K\rho$.
	However, $\rho H = \{ \rho, \rho^3, \mu\rho^3, \mu\rho \} = H\rho$.
\end{remark}

\begin{remark}[Lagrange's Theorem]%\S10.10
	Let $G$ be a finite group.
	If $H$ is a subgroup of $G$, then order of $H$ divides order of $G$.
\end{remark}
\begin{itemize}
	\item $aH \cap bH \ne \phi \implies aH = bH$.
	\item $\forall g \in G,\ g \in gH$.
	\item $\forall g \in G,\ |gH| = |H|$.
\end{itemize}

\begin{remark}[Cayley's Theorem]%\S8.16
	Every group is isomorphic to a group of isomorphisms.
\end{remark}

\begin{definition}
	Let $G,G'$ be groups.
	A \textbf{group homomorphism} is a function $\phi : G \to G'$ such that $\phi(x)\phi(y) = \phi(xy)$.
	Clearly $\phi(e) = e'$.
	A \textbf{trivial homomorphism} is a function $\phi : G \to G'$ such that $\phi(G) = \{ e' \}$.
\end{definition}


\begin{remark}
	Group homomorphism $\phi : G \to G'$ preserves identity, inverses and subgroups.
	And kernel of group homomorphism is a normal subgroup of $G$.%\S13.12
\end{remark}
\begin{remark}
	For example, $\phi : D_8 \to \mathbb{Z}_2$ defined by $\phi(\rho) = 1$ and $\phi(\mu) = 0$ is a group homomorphism with $\ker(\phi) = \{ 1,\rho^2,\mu,\mu\rho^2 \}$.
\end{remark}
%\pagebreak

%Week 02 Day 02 \S14.1-8
\section{Factor Groups}
\begin{remark}[factor group]
	Let $H$ be a normal subgroup of a group $G$.
	Then the \textbf{factor group} of $G$ over $H$, $G/H$ is the group of cosets of $H$ in $G$.
\end{remark}

\begin{remark}
	For example, $H = \{ 1,\rho^2,\mu,\mu\rho^2 \}$ is normal subgroup of $D_8$.
	And the factor group $D_8/H = \{ 1H,\rho H \}$.
\end{remark}

\begin{theorem}
	Let $\phi : G \to G'$ be a group homomorphism with kernel $H$.
	Then cosets of $H$ form a factor group, $G/H$ where $(aH)(bH) = (abH)$ Also, $\mu : G/H \to \phi[G]$ defined by $\mu(aH) = \phi(a)$ is an isomorphism.
\end{theorem}
\begin{proof}
	Let $\phi : G \to G'$ be a group homomorphism with $\ker(\phi) = H$.
	We have, $\phi^{-1}(\phi(a)) = \{ g \in G : \phi(g) = \phi(a) \}$.

	Let $x \in aH$.
	Then $x = ah$ for some $h \in H$.
	And $\phi(x) = \phi(ah) = \phi(a)\phi(h) = \phi(a)$, since $\phi(h) = e'$.
	Thus, $x \in \phi^{-1}(\phi(a))$ and $aH \subset \phi^{-1}(\phi(a))$.

	Let $x \in \phi^{-1}(\phi(a))$.
	Then $\phi(x) = \phi(a)$.
	And $\phi(a)^{-1} \phi(x) = e' \implies \phi(a^{-1}x) = e'$.
	Clearly, $a^{-1}x \in \ker(\phi)$.
	Thus, there exists $h \in H$ such that $a^{-1}x = h$.
	Therefore, $x = ah$ for some $h \in H$,.
	Thus, $x \in aH$ and $\phi^{-1}(\phi(a)) \subset aH$.
	Therefore, $\phi^{-1}(\phi(a)) = aH$.
	
	Similarly, $\phi^{-1}(\phi(a)) = Ha$.
	Thus $aH = Ha$ and $H$ is a normal subgroup of $G$.
	Therefore, we have the factor group $G/H$.

	To prove : $\mu : G/H \to \phi[G]$ is a one-one correspondence.
	ie, $aH \xleftrightarrow{\mu} \phi(a)$.
	To prove : $\mu$ is injective.
	Suppose $\mu(aH) = \mu(bH)$. Then $\phi(a) = \phi(b)$.
	And $b \in \phi^{-1}(\phi(a)) = aH$. Therefore, $bH = aH$.

	To prove : $\mu$ is surjective.
	Let $\phi(a) \in \phi[G]$.
	Then, there exists $aH$ such that  $\mu(aH) = \phi(a)$.

	We have, $\mu(aH) = \phi(a)$, $\mu(bH) = \phi(b)$, and $\mu((ab)H)  = \phi(ab)$.\\
	Therefore, $\mu((aH)(bH)) = \mu((ab)H) = \phi(ab) = \phi(a)\phi(b) = \mu(aH)\mu(bH)$.
	Thus $\mu$ is a homomorphism.
	Therefore $\mu$ is an isomorphism.
\end{proof}

\begin{theorem}
	Let $H$ be a subgroup of a group $G$.
	Then left coset multiplication is well-defined by $(aH)(bH) = (ab)H$ if and only if $H$ is a normal subgroup of $G$.
\end{theorem}
\begin{proof}
	Necessary part :
	Suppose $(aH)(bH) = (ab)H$ is well-defined.
	Let $a \in G$.
	It is enough to prove that $aH = Ha$.
	Let $x \in aH$.
	Then $(xH)(a^{-1}H) = (xa^{-1})H$.
	Also $(aH)(a^{-1}H) = eH = H$.
	We have, coset multiplication is well-defined.
	Thus $xa^{-1} = h \in H \implies x = ha \in Ha$.
	Then, $aH \subset Ha$.
	Similarly, $Ha \subset aH$ and $aH = Ha$.
	Therefore, $H$ is a normal subgroup of $G$.

	Sufficient part:
	Suppose $H$ is a normal subgroup of $G$, and let $x \in aH$ and $y \in bH$.
	$x \in aH \implies x = ah_1$ for some $h_1 \in H$
	$y \in bH \implies y = bh_2$ for some $h_2 \in H$.
	Therefore $xy = (ah_1)(bh_2) = (ah_1)(h_2b) = a(h_1(h_2b)) = a((h_1h_2)b)$
	Since $H$ is a group, $h_1h_2 = h_3 \in H$
	Thus, $xy = a(h_3b) = a(bh_3) = (ab)h_3 \in (ab)H$ for all $x \in aH$ and $y \in bH$
	Thus $(aH)(bH) = (ab)H$.
\end{proof}

\begin{corollary}
	Let $H$ be a normal subgroup of $G$.
	Then the cosets of $H$ form a group $G/H$ under the binary operation $(aH)(bH) = (ab)H$.
\end{corollary}
\begin{proof}
	Let $H$ be a normal subgroup and $aH,bH,cH$ are cosets of $H$ in $G$.
	\begin{enumerate}[label=G\arabic*]
		\item Associativity\\
			$(aH)[(bH)(cH)] = (aH)[(bc)H] = [a(bc)]H = [(ab)c]H = [(ab)H](cH) = [(aH)(bH)](cH)$
		\item Existence of identity, $eH$\\
			$(aH)(eH) = (ae)H = aH$ and $(eH)(aH) = (ea)H = aH$.
		\item Existence of inverse $(a^{-1}H)$\\
			$(aH)(a^{-1}H) = (aa^{-1})H = eH$ and $(a^{-1}H)(aH) = (a^{-1}a)H = eH$.
	\end{enumerate}
\end{proof}

\begin{remark}
	$n\mathbb{Z}$ is a normal subgroup of $\mathbb{Z}$.
	And $\mathbb{Z}/n\mathbb{Z} \approx \mathbb{Z}_n$.
	$\mathbb{Z}_n$ is a torsion group isomorphic to a factor group of torsion free group $\mathbb{Z}$.
\end{remark}

\begin{remark}
	Let $c \in \mathbb{R}^*$.
	Then the cyclic group generated by $c$ is a normal subgroup of $\mathbb{R}$ and $\mathbb{R}/<c> \approx \mathbb{R}_c$.
\end{remark}

\begin{question}
	Let $c = 0.31$.
	Find the coset $x\ +<0.31>$ containing $2$.
\end{question}
%\pagebreak

%Week 02 Day 03 \S14.9-15
\section{Fundamental Homomorphism \& Automorphisms}
\subsection{Fundamental Homomorphism Theorem}
\begin{theorem}
	Let $H$ be a normal subgroup of $G$.
	Then $\gamma : G \to G/H$ is defined by $\gamma(x) = xH$  is a homomorphism with kernel $H$.
\end{theorem}
\begin{proof}
	$\gamma(x)\gamma(y) = (xH)(yH)$
	Let $h_1,h_2 \in H$, $(xh_1)(yh_2) = xyh_3h_2 = xyh_4$ for some $h_3,h_4 \in H$.
	Therefore $(xH)(yH) = (xy)H$.
	$\gamma(xy) = (xy)H = \gamma(x)\gamma(y)$.
	$\gamma(x) = xH = H \iff x \in H$.
	Therefore, $\ker(\gamma) = H$.
\end{proof}

\begin{remark}
	Suppose $H$ is a normal subgroup of a group $G$.
	Then, a homomorphism $\gamma : G \to G/H$ is a natural homomorphism.
\end{remark}

\begin{theorem}[Fundamental Homomorphism]
	Let $\phi : G \to G'$ be a group homomorphism with kernel $H$.
	Then $\phi[G]$ is a group, and $\mu : G/H \to \phi[G]$ given by $\mu(gH) = \phi(g)$ is an isomorphism.
	If $\gamma : G \to G/H$ is the homomorphism given by $\gamma(g) = gH$, then $\phi(g) = \mu\gamma(g)$ for each $g \in G$.
\end{theorem}
\begin{proof}
	$\mu$ is an isomorphism $G/H \xleftrightarrow{\mu}\phi[G]$
	$\mu(\gamma(g)) = \mu(gH) = \phi(g)$.
	Thus $\mu\gamma = \phi$
\end{proof}
``Every group homomorphism $\phi : G \to G'$ has a unique natural group homomorphism $\gamma : G \to G/H$ and a unique isomorphism $\mu : G/H \to G'$ such that $\phi = \mu \circ \gamma$.

\subsection{Inner Automorphism}
\begin{theorem}
	Let $H$ be a subgroup of $G$, then the following statements are equivalent:
	\begin{enumerate}
		\item $ghg^{-1} \in H,\ \forall g \in G,\ h \in H$
		\item $gHg^{-1} = H,\ \forall g \in G$
		\item $gH = Hg,\ \forall g \in G$
	\end{enumerate}
\end{theorem}
\begin{proof}
	Let $G$ be a group and $H$ be a subgroup of $G$.
	By right multiplication,  $gHg^{-1} = H \iff gH = Hg$
	Trivially, $\forall h \in H,\ ghg^{-1} \in H \iff gHg^{-1} \subset H$
	Therefore, it is enough to prove that $H \subset gHg^{-1}$
	Let $h \in H$ and $x \in G$, then $xhx^{-1} = h'$ for some $h' \in H$.
	Then $h = x^{-1}h'x = x^{-1}h'(x^{-1})^{-1} \in gHg^{-1}$ where $x^{-1} = g \in G$.
	Thus $h \in gHg^{-1}$ and $H \subset gHg^{-1}$.
	Therefore $gHg^{-1} = H$.
\end{proof}

\begin{definition}Automorphisms : 
	\begin{itemize}
		\item An \textbf{automorphism} of $G$ is an isomorphism $\phi : G \to G$
		\item The \textbf{inner automorphism} of $G$ by $g \in G$ is the isomorphism\\
			$i_g : G \to G$ defined by $i_g(x) = gxg^{-1}$ for all $x \in G$
		\item The \textbf{conjugate} of $x$ by $g$ is the element $gxg^{-1} \in G$
		\item The \textbf{conjugate subgroup} of subgroup $H$, $i_g[H] = \{ ghg^{-1} : h \in H \}$
	\end{itemize}
\end{definition}

\begin{remark}
	For example, $i_\rho : D_8 \to D_8$ defined by $i_\rho(x) = \rho x \rho^{-1}$ is an inner automorphism.
	We have, $H = \{ 1,\mu \}$ is a subgroup of $D_8$.
	The conjugate subgroup $i_\rho[H] = \{ 1,\mu\rho^2 \}$.
\end{remark}

\begin{remark}
	Normal subgroups are invariant under any inner automorphism.
\end{remark}
%\pagebreak

%Week 02 Day 04 \S14E
\section{Exercise \S14}
%\begin{remark} Disprove the following results
%	\begin{itemize}
%		\item[$\star\bullet$] If $H$ is a subgroup and $N$ is a normal subgroup, then $H \cap N$ is normal. %\S14E35
%		\item Every factor group of a non-abelian group is also a non-abelian group.%\S14E23gh
%		\item If $c \in \mathbb{R}^*$, then the factor group $\mathbb{R}/n\mathbb{R}$ is cyclic.
%		\item Every factor group of a torsion free group is also a torsion free group.%\S14E22
%	\end{itemize}
%\end{remark}

\begin{remark}Prove the following results on normal subgroups
	\begin{itemize}
		\item Every subgroup of an abelian group $G$ is a normal subgroup of $G$. %\S14E23b
		\item Intersection of normal subgroups is normal. %\S14E31
		\item $SL(n,\mathbb{R})$ is a normal subgroup of $GL(n,\mathbb{R})$%\S14E40
	\end{itemize}
\end{remark}

\begin{remark}Prove the following results on factor groups
	\begin{itemize}
		\item $A_n$ is a normal subgroup of $S_n$.
			And $S_n/A_n \approx \mathbb{Z}_2$.%\S14E24
		\item If $H$ is a normal subgroup of $G$ and $(G:H) = m$, then $a^m \in H$ for all $a \in G$.%\S14E30
			(hint : $aH \in G/H\ \&\ |G/H| = m$)
		\item If a finite group $G$ has exactly one subgroup $H$ of order $m$, then $H$ is normal. %\S14E34
		\item If $G$ has a subgroup of order $m$, then the intersection of all subgroups of order $m$ is normal.%\S14E36
		\item Every factor group of an abelian group is also an abelian group.%\S14E23gh
		\item Every factor group of a torsion group is also a torsion group.%\S14E22
		\item The factor group of an abelian group $G$ over a torsion subgroup $T$, $G/T$ is torsion free.%\S14E26
	\end{itemize}
\end{remark}

\begin{definition}
	A \textbf{commutator} $c$ in group $G$ is an element in the form $c = aba^{-1}b^{-1}$ for some $a,b \in G$.
	The \textbf{commutator subgroup} is the smallest normal subgroup containing all commutators in $G$.
\end{definition}

\begin{remark}Prove the following results on commutator subgroups 
	\begin{itemize}
		\item For any subset $S$, there exists smallest normal subgroup containing $S$. %\S14E32
		\item If $C$ is the smallest normal subgroup containing all commutators in $G$, then $G/C$ is abelian.%\S14E33
	\end{itemize}
\end{remark}

\begin{remark}Prove the following results on automorphisms 
	\begin{itemize}
		\item Inner automorphism of an abelian group is an identity map.%\S14E23c
		\item Set of all $g \in G$ such that the inner automorpism $i_g$ is an identity map is normal. %\S14E38
		\item Set of automorphisms $\Gamma$ of a group $G$ is a group.
			And set of inner automorphisms is a normal subgroup of $\Gamma$.
		\item Subgroup conjugacy is an equivalance relation on the set of subgroups. %\S14E27
	\end{itemize}
\end{remark}
%\pagebreak

%Week 03 Day 01 \S15
\section{Simple Groups}
\begin{remark}Factor Group Computations :
	\begin{itemize}
		\item The converse of Lagrange's theorem is false.\\
			For example, $A_4$ has order $12$, but doesn't have a subgroup of order $6$.
		\item Factor group of a cyclic group is cyclic.\\
			If $a$ is a generator of $G$, then $aH$ is a generator of $G/H$.
	\end{itemize}
\end{remark}

\begin{question}
	Show that $\mathbb{Z}_4 \times \mathbb{Z}_6 / <(2,3)> \approx \mathbb{Z}_4 \times \mathbb{Z}_3$.
\end{question}

\begin{definition}
	A group is simple if it is non-trivial and has no proper, non-trivial normal subgroups.
\end{definition}

\begin{remark} Simple groups :
	\begin{itemize}
		\item Alternating groups $A_n$ are simple for $n \ge 5$. %\S15.15
		\item Every finite group can be factorised into simple groups. %\S15.15
		\item Every finite, non-abelian, simple group is of even order. %\S15.15
		\item Group homomorphism preserves normal subgroups. %\S15.16
		\item $M$ is maximal normal subgroup of $G$ if and only if $G/M$ is simple. %\S15.18
		\item Center $Z(G) = \{ z \in G : zg = gz,\ \forall g \in G \}$ is normal. %\S15.19
		\item Commutator $C(G) = \{ c \in G : c = aba^{-1}b{-1},\ a,b \in G \}$ is normal. %\S15.19
		\item Center of non-abelian groups of order $pq$ are trivial if $p,q$ are primes. %\S15.19
		\item If $N$ is normal, then $G/N$ is abelian iff $C$ is a subgroup of $N$. %\S15.20
	\end{itemize}
\end{remark}

%\pagebreak

%Week 03 Day 02 \S16.1-10
\section{Group Action on a Set}
\begin{definition}%\S16.1
	An action of a group $G$ on a set $X$ is a map.
	$\ast : G \times X \to X$ such that
	\begin{enumerate}
		\item $ex = x,\ \forall x \in X$
		\item $(g_1g_2)(x) = g_1(g_2x),\ \forall x \in X,\ \forall g_1,g_2 \in G$
	\end{enumerate}
	Then $X$ is a $G$-set.
\end{definition}

\begin{theorem}
	Let $X$ be a $G$-set.
	$\forall g \in G,\ \sigma_g : X \to X$ defined by $\sigma_g(x) = gx$ is a permutation of $X$.
	Also, the map $\phi : G \to S_X$ defined by $\phi(g) = \sigma_g$ is a homomorphism with the property that $\phi(g)(x) = gx$.
\end{theorem}
\begin{proof}
	We have, $X$ is a $G$-set.
	Let $g \in G$, and $x_1,x_2 \in X$.
	Suppose $\sigma_g(x_1) = \sigma_g(x_2)$.
	$\implies gx_1 = gx_2 \implies g^{-1}(gx_1) = g^{-1}(gx_2) \implies (g^{-1}g)x_1 = (g^{-1}g)x_2$.
	$\implies ex_1 = ex_2 \implies x_1 = x_2$.
	Thus, $\sigma_g$ is injective.
	Let $x \in X$.
	Then $\sigma_g(g^{-1}x) = g(g^{-1}x) = (gg^{-1})x = ex = x$.
	Thus, $\sigma_g$ is surjective.
	Therefore, $\sigma_g$ is a permutation of $X$, $\sigma_g \in S_X$.
	Let $g_1,g_2 \in G$.
	And $\phi(g_1)(x) = \sigma_{g_1}(x) = g_1x$, $\phi(g_2)(x) = \sigma_{g_2}(x) = g_2x$.
	$\phi(g_1g_2)(x) = \sigma_{g_1g_2}(x) = (g_1g_2)x = g_1(g_2x) = \sigma_{g_1}(g_2x) = \phi(g_1)(g_2x) = \phi(g_1)(\sigma_{g_2}(x)) = \phi(g_1)(\phi(g_2)(x)) = \phi(g_1)\phi(g_2)(x)$.
	Therefore, $\phi(g_1g_2) = \phi(g_1)\phi(g_2)$ and $\phi$ is a homomorphism.
\end{proof}

\begin{definition}Group Action,
	\begin{itemize}
		\item $G$ acts \textbf{faithfully} on $X$, if $e$ is the only element that leaves every $x \in X$ fixed. 
		\item $G$ is \textbf{transitive} on $X$ if for every $x_1,x_2 \in X,\ \exists g \in G$ such that $gx_1 = x_2$.
			$G$ is transitive on $X$ iff the subgroup $\phi[G]$ of $S_X$ is transitive.
	\end{itemize}
\end{definition}
%\pagebreak

%Week 03 Day 03 \S16.11-17
\section{Isotropy subgroups \& Orbits}
\begin{definition}Let $X$ be a $G$-set.
	\begin{itemize}
		\item The subset fixed by $g$, $X_g = \{ x \in X : gx = x \}$
		\item The isotropy subgroup of $x$, $G_x = \{ g \in G : gx = x \}$
		\item The orbit of $x$ in $X$ under $G$, $Gx = \{ gx \in X : g \in G \}$
	\end{itemize}
\end{definition}

\begin{theorem}
	Let $X$ be a $G$-set.
	Then $G_x$ is a subgroup of $G$, $\forall x \in X$.
\end{theorem}
\begin{proof}
	Let $x \in X$.
	And $g_1, g_2 \in G_x$.
	Then $g_1x = x$ and $g_2x = x$.

	Clearly, $(g_1g_2)x = g_1(g_2x) = g_1x = x$.
	Therefore, $g_1g_x \in G_x$.
	Also $ex = x \implies e \in G_x$.
	Let $g \in G_x$.
	Then $gx = x \implies g^{-1}(gx) = g^{-1}x \implies (g^{-1}g)x = g^{-1}x \implies x = g^{-1}x$.
	Thus, for any $g \in G_x$, $g^{-1} \in G_x$.
	Therefore, $G_x$ is a subgroup of $G$ for any $x \in X$.
\end{proof}

\begin{theorem}
	Let $X$ be a $G$-set and $x_1,x_2 \in X$.
	Then the relation $\sim$ defined by $x_1 \sim x_2$ iff $gx_1 = x_2$ is an equivalence relation.
\end{theorem}
\begin{proof}
	Let $x \in X$.
	Then $ex = x \implies x \sim x$.
	Let $x_1, x_2 \in X$ and $x_1 \sim x_2$.
	Then there exists some $g \in G$ such that $gx_1 = x_2$.
	We have, $g^{-1}x_2 = g^{-1}(gx_1) = (g^{-1}g)x_1 = ex_1 = x_1$.
	Therefore, $x_2 \sim x_1$.
	Let $x_1,x_2,x_3 \in X$ and $x_1 \sim x_2$ and $x_2 \sim x_3$.
	Then there are $g_1,g_2 \in G$ such that $g_1x_1 = x_2$ and $g_2x_2 = x_3$.
	Clearly, $g_2g_1 \in G$ and $(g_2g_1)x_1 = g_2(g_1x_1) = g_2x_2 = x_3$.
	Therefore, $x_1 \sim x_3$.
\end{proof}

\begin{theorem}
	Let $X$ be a $G$-set and $x \in X$.
	Then $|Gx| = (G:G_x)$.
	If $|G|$ is finite, then $|Gx|$ is a divisor of $|G|$.
\end{theorem}
\begin{proof}
	We have $Gx$ is the orbit of $x$ in $X$ under $G$ and ${L_G}_x$ is the left cosets of $G_x$ in $G$.
	Let $x_1 \in Gx$.
	Then there exists $g_1 \in G$ such that $x_1 = g_1x$.
	Define $\psi : Gx \to {L_G}_x$ by $\psi(x_1) = g_1G_x$.\\
	Step 1 : $\psi$ is well-defined.

	Let $x_1 \in Gx$. Suppose there exists $g_1,g_1' \in G$ such that $g_1x = x_1$ and $g_1'x = x_1$.
	Then we have, $g_1x = g_1'x \implies x = g_1^{-1}(g_1'x) = (g_1^{-1}g_1')x$.
	Thus, $g_1^{-1}g_1' \in G_x$.
	Therefore, $g_1(g_1^{-1}g_1') \in g_1G_x$.
	Clearly, $g_1(g_1^{-1}g_1') = (g_1g_1^{-1})g_1' = g_1' \in g_1G_x$.
	Therefore, $g_1G_x = g_1'G_x$.
	And $\psi(x_1) = g_1G_x$ is well-defined.\\
	Step 2 : $\psi$ is one-to-one.

	Suppose $\psi(x_1) = \psi(x_2)$.
	Let $x_1, x_2 \in Gx$ such that $x_1 = g_1x$ and $x_2 = g_2x$.
	Then we have $\psi(x_1) = \psi(x_2) \implies g_1G_x = g_2G_x$.
	Thus, $g_2 = g_1g$ for some $g \in G_x$.
	Clearly, $x_2 = g_2x = (g_1g)x = g_1(gx) = g_1x = x_1$.	\\
	Step 3 : $\psi$ is onto.

	Let $g_1G_x$ be a left coset of $G_x$ in $G$.
	Then we have, $g_1 \in G$ and $g_1x \in Gx$, say $x_1$.
	Therefore, there exists $x_1 \in Gx$ such that $\psi(x_1) = g_1G_x$.
\end{proof}

%\pagebreak

%Week 03 Day 04 \S16E
\section{Exercise \S16}

%\pagebreak

%Week 04 Day 01 \S17.1-7
\section{Application of $G$-Sets to Counting}
\begin{theorem}[Burnside]
	Let $G$ be a finite group and $X$ a finite $G$-set.
	If $r$ is the number of orbits in $X$ under $G$, then
	\begin{equation}
		r\cdot |G| = \sum_{g \in G}|X_g|
	\end{equation}
\end{theorem}
\begin{proof}
\end{proof}

\begin{corollary}
	If $G$ is a finite group and $X$ is a finite $G$-set, then
	\begin{equation}
		\text{number of orbits in X under G } = \frac{1}{|G|} \sum_{g \in G} |X_g|
	\end{equation}
\end{corollary}
