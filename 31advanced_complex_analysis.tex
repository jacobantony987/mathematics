%Text Books : \cite{ahlfors}
%Module 1:
%Harmonic Functions - Definitions and Basic Properties, The Mean-Value Property, Poisson's Formula, Schwarz's Theorem, The Reflection Principle. A closer look at Harmonic Functions - Functions with Mean Value Property, Harnack's Principle.
%The Dirichlet's Problem - Subharmonic Functions, Solution of Dirichlet's Problem ( Proof of Dirichlet's Problem and Proofs of Lemma 1 and 2 excluded )
%(Chapter 4 : Section 6: 6.1 - 6.5, Chapter 6 : Section 3 : 3.1 - 3.2 , Section 4 : 4.1 - 4.2)
%Module 2:
%Power Series Expansions - Weierstrass's theorem, The Taylor Series, The Laurent Series Partial Fractions and Factorization - Partial Fractions, Infinite Products, Canonical Products, The Gamma Function. Entire Functions - Jensen's Formula, Hadamard's Theorem ( Hadamard's theorem - proof excluded)
%(Chapter 5 : Section 1 : 1.1 - 1.3, Section 2 : 2.1 - 2.4, Section 3 : 3.1 - 3.2 )
%Module 3:
%The Riemann Zeta Function - The Product Development, The Extension of $\zeta(s)$ to the Whole Plane, The Functional Equation, The Zeroes of the Zeta Function Normal Families - Normality and Compactness, Arzela's Theorem
%(Chapter 5 : Section 4 : 4.1 - 4.4, Section 5 : 5.2 - 5.3)
%Module 4:
%The Riemann Mapping Theorem - Statement and Proof, Boundary Behaviour, Use of the Reflection Principle
%The Weierstrass's Theory - The Weierstrass's $\wp$-function, The functions $\zeta(s)$ and $\sigma(z)$, The Differential Equation
%(Chapter 6 : Section 1: 1.1-1.3, Chapter 7 : Section 3 : 3.1 - 3.3)

%Module 1 - \cite{ahlfors} 4.6, 6.3, 6.4
%Module 2 - \cite{ahlfors} 5.1, 5.2, 5.3
%Module 3 - \cite{ahlfors} 5.4, 5.5
%Module 4 - \cite{ahlfors} 6.1, 7.3
%Missing - \cite{ahlfors} (6.2, 6.5, 7.1, 7.2, 8.1, 8.2, 8.3, 8.4)

%Module 1 - 4.6, 6.3, 6.4
{\Large Module 1}
\section{Harmonic Functions(4.6)}
\subsection{Definition and Basic properties(4.6.1) pp. 162}
\begin{definition}[harmonic]
	A real-valued function $u(z)$ defined and single-valued in a region $\Omega$ is harmonic if it is continuous together with its partial derivatives of first two orders and satisfies Laplace's equation
	\[ \Delta u = \frac{\partial^2 u}{\partial x^2} + \frac{\partial^2 u}{\partial y^2} = 0 \]

	In polar form,
	\[ r \frac{\partial}{\partial r} \left( r \frac{\partial u}{\partial r}\right) + \frac{\partial^2 u}{\partial \theta^2} = 0 \]
\end{definition}
%Regularity conditions can be weakened. How ?
\begin{remark}
	The sum of two harmonic functions is harmonic. A constant multiple of a harmonic function is harmonic.\\

	From polar form, $\log r$ is harmonic and any harmonic function that depends only on $r$ must be of the form $a \log r + b$.  The argument $\theta$ is harmonic whenever it can be uniquely defined. \\

	If $u$ is harmonic in $\Omega$, then $f(z) = \frac{\partial u}{\partial x} - i \frac{\partial u}{\partial y}$ is analytic on $\Omega$.\\

	\[ f\ dz = f\ dx+i f\ dy = \left( \frac{\partial u}{\partial x} dx + \frac{\partial u}{\partial y}dy \right) + i \left( -\frac{\partial u}{\partial y}dx + \frac{\partial u}{\partial x}dy \right) \]
	The real part is the differential of $u$ and imaginary part is the conjugate differential of $du$.
	\[ f\ dz = du + i\ \underset{}{^\ast}du \]
\end{remark}

---to be continued---

\begin{theorem}
	If $u_1$ and $u_2$ are harmonic in a region $\Omega$, then
	\[ \int_\gamma u_1 ^\ast du_2 - u_2 ^\ast du_1 = 0 \]
\end{theorem}
\begin{proof}
\end{proof}

\subsection{The Mean-value Property(4.6.2) pp. 165}
\begin{theorem}
	The arithmetic mean of a harmonic function over concentric circles $|z|=r$ is a linear function of $\log r$,	
	\begin{equation}
		u(z_0) = \frac{1}{2\pi} \int_0^{2\pi} u\ d\theta = \alpha \log r + \beta
	\end{equation}
\end{theorem}
\begin{proof}
\end{proof}

\begin{theorem}
	A nonconstant harmonic function has neither a maximum nor a minimum in its region of definition. Consequently, the maximum and the minimum on a closed bounded set $E$ are taken on the boundary of $E$
\end{theorem}
\begin{proof}
\end{proof}

\subsection{Poisson's Formula(4.6.3) pp. 166}
\begin{remark}
	If $u(z)$ is continuous on a closed bounded set $E$ and harmonic in the interior of $E$, then it is uniquely determined by its values on the boundary of $E$.
\end{remark}
\begin{theorem}
	Suppose that $u(z)$ is harmonic for $|z| < R$, continuous for $|z| \le R$. Then
	\begin{equation}
		u(a) = \frac{1}{2\pi} \int_{|z|=R} \frac{R^2-|a|^2}{|z-a|^2}\ u(z)\ d\theta, \quad (|a| < R)
	\end{equation}
\end{theorem}
\begin{proof}
\end{proof}

\subsection{Schwarz's Theorem(4.6.4) pp. 168}
\begin{definition}[Poisson integral]
	Let $U$ be a piece-wise continuous function on $[0,2\pi]$.
	The Poisson integral of $U$ is given by,
	\begin{equation}
		P_U(z) = \frac{1}{2\pi} \int_0^{2\pi} \Re\left(\frac{e^{i\theta}+z}{e^{i\theta}-z}\right)\ U(\theta)\ d\theta
	\end{equation}
\end{definition}

\begin{remark} Clearly, Poisson integral is a positive linear functional since the Riemann integral is linear.
	\[ P_{U+V} = P_U + P_V \]
	\[ P_{cU} = cP_U \]
	\[ U \ge 0 \implies P_U \ge 0 \]
\end{remark}
\begin{proof}
\end{proof}

\begin{remark}
	Also we have, $P_c = c$. And since $P_U$ is a positive, linear functional, we have
	\[m \le U \le M \implies m \le P_U \le M\]
\end{remark}
\begin{proof}
\end{proof}

\begin{theorem}[Schwarz]
	The function $P_U(z)$ is harmonic for $|z|<1$, and
	\begin{equation}
		\lim_{z \to e^{i\theta_0}} P_U(z) = U(\theta_0)
	\end{equation}
	provided $U$ is continuous at $\theta_0$ and $P_U(z)$ is the Poisson integral of the piecewise continuous function $U$ on $[0,2\pi]$.
\end{theorem}
\begin{proof}
\end{proof}

\subsection{The Reflection Principle(4.6.5) pp. 172}
\begin{remark}
	Let $\Omega$ be a region and $\Omega^\ast$ be its reflection about real axis.
	In other words, if $z \in \Omega$, then $\bar{z} \in \Omega^\ast$.
	Let $u(z)$ be a harmonic function defined on $\Omega$, then $u(\bar{z})$ is a harmonic function on $\Omega^\ast$.
	And if $f(z)$ is analytic on $\Omega$, then $\overline{f(\bar{z})}$ is analytic on $\Omega^\ast$.
\end{remark}
\begin{proof}
\end{proof}

\begin{definition}[Symmetric Region]
	Region $\Omega$ is symmetric if it is symmetric about real axis.
	In other words, for every $z \in \Omega$, $\bar{z}$ is also in $\Omega$.
\end{definition}

\begin{theorem}[Reflection Principle]
	Let $\Omega^+$ be the part in the upper plane of a symmetric region $\Omega$ and $\sigma$ be its part on the real axis.
	Let $v(z)$ be continuous on $\Omega^+ \cup \sigma$, harmonic on $\Omega^+$ and vanishes on $\sigma$.
	Then $v$ has a harmonic extension to $\Omega$ satisfying $v(\bar{z}) = -v(z)$.
	And if $v$ is the imgainary part of an analytic function $f(z)$ in $\Omega^+$, then $f(z)$ has an analytic extension which satisfies $f(z) = \overline{f(\bar{z})}$.
\end{theorem}
\begin{proof}
\end{proof}

\section{A Closer look at Harmonic Functions(6.3)}
\subsection{Functions with the Mean-value Property (6.3.1) pp. 242}
\begin{definition}
	Let $u(z)$ be a real-valued continuous function in a region $\Omega$. Function $u$ satisfies mean-value property if
	\[ u(z_0) = \frac{1}{2\pi} \int_0^{2\pi} u(z_0+re^{i\theta})\ d\theta \]
	when the disk $|z-z_0| \le r$ is contained in $\Omega$.
\end{definition}
\begin{remark}
	Mean-value property implies maximum principle.
\end{remark}
\begin{theorem}
	A continuous function $u(z)$ satisfying the mean-value property is necessarily harmonic.
\end{theorem}
\begin{proof}
\end{proof}

\subsection{Harnack's Principle (6.3.2) pp. 243}
\begin{theorem}[Harnack's Inequality]
	\[ \frac{\rho-r}{\rho+r} u(0) \le u(z) \le \frac{\rho+r}{\rho-r}u(0) \]
	where $u(z)$ is harmonic in $|z| = r < \rho$ and continuous is $|z| \le \rho$.
\end{theorem}
\begin{proof}
\end{proof}

\begin{theorem}[Harnack's Principle]
	Increasing sequence of harmonic functions on regions $\Omega_n$ converges to either $+\infty$ or a harmonic function $u$ uniformly on every compact subset of $\Omega$ where every point $z \in \Omega$ belongs to all except finitely many $\Omega_n$.
\end{theorem}
\begin{proof}
\end{proof}

\section{The Dirichlet Problem(6.4)}
\begin{definition}[Dirichlet Problem]
	Find a harmonic function $u(z)$ on a region $\Omega$ with given boundary values.
\end{definition}

\subsection{Subharmonic Functions (6.4.1) pp. 245}
\begin{definition}[subharmonic]
	A continuous, real-valued function $v(z)$ is subharmonic in $\Omega$ if for any harmonic function $u(z)$ in $\Omega^\prime \subset \Omega$, $v-u$ satisfies the maximum principle on $\Omega^\prime$.
\end{definition}

\begin{remark}
	Harmonic functions are subharmonic.\\
	
	Sufficient condition for subharmonicity : If $v$ has a positive laplacian, then $v$ is subharmonic. The condition is not necessary as there exists subharmonic functions without partial dertivatives.
\end{remark}

\begin{theorem}
	A continuous function $v(z)$ is subharmonic in $\Omega$ if and only if 
	\[ v(z_0) \le \frac{1}{2\pi} \int_0^{2\pi} v(z_0+re^{i\theta})\ d\theta \]
	for every disk $|z-z_0| \le r$ contained in $\Omega$.
\end{theorem}
\begin{proof}
\end{proof}

\begin{remark}
	Elementary Properties of Subharmonic Functions,
	\begin{enumerate}
		\item If $v$ is subharmonic, then $kv$ is subharmonic for any constant $k \ge 0$.
			\begin{proof}\end{proof}
		\item If $v_1,v_2$ are subharmonic, then $v_1+v_2$ is subharmonic.
			\begin{proof}\end{proof}
		\item If $v_1,v_2$ are subharmonic, then $v=\max(v_1,v_2)$ is subharmonic.
			\begin{proof}\end{proof}
		\item Let $v$ be subharmonic on $\Omega$. Let $\Delta$ be a disk whose closure is contained in $\Omega$ and $P_v$ be the Poisson integral formed with $v$ on the boundary of $\Delta$. Then $v^\prime$ defined as $P_v$ on $\Delta$ and $v$ outside $\Delta$ is subharmonic.
			\begin{proof}\end{proof}
	\end{enumerate}
\end{remark}

\begin{definition}[semicontinuous]
	A function $v(z)$ is \textbf{upper semicontinuous} at $z_0$ if $\displaystyle \lim_{z \to z_0} \sup v(z) \le v(z_0)$.
	A function $v(z)$ is \textbf{lower semicontinuous} at $z_0$ if $\displaystyle \lim_{z \to z_0} \inf v(z) \ge v(z_0)$.
\end{definition}

\begin{remark}
	Upper continuous function attains maximum on any compact set.
\end{remark}

\subsection{Solution of Dirichlet's Problem (6.4.2) pp. 248}
\begin{definition}[$\mathfrak{B}(f)$]
	Let $f$ be continuous real-vallued function defined on the boundary $\Gamma$ of a bounded region $\Omega$.
	The family of functions, $\mathfrak{B}(f)$ contains all functions $v$ such that 
	\begin{enumerate}
		\item $v$ is subharmonic in $\Omega$ and
		\item $\displaystyle \varlimsup_{z \to \zeta} v(z) \le f(\zeta),\ \forall \zeta \in \Gamma$.
	\end{enumerate}
\end{definition}
\begin{lemma}
	The function $u$ defined as $u(z) = \inf \{ v(z) : v \in \mathfrak{B}(f)\}$ is harmonic in $\Omega$.
\end{lemma}
\begin{proof}
\end{proof}

\begin{remark}[Perron's method]
	Let $\Omega$ be a bounded\footnote{Bounded Set : A set $\Omega$ of complex numbers is bounded, if there exists $\rho \in \mathbb{R}$ such that $\Omega$ is contained in an open disk of radius $\rho$ about origin.} region.
	Let $\Gamma$ be the boundary of $\Omega$.
	Let $f$ be a continuous, real-valued function on $\Gamma$.
	Then, corresponding to each function $f$, there exists a harmonic function $u$ on $\Omega$ which can be obtained by Perron's method.\\

	Let $\Delta$ be a disk whose closure in contained in $\Omega$.
	Let $z_0 \in \Delta$ and sequence $\sequence{v_n}$ in $\mathfrak{B}(f)$ such that $\sequence{v_n(z_0)}$ converges.
	Define $u(z_0) = \lim_{n \to \infty} v_n(z_0)$.
	Without loss of generality, $\sequence{v_n}$ is nondecreasing.\footnote{If not, consider $V_n = \max\{v_1,v_2,\dots,v_n\}$}
	By Harnack's principle, there exists a harmonic function $U$ such that $U \le u$ and $U(z_0) = u(z_0)$.\\

	Let $z_1 \in \Delta$ and sequence $\sequence{w_n}$ in $\mathfrak{B}(f)$ converges.
	Define $u(z_1) = \lim_{n \to \infty} w_n(z_1)$.
	Then there exists a harmonic function $U_1$ such that $U \le U_1 \le u$ and $U_1(z_1) = u(z_1)$.
	However, $U - U_1$ hax maximum zero at $z_0$.
	Thus, $U = U_1$ and $\forall z_1 \in \Delta,\ U(z_1) = U_1(z_1) = u(z_1)$.
	Clearly, we have constructed the harmonic function $u = U$.
\end{remark}

\begin{lemma}
	Suppose there exists a harmonic function $w(z)$ in $\Omega$ whos continuous boundary values $w(\zeta)$ are strictly positive except at one point $\zeta_0 = 0$. Then, if $f(\zeta)$ is continuous at $\zeta_0$, the corresponding function $u$ determined by Perron's method satisfies $\displaystyle \lim_{z \to \zeta_0} u(z) = f(\zeta_0)$.
\end{lemma}
\begin{proof}
\end{proof}

\begin{theorem}
	The Dirchlet problem can be solved for any region $\Omega$ such that each boundary point is the end point of a line segment whose other points are exterior to $\Omega$.
\end{theorem}
\begin{proof}
\end{proof}

\pagebreak
%Module 2 - 5.1, 5.2, 5.3
{\Large Module 2}
\section{Power Series Expansions(5.1)}
\subsection{Weierstrass's Theorem(5.1.1) pp.175}
\begin{important}
	Weierstrass's theorem gives a sufficient condition for convergence to preserve analyticity, that is uniform convergence on every compact subset of the region. 
	And Hurwitz's theorem characterises such analytic, limit functions.
\end{important}
\begin{theorem}[Weierstrass]
	Suppose $f_n(z)$ is analytic in the region $\Omega_n$, and the sequence $\sequence{f_n(z)}$ converges to $f(z)$ in a region $\Omega$ uniformly on every compact subset of $\Omega$. Then the limit function $f(z)$ is analytic in $\Omega$. Moreover, $f_n^\prime(z)$ converges to $f^\prime(z)$ on every compact subset of $\Omega$.
\end{theorem}
\begin{proof}
\end{proof}

\begin{remark}
	A series with analytic terms
	\[ f(z) = f_1(z) + f_2(z) + \dotsb + f_n(z) + \dotsb \]
	converges uniformly on every compact subset of a region $\Omega$, then the sum $f(z)$ is analytic in $\Omega$. And the series can be differentiated term by term.
\end{remark}
\begin{proof}
\end{proof}
\begin{remark}
	Uniform convergence on the boundary of a compact set $A$ implies unform convergence on $A$.
\end{remark}
\begin{proof}
	Hint : maximum principle
\end{proof}

\begin{remark}
	If $f_n(z)$ are analytic in $|z|<1$ and convergence is uniform on each circle $C : |z| = r_m$ where $r_m \to 1$ as $m \to \infty$, then limit function $f(z)$ is analytic.
\end{remark}
\begin{proof}
\end{proof}

\begin{theorem}[Hurwitz]
	If the functions $f_n(z)$ are analytic in a region $\Omega$, $f_n(z) \ne 0, \forall z \in \Omega$ and $f_n(z)$ converges to $f(z)$ uniformly on every compact subset of $\Omega$, then $f(z)$ is either identitally zero in $\Omega$ or never assumes zero in $\Omega$.
\end{theorem}
\begin{proof}
\end{proof}
	
\subsection{The Taylor Series (5.1.2) pp 179}
\begin{important}
	Every analytic function has a convergent Taylor series representation.
\end{important}
\begin{theorem}
	If $f(z)$ is analytic in a region $\Omega$ containing $z_0$, then the representation
	\begin{equation}
		f(z) = f(z_0) + \frac{f^\prime(z_0)}{1!}(z-z_0) + \dotsb + \frac{f^{(n)}(z_0)}{n!}(z-z_0)^n + \dotsb
	\end{equation}
	is valid in the largest open disk of center $z_0$ contained in $\Omega$.
\end{theorem}
\begin{proof}
\end{proof}

Taylor series representation for a few analytic functions,
\begin{enumerate}
	\item $e^z = 1 + z + \frac{z^2}{2!} + \dotsb + \frac{z^n}{n!} + \dotsb$
	\item $\cos z = 1 - \frac{z^2}{2!} + \frac{z^4}{4!} + \dotsb + $
	\item $\sin z = z - \frac{z^3}{3!} + \frac{z^5}{5!} + \dotsb + $
	\item $(1+z)^\mu = 1 + \mu z + \binom{\mu}{2} z^2 + \dotsb$ where $\binom{\mu}{n} = \frac{\mu(\mu-1)(\mu-2) \dotsm (\mu-n+1)}{1 \cdot 2 \cdot 3 \dotsm n}$
	\item $\log (1+z) = z - \frac{z^2}{2} + \frac{z^3}{3} - \frac{z^4}{4} + \dotsb$
	\item $\arctan z = z - \frac{z^3}{3} + \frac{z^5}{5} - \frac{z^7}{7} + \dotsb$
	\item $\frac{1}{\sqrt{1-z^2}} = 1 + \frac{1}{2}z^2 + \frac{1}{2}\frac{3}{4} z^4 + \frac{1}{2} \frac{3}{4} \frac{5}{6} z^6 + \dotsb$ 
	\item $\arcsin z = z + \frac{1}{2} \frac{z^3}{3} + \frac{1}{2} \frac{3}{4} \frac{z^5}{5} + \frac{1}{2} \frac{3}{4} \frac{5}{6} \frac{z^7}{7} + \dotsb$
	\item $\tan z = z + \frac{z^3}{3}+\frac{2}{15}z^5+\dotsb$
	\item $\log \frac{\sin z}{z} = $
\end{enumerate}

\subsubsection*{}
The procedure to find Taylor representation of an analytic function,
\begin{enumerate}
	\item Choose a well-defined branch so that we have a single-valued function.
	\item Choose a center $z_0$ which is not a branch point about which the series is to be developed.
\end{enumerate}

\subsubsection*{}
Computational techniques to obtain Taylor series representation,
\begin{enumerate}
	\item Comparing Real and Imaginary Parts of a Taylor Series
	\item Differentiating/Intergrating a Taylor Series
	\item $f(z)g(z) = P_n(z)Q_n(z) + [z^{n+1}]$ \\ where $f(z) = P_n(z) + [z^{n+1}]$ and $g(z) = Q_n(z) + [z^{n+1}]$.
	\item $f(z)/g(z) = R_n(z) + [z^{n+1}]$\\ such that $P_n(z) + [z^{n+1}] = Q_n(z)R_n(z) + [z^{n+1}]$.
	\item $f(g(z)) = P_n(Q_n(z)) + [z^{n+1}]$
	\item $g^{-1}(w) = P_n(w) + [w^{n+1}]$ \\ such that $P_n(Q_n(z)) + [z^{n+1}] = z$.
\end{enumerate}

\begin{definition}[Legendre Polynomials]
	Legendre polynomials in $\alpha$, $P_n(\alpha)$ are the coefficients of the Taylor series of $(1-2\alpha z + z^2)^{-\frac{1}{2}}$ about origin.
	\begin{equation}
		(1-2\alpha z + z^2)^{-\frac{1}{2}} = 1 + P_1(\alpha)z + P_2(\alpha)z^2+\dotsb
	\end{equation}
\end{definition}

\subsection{The Laurent Series (5.1.3) pp. 184}
\begin{remark}
	If the function $f(z)$ is analytic in the annulus $R_1 < |z-a| < R_2$, then the Laurent Series representation
	\begin{equation}
		f(z) = \sum_{n = -\infty}^\infty A_n (z-a)^n
	\end{equation}
	is valid for any $z$ in the annulus.
\end{remark}
\begin{proof}
\end{proof}

\begin{definition}[Schwarzian Derivative]
	Schwarzian Derivative of $f$ is defined as,
	\[ \{f,z\} = \frac{f^{\prime\prime\prime}(z)}{f^\prime(z)} - \frac{3}{2} \left(\frac{f^{\prime\prime}(z)}{f^\prime(z)}\right)^2 \]
	The leading term in the Laurent Series representation of Schwarzian derivative should be of some importance.
\end{definition}

\begin{remark}
	Laurent series representation,
	\[ (e^2-1)^{-1} = \frac{1}{z} - \frac{1}{2} + \sum_{k=1}^\infty (-1)^{k-1}\frac{B_k}{(2k)!} z^{2k-1} \]
	where $B_k$ are the Bernoulli numbers.
\end{remark}
\section{Partial Fractions and Factorisation(5.2)}
\subsection{Partial Fractions (5.2.1) pp. 187}
\begin{theorem}[Mittag-Leffler]
	Let $\sequence{b_\nu}$ be a sequence of complex numbers with $\displaystyle \lim_{\nu \to \infty} b_\nu = \infty$, and let $P_\nu(\zeta)$ be polynomials without constant term. Then there are functions which are meromorphic in the whole plane with poles are the points $b_\nu$ and the corresponding singular parts $P_\nu(\frac{1}{z-b_\nu})$. Moreover, the most general meromorphic function of this kind can be written in the form
	\begin{equation}
		f(z) = \sum_\nu \left[ P_v\left(\frac{1}{z-b_\nu}\right) - p_\nu(z) \right] + g(z)
	\end{equation}
	where $p_\nu(z)$ are suitably chosen polynomials and $g(z)$ is analytic in the whole plane.
\end{theorem}
\begin{proof}
\end{proof}

\begin{remark}
	\begin{enumerate}
		\item \[ \frac{\pi^2}{\sin^2 \pi z} = \sum_{n=-\infty}^\infty \frac{1}{(z-n)^2} \]
		\item \[ \pi \cot \pi z = \frac{1}{z} + \sum_{n \ne 0} \left(\frac{1}{z-n} + \frac{1}{n} \right) \]
	\end{enumerate}
\end{remark}

\subsection{Infinite Products (5.2.2) pp. 191}
\begin{definition}[Infinite Product]
	Infinite product of complex numbers $\sequence{p_n}$ is defined as
	\begin{equation}
		P = \lim_{n \to \infty} P_n = \lim_{n \to \infty} \prod_{k=1}^n p_k
	\end{equation}
	if the limit exists and is nonzero. The infinite product converges if and only if at most finitely many terms are zero and infinite product of the nonzero terms converges to a finite limit.	
\end{definition}

\begin{remark}
	In a convergent infinite product, the general term $p_n \to 1$.\\
	Thus, we may represent an infinite product as
	\[ \prod_{n=1}^\infty p_n = \prod_{n = 1}^\infty (1+a_n) \text{ where } a_n \to 0 \]
\end{remark}

\begin{theorem}
	The infinite product $\prod (1+a_n)$ with $1+a_n \ne 0$ converges simultaneously with the series $\sum \log (1+a_n)$ whose terms represent the value of the pricipal branch of the logarithm.
\end{theorem}
\begin{proof}
\end{proof}

\begin{theorem}
	The infinite product $\prod (1+a_n)$ converges absolutely if and only if the respective series $\sum |a_n|$ converges.
\end{theorem}
\begin{proof}
\end{proof}

\subsection{Canonical Products (5.2.3) pp. 193}
\begin{remark}[Canonical Product]
	An entire function $f(z)$ may be represented using another entire function $g(z)$, 
	\begin{equation}
		f(z) = z^m e^{g(z)} \prod(1-\frac{z}{a_n})
	\end{equation}
	if the infinite product converges absolutely on every compact subset.
\end{remark}

\begin{theorem}[Weierstrass]
	There exists an entire function with arbitrarily prescribed zeroes $a_n$ provided that, in the case of infinitely many zeroes, $a_n \to \infty$. Every entire functions with these and no other zeroes can be written in the form
	\begin{equation}
		f(z) = z^m e^{g(z)} \prod_{n=1}^\infty \left( 1- \frac{z}{a_n} \right) e^{\frac{z}{a_n} + \frac{1}{2}(\frac{z}{a_n})^2 + \dotsb + \frac{1}{m_n}(\frac{z}{a_n})^{m_n}}
	\end{equation}
	where all the product is taken over all $a_n \ne 0$, the $m_n$ are certain integers, and $g(z)$ is an entire function.
\end{theorem}
\begin{proof}
\end{proof}

\begin{corollary}
	Every function which is meromorphic in the whole plane is the quotient of two entire functions.
\end{corollary}
\begin{proof}
\end{proof}

\subsection{The Gamma Function (5.2.4) pp 198}
\begin{definition}
	Euler's gamma function is defined as,
\begin{equation}
	\Gamma(z) = \frac{e^{-\gamma z}{z}}{z} \prod_{n=1}^\infty \left( 1+\frac{z}{n}\right)^{-1} e^\frac{z}{n}
\end{equation}
	where $\gamma = \displaystyle \lim_{n \to \infty} \left( 1+ \frac{1}{2} + \frac{1}{3} + \dotsb + \frac{1}{n} - \log n \right) \approx 0.57722 $ is the Euler's constant.
\end{definition}
\begin{remark}
	\begin{equation}
		\Gamma(z+1) = z \Gamma(z)
	\end{equation}
\end{remark}
\begin{remark}
	\begin{equation}
		\Gamma(z) \Gamma(1-z) = \frac{\pi}{\sin \pi z}
	\end{equation}
\end{remark}
\begin{remark}[Legendre's Duplication Formula]
	\begin{equation}
		\sqrt{\pi} \Gamma(2z) = 2^{2z-1} \Gamma\left(z+\frac{1}{2}\right)
	\end{equation}
\end{remark}
\subsection*{Exercise}
\begin{remark}[Gauss's Formula]
	\begin{equation}
		(2\pi)^\frac{n-1}{2} \Gamma(z) = n^{z-\frac{1}{2}} \ \Gamma\left(\frac{z}{n}\right) \Gamma\left(\frac{z+1}{n}\right) \Gamma\left(\frac{z+1}{n}\right) \ \dotsm \ \Gamma\left(\frac{z+n+1}{n}\right)
	\end{equation}
\end{remark}

\begin{remark}
	\begin{equation}
		\Gamma\left(\frac{1}{6}\right) = 2^{-\frac{1}{2}} \left( \frac{3}{\pi}\right)^\frac{1}{2} \Gamma\left(\frac{1}{3}\right)^2
	\end{equation}
\end{remark}

%\subsection{Stirling's Formula (5.2.5) pp 201}
\section{Entire Functions(5.3)}
\subsection{Jensen's Formula (5.3.1) pp 207}
\begin{remark}[Jensen's Formula]
	Let $f$ be an entire function with zeroes $a_1,a_2,\dots,a_n$ in the interior of $|z| < \rho$.
	Then,
	\begin{equation}
		\log |f(0)| = -\sum_{i=1}^n \log \left( \frac{\rho}{|a_i|} \right) + \frac{1}{2\pi} \int_0^{2\pi} \log \left| f\left(\rho e^{i\theta}\right) \right| \ d\theta
	\end{equation}
\end{remark}
\begin{proof}
\end{proof}
\begin{remark}[Poisson-Jensen Formula]
	Let $f$ be an entire function with zeroes $a_1,a_2,\dots,a_n$ in the interior of $|z| < \rho$ and $f(z) \ne 0$.
	Then,
	\begin{equation}
		\log |f(0)| = -\sum_{i=1}^n \log \left| \frac{\rho^2-\bar{a_i}z}{\rho(z-a_i)} \right| + \frac{1}{2\pi} \int_0^{2\pi} \Re \frac{\rho e^{i \theta}+z}{\rho e^{i \theta} - z}\ \log \left| f\left(\rho e^{i\theta}\right) \right| \ d\theta
	\end{equation}
\end{remark}
\begin{proof}
\end{proof}

\subsection{Hadamard's Theorem (5.3.2) pp 208}
\begin{definition}[Order]
	The order of an entire function $f$ is defined by
	\begin{equation}
		\lambda = \varlimsup_{r \to \infty} \frac{\log \log M(r)}{\log r} \text{ where } M(r) = \max \{ f(z) : |z| = r\}
	\end{equation}
\end{definition}
\begin{remark}
	The order is the smallest number such that
	\begin{equation}
		M(r) \le e^{r^{\lambda+\varepsilon}},\ \forall \varepsilon > 0 \text{ and sufficiently large } r
	\end{equation}
\end{remark}

\begin{theorem}
	The genus $h$ and the order $\lambda$ of an entire function \textcolor{blue}{of finite order} satisfies the double inequality $h \le \lambda \le h+1$.
\end{theorem}
\begin{proof}
\end{proof}

\begin{remark}
	If the genus of an entire function is fractional, then the genus $h$ and the form of the product is uniquely determined.
	If the genus is integral, then there is an ambiguity.
\end{remark}
\begin{proof}
\end{proof}

\begin{corollary}
	An entire function of fractional order assumes every finite value infinitely many times.
\end{corollary}
\begin{proof}
\end{proof}

\pagebreak
%Module 3 - 5.4, 5.5
{\Large Module 3}
\section{The Riemann Zeta Function (5.4) pp 212}
\begin{definition}[Euler Zeta Function]
\begin{equation}
	\zeta(s) = \sum_{n=1}^\infty n^{-s},\ s > 1
\end{equation}
\end{definition}

\begin{definition}[Riemann Zeta Function]
	The Riemann Zeta Function given by,
\begin{equation}
	\zeta(s) = \sum_{n=1}^\infty n^{-s},\ \Re(s) = \sigma > 1 \quad (s = \sigma + it)
\end{equation}
	is analytic in the half plane $\sigma > 1$.
\end{definition}

\subsection{The Product Development (5.4.1) pp 213}
\subsection{Extension of $\zeta(s)$ to the Whole Plane (5.4.2) pp 214}
\subsection{The Functional Equation (5.4.3) pp 218}
\subsection{The Zeroes of the Zeta Function (5.4.4) pp 218}

\section{Normal Families (5.5) pp 219}
\subsection{Equicontinuity (5.5.1) pp 219}
\subsection{Normality and Compactness (5.5.2) pp 220}
\subsection{Arzela's Theorem (5.5.3) pp 222}
%\subsection{The Classical Definition (5.5.4) pp 225}

%Module 4 - 6.1, 7.3
\pagebreak
{\Large Module 4}
\section{The Riemann Mapping Theorem(6.1)}
\subsection{Statement and Proof (6.1.1) pp 229}
\subsection{Boundary Behaviour (6.1.2) pp 232}
\begin{commentary}
\begin{theorem}
	Let $f : \Omega \to \Omega^\prime$ be a surjective, conformal mapping.
	If $\Omega$ and $\Omega^\prime$ are Jordan regions, then $f$ can be extended to a topological mapping from the closure of $\Omega$ onto the closure of $\Omega^\prime$.
\end{theorem}
\begin{proof}
	Out of scope.
\end{proof}
\end{commentary}

\begin{definition}
	Let sequence $\sequence{z_n}$ ben a sequence in $\Omega$. The \textbf{sequence tends to the boundary of $\Omega$} if the points eventually stays away from any point in $\Omega$.
\end{definition}

In other words, the sequence $\sequence{z_n}$ tends to the boundary of $\Omega$ if for any point $z_0 \in \Omega$ and any neighbourhood $|z - z_0| < \varepsilon$, there exists a natural number $N$ such that $\forall n > N,\ |z_n - z_0|>\varepsilon$.

\begin{definition}
	\textbf{An arc $z(t)$ tends to the boundary of $\Omega$} if $z(t)$ eventually stays away from any point in $\Omega$.
\end{definition}
In other words, the arc $z : [a,b] \to \mathbb{C}$ tends to the boundary of $\Omega$ if $\forall \varepsilon > 0$ there exists $t_0 \in [a,b]$ such that $\forall t > t_0,(t \in [a,b])\ \forall z \in \Omega,\ |z(t)-z| > \varepsilon$.

\begin{theorem}
	Let $f$ be a topological mapping of a region $\Omega$ onto $\Omega^\prime$. If a sequence $\sequence{z_n}$ or an arc $z(t)$ tends to the boundary of $\Omega$, then the sequence $\sequence{f(z_n)}$ or the arc $f(z(t))$ tends to the boundary of $\Omega^\prime$.
\end{theorem}
\begin{proof}
	Suppose sequence $\sequence{z_n}$ is a sequence which tends to boundary of $\Omega$.
	Let $K$ be a compact subset of $\Omega^\prime$.
	Then $f^{-1}(K)$ is a compact subset of $\Omega$ (which is preserved by the continuous function).\\
	
	The family of balls with center at $z \in K$ and radius $\varepsilon$ is a cover of $K$.
	The radius of these balls may depend on the point $z \in K$.
	Since $f^{-1}(K)$ is compact, the above cover has a finite subcover.
	Thus, there are finite balls with center $z_1,z_2,\dots,z_k$ and radius $\varepsilon_1, \varepsilon_2,\dots,\varepsilon_k$.
	Since $\sequence{z_n}$ stays away from each of these points, there exists natural numbers $N_1,N_2,\dots,N_k$ such that $|z_n-z_j|>\varepsilon_j$ for every $n > N_k$ and $j=1,2,\dots,k$.
	Define $n_0 = \max\ \{N_1,N_2,\dots,N_k\}$ and $\varepsilon_0 = \min\{\varepsilon_1,\varepsilon_2,\dots,\varepsilon_k\}$.
	Then $|z_n-z_j| > \varepsilon_0$ for every $n > n_0$ and $j = 1,2,\dots,k$.\\

	Clearly, $z_n \notin K$ for $n > n_0$ and $f(z_n) \notin K$ for $n > n_0$.
	Thus, the sequence $\sequence{f(z_n)}$ stays away from each point in $K$.
	Since $K$ is arbitrary, the sequence stays away from every compact subset of $\Omega^\prime$.
	Thus, the sequence stays away from any point in $\Omega^\prime$.
	Clearly, the sequence tends to the boundary of $\Omega^\prime$.
\end{proof}

\subsection{Use of the Reflection Principle (6.1.3) pp 233}
\begin{definition}[free boundary arc]
	Let $\Omega$ be a region and $\gamma$ be a arc.
	Arc $\gamma$ is a free boundary arc if every point of $\gamma$ has a neighbourhood whose intersection with $\partial \Omega$ is same as its intersection with $\gamma$.
\end{definition}

\begin{definition}[one-sided/two-sided boundary arc]
	Let $\gamma$ be a free-boundary arc with respect to $\Omega$.
	Then each point $z(t_0)$ on $\gamma$ has a disk about $z(t_0)$ such that $\gamma$ divides the disk into two half disks in such a way that at least one of the half disks is contained in $\Omega$.\\

	A point on $\gamma$ is a one-sided boundary point if one half disk is outside $\Omega$ and two-side boundary point if both half disks are inside $\Omega$.\\

	A arc is one-side boundary arc if all its points are one-sided boundary points. And two-sided boundary arc if all its points are two-sided boundary points.
\end{definition}

\begin{theorem}
	Let $\Omega$ be a simply connect region.
	Let $\gamma$ be line segement in $\partial \Omega$ and is a one-sided boundary arc.
	Then the function which maps $\Omega$ onto the unit disk can be extended to a function analytic and one to one on $\Omega \cup \gamma$. The image of $\gamma$ is an arc $\gamma^\prime$ on the unit circle.
\end{theorem}
\begin{proof}
	Let $f$ be a conformal mapping from $\Omega$ onto unit disk, normalised by the condition $f(z_0) = 0$.
	Consider a disk about $x_0 \in \gamma$ such that the half disk in $\Omega$ does not contain $z_0$ with $f(z_0) = 0$.	
	Then $\log f(z)$ has a single valued branch in the half disk, and its real part tends to $0$ as $z$ tends to the diameter.\footnote{If the intersection of neighbourhood of $x_0$ with $\partial \Omega$ is its intersection with $\gamma$, then the intersection is a diameter of the disk}
	From the boundary behaviour of the conformal map $f$ suggested by Riemann mapping theorem, we know that ``as $z$ tends to the boundary of $\Omega$, $f(z)$ tends to the boundary of the unit disk.''
	Thus, $|f(z)|$ tends to $1$.\\

	By reflection principle, $\log f(z)$ has an analytic extension to the whole disk. Therefore, $\log f(z)$ is analytic at $x_0$. Clearly, $f(z)$ is analytic at $x_0$. Suppose $x_1 \in \gamma$ belongs to the disk. Then the extension to the neighbourhood of $x_1$ coincides with the extension to the neighbourhood of $x_0$ on their intersection and $f$ is analytic at $x_1$. Continuing like this, $f(z)$ can be extended to $\Omega \cup \gamma$ preserving its analyticity.\\

	We claim that, $f'(z) \ne 0$ for any $z \in \gamma$. Suppose $f'(x_0) = 0$ for some $x_0 \in \gamma$. Then $f(x_0)$ is a branch point and the two subarcs of $\gamma$ at $x_0$ form an angle $\pi/n$ where $n \ge 2$. Then the unit cirlcle should have two arcs at $\pi/n$ angles which is impossible.\\

	Without loss of generality suppose that the upper half disks are in $\Omega$. Then 
	\[ \frac{\partial \log |f|}{\partial y} = -\frac{\partial \arg f}{\partial x} < 0 \]
	 on $\gamma$. Thus, $\gamma$ and $\arg f$ move in the same direction.
	 Therefore, $f$ is injective on $\gamma$.
\end{proof}
\begin{remark}
	For two-side (boundary) arcs
	---to continue--- find `obvious' modifications
\end{remark}

\subsection{Analytic Arcs (6.1.4) pp 234}
\begin{theorem}
	If the boundary of $\Omega$ contains a free one-sided analytic arc $\gamma$, then the mapping function has an analytic extension to $\Omega \cup \gamma$, and $\gamma$ is mapped on an arc of the unit circle.
\end{theorem}
\begin{proof}
\end{proof}

\section{The Weierstrass Theory(7.3)}
\subsection{The Weierstrass $\wp$ function (7.3.1) pp 272}
\subsection{The Functions $\zeta(z)$ and $\sigma(z)$ (7.3.2) pp 273}
\subsection{The Differential Equation (7.3.3) pp 275}
