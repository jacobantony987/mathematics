%Text Books : \cite{ahlfors}
%Module 1:
%Harmonic Functions - Definitions and Basic Properties, The Mean-Value Property, Poisson's Formula, Schwarz's Theorem, The Reflection Principle. A closer look at Harmonic Functions - Functions with Mean Value Property, Harnack's Principle.
%The Dirichlet's Problem - Subharmonic Functions, Solution of Dirichlet's Problem ( Proof of Dirichlet's Problem and Proofs of Lemma 1 and 2 excluded )
%(Chapter 4 : Section 6: 6.1 - 6.5, Chapter 6 : Section 3 : 3.1 - 3.2 , Section 4 : 4.1 - 4.2)
%Module 2:
%Power Series Expansions - Weierstrass's theorem, The Taylor Series, The Laurent Series Partial Fractions and Factorization - Partial Fractions, Infinite Products, Canonical Products, The Gamma Function. Entire Functions - Jensen's Formula, Hadamard's Theorem ( Hadamard's theorem - proof excluded)
%(Chapter 5 : Section 1 : 1.1 - 1.3, Section 2 : 2.1 - 2.4, Section 3 : 3.1 - 3.2 )
%Module 3:
%The Riemann Zeta Function - The Product Development, The Extension of $\zeta(s)$ to the Whole Plane, The Functional Equation, The Zeroes of the Zeta Function Normal Families - Normality and Compactness, Arzela's Theorem
%(Chapter 5 : Section 4 : 4.1 - 4.4, Section 5 : 5.2 - 5.3)
%Module 4:
%The Riemann Mapping Theorem - Statement and Proof, Boundary Behaviour, Use of the Reflection Principle
%The Weierstrass's Theory - The Weierstrass's $\rho$-function, The functions $\zeta(s)$ and $\sigma(z)$, The Differential Equation
%(Chapter 6 : Section 1: 1.1-1.3, Chapter 7 : Section 3 : 3.1 - 3.3)

%Module 1 - \cite{ahlfors} 4.6, 6.3, 6.4
%Module 2 - \cite{ahlfors} 5.1, 5.2, 5.3
%Module 3 - \cite{ahlfors} 5.4, 5.5
%Module 4 - \cite{ahlfors} 6.1, 7.3
%Missing - \cite{ahlfors} (6.2, 6.5, 7.1, 7.2, 8.1, 8.2, 8.3, 8.4)

%Module 1 - 4.6, 6.3, 6.4
{\Large Module 1}
\section{Harmonic Functions(4.6)}
\subsection{Definition and Basic properties(4.6.1)}
\begin{definition}[harmonic]
	A real-valued function $u(z)$ defined and single-valued in a region $\Omega$ is harmonic if it is continuous together with its partial derivatives of first two orders and satisfies Laplace's equation
	\[ \Delta u = \frac{\partial^2 u}{\partial x^2} + \frac{\partial^2 u}{\partial y^2} = 0 \]

	In polar form,
	\[ r \frac{\partial}{\partial r} \left( r \frac{\partial u}{\partial r}\right) + \frac{\partial^2 u}{\partial \theta^2} = 0 \]
\end{definition}
%Regularity conditions can be weakened. How ?
\begin{remark}
	The sum of two harmonic functions is harmonic. A constant multiple of a harmonic function is harmonic.\\

	From polar form, $\log r$ is harmonic and any harmonic function that depends only on $r$ must be of the form $a \log r + b$.  The argument $\theta$ is harmonic whenever it can be uniquely defined. \\

	If $u$ is harmonic in $\Omega$, then $f(z) = \frac{\partial u}{\partial x} - i \frac{\partial u}{\partial y}$ is analytic on $\Omega$.\\

	\[ f\ dz = f\ dx+i f\ dy = \left( \frac{\partial u}{\partial x} dx + \frac{\partial u}{\partial y}dy \right) + i \left( -\frac{\partial u}{\partial y}dx + \frac{\partial u}{\partial x}dy \right) \]
	The real part is the differential of $u$ and imaginary part is the conjugate differential of $du$.
	\[ f\ dz = du + i\ \underset{}{^\ast}du \]
\end{remark}

---to be continued---

\subsection{The Mean-value Property(4.6.2)}
\subsection{Poisson's Formula(4.6.3)}
\subsection{Schwarz's Theorem(4.6.4)}
\subsection{The Reflection Principle(4.6.5)}

\section{A Closer look at Harmonic Functions(6.3)}
\subsection{Functions with the Mean-value Property (6.3.1)}
\begin{definition}
	Let $u(z)$ be a real-valued continuous function in a region $\Omega$. Function $u$ satisfies mean-value property if
	\[ u(z_0) = \frac{1}{2\pi} \int_0^{2\pi} u(z_0+re^{i\theta})\ d\theta \]
	when the disk $|z-z_0| \le r$ is contained in $\Omega$.
\end{definition}
\begin{remark}
	Mean-value property implies maximum principle.
\end{remark}
\begin{theorem}
	A continuous function $u(z)$ satisfying the mean-value property is necessarily harmonic.
\end{theorem}
\begin{proof}
\end{proof}

\subsection{Harnack's Principle (6.3.2)}
\begin{theorem}[Harnack's Inequality]
	\[ \frac{\rho-r}{\rho+r} u(0) \le u(z) \le \frac{\rho+r}{\rho-r}u(0) \]
	where $u(z)$ is harmonic in $|z| = r < \rho$ and continuous is $|z| \le \rho$.
\end{theorem}
\begin{proof}
\end{proof}

\begin{theorem}[Harnack's Principle]
	Increasing sequence of harmonic functions on regions $\Omega_n$ converges to either $+\infty$ or a harmonic function $u$ uniformly on every compact subset of $\Omega$ where every point $z \in \Omega$ belongs to all except finitely many $\Omega_n$.
\end{theorem}
\begin{proof}
\end{proof}

\section{The Dirichlet Problem(6.4)}
\begin{definition}[Dirichlet Problem]
	Find a harmonic function $u(z)$ on a region $\Omega$ with given boundary values.
\end{definition}

\subsection{Subharmonic Functions (6.4.1)}
\begin{definition}[subharmonic]
	A continuous, real-valued function $v(z)$ is subharmonic in $\Omega$ if for any harmonic function $u(z)$ in $\Omega' \subset \Omega$, $v-u$ satisfies the maximum principle on $\Omega'$.
\end{definition}

\begin{remark}
	Harmonic functions are subharmonic.\\
	
	Sufficient condition for subharmonicity : If $v$ has a positive laplacian, then $v$ is subharmonic. The condition is not necessary as there exists subharmonic functions without partial dertivatives.
\end{remark}

\begin{theorem}
	A continuous function $v(z)$ is subharmonic in $\Omega$ if and only if 
	\[ v(z_0) \le \frac{1}{2\pi} \int_0^{2\pi} v(z_0+re^{i\theta})\ d\theta \]
	for every disk $|z-z_0| \le r$ contained in $\Omega$.
\end{theorem}
\begin{proof}
\end{proof}

\begin{remark}
	Elementary Properties of Subharmonic Functions,
	\begin{enumerate}
		\item If $v$ is subharmonic, then $kv$ is subharmonic for any constant $k \ge 0$.
		\item If $v_1,v_2$ are subharmonic, then $v_1+v_2$ is subharmonic.
		\item If $v_1,v_2$ are subharmonic, then $v=\max(v_1,v_2)$ is subharmonic.
		\item Let $v$ be subharmonic on $\Omega$. Let $\Delta$ be a disk whose closure is contained in $\Omega$ and $P_v$ be the Poisson integral formed with $v$ on the boundary of $\Delta$. Then $v'$ defined as $P_v$ on $\Delta$ and $v$ outside $\Delta$ is subharmonic.
	\end{enumerate}
\end{remark}

\begin{definition}[semicontinuous]
	A function $v(z)$ is upper semicontinuous at $z_0$ if $\displaystyle \lim_{z \to z_0} \sup v(z) \le v(z_0)$.
	A function $v(z)$ is lower semicontinuous at $z_0$ if $\displaystyle \lim_{z \to z_0} \inf v(z) \ge v(z_0)$.
\end{definition}

\begin{remark}
	Upper continuous function attains maximum on any compact set.
\end{remark}

\subsection{Solution of Dirichlet's Problem (6.4.2)}
\begin{theorem}
\end{theorem}
\begin{proof}
\end{proof}
\pagebreak
%Module 2 - 5.1, 5.2, 5.3
{\Large Module 2}
\section{Power Series Expansions(5.1)}
\subsection{Weierstrass's Theorem(5.1.1) pp.175}
\begin{important}
	Weierstrass`s theorem gives a sufficient condition for convergence to preserve analyticity, that is uniform convergence on every compact subset of the region. 
	And Hurwitz`s theorem characterises such analytic, limit functions.
\end{important}
\begin{theorem}[Weierstrass]
	Suppose $f_n(z)$ is analytic in the region $\Omega_n$, and the sequence $\sequence{f_n(z)}$ converges to $f(z)$ in a region $\Omega$ uniformly on every compact subset of $\Omega$. Then the limit function $f(z)$ is analytic in $\Omega$. Moreover, $f_n'(z)$ converges to $f'(z)$ on every compact subset of $\Omega$.
\end{theorem}
\begin{proof}
\end{proof}

\begin{remark}
	A series with analytic terms
	\[ f(z) = f_1(z) + f_2(z) + \dots + f_n(z) + \dots \]
	converges uniformly on every compact subset of a region $\Omega$, then the sum $f(z)$ is analytic in $\Omega$. And the series can be differentiated term by term.
\end{remark}
\begin{proof}
\end{proof}
\begin{remark}
	Uniform convergence on the boundary of a compact set $A$ implies unform convergence on $A$.
\end{remark}
\begin{proof}
	Hint : maximum principle
\end{proof}

\begin{remark}
	If $f_n(z)$ are analytic in $|z|<1$ and convergence is uniform on each circle $C : |z| = r_m$ where $r_m \to 1$ as $m \to \infty$, then limit function $f(z)$ is analytic.
\end{remark}
\begin{proof}
\end{proof}

\begin{theorem}[Hurwitz]
	If the functions $f_n(z)$ are analytic in a region $\Omega$, $f_n(z) \ne 0, \forall z \in \Omega$ and $f_n(z)$ converges to $f(z)$ uniformly on every compact subset of $\Omega$, then $f(z)$ is either identitally zero in $\Omega$ or never assumes zero in $\Omega$.
\end{theorem}
\begin{proof}
\end{proof}
	
\subsection{The Taylor Series (5.1.2) pp 179}
\begin{important}
	Every analytic function has a convergent Taylor series representation.
\end{important}
\begin{theorem}
	If $f(z)$ is analytic in a region $\Omega$ containing $z_0$, then the representation
	\begin{equation}
		f(z) = f(z_0) + \frac{f'(z_0)}{1!}(z-z_0) + \dots + \frac{f^{(n)}(z_0)}{n!}(z-z_0)^n + \dots
	\end{equation}
	is valid in the largest open disk of center $z_0$ contained in $\Omega$.
\end{theorem}
\begin{proof}
\end{proof}

Taylor series representation for a few analytic functions,
\begin{enumerate}
	\item $e^z = 1 + z + \frac{z^2}{2!} + \dots + \frac{z^n}{n!} + \dots$
	\item $\cos z = 1 - \frac{z^2}{2!} + \frac{z^4}{4!} + \dots + $
	\item $\sin z = z - \frac{z^3}{3!} + \frac{z^5}{5!} + \dots + $
	\item $(1+z)^\mu = 1 + \mu z + \binom{\mu}{2} z^2 + \dots$ where $\binom{\mu}{n} = \frac{\mu(\mu-1)(\mu-2)\dots(\mu-n+1)}{1 \cdot 2 \cdot 3 \cdots n}$
	\item $\log (1+z) = z - \frac{z^2}{2} + \frac{z^3}{3} - \frac{z^4}{4} + \dots$
	\item $\arctan z = z - \frac{z^3}{3} + \frac{z^5}{5} - \frac{z^7}{7} + \dots$
	\item $\frac{1}{\sqrt{1-z^2}} = 1 + \frac{1}{2}z^2 + \frac{1}{2}\frac{3}{4} z^4 + \frac{1}{2} \frac{3}{4} \frac{5}{6} z^6 + \dots$ 
	\item $\arcsin z = z + \frac{1}{2} \frac{z^3}{3} + \frac{1}{2} \frac{3}{4} \frac{z^5}{5} + \frac{1}{2} \frac{3}{4} \frac{5}{6} \frac{z^7}{7} + \dots$
	\item $\tan z = z + \frac{z^3}{3}+\frac{2}{15}z^5+\dots$
	\item $\log \frac{\sin z}{z} = $
\end{enumerate}

\subsubsection*{}
The procedure to find Taylor representation of an analytic function,
\begin{enumerate}
	\item Choose a well-defined branch so that we have a single-valued function.
	\item Choose a center $z_0$ which is not a branch point about which the series is to be developed.
\end{enumerate}

\subsubsection*{}
Computational techniques to obtain Taylor series representation,
\begin{enumerate}
	\item Comparing Real and Imaginary Parts of a Taylor Series
	\item Differentiating/Intergrating a Taylor Series
	\item $f(z)g(z) = P_n(z)Q_n(z) + [z^{n+1}]$ \\ where $f(z) = P_n(z) + [z^{n+1}]$ and $g(z) = Q_n(z) + [z^{n+1}]$.
	\item $f(z)/g(z) = R_n(z) + [z^{n+1}]$\\ such that $P_n(z) + [z^{n+1}] = Q_n(z)R_n(z) + [z^{n+1}]$.
	\item $f(g(z)) = P_n(Q_n(z)) + [z^{n+1}]$
	\item $g^{-1}(w) = P_n(w) + [w^{n+1}]$ \\ such that $P_n(Q_n(z)) + [z^{n+1}] = z$.
\end{enumerate}

\begin{definition}[Legendre Polynomials]
	Legendre polynomials in $\alpha$, $P_n(\alpha)$ are the coefficients of the Taylor series of $(1-2\alpha z + z^2)^{-\frac{1}{2}}$ about origin.
	\begin{equation}
		(1-2\alpha z + z^2)^{-\frac{1}{2}} = 1 + P_1(\alpha)z + P_2(\alpha)z^2+\dots
	\end{equation}
\end{definition}

\subsection{The Laurent Series (5.1.3) pp. 184}
\begin{remark}
	If the function $f(z)$ is analytic in the annulus $R_1 < |z-a| < R_2$, then the Laurent Series representation
	\begin{equation}
		f(z) = \sum_{n = -\infty}^\infty A_n (z-a)^n
	\end{equation}
	is valid for any $z$ in the annulus.
\end{remark}
\begin{proof}
\end{proof}

\begin{definition}[Schwarzian Derivative]
	Schwarzian Derivative of $f$ is defined as,
	\[ \{f,z\} = \frac{f^{\prime\prime\prime}(z)}{f'(z)} - \frac{3}{2} \left(\frac{f^{\prime\prime}(z)}{f'(z)}\right)^2 \]
	The leading term in the Laurent Series representation of Schwarzian derivative should be of some importance.
\end{definition}

\begin{remark}
	Laurent series representation,
	\[ (e^2-1)^{-1} = \frac{1}{z} - \frac{1}{2} + \sum_{k=1}^\infty (-1)^{k-1}\frac{B_k}{(2k)!} z^{2k-1} \]
	where $B_k$ are the Bernoulli numbers.
\end{remark}
\section{Partial Fractions and Factorisation(5.2)}
\subsection{Partial Fractions (5.2.1) pp. 187}
\begin{theorem}[Mittag-Leffler]
	Let 
\end{theorem}
\section{Entire Functions(5.3)}
\pagebreak
%Module 3 - 5.4, 5.5
{\Large Module 3}
\section{The Riemann Zeta Function(5.4)}
\section{Normal Families(5.5)}
%Module 4 - 6.1, 7.3
\pagebreak
{\Large Module 4}
\section{The Riemann Mapping Theorem(6.1)}
\section{The Weierstrass Theory(7.3)}
