%Text Books : \cite{balakrishnan}
%Module I:
%Introduction, Basic concepts. Sub graphs. Degrees of vertices. Paths and Connectedness, Automorphism of a simple graph, line graphs, Operations on graphs, Graph Products.
%Directed Graphs : Introduction, basic concepts and tournaments.
%(Chapter 1 Sections 1.1 – 1.7( Upto 1.7.3 including ) 1.8, 1.9)
%(Chapter 1 Sections 2.1, 2.2, 2.3) (20Hours)
%Module II:
%Connectivity : Introduction, Vertex cuts and edge cuts, connectivity and edge connectivity, blocks, Cyclical edge Connectivity of a graph.
%Trees; Introduction, Definition, characterization and simple properties, centres and cancroids, counting the number of spanning trees, Cayley’s formula.
%Applications
%(Chapter 3 Sections 3.1, 3.2 , 3.3, 3.4 and 3.5 )
%(Chapter 4 Sections4.1, 4.2, 4.3, 4.4 (Up to 4.4.3 including ) and 4.5, 4.7) (25Hours)
%Module III:
%Eulerian and Hamiltonian Graphs: Introduction, Eulereian graphs,
%Hamiltonian Graphs, Hamiltonian around’ the world’ game
%Graph Colorings: Introduction, Vertex Colorings, Applications of Graph Coloring, Critical Graphs, Brooks’ Theorem
%(Chapter 6 Sections 6.1, 6.2 and 6.3 )
%(Chapter 7 Sections 7.1, 7.2 and7.3(Up to 7.3.1 including ) (20Hours)
%Module IV:
%Planarity: Introduction, Planar and Nonplanar Graphs, Euler Formula and Its Consequences, K5 and K 3,3 are Nonplanar Graphs, Dual of a Plane Graph, The Four-Color Theorem and the Heawood Five-Color Theorem .
%Spectral Properties of Graphs: Introduction, The Spectrum of a Graph, Spectrum of the Complete Graph Kn, Spectrum of the Cycle Cn,
%(Chapter 8 Sections 8.1, 8.2 , 8.3, 8.4, 8.5 and 8.6 )
%(Chapter 11 Sections 11.1, 11.2 , 11.3 and 11.4) (25Hours)

%Module 1 - \cite{balakrishnan} 1, 2
%Module 2 - \cite{balakrishnan} 3, 4
%Module 3 - \cite{balakrishnan} 6, 7
%Module 4 - \cite{balakrishnan} 8, 11
%Missing - \cite{balakrishnan} 5, 9, 10, ?

{\Large Module 1}
%\chapter{Basic Results}
\section{Basic Results}
\subsection{Introduction}
\cite{balakrishnan}
\subsection{Basic Concepts}
\begin{definition}
	A \textbf{graph} is an ordered triple $G = (V,E,I)$ where $V$ is a nonempty set of vertices, $E$ is a set of edges and $I$ is an incidence map which associates each edge with an unordered pair of vertices.
\end{definition}

\begin{description}
	\item[end vertices] Let $e$ be an edge with $I_G(e) = \{ u,v \}$. Then $u,v$ are the end vertices of $e$. We may write $e = uv$.
	\item[incident] An edge $e =uv $ said to be incident on both vertices $u$ and $v$. Then vertices $u$ and $v$ are incident on edge $e$ as well. 
	\item[parallel edges] are those edges which have same pair of end vertices.
	\item[loop] is an edge whose both end vertices are the same.
	\item[neighbour] Vertex $u$ is neighbour of vertex $v$ if $uv$ is an edge of the graph.
	\item[Open neighbourhood] $N(u)$ is the set of all neighbours of the vertex $u$.
	\item[Closed neighbourhood] $N[u] = N(u) \cup \{ u \}$.
	\item[simple] graph does not have any parallel edges or loops.
	\item[adjacent] Two vertices $u,v$ are adjacent if both $u,v$ are incident on an edge, say $uv$. Two edges are adjacent if they have a common end vertex.
\end{description}
\begin{definition}
	A graph is \textbf{finite} if both vertex set $V$ and edge set $E$ are finite. A finite graph, $G$ order, $n(G) = |V(G)|$ and size, $m(G) = |E(G)|$. Or simply $n = |V(G)|$ and $m = |E(G)|$.
\end{definition}

\begin{definition}
	A graph $G$ is \textbf{labeled} if its vertices are distinguished from one another by means of distinct labels.
\end{definition}

\begin{definition}
	Two graphs $G$ and $H$ are \textbf{isomorphic} if there exists a pair $(\phi,\theta)$ where $\phi : V(G) \to V(H)$ and $\theta : E(G) \to E(H)$ are bijections such that $I_G(e) = \{ u,v \} \iff I_H(\theta(e)) = \{ \phi(u),\phi(v) \}$.\\

	Two simple graphs $G$ and $H$ are \textbf{isomorphic} if there exists a bijection $\phi : V(G) \to V(H)$ which induced another bijection $\theta : E(G) \to E(H)$ such that $I_G(e) = \{ u,v \} \iff I_H(\theta(e)) = \{ \phi(u),\phi(v) \}$.
\end{definition}

\begin{exercise}
	Let $G$, $H$ be simple graph and let $\phi : V(G) \to V(H)$ be a bijection such that $uv \in E(G) \implies \phi(u)\phi(v) \in E(H)$. Show that $\phi$ is not an isomorphism.
\end{exercise}
	Let $G,H$ be simple graph with bijection $\phi : V(G) \to V(H)$.
	If there exists two vertices $u,v$ which are non-adjacent in $G$, but $\phi(u),\phi(v)$ are adjacent in $H$.
	Then $\phi$ is not an isomorphism.

\begin{definition}
	A simple graph is \textbf{complete} if every pair of distinct vertices are adjacent.
	A complete graph with $n$ vertices is denoted by $K_n$.
	Then $m(K_n) = \binom{n}{2} = \frac{n(n-1)}{2}$.
	A \textbf{totally} disconnected graph has no edges.
	Thus, $0 \le m \le \binom{n}{2}$.
	A graph $G$ is trivial if it has only one vertex and no edges.
\end{definition}

\begin{definition}
	A graph $G$ is \textbf{bipartite} if its vertex set can be partitioned into two nonempty sets $X$ and $Y$ such that every end of $G$ has one end in $X$ and the other end in $Y$. We write $G(X,Y)$ is bipartite with partition $X,Y$.
	A simple, bipartite graph $G(X,Y)$ is \textbf{complete bipartite} if every vertex in $X$ is adjacent to every vertex in $Y$. Let $G(X,Y)$ be a complete bipartite graph with $|X| = p$ and $|Y|=q$, then we write $K_{p,q}$. A graph of the form $K_{1,q}$ is called a \textbf{star}.
\end{definition}

\begin{remark}
	Let $G$ be a complete bipartite graph, $K_{p,q}$.
	Then it has order $n = p+q$ and size $m = pq$.
\end{remark}

\begin{definition}
Let $G$ be a graph. The complement $G^c$ of graph $G$ is the graph with same vertex set. Two vertices $u,v$ of $G^c$ are adjacent if and only if $u,v$ are non-adjacent in $G$.
	$$ V(G^c) = V(G) \quad \& \quad uv \in E(G^c) \iff uv \notin E(G) $$
\end{definition}
\begin{remark}
	We have, $(G^c)^c = G$, since ${uv \in G \iff uv \notin G^c \iff uv \in (G^c)^c}$ and $V(G) = V(G^c) = V((G^c)^c)$.
	Let $G$ be a graph of order $n$, then $E(G) + E(G^c) = E(K_n)$ as each edge of $K_n$ is either an edge of $G$ or an edge of $G^c$.
\end{remark}

\begin{definition}
	A simple graph is self-complementary if $G \cong G^c$.
\end{definition}
\begin{remark}
	The order of a self complementary graph $G$ of order $n$ is $n(n-1)/4$ since $m(G) = m(G^c)$ and $m(G) + m(G^c) = m(K_n)$.
\end{remark}

\subsection{Subgraphs}
\begin{description}
	\item[subgraph] A graph $H$ is a subgraph of $G$ if $V(H) \subset V(G)$, $E(H) \subset E(G)$ and $I_H$ is $I_G$ restricted to $E(H)$. Then $G$ is a \textbf{supergraph} of $H$.
	\item[induced subgraph] Let $G$ be a graph and $S$ be a subset of the vertex set of $G$. The subgraph induced by $S$, $G[S]$ has vertex set $S$ and two vertices are adjacent only if they are adjacent in $G$.
	\item[edge induced subgraph] Let $G$ be a graph and $S$ be a subset of the edge set of $G$. The subgraph induced by the edge set $S$ is denoted by $G[S]$. Vertex $u$ is a vertex of $G[S]$ only if $S$ has an edge incident on it.
	\item[spanning subgraph] Let $H$ be a subgraph of graph $G$. If $V(H) = V(G)$, then $H$ is a spanning subgraph.
	\item[clique] is a subgraph which is complete. A clique is maximal if it is not contained in another clique.
\end{description}

\begin{remark}
	If an edge $e$ is deleted from a graph $G$, then vertex set remains the same. The graph $G-\{e\}$ is a spanning subgraph of $G$.
	If a vertex $u$ is deleted from a graph $G$, then all the edge incident on $u$ are also deleted. The graph $G-\{u\}$ is an induced subgraph of $G$.
\end{remark}

\subsection{Degree of Vertices}
\begin{definition}
	Let $G$ be a graph and $u$ be a vertex of $G$.
	The \textbf{degree} of $u$ is the number of edges incident on it with multiplicities. That is, every loop incident on $u$ is counted twice.
\end{definition}
\begin{remark}
	In a simple graph, the degree of a vertex is the cardinality of its open neighbourhood.
	$\deg_G(u) = |N_G(u)|$.
\end{remark}

\begin{definition}
	A graph $G$ is $\mathbf{k}$-\textbf{regular} if every vertex is of degree $k$. Graph $G$ is \textbf{regular} if it is $k$-regular for some $k$. \textbf{Cubic} graphs are the $3$-regular graphs.
\end{definition}
\begin{remark}
	Complete graph $K_{n+1}$ are $n$-regular.
	And complete graphs are the smallest regular graphs.
	$K_4$ is cubic.
	Petersen graph is cubic.
\end{remark}
\begin{figure}
\centering
\scalebox{0.9}{
\begin{tikzpicture}
	\node (u){};
	\tikzstyle every node=[draw,circle]
	\draw (u) ++(90:1.5) node (b1)[label=above right:$v_6$]{};
	\draw (u) ++(162:1.5) node (b2)[label=above:$v_7$]{};
	\draw (u) ++(234:1.5) node (b3)[label=left:$v_8$]{};
	\draw (u) ++(306:1.5) node (b4)[label=right:$v_9$]{};
	\draw (u) ++(18:1.5) node (b5)[label=above:$v_{10}$]{};

	\draw (u) ++(90:3) node (a1)[label=above right:$v_1$]{};
	\draw (u) ++(162:3) node (a2)[label=above:$v_2$]{};
	\draw (u) ++(234:3) node (a3)[label=left:$v_3$]{};
	\draw (u) ++(306:3) node (a4)[label=right:$v_4$]{};
	\draw (u) ++(18:3) node (a5)[label=above:$v_5$]{};

	\draw (b1)--(b3)--(b5)--(b2)--(b4)--(b1);
	\draw (a1)--(a2)--(a3)--(a4)--(a5)--(a1);

	\draw (a1)--(b1);
	\draw (a2)--(b2);
	\draw (a3)--(b3);
	\draw (a4)--(b4);
	\draw (a5)--(b5);
\end{tikzpicture}
}
\end{figure}
\begin{remark}
	The complement of a regular graph is also regular. If $G$ is $k$-regular, then $G^c$ is $r$-regular where $k+r = n-1$.
\end{remark}
\begin{description}
	\item[1-factor] is a spanning, $1$-regular subgraph.
	\item[isolated vertex] is a vertex with degree $0$.
	\item[pendent vertex] is a vertex with degree $1$.
	\item[pendent edge] is the only edge incident on a pendent vertex.
\end{description}

\begin{theorem}[Euler]
	The degree sum of a graph is twice its size.
\end{theorem}
\begin{proof}
	Every edge $e= uv$ contributes $1$ to the degree of both the end vertices $u$ and $v$. Thus every edge contributes $2$ to the degree sum. There are $m$ edges, thus degree sum is $2m$.
\end{proof}

\begin{corollary}
	In a grpah $G$, the number of vertices of odd degree is even.
\end{corollary}
\begin{proof}
	Let $G$ be a graph of order $n$.
	Let $V_1,V_2$ be the set of vertices of even,odd degree respectively.
	Then, $\sum d_i = \sum_{v \in V_1} \deg(v) + \sum_{v \in V_2} \deg(v)$.
	The degree sum is even and the first sum on RHS is also even. Thus the second sum should be even. Since, each term in the second sum is an odd integer, there are even number of terms in the second sum. In other words, there are even number of vertices with an odd degree.
\end{proof}
\begin{exercise} 
	If $G \overset{\phi}{\cong} H$, then each pair $u,\phi(u)$ have the same degree.
\end{exercise}
Let $G,H$ be isomorphic graphs.
Let $u$ be a vertex of $G$.
A vertex $v$ is adjacent of $u$ in $G$ if and only if $\phi(v)$ is adjacent to $\phi(u)$ in $H$. Thus, $\deg_G(u) = \deg_H(\phi(u))$.

\begin{remark}
	Clearly, a graph isomorphism preserves adjacency, degree of vertices and neighbourhoods, etc.
	$N_H(\phi(u)) = \{ \phi(v) : v \in N_G(u) \}$.
\end{remark}

\begin{exercise}
	Let $d : d_1,d_2,\dots,d_n$ be the degree sequence of a graph $G$.
	Let $r$ be a positive integer, then $\sum d_i^r$ is even.
\end{exercise}
Let $G$ be a graph of order $n$.
Let $V_1,V_2$ be the set of vertices of even and odd degree respectively.
Clearly, $\sum d_i^r = \sum_{v \in V_1} d_i^r + \sum_{v \in V_2} d_i^r$.
Once again, the first sum on RHS is even as $d_i^r$ is even when $d_i$ is even. And the second sum is odd as $d_i^r$ is odd when $d_i$ is odd. By Euler's theorem, there are an even number of such odd terms. Thus, the second sum on RHS is also even. Therefore, $\sum d_i^r$ is even.

\begin{definition}
	A sequence of nonnegative integers $d : d_1,d_2,\dots,d_n$ is \textbf{graphical} if there exists a \textit{simple} graph with degree sequence $d$.
\end{definition}

\begin{example}
	The sequence $d : 7,6,3,3,2,1,1,1$ is nongraphical. Let $v_0,v_1$ be the vertics with degree $7,6$ respectively. Then for $d$ to be graphical, there should be at least another $4$ vertices with degree greater than or equal to $2$.
	This is not the case, therefore $d$ is not graphical.
\end{example}

\begin{exercise}
	Let $d : d_1,d_2,\dots,d_n$ is a sequence of nonnegative integers with $\sum d_i$ even. Show that there exists a non-simple graph with $d$ as its degree sequence.
\end{exercise}
	Let $d : d_1,d_2,\dots,d_n$ be a sequence of nonnegative integers with $\sum d_i$ even. Then there are even number of odd integers (if any).\\

	Step 1 : Draw a vertex $u_i$ for each term $d_i$.
	If a pair $(d_i,d_j)$ is odd\footnote{If you want maximal simple graph, you may add an edge $u_iu_j$(if it not already there) and subtract $1$ from both $d_i$ and $d_j$, provided both $d_i$ and $d_j$ are nonzero.}, then draw an edge $u_iu_j$ and subtract $1$ from both $d_i$ and $d_j$.\\

	Step 2 : If $d_i = 2k$. Draw $k$ loops on $u_i$.\\

\begin{challenge}
	Draw maximal connected graph for a degree sequence ?
\end{challenge}

\begin{application}
	In any group of $n$ persons ($n \ge 2$), there are at least two with same number of friends.
\end{application}
\begin{proof}
	Suppose group of $n$ persons is modelled by a graph with $n$ vertices and two vertices are adjacent if the respective persons are friends. It is assumed that friendship is mutual, otherwise it is not considered. Suppose all of them have different number of friends. Then every vertex in the graph has different degree.\\

	The possible degree of a vertex in a graph of order $n$ is $0,1,\dots,(n-1)$. Since all $n$ vertices have different degree and we have only $n$ options. There exists a vertex of degree $j$ for each $0 \le j < n$. This leads to a contradiction.\\

	Let $v_1,v_n$ be the vertices with degree $0,(n-1)$. Then $v_n$ is adjacent to every other vertex and $v_1$ is non-adjacent to every other vertex which is not possible. Thus, at least two vertices should have same degree. Therefore, in a group of $n$ persons, at least two of them have same number of friends.
\end{proof}

\begin{exercise}
	Let $G$ be a graph in which every vertex is of degree $k$ or $k+1$. Then the number of vertices with degree $k$ is $(k+1)n-2m$.
\end{exercise}
\begin{proof}
	Let $G$ be a graph of order $n$ and size $m$.
	Let $x$ be the number of vertices with degree $k$.
	Then there are $n-x$ vertices with degree $k+1$.
	By Euler's theorem, $ xk + (n-x)(k+1) = 2m \implies x = (k+1)n-2m$.
\end{proof}
	
\subsection{Paths and Connectedness}
\begin{definition}
	Let $G$ be a graph. A \textbf{walk} on $G$ is an alternating finite sequence of vertices and edges which starts and ends at some vertices, say $W : v_0 e_0 v_1 e_1 \dots e_k v_k$.
	The vertex $v_0$ is the \textbf{origin} and $v_k$ is the \textbf{terminus} of the walk $W$.
\end{definition}
\begin{description}
	\item[closed walk] A walk is \textbf{closed} if its origin and terminus are identical. Otherwise the walk is \textbf{open}.
	\item[trail] is a walk in which edges are distinct.
	\item[path] is a trail in which vertices are distinct.
	\item[cycle] is a closed trail in which vertices are distinct.
\end{description}

\begin{remark}
	A walk of length zero is a single vertex. And this walk is called a \textbf{trivial path}.
	Let $P : v_0,e_1,v_1,e_2,v_2,\dots,v_{k-1},e_k,v_k$ be a path in $G$. Then the \textbf{inverse of the path} is given by, $P^{-1} : v_k,e_k,v_{k-1},\dots,v_1,e_1,v_0$.
	And $v_j,e_j,v_{j+1},\dots,v_{k-1},e_{k-1},v_k$ is the $v_j-v_k$ \textbf{section} of the path $P$.
\end{remark}

\begin{definition}
	Let $G$ be a graph. Then connectedness is an equivalence relation on $V(G)$. Let $V_1,V_2,\dots,V_\omega$ be the equivalence classes. Then the induced subgraphs $G[V_1], G[V_2], \dots, G[V_\omega]$ are the \textbf{components} of $G$.
\end{definition}

\begin{remark}
	For a connected graph, $\omega = 1$. And for disconnected graphs $\omega \ge 2$. And components are maximal connected subgraphs of $G$. The number of components of $G$ is denoted by $\omega(G)$.
\end{remark}

\begin{definition}
	Let $d : V(G) \to V(G),\ d(u,v)$ is the length of the shortest $u-v$ path. If there is no $u-v$ path, then $d(u,v) = \infty$.
\end{definition}
\begin{exercise}
	Let $G$ be a simple graph.
	The vertex set $V(G)$ together with distance function $d(u,v)$, length of shortest $u-v$ path is a metric space.
\end{exercise}
\begin{proof}
	Let $G$ be a simple graph with vertex set $V(G)$ and $d : V(G) \to V(G),\ d(u,v)$ is the length of a shortest $u-v$ path.
\begin{enumerate}
	\item Every path has non-negative length. Thus, $\forall u,v \in V(G),\ d(u,v) \ge 0$.\\
	And $d(u,u) = 0$ since trivial path has zero length.
	\item Suppose $u,v$ are connected in $G$. Otherwise $d(u,v) = d(v,u) = \infty$.\\
	Let $P$ be a shortest $u-v$ path. Then $P^{-1}$ is a shortest $v-u$ path.
	Suppose there exists a $v-u$ path, $Q$ which is shorter than $P^{-1}$. Then $Q^{-1}$ is shorter than $P$ which is a contradiction.
	Thus, $d(u,v) = d(v,u)$.
	\item Suppose $d(u,v) < \infty$. Otherwise the result is trivial.\\
	Let $P,Q$ be shortest $u-w$ path, $w-v$ path in $G$. Then $P+Q$ is a $u-v$ walk.
	Thus, there exists a $u-v$ path\footnote{Every $u-v$ walk, contains a $u-v$ path which can be obtained by subsequently replacing each $w-w$ section of the walk with $w$.} of length less than or equal to the length of $P+Q$.
	Therefore, $d(u,v) \le d(u,w) + d(w,v)$.
\end{enumerate}
\end{proof}

\begin{proposition}
	If $G$ is a simple graph with $\delta(G) \ge \frac{n-1}{2}$, then $G$ is connected.
\end{proposition}
\begin{proof}
	Let $G$ be a graph of order $n$.
	Suppose $G$ is not connected. Then $G$ has at least two components, say $G_1$ and $G_2$.
	Let $v \in V(G_1)$. Since $\deg(v) \ge \frac{n-1}{2}$, there are at least $\frac{n-1}{2}$ other vertices in $G_1$.
	Thus, each component of $G$ has at least $\frac{n+1}{2}$ vertices.
	And $G$ has at least $\frac{n+1}{2} + \frac{n+1}{2}$ vertices which is a contradiction.
	Therefore, $G$ is connected.
\end{proof}

\begin{exercise}
	A simple graph with $\delta(G) \ge \frac{n-2}{2}$ is not necessarily connected.
\end{exercise}
\begin{proof}
	Let $G$ be a simple graph with $\delta(G) = \frac{n-2}{2}$ where $n$ is even. Then $G$ is not necessarily connected. For example, $2K_2$ has $\delta = 1$ and is disconnected.
\end{proof}

\begin{exercise}
	In a group of six people, there must be three people who are mutually acquainted or three people who are mutually nonacquainted.
\end{exercise}
\begin{proof}
	Suppose the result is not true.
	Then every person $u$ has at least one friend. Suppose $u$ has no friends. Then $(v,w,x)$ are mutual friends. Otherwise, either $(u,v,w)$, $(u,v,x)$ or $(u,w,x)$ are mutually non-acquainted.\\
	
\begin{figure}[hbt]
\centering
\scalebox{0.9}{
\begin{tikzpicture}
	\tikzstyle every node=[draw,circle]
	\node (u)[label=above:$u$]{};
	\node (v)[above right=1cm of u,label=above:$v$]{};
	\node (w)[below right=1cm of u,label=below:$w$]{};
	\node (x)[right=1cm of v,label=above:$x$]{};
	\node (y)[right=1cm of w,label=below:$y$]{};
	\node (z)[below right=1cm of x,label=above:$z$]{};
	\draw[dotted] (w)--(u)--(v);
	\draw[dotted] (x)--(u)--(y);
	\draw[dotted] (u)--(z);
	\draw[dashed,red] (v)--(w)--(z)--(v);
\end{tikzpicture}
}
\caption{Minimum Degree $\delta(G) > 0$}
\end{figure}
	There exists a person $u$ with at least two friends. Suppose every person has exactly one friend, say $(u,v), (w,x), (y,z)$ are friends. Then $(u,w,y)$ are mutually non-acquainted.\\

\begin{figure}[hbt]
\centering
\scalebox{0.9}{
\begin{tikzpicture}
	\tikzstyle every node=[draw,circle]
	\node (u)[label=above:$u$]{};
	\node (v)[above right=1cm of u,label=above:$v$]{};
	\node (w)[below right=1cm of u,label=below:$w$]{};
	\node (x)[right=1cm of v,label=above:$x$]{};
	\node (y)[right=1cm of w,label=below:$y$]{};
	\node (z)[below right=1cm of x,label=above:$z$]{};
	\draw (u)--(v);
	\draw (w)--(x);
	\draw (y)--(z);
	\draw[dotted,red] (u)--(w)--(y)--(u);
\end{tikzpicture}
}
\caption{Maximum degree $\Delta(G) > 1$}
\end{figure}
	Suppose $u$ has at least two friends, say $v,w$. Then $v,w$ are not friends, otherwise $(u,v,w)$ are mutual friends.
	Every other person $x,y$ and $z$ is friend of either $v$ or $w$. Suppose $x$ is not a friend of both $v$ and $w$. Then $(v,w,x)$ are mutually non-acquainted. Suppose $x$ is a friend of $v$ or $w$ or both. Then $u$ is not a friend of $x$. Otherwise either $(u,v,x)$ or $(u,w,x)$ are mutual friends.\\

\begin{figure}[hbt]
\centering
\scalebox{0.9}{
\begin{tikzpicture}
	\tikzstyle every node=[draw,circle]
	\node (u)[label=above:$u$]{};
	\node (v)[above right=1cm of u,label=above:$v$]{};
	\node (w)[below right=1cm of u,label=below:$w$]{};
	\node (x)[right=1cm of v,label=above:$x$]{};
	\node (y)[right=1cm of w,label=below:$y$]{};
	\node (z)[below right=1cm of x,label=above:$z$]{};
	\draw (v)--(u)--(w);
	\draw[dotted,red] (v)--(x)--(w)--(v);
\end{tikzpicture}
\hspace{2cm}
\begin{tikzpicture}
	\tikzstyle every node=[draw,circle]
	\node (u)[label=above:$u$]{};
	\node (v)[above right=1cm of u,label=above:$v$]{};
	\node (w)[below right=1cm of u,label=below:$w$]{};
	\node (x)[right=1cm of v,label=above:$x$]{};
	\node (y)[right=1cm of w,label=below:$y$]{};
	\node (z)[below right=1cm of x,label=above:$z$]{};
	\draw (v)--(u)--(w);
	\draw[dotted] (v)--(w);
	\draw[dashed,blue] (v)--(x);
	\draw[dashed,blue] (w)--(x);
	\draw[dotted] (u)--(x);
	\draw[dotted] (u)--(y);
	\draw[dotted] (u)--(z);
\end{tikzpicture}
}
\caption{$\deg_G(u) = 2$ or $\Delta(G) = 2$}
\end{figure}
	Similarly, $u$ is not a friend of $y$ and $z$. Thus, $\deg(u) = 2$. And $\Delta(G) = 2$.\\

	Suppose $(x,y,z)$ are not mutual friends. Then either $(u,x,y), (u,y,z)$, or $(u,z,x)$ are mutually non-acquainted which is a contradiction.
\begin{figure}[hbt]
\centering
\scalebox{0.9}{
\begin{tikzpicture}
	\tikzstyle every node=[draw,circle]
	\node (u)[label=above:$u$]{};
	\node (v)[above right=1cm of u,label=above:$v$]{};
	\node (w)[below right=1cm of u,label=below:$w$]{};
	\node (x)[right=1cm of v,label=above:$x$]{};
	\node (y)[right=1cm of w,label=below:$y$]{};
	\node (z)[below right=1cm of x,label=above:$z$]{};
	\draw (v)--(u)--(w);
	\draw[dotted] (v)--(w);
%	\draw[blue] (v)--(x);
	\draw[dotted,blue] (u)--(x);
	\draw[dotted,blue] (u)--(y);
	\draw[dotted,blue] (u)--(z);
	\draw[dashed,red] (x)--(y)--(z)--(x);
\end{tikzpicture}
}
\end{figure}
\end{proof}

\begin{theorem}
	If a simple graph $G$ is not connected, then $G^c$ is connected.
\end{theorem}
\begin{proof}
	Let $G$ be a disconnected graph with at least two components $G_1,G_2$.
	Let $u,v \in V(G)$. It is enough to prove that there exists a $u-v$ path in $G^c$.\\

	Case 1 : Two vertices $u,v$ belong to different components of $G$.
	Then $u,v$ are non-adjacent in $G$. Thus, $u,v$ are adjacent in $G^c$. Therefore, there exists a $u-v$ path in $G^c$.\\

	Case 2 : Suppose $u,v$ belongs to the same component of $G$. Let $w$ be a vertex from another component of $G$. Then $u,w,v$ is a $u-v$ path in $G^c$.\\

	Therefore, there exists a $u-v$ path between any pair of vertices in $G^c$.
\end{proof}

\begin{corollary}
	If a graph is self-complementary, then it is connected.
\end{corollary}
\begin{proof}
	Suppose $G$ is disconnected. Then its complement is connected. Therefore, $G$ is not self-complementary.
\end{proof}

\begin{exercise}
	Let $G$ be a self-complementary graph.\\ Then $n(G) \cong 0 \text{ or } 1 \pmod{4}$.
\end{exercise}
\begin{proof}
	Let $G$ be a self-complementary graph of order $n$.
	Then $m(G) = m(G^c)$ since $G \cong G^c$.
	And $m(G)+m(G^c)= m(K_n)$ since every edge in $K_n$ is either an edge of $G$ or an edge of $G^c$.
	We know that, $m(K_n) = n(n-1)/2$.
	Therefore, $m(G) = n(n-1)/4$.
	Since $n,n-1$ are consecutive, at most one of them is even.
	Therefore, either $n \cong 0 \pmod{4}$ or $n-1 \cong 0 \pmod{4}$.
\end{proof}

\begin{exercise}
	Let $G$ be a self-complementary graph.
	If $G$ has one pendent vertex, then it has another pendent vertex.
\end{exercise}
\begin{proof}
	Let $G$ be a self-complementary graph with isomorphism $\phi$ from $V(G)$ to $V(G^c)$.
	Let $u$ be a pendent vertex of $G$.
	Then $\phi(u)$ is also a pendent vertex of $G$.
	Suppose $u = \phi(u)$. Then, $n-1 = \deg(u) + \deg(\phi(u)) = 2$. However, no graph of order $3$ is self-complementary.
	Therefore, $u \ne \phi(u)$.
\end{proof}

\begin{exercise}
	Let $G$ be a simple graph.
	Then $m < \frac{(n-\omega)(n-\omega+1)}{2}$.
\end{exercise}
\begin{proof}
	Let $G_1,G_2,\dots,G_\omega$ be the components of $G$.
	Let $n_i,m_i$ be the order and size of $G_i$.
	Since every component is nonempty, $n-\omega+1$ is an upperbound for order of any component of $G$.
	\begin{align*}
		m 
		& \le \sum_{i=1}^\omega m_i\\
		& \le \sum_{i=1}^\omega n_i(n_i-1)/2,\quad \text{by Euler's theorem}\\
		& \le \frac{n-\omega+1}{2} \sum_{i=1}^\omega (n_i - 1),\quad \text{applying upperbound for $n_i$}\\
		& \le \frac{n-\omega+1}{2} (n - \omega)
	\end{align*}
\end{proof}

\begin{definition}
	A graph $G$ is \textbf{locally connected} at a vertex $v$ if the subgraph induced by open neighbhourhood of $v$ in $G$ is connected.
	And $G$ is locally connected if it is locally connected at every vertex.
\end{definition}

\begin{theorem}[Odd cycle characterisation of Bipartite graphs].\\
	A graph is bipartite if and only if it has no odd cycles.
\end{theorem}
\begin{proof}
	Let $G$ be a bipartite graph with partite sets $X,Y$.
	Let $C : v_0,v_1,\dots,v_k,v_0$ be a cycle in $G$.
	WLOG suppose $v_0 \in X$.
	Then $v_1 \in Y$, $v_2 \in X$, \dots.
	Clearly, if $j$ is even, then $v_j \in X$.
	Since $v_k,v_0$ are adjacent $k$ is odd and the cycle is of even length.\\

	Suppose $G$ has no odd cycles.
	Supppose $G$ is connected.
	Let $u$ be vertex of $G$.
	Define $X = \{ v \in V(G) : d(u,v) \text{ is even } \}$ and $Y = \{ v \in V(G) : d(u,v) \text{ is odd }\}$.
	Suppose vertices $v,w \in X$ are adjacent.
	Since $v,w \in X$, there exists $u-v$ path $P$ and $u-w$ path $Q$ both are of even length. Then $P+Q$ is not necessarily a $v-w$ path, let $w'$ be the last vertex they have in common. Then the $w'-v$ section of $P$, say $P'$ and $w'-w$ section of $Q$, say $Q'$ are disjoint. If $u-w'$ is of even(odd) length, then both $P',Q'$ are even(odd). And $P'+Q'+vw$ is an odd cycle which is a contradiction.\\

	Similary, if $v,w \in Y$ are adjacent then respective disjoint path $P',Q'$ are both of odd length. And $P'+Q'+vw$ is again an odd cycle which is a contradiction. Thus, $G(X,Y)$ is bipartite.\\

	Suppose $G$ is disconnected. Then each component $G_i(X_i,Y_i)$ is bipartite. And $G(X,Y)$ is bipartite where $X = \cup X_i$ and $Y = \cup Y_i$.
\end{proof}

\begin{exercise}
	Let $G$ be a simple, nontrivial graph.
	Graph $G$ is connected if and only if there exists an edge between any two nonempty partitions of $V(G)$.
\end{exercise}
\begin{proof}
	Suppose $G$ is connected.
	Let $V_1,V_2$ be two nonempty partitions of $V(G)$.
	Let $v_1 \in V_1$ and $v_2 \in V_2$.
	Then there exists a $v_1-v_2$ path, $P$ in $G$.
	Let $u$ be the first vertex in $P$ which is not from $V_1$ and $w$ be its preceeding vertex in $P$. Then $wu$ is an edge between $V_1$ and $V_2$.\\

	Suppose every nonempty parition $V_1,V_2$ has an edge between them.
	Let $u,v \in V(G)$ and $V_1 = \{ u \}$ and $V_2 = V(G)-V_1$. Then there exists an edge $uw$ between $V_1$ and $V_2$. If $w = v$, then we have a $u-v$ path. Suppose $w \ne v$. Consider $V_1 = \{u,w\}$ and $V_2 = V(G)-V_1$. Again there exists and edge between $V_1$ and $V_2$. Continuing like this, we get a $u-v$ path as the vertex set of $G$ is finite. Since every pair of vertices $(u,v)$ are connected by a $u-v$ path, the graph $G$ is connected.
\end{proof}

\begin{exercise}
	Let $G$ be a connected graph of order at least $3$.
	Then any two longest paths in $G$ has a common vertex.
\end{exercise}
\begin{proof}
	Let $G$ be a connected graph of order at least $3$.
	Let $P:u_1,u_2,\dots,u_k$, $Q : v_1,v_2,\dots,v_k$ be two longest paths in $G$ which are disjoint.
	Since $G$ is connected, there exists a $u_1-v_1$ path, $P'$ in $G$.
	Let $u_r,v_s$ be vertices in $P'$ such that $u_r-u_k$ section of $P$ and $u_r-v_1$ section of $P'$ are disjoint and $v_s-v_k$ section of $Q$ and $u_1-v_s$ section of $Q$ are disjoint. Let $P''$ be the $u_r-v_s$ section of $P'$ of length at least $1$. \\

	WLOG $u_1-u_r$ section of $P$, say $P_1$ and $v_s-v_1$ section of $Q$, say $Q_1$ are of length at least $k/2$. Then $P_1 + P''+Q_1$ is path of length at least $k/2+1+k/2$. This is a contradiction as this path is longer than the longest paths in $G$. Therefore, two longest paths in $G$ must have a common vertex.
\end{proof}

\begin{exercise}
	The union of two distinct paths joining two vertices contains a cycle.
\end{exercise}
\begin{proof}
	Let $P$, $Q$ be two distinct paths between two distinct vertices $u,v$ in $G$.
	The vertex $u$ belongs to both $P$ and $Q$.
	Let $w$ be the first vertex such that the vertex after $w$ is different in $P$ and $Q$. Let $x$ be a vertex after $w$ which is in both $P$ and $Q$. There exists such a vertex since $v$ belongs to both $P$ and $Q$. Then $w-x$ section of $P,Q$ are disjoint $w-x$ paths. Therefore, they form a cycle.
\end{proof}
\begin{exercise}
	If a simple, connected graph $G$ of order $n \ge 3$ is not complete. Then there exists three vertices $u,v,w$ such that $uv,vw$ are edges of $G$, but $uw$ is not an edge of $G$.
\end{exercise}
\begin{proof}
	Let $n \ge 3$. Since $G$ is not complete, there exists two vertices $u,v$ which are non-adjacent in $G$. Since $G$ is connected, there exists a $u-v$ path $P$ in $G$, let $w$ be the vertex preceeding $v$ in $P$. If $u,w$ are adjacent, then the proof is complete.\\

	Suppose $u,w$ are not adjacent. Rename $w$ as $v$. Again, we have a pair of vertices $u,v$ which are adjacent and a vertex $w$ preceeding $v$ on the $u-v$ path $P'$. The length of $P'$ is one less than the length of $P$. Suppose there is no such vertex for every path of length $\ge 3$. Then a $u-v$ path of length two, say $u,w,v$ has a such a vertex.
\end{proof}

\begin{definition}
	A simple graph $G$ is \textbf{highly irregular} at a vertex $v$ if all its neighbours have different degrees. A simple graph $G$ is highly irregular if it is highly irregular at every vertex of $G$.
\end{definition}
\begin{exercise}
	There does not exists highly irregular, simple connected graphs of order $3$ and $5$. 
\end{exercise}
\begin{proof}
	Let $G$ be a simple connected graph of order $3$.
	Then $G$ has a vertex $v$ with degree $2$.
	Let $u,w$ be the neighbours of $v$.
	Since $G$ is connected, degree of these vertices are $1$ or $2$.
	By Euler's theorem, the sum of degrees should be even. Therefore, the degree of $u,w$ are the same. Therefore simple connected graphs of order $3$ are not highly irregular.\\

	Let $G$ be a simple connected graph of order $5$.
	Suppose $G$ is highly irregular.
	If $G$ has a vertex $v$ with degree $4$, then the degree of its neighbours are $1,2,3$ and $4$.
	Then the degree sequence of $G$ is $4,4,3,2,1$
	This is a contradiction as $G$ has two vertices of degree $4$ and one vertex of degree $1$.
	Therefore, $G$ does not have a vertex of degree $4$.\\

	If $G$ has a vertex $v$ with degree $3$, then degree of its neighbours are $1,2,3$ as $G$ does not have a vertex of degree $0$ or $4$. By Euler's theorem, the vertex which is not adjacent to $v$ must have an odd degree.
	Therefore the degree sequence of $G$ is either $3,3,3,2,1$ or $3,3,2,1,1$.
	Let $u$ be the vertex with degree $2$. Then the neighbours $u$ of must have degree different degree say, $1$ and $3$ which is not possible as the remaining vertex needs three neighbours.\\

	Then $G$ has vertices of degree $2$ or $1$.
	By Euler's theorem, the degree sequences are $(2,1,1,1,1)$, $(2,2,2,1,1)$ or $(2,2,2,2,2)$. Clearly, not every vertex of degree $2$ have neighbours with different degree. Therefore, no graph of order $5$ is highly irregular.
\end{proof}
\begin{challenge}
	Prove that : highly irregular graphs of all orders exists, except $3,5$ and $7$.
\end{challenge}
\begin{remark}
	$P_4$ is highly irregular.\footnote{I thought, we should call it $P_3$.}
	And there exists a unique connected simple graph of order $6$ with degree sequence $3,3,2,2,1,1$ which is  highly irregular.
\end{remark}

\begin{definition}
	The \textbf{generalised Petersen graph} $P(n,k)$ is given by
	$$ V(P(n,k)) = \{ a_i,b_i : 0 \le i \le n-1 \} $$
	$$ E(P(n,k)) = \{ a_ia_{i+1}, a_ib_i, b_ib_{i+k} : 0 \le i \le (n-1) \}$$
	where the additions $i+1, i+k$ are modulo $n$.
\end{definition}
\begin{exercise}
	The generalised Petersen graph $P(n,k)$ is bipartite if $n$ is even and $k$ is odd.
\end{exercise}
\begin{proof}
	Consider the sets $X = \{ a_{2j}, b_{2j-1} : 1 \le j \le n/2 \}$ and $Y = \{ a_{2j-1}, b_{2j} : 1 \le j \le n/2 \}$. Clearly, edges of the form $a_ib_i$ has one end vertex in $X$ and other in $Y$.\\

	\textbf{Suppose $n$ is even}. Then $a_1 \in Y$ and $a_n \in X$. Therefore, edge of the form $a_i,a_{i+1}$ has one end vertex in $X$ and other in $Y$.\\

	\textbf{Suppose $k$ is odd}. If $i$ is odd, then $i+k \pmod{n}$ is even. And $b_i \in Y$ and $b_{i+k} \in X$. Similarly, if $i$ is even, then $b_i \in X$ and $b_{i+k} \in Y$. Therefore, edges of the form $b_ib_{i+k}$ has one end vertex in $X$ and other in $Y$.
	Therefore, $P(n,k)$ is bipartite if $n$ is odd and $k$ is even.
\end{proof}

\begin{exercise}
	If $G$ is simple and $\delta(G) \ge k$, then $G$ contains a path of length at least $k$.
\end{exercise}
\begin{proof}
	Let $P : v_0,v_1,\dots,v_r$ be a longest path in $G$.
	Then $v_r$ is at most adjacent to $v_1,v_2,\dots,v_{r-1}$. Otherwise, there exists a longer path in $G$.
	Therefore, $\deg(v_r) < r$. And $G$ has a vertex with minimum degree less than the length of its longest path.
	In other words, if $G$ has a vertex with minimum degree $k$ then it has a path which is at least $k$ long.
\end{proof}

\subsection{Automorphisms of a simple graph}
\begin{definition}
	An \textbf{automorphism} on a graph $G$ is an isomorphism of $G$ onto itself.
	The set $aut(G)$ is the set of all automorphisms on $G$.
\end{definition}

\begin{theorem}
	The set $aut(G)$ of all automorphisms on a simple graph $G$ together with function composition is a group.
\end{theorem}
\begin{proof}
	Let $G$ be a simple graph. Then the identity map $\phi : G \to G$ defined by $\phi(v) = v$ is a trivial automorphism on $G$. Therefore the $aut(G)$ is nonempty.\\

	We need to prove that the proposed group operation is closed. Let $\phi_1,\phi_2$ be two automorphisms on $G$. Let $u,v$ be two vertices of $G$.
	\begin{align*}
		u,v \text{ are adjacent } 
		& \iff \phi_2(u),\phi_2(v) \text{ are adjacent}\\
		& \iff \phi_1(\phi_2(u)),\phi_1(\phi_2(v)) \text{ are adjacent}
	\end{align*}
	Clearly, $\phi_1 \circ \phi_2$ are bijective and preserves adjacency and non-adjacency. Therefore, $\phi_1 \circ \phi_2$ is an automorphism. Thus, the group operation is closed.\\

	We know that function composition is associative. The trivial automorphism is the identity element of the group. And the inverse of each automorphism $\phi$ is the inverse map $\phi^{-1}$. Therefore, $\entity{aut(G),\circ}$ is a group.
\end{proof}

\begin{theorem}
	For any simple graph $aut(G) = aut(G^c)$.
\end{theorem}
\begin{proof}
	Let $G$ be a simple graph and $\phi$ be an automorphism on $G$. Let $u,v$ be two vertices of $G$.
	\begin{align*}
		u,v \text{ are non-adjacent in } G^c 
		& \iff u,v \text{ are adjacent in } G \\
		& \iff \phi(u),\phi(v) \text{ are adjacent in } G \\
		& \iff \phi(u),\phi(v) \text{ are nonadjacent in } G^c
	\end{align*}
	Clearly, every automorphism $\phi$ on $G$ is essentially a bijection on the vertex set of both $G$ and $G^c$. The $\phi$ preserves adjacency/non-adjacency of $G^c$ as well.\\

	Every autmorphism $\phi$ on $G$ is an automorphism on $G^c$.
	If $\phi \in aut(G)$ then $\phi \in aut(G^c)$. Thus, $aut(G^c) \subset aut(G)$.
	Similarly, every automorphism on $G^c$ is an automorphism on $G$ as well. 
	If $\phi \in aut(G^c)$ then $\phi \in aut(G)$. Thus, $aut(G) \subset aut(G^c)$.
	Therefore, $aut(G) \cong aut(G^c)$.
\end{proof}

\begin{exercise}
	$aut(K_n) \cong S_n$.
\end{exercise}
\begin{proof}
	Let $K_n$ be a complete graph with vertices $v_1,v_2,\dots,v_n$.
	Any automorphism $\phi$ on $K_n$ is a bijection/permutation of $n$ vertices. Thus, $\phi \in S_n$.
\end{proof}
\subsection{Line Graph}
\begin{definition}
	Let $G$ be a graph. The \textbf{line graph} of $G$ is graph $L(G)$ with vertex set in $1-1$ correspondence with edge set of $G$ and two vertices of $L(G)$ are adjacent if the corresponding edges are adjacent in $G$.
\end{definition}
\begin{commentary}
	Let $G$ be a graph with vertex set $V(G) = \{ u,v,w,x \}$ and edge set $E(G) = \{ uv, vw, wx, uw \}$. Then the line graph $L(G)$ is a graph with vertex set $V(L(G)) = \{ a,b,c,d \}$ with $1-1$ correspondence $a \to uv, b \to vw, c \to wx, d \to uw$ and edge set $E(L(G)) = \{ ab, ad, bc, bd, cd \}$ with adjacency at vertices $v,u,w,w,w$ respectively.
\end{commentary}

\subsubsection{Properties of Line graph}
Suppose $G$ has no isolated vertices.
\begin{enumerate}
	\item $G$ is connected if and only if $L(G)$ is connected.
	\begin{proof}
		Let $G$ be a connected graph. Let $e,f$ be itwo vertices of $L(G)$. Then $e,f$ corresponds to edges of $G$ say $e = uv$ and $f = wx$. Since $G$ is connected, there exists a $u-v$ path in $G$. The edges of $u-v$ path corresponds to the vertices of an $e-f$ path in $L(G)$. Therefore, two arbitrary vertices $e,f$ in $L(G)$ have a path between them. Thus, $L(G)$ is a connected graph.\\

		Let $G$ be a graph and $L(G)$ is the line graph of $G$. Suppose $L(G)$ is connected. Let $u,v$ be two vertices of $G$. Since $G$ has no isolated vertices. There exists some edges $e,f$ incident of $u,v$. Suppose $e \ne f$. Otherwise edge $uv$ is a $u-v$ path of length $1$. Since $L(G)$ is connected, there exists a $e-f$ path in $L(G)$. The vertices of $e-f$ path corresponds to edges in a $u-v$ path in $G$. Therefore, two arbitrary vertices of $G$ have a path between them. Thefore, $G$ is connected.
	\end{proof}
	\item If $H$ is a subgraph of $G$, then $L(H)$ is a subgraph of $L(G)$.
	\begin{proof}
		Let $H$ be a subgraph of $G$.
		The vertices of $L(H)$ corresponds to edges of $H$. Since $H$ is a subgraph of $G$, these edges also belong to $G$. Thus, $L(G)$ have vertices corresponding to these edges as well. In other words, $V(L(H)) \subset V(L(G))$.\\

		Two vertices in $L(H)$ are adjacent if the corresponding edges are adjacent in $H$. Clearly, these edges are adjacent in $G$ as well. Therefore, the vertices are adjacent in $L(G)$ as well. That is, $E(L(H)) \subset E(L(G))$. Therefore $L(H)$ is a subgraph of $L(G)$.
	\end{proof}
	\item The edges incident at a vertex of $G$ give rise to a clique of $L(G)$.
	\begin{proof}
		Suppose $k$ edges are incident on a vertex of $G$. Let $e_1,e_2,\dots,e_k$ be the vertices of $L(G)$ corresponding to those edges in $G$. Every pair of vertices $e_i, e_j$ among them are adjacent in $L(G)$ as the respective edges are adjacent in $G$. Thus, we have a complete subgraph of $L(G)$. Since $G$ is simple, no other edge in $G$ is simultaneously adjacent to all those edges. Therefore, the complete subgraph induced by $\{ e_1,e_2,\dots,e_k \}$ is maximal in $L(G)$.
	\end{proof}
\item Let $G$ be a simple graph. Then the degree of the vertex $e$ of $L(G)$ corresponding to an edge $e = uv$ of $G$ is given by
		$$d_{L(G)}(e) = d_G(u) + d_G(v) - 2$$
	\begin{proof}
		Let $e = uv$ be an edge in $G$. The number of edges adjacent to $e$ at $u$ is $d(u)-1$. Similarly, the number of edges adjacent to $e$ at $v$ is $d(v)-1$. Therefore, there are $d(u)+d(v)-2$ edges adjacent to $e$ in $G$. Clearly, the correpsonding vertex $e$ is adjacent to $d(u)+d(v)-2$ vertices in $L(G)$. In other words, $d_{L(G)}(e) = d_G(u)+d_G(v)-2$.
	\end{proof}
	\item Let $G$ be a simple graph with degree sequence $\mathscr{D}(G) = d_1,d_2,\dots,d_n$.
		Then size of the line graph $L(G)$ is given by $m(L(G)) = \frac{1}{2} \sum d_i^2 - m(G)$.
	\begin{proof}
	By Euler's theorem,
	\begin{align*}
		m(L(G)) 
		& = \frac{1}{2} \sum_{e \in V(L(G))} d_{L(G)}(e) \\
		& = \frac{1}{2} \sum_{e \in E(G)} \left[ d_G(u)+d_G(v)-2 \right] \\
		& = \frac{1}{2} \left[ \sum_{v \in V(G)} d_G(v)^2 \right] - m(G) 
	\end{align*}
	\end{proof}
\end{enumerate}

\begin{exercise}
	Line graph of the star $K_{1,n}$ is the complete graph $K_n$.
	$$L(K_{1,n}) \cong K_n$$
\end{exercise}
\begin{proof}
	Let $e_1,e_2,\dots,e_n$ be the edges of $K_{1,n}$.
	Clearly, $L(K_{1,n})$ has $n$ vertices.
	Since every pair of edges in $K_{1,n}$ are adjacent, the respective vertices are adjacent in $L(K_{1,n})$.
	Therefore, line graph $L(K_{1,n})$ is a complete graph with $n$ vertices, say $K_n$.
\end{proof}

\begin{exercise}
	Line graph of the cycle $C_n$ ($n \ge 3$) is $C_n$ itself.
	$$ L(C_n) = C_n$$
\end{exercise}
\begin{proof}
	Let $C_n$ be a cycle of length $n \ge 3$. Let $e_1,e_2,\dots,e_n$ be the edges of $C_n$ where $e_j,e_{j+1}$ are adjacent. Let $G$ be the line graph of $C_n$. Then $G$ has $n$ vertices corresponding to $n$ edges of $C_n$. We know that $d_{L(G)}(e) = d(u) + d(v) - 2$. Since $C_n$ are $2$-regular, $d(u) = d(v) = 2$. Then the line graph $G$ is $2$-regular. We know that, line graph of a connected graph is connected. Since $C_n$ is connected, its line graph $G$ is also connected. Since $G$ is a connected $2$-regular graph, $G$ is isomorphic to $C_n$ itself.
\end{proof}

\begin{theorem}
	Line graph of a simple graph is a path if and only if $G$ is a path.
	$$ G \cong P_n \iff L(G) \cong P_{n-1} $$
\end{theorem}
\begin{proof}
	Let $G$ be a path of length $n$, say $P_n$. Let $e_1,e_2,\dots,e_n$ be the edges of $P_n$ such that each pair of edges $e_j,e_j+1$ are adjacent except the pair $e_n,e_1$.
	\begin{commentary} (We usually consider addition modulo $n$.\end{commentary}) Then line graph $L(G)$ has $n$ vertices. Since $G$ is connected, $L(G)$ is also connected. Clearly, every vertex $e_j$ is adjacent to $e_{j+1}$ except $e_n,e_1$. Therefore, $L(P_n)$ is a path of length $n-1$, say $P_{n-1}$.\\

		Let $G$ be a simple graph with no isolated vertices. Suppose $L(G)$ is a path of length $m$, say $P_m$. By the definition of line graph the size of $G$ is $m$. Since $L(G)$ is connected, $G$ is also connected. Since $L(G)$ has two pendent vertices, $G$ has two pendent edges. Since the order of maximal subgraphs are at most $2$, the degree of vertices of $G$ is at most $2$. Therefore, $G$ is a path of length $m+1$, say $P_{m+1}$.
\end{proof}

\begin{exercise}
	\label{ex:L2G}
	Let $H = L^2(G)$ be defined as $L(L(G))$. Find $m(H)$ if $G$ is the graph in figure \ref{dia:L2G}.
\begin{figure}[hbt]
\centering
\begin{tikzpicture}
	\tikzstyle every node=[draw,circle]
	\node (u){};
	\node (v)[below=0.6cm of u]{};
	\node (w)[below left=1cm of v]{};
	\node (x)[below right=1cm of v]{};
	\node (y)[below left=1cm of w]{};
	\node (z)[below right=1cm of x]{};
	\draw (u)--(v)--(w)--(y);
	\draw (v)--(x);
	\draw (x.360) to[out=10,in=100] (z.90);
	\draw (x.270) to[out=280,in=180] (z.170);
\end{tikzpicture}
	\caption{Graph $G$ in exercise \ref{ex:L2G} }
	\label{dia:L2G}
\end{figure}
\end{exercise}
\begin{proof}
	(do it yourself)
\end{proof}

\begin{exercise}
	Show that $d_{L(G)}(uv) = d_G(u) + d_G(v)-2$ is not valid if $G$ has a loop.
\end{exercise}
\begin{proof}
	(do it yourself)
\end{proof}

\begin{exercise}
	Prove that a simple connected graph $G$ is isomorphic to its line graph if and only if it is a cycle.
\end{exercise}
\begin{proof}
	We know that the line graph of a cycle is always a cycle of same length. Thus, if $G$ is a cycle then it is isomorphic to its line graph.\\

	Let $G$ be a connected graph. Suppose $G$ is isomorphic to its line graph. Then order and size of $G$ are the same.\\

	Let $\Delta$ be the maximum degree of $G$. Then $G$ has a vertex $u$ with degree $\Delta$. Then $u$ has at least one neighbour with degree greater than $1$. Otherwise, $G \cong K_{1,\Delta}$ and $L(G) = K_{\Delta}$ is a contradiction. If degree of a vertex $v$ adjacent to $u$ has degree greater than $2$. Then $L(G)$ has a vertex $uv$ with degree greater than $\Delta$ which is a contradiction.\\

	Suppose $G$ has a vertex $u$ with $d(u) = \Delta$ and a vertex $v$ adjacent to $u$ with $d(v) = 2$. Then $L(G)$ has a vertex $x'$ with $d(x') = \Delta$ and a vertex $y'$ adjacent to $x'$ with $d(y') = \Delta-1$. Since $G \cong L(G)$. $G$ has vertices $x,y$ which adjacent and $d(x) = \Delta,\ d(y) = \Delta-1$. Then $L(G)$ has a vertex $xy$, corresponding to the edge $xy$ of $G$, with $d(xy) = 2\Delta - 3$. Clearly, $\Delta \le 3$ otherwise $L(G)$ has a vertex with degree greater than $\Delta$ which is a contradiction.\\

	Suppose $\Delta  = 3$. Then $G$ has two adjacent vertices $u,v$ with $d(u) = 3,\ d(v) = 2$. Then $L(G)$ has two adjacent vertices $x',y'$ with $d(x') = 3$ and $d(y') = 3$. Since $G \cong L(G)$, $G$ has two adjacent vertices $x,y$ with $d(x) = 3,\ d(y) = 3$. Thus $L(G)$ has a vertex $xy$ with degree $4$ which is a contradiction.\\
	
	We know that $\Delta \ne 1$, otherwise $G = K_2$ and $L(G) = K_1$ which is a contradiction. Thus, $\Delta = 2$. Since $G$ is connected, $G$ is either a path or a cycle. Suppose $G$ is a path $P_n$, then $L(G) = P_{n-1}$ which is a contradiction. Therefore $G$ is a cycle.
\end{proof}

\begin{exercise}
	Disprove : If the graph $H$ is a spanning subgraph of $G$, then $L(H)$ is a spanning subgraph of $G$.
\end{exercise}
\begin{proof}
	(do it yourself)
\end{proof}

\begin{theorem}
	If the simple graphs $G_1,G_2$ are isomorphic, then $L(G_1),L(G_2)$ are isomorphic.
\end{theorem}
\begin{proof}
	Let two simple graphs $G_1,G_2$ are isomophic.
	Then there exists an isomorphic $\phi : V(G_1) \to V(G_2)$ which gives rise to an edge set bijection $\theta : E(G_1) \to E(G_2)$ such that $\phi(u)\phi(v) = \theta(uv)$.\\

	Let $e$ be an edge of $G_1$. Then $e$ corresponds to a vertex $e$ of $L(G_1)$. By graph isomorphism, $\theta(e)$ is an edge of $G_2$ and $\theta(e)$ corresponds to a vertex $\theta(e)$ of $L(G_2)$. Consider the map $\phi' : V(L(G_1)) \to V(L(G_2))$ defined by $\phi'(e) = \theta(e)$. Clearly, $\phi'$ is a bijection.\\

	Suppose two edges $e,e'$ are adjacent in $G_1$. Then by isomorphism $\theta(e),\theta(e')$ are adjacent in $G_2$. Therefore, if two vertices $e,e'$ are adjacent in $L(G_1)$, then the respective vertices $\phi'(e),\phi'(e')$ are adjacent in $L(G_2)$. Therefore $\phi'$ is a graph isomorphism between $L(G_1)$ and $L(G_2)$. Clearly, $L(G_1) \cong L(G_2)$.
\end{proof}

\begin{remark}
	Suppose $G_1,G_2$ be simple, connected graphs. Then $L(G_1) \cong L(G_2)$ does not imply that $G_1 \cong G_2$.
\end{remark}
\begin{proof}
	Suppose $G_1 = K_{1,3}$ and $G_2 = K_3$. Then $L(G_1) \cong L(G_2) \cong K_3$ and $G_1 \not\cong G_2$.
\end{proof}

\subsection{Graph Operations}
\begin{definition}
	Let $G_1,G_2$ be two simple graph.
	Their \textbf{union} $G_1 \cup G_2$ is given by 
	$$ V(G_1 \cup G_2) = V(G_1) \cup V(G_2) \qquad E(G_1 \cup G_2) = E(G_1) \cup E(G_2) $$
	If $V(G_1) \cap V(G_2) = \phi$, then the graph $G_1 \cup G_2$ is their \textbf{sum} $G_1 + G_2$.
\end{definition}

\begin{remark}
	The finite union of graph is associative.
\end{remark}

\begin{definition}
	Let $G_1,G_2$ be two simple graph with $V(G_1) \cap V(G_2) \ne \phi$.
	Their \textbf{intersection} $G_1 \cap G_2$ is given by 
	$$ V(G_1 \cap G_2) = V(G_1) \cap V(G_2) \qquad E(G_1 \cup G_2) = E(G_1) \cap E(G_2) $$
\end{definition}

\begin{definition}
	Let $G_1,G_2$ be two vertex-disjoint graphs. Then the \textbf{join} $G_1 \vee G_2$ is the supergraph of $G_1 + G_2$ in which each vertex of $G_1$ is adjacent to every vertex of $G_2$.\\

	If $G_1 = K_1$ and $G_2 = C_n$, then $G_1 \vee G_2 \cong W_n$ a \textbf{wheel} of order $n+1$.
\end{definition}

\subsubsection{Properties of Graph Operations}
\begin{enumerate}
	\item $n(G_1 \cup G_2) = n(G_1) + n(G_2) - n(G_1 \cup G_2)$ and\\
		$m(G_1 \cup G_2) = m(G_1) + m(G_2) - m(G_1 \cup G_2)$.
		\begin{proof}
			do it yourself
		\end{proof}
	\item $n(G_1+G_2) = n(G_1) + n(G_2)$ and \\ $m(G_1+G_2) = m(G_1) + m(G_2)$.
		\begin{proof}
			do it yourself
		\end{proof}
	\item $n(G_1 \vee G_2) = n(G_1)+n(G_2)$ and \\ $m(G_1 \vee G_2) = m(G_1)+m(G_2)+n(G_1)n(G_2)$.
		\begin{proof}
			do it yourself
		\end{proof}
\end{enumerate}

\subsection{Graph Products}
\begin{definition}
	The \textbf{general graph product} $G_1 \ast G_2$ is defined as a graph with vertex set $V(G_1 \ast G_2) = V(G_1) \times V(G_2)$ and the adjacency/non-adjacency among a pair of vertices is given by the structure table $S$, 
$$ S = \begin{bmatrix} \circ & \circ & \circ \\ \circ & = & \circ \\ \circ & \circ & \circ \end{bmatrix} $$
	where $\circ$'s are either $N$ or $E$. The letter `E' means Edge and `N' means No edge.\\

	The first, second and third rows of $S$ corresponds to adjacency, equality and non-adjacency of respective vertices in $G_1$. Similarly, first, second and third columns of $S$ corresponds to adjacency, equality and non-adjacency of respective vertices in $G_2$.
\end{definition}

\begin{commentary}
	For example, $G_1,G_2$ be two simple graph given in diagram \ref{dia:generalProduct}. Let $G_1 \star G_2$ be a graph product with the following strcture table $S$ given in table \ref{tb:generalProduct}.\\

\begin{figure}[hbt]
\centering
\begin{tikzpicture}
\tikzstyle every node=[draw,black,circle]
	\node (u)[label=left:$u_1$]{};
	\node (v)[above right=1cm of u,label=left:$v_1$]{};
	\node (w)[below right=1cm of u,label=left:$w_1$]{};
	\draw[black] (v)--(u)--(w);
\end{tikzpicture}
\hfill
\begin{tikzpicture}
\tikzstyle every node=[draw,black,circle]
	\node (u)[label=above:$u_2$]{};
	\node (v)[right = 1cm of u,label=above:$v_2$]{};
	\draw[black] (u)--(v);
\end{tikzpicture}
\hfill
\begin{tikzpicture}
\tikzstyle every node=[draw,black,circle]
	\node (u1)[label=left:$u_1u_2$]{};
	\node (v1)[above right=1cm of u1,label=left:$v_1u_2$]{};
	\node (w1)[below right=1cm of u1,label=left:$w_1u_2$]{};
	\node (u2)[right=2cm of u1,label=right:$u_1v_2$]{};
	\node (v2)[above right=1cm of u2,label=right:$v_1v_2$]{};
	\node (w2)[below right=1cm of u2,label=right:$w_1v_2$]{};
%	\draw[black] (v1)--(u1)--(w1);
%	\draw[black] (v2)--(u2)--(w2);
	\draw[green] (v1)--(u1)--(w1);
	\draw[green] (w2)--(u2)--(v2);
	\draw[cyan] (u1)--(u2);
	\draw[cyan] (v1)--(v2);
	\draw[cyan] (w1)--(w2);
	\draw[blue] (v1.300) to[out=290,in=160] (w2.150);
	\draw[blue] (w1.60) to[out=70,in=200] (v2.210);
\end{tikzpicture}
	\caption{A graph product $G_1 \star G_2$}
	\label{dia:generalProduct}
\end{figure}

	For example, $v_1u_2, w_1v_2$ are two vertices of $G_1 \star G_2$. We know that $v_1,w_1$ are non-adjacent in $G_1$. Thus, we need to look at the third row of the structure table $S$. Again, $u_2,v_2$ are adjacent in $G_2$. Thus, we need to look at the first column of the structure table $S$. Clearly, $S_{3,1}$ which is $E$. Therefore, they are adjacent in $G_1 \star G_2$. Refer table \ref{tb:generalProduct} to find element of $S$ corresponding to each pair of vertices in $G_1 \star G_2$.

\begin{table}[hbt]
\centering
$ S = \displaystyle\begin{bmatrix} N & E & N \\ E & = & N \\ E & N & N \end{bmatrix}$
\hfill
\begin{tabular}{|c|c|c|c|c|c|c|} \hline
	& $u_1u_2$ & $v_1u_2$ & $w_1u_2$ & $u_1v_2$ & $v_1v_2$ & $w_1v_2$ \\ \hline
	$u_1u_2$ & $S_{2,2}$ & \textcolor{green}{$S_{1,2}$} & \textcolor{green}{$S_{1,2}$} & \textcolor{cyan}{$S_{2,1}$} & $S_{1,1}$ & $S_{1,1}$ \\ \hline
	$v_1u_2$ & \textcolor{green}{$S_{1,2}$} & $S_{2,2}$ & $S_{3,2}$ & $S_{1,1}$ & \textcolor{cyan}{$S_{2,1}$} & \textcolor{blue}{$S_{3,1}$} \\ \hline
	$w_1u_2$ & \textcolor{green}{$S_{1,2}$} & $S_{3,2}$ & $S_{2,2}$ & $S_{1,1}$ & \textcolor{blue}{$S_{3,1}$} & \textcolor{cyan}{$S_{2,1}$} \\ \hline
	$u_1v_2$ & \textcolor{cyan}{$S_{2,1}$} & $S_{1,1}$ & $S_{1,1}$ & $S_{2,2}$ & \textcolor{green}{$S_{1,2}$} & \textcolor{green}{$S_{1,2}$} \\ \hline
	$v_1v_2$ & $S_{1,1}$ & \textcolor{cyan}{$S_{2,1}$} & \textcolor{blue}{$S_{3,1}$} & \textcolor{green}{$S_{1,2}$} & $S_{2,2}$ & $S_{3,2}$ \\ \hline
	$w_1v_2$ & $S_{1,1}$ & \textcolor{blue}{$S_{3,1}$} & \textcolor{cyan}{$S_{2,1}$} & \textcolor{green}{$S_{1,2}$} & $S_{3,2}$ & $S_{2,2}$ \\ \hline
\end{tabular}
	\caption{Adjacency/Non-adjacency Table for $G_1 \star G_2$}
	\label{tb:generalProduct}
\end{table}
\end{commentary}

\begin{definition}
	The \textbf{cartesian product} $G_1 \square G_2$ is the graph product with structure table $S$ given by
	$$ S = \begin{bmatrix} N & E & N \\ E & = & N \\ N & N & N \end{bmatrix} $$
\end{definition}

\begin{definition}
	The \textbf{direct product}(tensor/Kronecker product) $G_1 \times G_2$ is the graph product with structure table $S$ given by
	$$ S = \begin{bmatrix} E & N & N \\ N & = & N \\ N & N & N \end{bmatrix} $$
\end{definition}

\begin{definition}
	The \textbf{composition product} (wreath/lexicographic product) $G_1[G_2]$ is the graph product with structure table $S$ given by
	$$ S = \begin{bmatrix} E & E & E \\ E & = & N \\ N & N & N \end{bmatrix} $$
\end{definition}

\begin{definition}
	The \textbf{strong product} (normal product) $G_1 \boxtimes G_2$ is the graph product with structure table $S$ given by
	$$ S = \begin{bmatrix} E & E & N \\ E & = & N \\ N & N & N \end{bmatrix} $$
\end{definition}

\begin{remark}
	From structure table $S$, it is evident that all the above graph products are associative and all except the composition product are commutative.
\end{remark}

\begin{challenge}
	Can you find a non-associative graph product ?
\end{challenge}

\begin{exercise}
	Show that $G_1[G_2] \not\cong G_2[G_1]$
\end{exercise}
\begin{proof}
	$K_2[K_{1,3}] \not\cong K_{1,3}[K_2]$
\end{proof}

\begin{exercise}
	Prove that $G_1 \square G_2 \cong G_2 \cong G_1$.
\end{exercise}
\begin{proof}
	Since the structure table $S$ of the cartesian graph product is symmetric the graph product is commutative.
\end{proof}

\begin{exercise}
	Prove that $G_1 \boxtimes G_2 \cong G_2 \boxtimes G_1$
\end{exercise}
\begin{proof}
	Since the structure table $S$ of the strong product is symmetric the graph product is commutative.
\end{proof}

\begin{definition}
	Let $(x,y)$ be a vertex of $G_1 \square G_2$. The $\mathbf{G_2}$ \textbf{fiber at} $\mathbf{x}$ in $G_1 \square G_2$ denoted by $(G_2)_x$ is the subgraph of $G_1 \square G_2$ induced by the set of all vertices $\{ (x,z) : z \in V(G_2) \}$. Similarly, the $\mathbf{G_1}$\textbf{-fiber at} $\mathbf{y}$, $(G_1)_y)$ is the subgraph of $G_1 \square G_2$ induced by $\{ (z,y) : z \in V(G_1) \}$.
\end{definition}


\begin{exercise}Properties of graph products
\begin{enumerate}
	\item The order of a general graph product is the product of their orders.
	\begin{align*}
		n(G_1 \square G_2) 
		= n(G_1[G_2]) = n(G_2[G_1]) & =\\
		n(G_1 \times G_2) = n(G_1 \boxtimes G_2) & = n(G_1) n(G_2)
		\end{align*}
	\begin{proof}
		The vertex set of a general graph product is the cartesian product of the vertex sets, $G_1 \ast G_2$ is $V(G_1) \times V(G_2)$. Therefore, order of a general graph product is $n(G_1)n(G_2)$.
	\end{proof}
	\item $m(G_1 \square G_2) = n(G_1) m(G_2) + m(G_1) n(G_2)$.
	\begin{proof}
		The structure table of cartesian graph product has entry $E$ at $S_{1,2}$ and $S_{2,1}$.
		Thus the edge conditions for cartesian graph product $G_1 \square G_2$ are
		\begin{enumerate*}
			\item $u_1 = v_1$ and $u_2,v_2$ are adjacent or 
			\item $u_2 = v_2$ and $u_1,v_1$ are adjacent.
		\end{enumerate*}

		\textbf{Case 1 :} Let $xy$ be an edge of $G_2$ and $v$ be a vertex of $G_1$. Then $(v,x),(v,y)$ are adjacent in $G_1 \square G_2$. Every edge of type 1 is of this form. Therefore, there are $n(G_1)m(G_2)$ edges of type $1$ in $G_1 \square G_2$.\\

		\textbf{Case 2 :} Let $xy$ be an edge of $G_1$ and $v$ be a vertex of $G_2$. Then $(x,v),(y,v)$ are adjacent in $G_1 \square G_2$. Every edge of type 2 is of this form. Therefore, there are $m(G_1)n(G_2)$ edges of type $2$ in $G_1 \square G_2$.
		Therefore, $m(G_1 \square G_2) = n(G_1)m(G_2) + m(G_1)n(G_2)$.
	\end{proof}
	\item $m(G_1[G_2]) = n(G_1) m(G_2) + n(G_2)^2 m(G_1)$.
	\begin{proof}
		The structure table for composition product has entry $E$ at first row of $S$ and $S_{2,1}$. Thus the condition for adjacency of two vertices $(u_1,u_2),(v_1,v_2)$ in $G_1[G_2]$ are
	\begin{enumerate*}
		\item $u_1,v_1$ are adjacent or
		\item $u_1 = v_1$ and $u_2,v_2$ are adjacent.
	\end{enumerate*}

		\textbf{Case 1 :} Let $xy$ be an edge of $G_1$ and $u,v$ be two vertices of $G_2$. Then an edge of type $1$ is of the form $(x,u)(y,v)$. Thus, there are $m(G_1)n(G_2)^2$ edges of type $1$ in $G_1[G_2]$.

		\textbf{Case 2 :} Let $xy$ be an edge in $G_2$ and $v$ be a vertex in $G_1$. Then an edge of type $2$ is of the form $(v,x)(v,y)$. Thus, there are $m(G_2)n(G_1)$ edges of type $2$ in $G_1[G_2]$.

		Therefore $m(G_1[G_2]) = m(G_1)n(G_2)^2 + m(G_2)n(G_1)$.
	\end{proof}
	\item $m(G_1 \times G_2) = 2m(G_1) m(G_2)$.
	\begin{proof}
		The only E entry in the structure table of direct product is $S_{1,1}$. Thus the condition for adjacency between $(u_1,u_2),(v_1,v_2)$ in $G_1 \times G_2$ is $u_1,v_1$ are adjacent and $u_2,v_2$ are adjacent.\\

		Let $uv,xy$ be edges in $G_1,G_2$ respectively. Then an edge of direct product is of the form $(u,x)(v,y)$ or $(u,y)(v,x)$. There are $m(G_1)m(G_2)$ edges of both kind.
		Therefore, $m(G_1 \times G_2) = 2m(G_1)m(G_2)$.
	\end{proof}
	\item $m(G_1 \boxtimes G_2) = n(G_1) m(G_2) + m(G_1) n(G_2) + 2m(G_1) m(G_2)$.
	\begin{proof}
		We know that the strong product is the disjoint union of the cartesian and direct graph products, $G_1 \boxtimes G_2 = G_1 \square G_2 \cup G_1 \times G_2$. 
		By the properties of sum of two graph, $E(G_1 \boxtimes G_2) = E(G_1 \square G_2) \cup E(G_1 \times G_2)$.
		Therefore, $m(G_1 \boxtimes G_2) = m(G_1 \square G_2) + m(G_1 \times G_2)$.
	\end{proof}
\end{enumerate}
\end{exercise}

\begin{exercise}
	Prove that $L(K_{m,n}) \cong K_m \square K_n$.
\end{exercise}
\begin{proof}
	Let $u_j,v_k$ be labels for the vertices of $K_m,K_n$ respectively. Abusing the language we may write that a vertex of $K_m \square K_n$ is of the form $u_jv_k$. By the definition of cartesian graph product, $u_jv_s, u_kv_t$ are adjacent if and only if $u_j,u_k$ are adjacent and $v_s,v_t$ are adjacent.\\

	Consider the line graph $L(K_{m,n})$ of the bipartite graph $K_{m,n}$ with partite set $X,Y$. Let $x_j,y_k$ be labels for vertices in the partite sets $X,Y$ respectively. The edge of $K_{m,n}$ are of the form $x_jy_k$. Two vertices $x_jy_s,x_ky_t$ are adjacent in $L(K_{m,n})$ if the respective edges have a common vertex. In other words, either $x_j = x_k$ or $y_s = y_t$.\\

	\begin{commentary}
	Define $\psi : X \to V(K_m)$ where $\psi(x_j) = u_j$ and define $\xi : Y \to V(K_n)$ where $\xi(y_k) = v_k$.
	\end{commentary}
	Define $\phi : V(L(K_{m,n})) \to V(K_m \square K_n)$ by $\phi(x_jy_k) = u_jv_k$. Clearly, $x_jy_s,x_ky_t$ are adjacent if and only if $\phi(x_jy_s)=u_jv_s$ and $\phi(x_ky_t)=u_kv_t$ are adjacent. Therefore, $L(K_{m,n}) \cong K_m \square K_n$.
\end{proof}

\begin{definition}
	The $\mathbf{k}$\textbf{th power} $G^k$ of $G$ has $V(G^k) = V(G)$ where two vertices $u,v$ are adjacent in $G^k$ whenever $d_G(u,v) \le k$.
\end{definition}

\begin{exercise}
	Show that if $G$ is a connected graph with at least two edges, then each edge of $G^2$ belongs to a triangle.
\end{exercise}
\begin{proof}
	Let $e=uv$ be an edge of $G^2$. Then there are two possibilities, either $d_G(u,v) = 1$ or $d_G(u,v) = 2$.\\

	\textbf{Case 1 :} Suppose $d_G(u,v) = 1$. Since $G$ is connected, without loss of generality there exists an edge adjacent to $uv$ at $v$, say $vw$. Then $d_G(u,w) \le 2$ and $uw \in E(G^2)$. Clearly, the edge $uv$ belongs to the triangle $\triangle uvw$ in $G^2$.\\

	\textbf{Case 2 :} Suppose $d_G(u,v)=2$. Then there exists $w \in V(G)$ such that $uw,wv \in E(G)$. Clearly, the edge $uv$ belongs to the triangle $\triangle uwv$ in $G^2$.
\end{proof}

\begin{exercise}
	If $d_G(u,v) = p$, determine $d_{G^k}(u,v)$.
\end{exercise}
\begin{proof}
	Suppose $p$ is finite. Otherwise $d_G(u,v) = \infty$ and  $d_{G^k}(u,v) = \infty$.
	Since $p,k$ are natural numbers, there exists unique integers $q,r$ such that $p = qk+r$, $q \ge 0$ and $0 \le r < k$. Then the shortest $u-v$ path contains vertices $v_1,v_2,\dots,v_q$ such that $d_G(u,v_1) = k$ and $d_G(v_j,v_{j+1}) = k$ and $d_G(v_{j+1},v) = r$. Then $u,v_1$ are adjacent in $G^k$. Similarly, $v_j,v_{j+1}$ are adjacent in $G^k$.\\

	Suppose $r = 0$. Then $v_q = v$ and $u,v_1,v_2,\dots,v_{q-1},v$ is the shortest $u-v$ path in $G^k$. Thus, $d_{G^k}(u,v) = q$. Suppose $r > 0$. Then $d_G(v_q,v) = r < k$. Thus, $v_q,v$ are adjacent in $G^k$ and $u,v_1,v_2,\dots,v_q,v$ is the shortest $u-v$ path in $G^k$. Therefore, $d_{G^k}(u,v) = q+1$.
\end{proof}

\setcounter{subsection}{11}
\subsection{Miscellaneous Exercises}
\begin{enumerate}
	\item Show that the sequences \begin{enumerate*} \item $(7,6,5,4,3,3,2)$ and \item $(6,6,5,4,3,3,1)$ are not graphical \end{enumerate*}
	\item Give an example of a degree sequence that is realizable as the degree sequence by only a disconnected graph.
	\begin{proof}[Solution]
		$(1,1,1,1)$ or $(1,1,0)$.
	\end{proof}
	\item Show that for a simple bipartite graph, $m \le n^2/4$.
	\begin{proof}[Solution]
		Let $G$ be a bipartite graph. The size of a bipartite graph is maximum when it is a complete bipartite graph with cardinality of partite set $X,Y$ almost equal.\\

		Suppose order of $G$ is even, $n = 2k$. Then the size of the bipartite graph is maximum when $p = q = k$. And size 
		$$ m \le  pq = k^2 = \frac{n^2}{4} $$
		Suppose order of $G$ is odd, $n = 2k+1$. The the size of the bipartite graph is maximum when $p = k$ and $q = k+1$. And size 
		$$ m \le pq = k(k+1) \le \left(k+\frac{1}{2}\right)^2 = \frac{n^2}{4} $$
	\end{proof}
	\item Show that $\delta \le \frac{2m}{n} \le \Delta$.
	\begin{proof}[Solution]
		By Euler's theorem, the sum of degrees of a graph is twice its size. Thus $2m/n$ is the average of the degrees of vertices. We know that, $minimum \le average \le maximum$ is true for any finite sequence.
	\end{proof}
	\item For every $n \ge 3$, construct a $3$-regular simple graph on $2n$ vertices containing no triangles.
	\begin{proof}[Solution]
		The generalised Petersen graphs $P(n,k)$ where $0 \le k \le n$ and $\gcd(n,k)=1$ are $3$-regular and does not contain any cycles.
	\end{proof}
\item If a bipartite graph $G(X,Y)$ is regular, show that $|X| = |Y|$.
	\begin{proof}[Solution]
		Since $G$ is bipartite the sum of degree of vertices both partite sets are equal, as every edge contributes one to both the sum.
		If $G$ is regular, then the degree of every vertex is the same, say $k$.\\

		Let $X,Y$ be the partite sets of $G$ with cardinality $p,q$.
		Then the degree sum of vertices in $X$ is $kp$ and the degree sum of vertices in $Y$ is $kq$.
		We know that $G$ is bipartite, thus every edge in $G$ has one end in $X$ and the other end in $Y$.
		Thus, the degree sum of $X$ and $Y$ should be the same.
		That is, $kp = kq$.
		Therefore $p = q$.
		In other word, both partite sets have the same number of vertices.
	\end{proof}
	\item Show that in a simple graph, any closed walk of odd length contains a cycle.
	\begin{proof}[Solution]
		Let $G$ be a simple graph and $W : u_0,e_1,u_1,\dots,e_n,u_0$ be a closed walk of odd length in $G$. If $W$ does not visit any vertex twice(other than the origin $u_0$), then $W$ is a cycle.\\

		Suppose $W$ visits another vertex $v$ multiple times. Suppose every such sections are of even length, then delete all those $u_j-u_k$ sections of $W$ to obtain an odd cycle contained in $W$. If the number of edges of $W$ between two consecutive visits of a vertex $v = u_j,u_k$ is odd, then $u_j-u_k$ section of $W$ is a closed walk of odd length. Repeating like this, we get an odd cycle contained in $W$.
	\end{proof}
\item Give an example of a disconnected simple graph with $\omega$ components, $n$ vertices and $\frac{(n-\omega)(n-\omega+1)}{2}$ edges
	\begin{proof}[Solution]
		Do it yourself (hint : check the proof)
	\end{proof}
\item Prove or disprove : If $H$ is a subgraph of $G$, then $\delta(H) \le \delta(G)$, $\Delta(H) \le \Delta(G)$.
	\begin{proof}[Solution]
		Do it yourself
	\end{proof}
\item If $m > \frac{(n-1)(n-2)}{2}$, then show that $G$ is connected.
	\begin{proof}[Solution]
		Check exercise $8$ with $\omega = 2$.
	\end{proof}
	\item If $\delta \ge 2$, then show that $G$ contains a cycle.
	\begin{proof}[Solution]
		Let $G$ be a graph with $\delta \ge 2$.
		Suppose $G$ doesnot contain a cycle.
		Let $v$ be a vertex of $G$ then $v$ has at least two neighbors, say $u,w$. Since $G$ has no cycles, vertices $u,w$ are non-adjacent. Then $w$ have another neighbor  $x$ and $x$ is non-adjacent to $u,v$. Thus, $x$ have another neighbour $y$, which is not adjacent to $u,v$ or $w$. Continuing like this, we get a vertex $z$ with degree $1$ since $G$ has only finite number of vertices. This is a contradiction. Therefore, every finite graph $G$ with $\delta \ge 2$ contains a cycle.
	\end{proof}
\end{enumerate}

%\chapter{Directed Graphs}
\section{Directed Graphs}
\setcounter{subsection}{1}
\subsection{Basis Concepts}
\begin{definition}
	A \textbf{directed graph} $D$ is a ordered triple $(V(D),A(D),I_D)$ where $V(D)$ is a nonempty set of vertices, $A(D)$ is a set of arcs and $I_D$ is an incidence map that associates each arc of $D$ with an ordered pair of vertices.
\end{definition}

\begin{remark}
	A directed graph is also called a digraph. For every digraph $D$, there exists a graph $G$ on the same vertex set which has an edge for each arc of $D$ with same end vertices. This graph $G$ is the \textbf{underlying graph} of $D$.\\

	Let $G$ be a graph. If we assign an arc for each edge of $G$ with same end vertices, then we get a digraph $D$. This assignment/specification is an \textbf{orientation} of $G$. If the size of $G$ is $m$, then $2^m$ orientations are possible.
\end{remark}

\begin{definition}
	Let $a = (u,v)$ be an arc of $D$. Then $a$ is \textbf{incident out} of $u$ and \textbf{incident into} $v$. And $v$ is an \textbf{outneighbor} of $u$ and $u$ is an \textbf{inneighbor} of $v$. The sets $\mathbf{N^+_D(u), N^-_D(u)}$ are the outneighborhood and inneighborhood of $u$ respectively. And their cardinalities $d^+_D(u),d^-_D(u)$ are the \textbf{outdegree} and \textbf{indegree} of $u$. Thus, degree of a vertex $u$ is $d_D(u) = d^+_D(u) + d^-_D(u)$.\\

	Since each arc contributes one to sum of indegrees and outdegrees, we have
	$$ \sum_{v \in V(D)} d^+(v) = \sum_{v \in V(D)} d^-(v) = m(D) $$
	A vertex $v$ is pendent if $d(v) = 1$. Then either $d^+(v) = 1, d^-(v) = 0$ or $d^+(v) = 0, d^-(v) = 1$. A vertex $v$ is isolated if $d(v) = 0$.
\end{definition}

\begin{definition}
	A digraph $D'$ is \textbf{subdigraph} of a digraph $D$ if $V(D') \subset V(D)$, $A(D') \subset A(D)$ and $I_D'$ is the restriction of $I_D$ of $A(D')$.
\end{definition}

\begin{description}
	\item[directed walk] is an alternating sequence $W : v_0,a_1,v_1,a_2,\dots,a_n,v_n$ where $a_j = (v_{j-1},v_j)$.
	A walk is closed if $v_0 = v_n$ otherwise it is open.
	\item[directed trail] is a directed walk with no arcs repeated.
	\item[directed path] is a directed walk with no vertices repeated.
	\item[directed cycle] is a closed directed trail with no vertices repeated.
	\item[induced subdigraph] $D[S]$ is the maximal subdigraph of the digraph $D$ with vertex set $S$.
	\item[induced subdigraph] $D[A']$ is the minimal subdigraph of the digraph $D$ with edge set $A'$.
	\item[reachable] A vertex $v$ is reachable from $u$ if there exists a directed $u-v$ path in $D$.
	\item[diconnected] Two vertices $u,v$ are diconnected if there exists both $u-v$ directed path and $v-u$ directed path in $D$.\\

		The diconnectedness is an equivalence relation on vertex set of a digraph of $D$. A digraph $D$ is \textbf{strongly connected/diconnected} if it has exactly one dicomponent.
	\item[strict] A digraph is strict if its underlying graph is simple.
	\item[symmetric] a digraph is symmetric if $(u,v)$ is an arc of $D$ whenever $(v,u)$ is an arc of $D$.
\end{description}

\begin{exercise} How many orientations does a simple graph of $m$ edges have ? $2^m$
\end{exercise}

\begin{exercise} Let $D$ be a directed graph with no directed cycle. Prove that there exists a vertex whose indegree is $0$. Deduce that there is an ordering $v_1,v_2,\dots,v_n$ of $V$ such that for $2 \le i \le n$ every arc of $D$ with terminal vertex $v_i$ has its initial vertex in $\{v_1,v_2,\dots,v_{i-1}\}$.
\end{exercise}
\begin{proof}[Solution]
	See Chapter 1 Problem 12.11
\end{proof}

\subsection{Tournament}
\begin{definition}
	A digraph $D$ is a \textbf{tournament} if its underlying graph is complete.
\end{definition}

\begin{theorem}[R\'edei]
	Every tournament contains a directed Hamilton path.
\end{theorem}
\begin{proof}
	The proof is by induction on number of vertices of the tournament. The result is trivial for $n = 2,3$. Suppose the result is true for $n-1$. Let $T$ be a tournament of order $n$. Let $v \in V(T)$. Then $T'=T-\{v\}$ is tournament of order $n-1$. By induction hypothesis $T'$ has a Hamilton path $v_1,v_2,\dots,v_{n-1}$. Suppose $T$ does not have the arcs $(v,v_1)$ or $(v_{n-1},v)$. Otherwise $T$ has a directed Hamilton path. Since $T$ is a tournment it has the arcs $(v_1,v)$ and $(v,v_{n-1})$.\\

	Then there exists a vertex $v_j$ such that $(v_j,v),(v,v_{j+1})$ belongs to the arc set of $T$. Thus, $v_1,v_2,\dots,v_j,v,v_{j+1},\dots,v_{n-1}$ is a Hamilton path in $T$.
\end{proof}

\begin{theorem}[Moon]
	Every vertex of a diconnected tournament $T$ on $n$ vertices with $n \ge 3$ is contained in a directed $k$-cyclewhere $3 \le k \le n$.\\ In other words, every diconnected tournament is vertex-pancyclic.
\end{theorem}
\begin{proof}
	Let $T$ be a diconnected tournament with order $n \ge 3$. Let $u \in V(T)$, $S = N^+(u)$ and $S' = N^-(u)$. Since $T$ is diconnected, the set of all arcs from $S$ to $S'$, $[S,S']$ is nonempty, say $vw \in [S,S']$. Then $uvwu$ is a $3$-cycle in $T$. Thus, every vertex in a diconnected tournament belongs to a $3$-cycle.\\

	Suppose $u$ belongs to cycles of length $3 \le k \le p$ where $p < n$. It is enough to prove that $u$ belongs to a cycle of length $p+1$.\\

	By induction hypothesis, there exists a cycle $C : v_0,v_1,\dots,v_{p-1},v_0$ of length $p$ containing $v_0 = u$. Since $p < n$, $T$ has vertices which does not belong to the cycle $C$. Suppose $T$ has a vertex $v$ such that there exists arcs $v_i,v$ and $v,v_j$. Then there would exists two adjacent vertices $v_i,v_{i+1}$ such that arcs $v_iv,vv_{i+1}$ exists. Thus $u$ belongs to a cycle $v_0,v_1,\dots,v_i,v,v_{i+1},\dots,v_{p-1},v_0$ of length $p+1$.\\

	Suppose	no such vertex exists. Let $S$ be the set of vertices such that their inneighbourhood contains $C$ and $S'$ be the set of vertices such that their outneighbourhood contains $C$. Since $T$ is diconnected, both $S,S'$ are nonempty and there exists an arc $vw \in [S,S']$. Then $v_0,v,w,v_2,v_3,\dots,v_{p-1},v_0$ is a cycle of length $p+1$. By induction, every vertex of a diconnected graph belongs to cycles of length $k$ where $3 \le k \le n$.
\end{proof}

\begin{remark}
	Every diconnected tournament is Hamiltonian.\footnote{A digraph is Hamiltonian if it contains a directed spanning cycle.}
\end{remark}

\begin{exercise}
	Show that every tournament $T$ is diconnected or can be made into one by the reorientation of just one arc of $T$
\end{exercise}
\begin{proof}[Solution]
	By R\'edei's theorem, every tournament contains a directed spanning path, say $v_1,v_2,\dots,v_n$. Suppose $v_nv_1 \in A(T)$. Then $T$ is diconnected. Otherwise, $v_1v_n \in A(T)$ and we can obtain a diconnected tournament by reorienting the arc $v_1v_n$.
\end{proof}

\begin{exercise}
	Show that a tournament is diconnected if and only if it has a spanning directed cycle.
\end{exercise}
\begin{proof}[Solution]
	By Moon's theorem, every diconnected tournament is Hamiltonian. Suppose a digraph $T$ is Hamiltonian with directed spanning cycle $C$. Let $u,v \in V(T)$. Then $u-v$, $v-u$ sections of $C$ are directed $u-v$ path and directed $v-u$ path respectively. Therefore, every Hamiltonian digraph $T$ is diconnected.
\end{proof}

\begin{exercise}
	Show that every tournament of order $n$ has at most one vertex $v$ with $d^+(v) = n-1$.
\end{exercise}
\begin{proof}[Solution]
	Let $T$ be a tournament of order $n$. Since the underlying graph is complete,  the degree of every vertex of $T$ is $n-1$. Suppose there exists a vertex $u$ with outdegree $n-1$. Then indegree of $u$ is zero. Thus, $u$ is not an outneighbour for any vertex in $T$. Thus, $T$ has at most one vertex with outdegree $n-1$.
\end{proof}

\begin{exercise}
	Show that for each positive integer $n \ge 3$, there exists a non-Hamiltonian tournament of order $n$.
\end{exercise}
\begin{proof}[Solution]
	Let $T$ be a tournament of order $n \ge 3$. Suppose $T$ has a vertex $v$ with outdegree( or indegree) $n-1$. Then $T$ is not Hamiltonian.
\end{proof}

\begin{exercise}
	Show that if a tournament contains a directed cycle, then it contains a directed cycle of length $3$.
\end{exercise}
\begin{proof}[Solution]
	Suppose a tournament $T$ is Hamiltonian. Then $T$ is diconnected. By Moon's theorem, a diconnected tournament $T$ of order $n \ge 3$ contains a cycle of length $3$.
\end{proof}

\begin{exercise}
	Show that every tournament $T$ contains a vertex $v$ such that every other vertex of $T$ is reachable from $v$ by a directed path of length at most $2$.
\end{exercise}
\begin{proof}[Solution]
	Suppose there are no such vertices in a tournament $T$. Then every vertex in $T$ must have a vertex at distance $3$.\\

	Let $u$ be vertex of $T$, there should exists at least one vertex $v$ such that $d(u,v) = 3$ and $u$ in an out-neighbour of $v$. By hypothesis, there is another vertex $w$ such that $d(v,w) = 3$ and arcs $(w,v), (w,u)$ belongs to $T$. By hypothesis, there should exists a vertex different from $u$ and $v$, which is at a distance $3$ from $w$ and the vertices $u,v,w$ are out-neighbours of $x$.\\

	Continuing like this, we get a tournament $T$ with infinite number of vertices which is a contradiction.
\end{proof}
\pagebreak

{\Large Module 2}
%\chapter{Connectivity}
\section{Connectivity}
\setcounter{subsection}{1}
\subsection{Vertex/Edge Cut}
\begin{definition}
	Let $G$ be a connected graph. A proper subset $S \subset V(G)$ is a \textbf{vertex cut} if $G-S$ is disconnected. Let $S' = V(G)-S$. Then the set of all edge $[S,S']$ is an \textbf{edge cut}.\\

	If vertex cut is singleton, then its a \textbf{cut vertex}. Similarly, if $[S,S']$ is singleton, then the edge cut is a \textbf{cut edge}.
\end{definition}
\begin{remark}
	If $e = uv$ belongs to an edge cut $[S,S']$, then every edge parallel to $uv$ also belongs to the edge cut. Edge cuts won't contain any loop.\\

	The set of edges $E$ such that $G-E$ is disconnected is a \textbf{separating set of edges}.
\end{remark}

\begin{exercise}
	Show that if $\{x,y\}$ is a $2$-edge cut of a graph $G$, then every cycle of $G$ that contains $x$ also contains $y$.
\end{exercise}
\begin{proof}[Solution]
	Let $\{x,y\}$ be an edge cut $[S,S']$ of a graph $G$. Suppose there exists a cycle $C$ containing the edge $x = uv$. Then the cycle should contain another edge from $[S,S']$. Since $y$ is the only remaining edge in $[S,S']$, the cycle should contain $y$. Thus, every cycle containing $x$ (if any) contains $y$.
\end{proof}

\begin{theorem}
	A vertex $v$ of a connected graph $G$ with at least three vertices is a cut vertex if and only if there exists vertices $u$ and $w$ of $G$ distinct from $v$ such that $v$ is in every $u-w$ path in $G$.
\end{theorem}
\begin{proof}
	Suppose $v$ belongs to every $u-w$ path in $G$. Then $G-v$ won't have any $u-v$ path. Therefore, $G-v$ is disconnected and $v$ is a cut vertex.\\

	Suppose $v$ is a cut vertex of $G$. Then $G-v$ has at least two components, $G_1,G_2$. Let $u \in V(G_1)$ and $w \in V(G_2)$. Since $G$ is connected, there exists a $u-w$ path in $G$. Clearly, $v$ belongs to every such $u-w$ path in $G$. Otherwise we have a contradiction since the vertices $u,w$ are chosen from two different components of $G-v$.
\end{proof}

\begin{theorem}
	An edge $e = xy$ of a connected graph $G$ is a cut edge of $G$ if and only if $e$ belongs to no cycle of $G$.
\end{theorem}
\begin{proof}
	Let $e = xy$ be a cut edge of $G$. Then $G-e$ has two components $G_1$ and $G_2$. Suppose $e$ belongs to a cycle $C : x,y,\dots,w,x$. Then $C$ must contain another edge between $G_1$ and $G_2$ which is a contradiction.\\

	Suppose $e = xy$ is not a cut edge of $G$. Then $G-e$ is connected. Therefore, between $x$ and $y$ there exists a path $P : x,u,\dots,y$ in $G-e$. Clearly, the path $P$ together with the edge $e$ is a cycle containing $e$. By contrapositivity, if there is no cycle containing $e$, then $e$ is a cut edge of $G$.
\end{proof}

\begin{theorem}
	An edge $e = xy$ is a cut edge of a connected graph $G$ if and only if there exists vertices $u,v$ such that $e$ belongs to every $u-v$ path in $G$.
\end{theorem}
\begin{proof}
	Let $G$ be a connected graph. Let $e = xy$ be a cut ege of $G$. Since $G$ is connected, there exists at least one $u-v$ path in $G$. Since $e$ is a cut edge of $G$, $G-e$ is a disconnected graph with components $G_1$ and $G_2$. Thus, there exists vertices $u \in V(G_1)$ and $v \in V(G_2)$. Clearly, there is not $u-v$ path in $G-e$. Therefore, it is evident that every $u-v$ path in $G$ contains the edge $e$.\\

	Let $G$ be a connected graph. Suppose there exists two vertices $u,v$ such that an edge $e$ belongs to every $u-v$ path in $G$. Then there is not $u-v$ path in $G-e$. Thus, $G-e$ is disconnected and edge $e$ is a cut edge of $G$.
\end{proof}

\begin{theorem}
	A connected graph $G$ with at least two vertices contains at least two vertices that are not cut vertices.
\end{theorem}
\begin{proof}
	Let $G$ be a connected graph. If $G$ has only two vertices, then $G \cong K_2$. And both vertices are not cut vertices.\\

	Suppose $n \ge 3$. Let $u,v$ be two vertices of $G$ such that $d(u,v)$ is maximum. We claim that both $u,v$ are not cut vertices of $G$. Suppose that $u$ is a cut vertex of $G$. Then $G$ has at least two components $G_1$ and $G_2$ such that $v \in G_1$ and $w \in G_2$. Then every $v-w$ path in $G$ contains $u$. Then $d(u,v) < d(w,v)$ which is a contradiction. Therefore, $u$ is not a cut vertex. Similarly, $v$ is not a cut vertex.
\end{proof}

\begin{proposition}
	A simple cubic connected graph $G$ has a cut vertex if and only if it has a cut edge.
\end{proposition}
\begin{proof}
	Let $G$ be a simple cubic connected graph.\\

	Suppose $G$ has a cut vertex $u$ with adjacent vertices $v,w,x$. Then $G-u$ has either two or three component. If $G-u$ has only two components, then $G$ has a cut edge $uv$. If $G-u$ has three components, WLoG suppose that $v,w$ belongs the the same component. Then $G$ has a cut edge $ux$.\\

	Suppose $G$ a cut edge $uv$. Then $G$ has a cut vertex $u$.
\end{proof}

\begin{exercise}
	Find the vertex cuts and edge cuts of the graph
\end{exercise}
\begin{proof}[Solution]
	do it yourself
\end{proof}

\begin{exercise}
	Prove or disprove : Let $G$ be a simple connected graph with $n \ge 3$. Then $G$ has a cut edge if and only if $G$ has a cut vertex.
\end{exercise}
\begin{proof}[Solution]
	do it yourself
\end{proof}

\begin{exercise}
	Show that in a graph, the number of edges common to a cycle and an edge cut is even.
\end{exercise}
\begin{proof}[Solution]
	Let $G$ be a graph and $C$ be a cycle in $G$. Let $S,S'$ be a partition of $V(G)$ such that $C$ contains at least one edge from $[S,S']$. Since $C$ is a closed trail, for each edge in $C$ from $[S,S']$, $C$ must contain another edge from $[S,S']$. Thus the number of edges common to a cycle and an edge cut is always an even integer.
\end{proof}

\subsection{Connectivity and Edge connectivity}
\begin{definition}
	The (vertex) \textbf{connectivity}, $\kappa(G)$ is the minimum number $k$ for which there exists a $k$-vertex cut.\\

	If $G$ is trivial or disconnected, then $\kappa = \lambda = 0$. Thus, $\kappa(K_n) = n-1$.
\end{definition}

\begin{exercise}
	Prove that a simple graph $G$ with $n$ vertices is complete if and only if $\kappa(G) = n-1$.
\end{exercise}
\begin{proof}[Solution]
	We know that $\kappa(K_n) = n-1$. Suppose $\kappa(G) = n-1$. Suppose $G$ has a non-adjacent pair of vertices $u,v$. Then $V(G)-\{u,v\}$ is a vertex cut of $G$ and $\kappa(G) \le n-2$. This is a contradiction. Thus, every pair of vertices are adjacent in $G$. Therefore $G$ is complete.
\end{proof}
\begin{definition}
	The \textbf{edge connectivity}, $\lambda(G)$ is the minimum number $k$ for which there exists a $k$-edge cut.
\end{definition}

\begin{exercise}
	Let $[S,S']$ be a minimum edge cut. Then $G-[S,S']$ has exactly two components.
\end{exercise}
\begin{proof}[Solution]
	Let $G$ be a connected graph with minimum edge cut $[S,S']$. Suppose $G-[S,S']$ has more than two components. Let $G_1$ be one of the components. Define $M = V(G_1)$. Then $[M,M']$ is a proper subset of $[S,S']$ and $G-[M,M']$ is disconnected. This is a contradiction as $[S,S']$ is a minimum edge cut. Therefore, $G - [S,S']$ has only two components.
\end{proof}

\begin{definition}
	A graph $G$ is $r$-connected if $\kappa(G) \ge r$. Similarly, $G$ is $r$-edge connected if $\lambda(G) \ge r$.
\end{definition}

\begin{theorem}
	For any loopless connected graph $G$,
	$$ \kappa(G) \le \lambda(G) \le \delta(G) $$
\end{theorem}
\begin{proof}
	Suppose $G$ is connected. Otherwise the result is trivial. Let $\delta$ be the minimum degree $\delta(G)$. Then there exists a vertex $v$ such that $d(v) = \delta$. Consider $S = \{v\}$ and $S'= V(G)-\{v\}$. Then $[S,S']$ is the set of all edges incident on $v$. Clearly, $[S,S']$ is an edge cut which is not necessarily one of the smallest edge cuts. Thus, $\lambda(G) \le k = \delta(G)$.\\

	Let $E$ be an edge cut of $G$ with cardinality $\lambda = \lambda(G)$. Since $G$ is connected, $\lambda \ge 1$. Thus there exists an edge $uv \in E$. For each edge in $G$, delete an end vertex which is neither $u$ nor $v$. Clearly, the number of vertices deleted, $t \le \lambda-1$ and the deletion of $t$ vertices results in the deletion of every edge in $E$ except the edge $uv$. The deletion of either $u$ or $v$ from this graph gives a disconnected/trivial graph. Thus $\kappa(G) = t+1 \le \lambda$.
\end{proof}

\begin{exercise}
Prove or disprove : If $H$ is a subgraph of $G$, then\\ \begin{enumerate*} \item $\kappa(H) \le \kappa(G)$ \item $\lambda(H) \le \lambda(G)$ \end{enumerate*}
\end{exercise}
\begin{proof}[Solution]
	do it yourself
\end{proof}

\begin{exercise}
	Determine $\lambda(K_n)$ 
\end{exercise}
\begin{proof}[Solution]
	The complete graph $K_n$ is $(n-1)$-regular and $\lambda(K_n) \le n-1$.
	Suppose $[S,S']$ be an edge cut of $K_n$. Then $[S,S'] = \binom{n-1}{k}$ where $|S| = k$. Clearly, $[S,S']$ is minimum, when $k$ is extremum. Therefore, $k = 1$ and $\lambda(K_n) = n-1$.
\end{proof}

\begin{exercise}
	Determine the connectivity and edge connectivity of the Petersen graph $P$.
\end{exercise}
\begin{proof}[Solution]
	The Petersen graph $P$ is a simple, connected, cubic graph.  \footnote{This theorem is proved next, but the proof does not depend on this result.} Therefore, $1 \le \kappa = \lambda \le 3$. Clearly, every edge cut $[S,S']$ of $P$ has cardinality $\ge 3$. Therefore, $\kappa = \lambda = 3$.
\end{proof}

\begin{theorem}
	The connectivity and edge connectivity of a simple cubic graph $G$ are equal.
\end{theorem}
\begin{proof}
	Trivially, if simple cubic graph $G$ is disconnected, then $\lambda = \kappa = 0$.
	We know that a simple cubic connected graph has a cut vertex if and only if it has a cut edge. Therefore, $\lambda = 1$ if and only if $\kappa = 1$.
	Also we know that $\kappa \le \lambda \le \delta$. Thus, if $\kappa = 3$ then $\lambda = 3$. It is enough to prove that if $\kappa = 2$, then $\lambda = 2$.\\

	Let $\kappa(G) = 2$ and $\{ u,v \}$ be a vertex cut of $G$. Since both $G-u$ and $G-v$ are connected, both $u,v$ must have an edge with every component of $G-\{ u,v \}$. If $G-\{ u, v \}$ has three components $G_1,G_2,G_3$, then the two edges incident on $G_1$ is an edge cut of $G$. If $G-\{ u,v\}$ has only two components $G_1$ and $G_2$, then there are three possibilities
	\begin{enumerate}
		\item WLoG $u$ has two edges incident with $G_1$ and one edge incident with $G_2$, $v$ has two edges incident with $G_1$ and one edge incident with $G_2$.
		\item WLoG $u$ has two edges incident with $G_1$ and one edge incident with $G_2$, $v$ has two edge incident with $G_2$ and one edge incident with $G_1$.
		\item $u$ is adjacent to $v$ and both $u$ and $v$ have one edge each incident with $G_1$ and $G_2$.
	\end{enumerate}
	In the first case, the two edges incident with $G_2$ forms an edge cut of $G$. In the second case, the edge from $u$ incident with $G_2$ and the edge from $v$ incident with $G_1$ forms an edge cut of $G$. In the third case, the two edges incident on $G_1$ forms an edge cut of $G$. Therefore, if $\kappa = 2$ then $\lambda = 2$. Thus, $\kappa = 2$ if and only if $\lambda = 2$.\\

	Suppose $\lambda = 3$. Then $\kappa \ne 0$ since $G$ is connected, $\kappa \ne 1$ since $\lambda \ne 1$, and $\kappa \ne 2$ since $\lambda \ne 2$. Therefore, $\kappa = 3$. Therefore, $\kappa = 3$ if and only if $\lambda = 3$. Clearly, $\kappa = \lambda$ for a simple, cubic graph.
\end{proof}

\begin{exercise}
	Give examples of cubic graphs $G_1,G_2,G_3$ with $\kappa = 1,2,3$.
\end{exercise}
\begin{proof}[Solution]
	do it yourself
\end{proof}

\begin{definition}
	A family of paths are \textbf{internally disjoint} if no vertex is an internal vertex for more than one path in it.
\end{definition}

\begin{theorem}[Whitney]
	A graph with at least three vertices is $2$-connected if and only if any two vertices of $G$ are connected by at least two internally disjoint paths.
\end{theorem}
\begin{proof}
	Let $G$ be a $2$-connected graph with at least three vertices. Let $u,v$ be two distinct vertices of $G$. The proof is by induction on the distance $d(u,v)$.\\
	
	Suppose $d(u,v) = 1$. Then $e = uv$ is an edge of $G$. Since $G$ is $2$-connected graph with at least three vertices, $G$ does not have a cut edge. Therefore $G-e$ is connected. That is, there is $u-v$ path $P$ in $G-e$. Then the path $P$ is internally disjoint from the the path $u-v$. Clearly, there two internally disjoint path between $u$ and $v$.\\

	Suppose a $2$-connected graph $G$ has two internally disjoint paths between vertices $x,y$ whenever $d(x,y) = k-1$. Let $d(u,v) = k$. Then there exists a vertex $w$ such that $w$ is adjacent to $v$ and $d(u,w) = k-1$. By induction hypothesis, there exists two internally disjoint $u-w$ paths $P,Q$ in $G$. Since $G$ is $2$-connected, $G-w$ is connected and there exists a $u-v$ path $R$ in $G-w$.\\

	Now using the path $P_1,P_2$ and $Q$ we can construct two internally disjoint $u-v$ paths in $G$. Let $x$ be a vertex on path $Q$ such that the $x-v$ section of $Q$ contains only $x$ in common with $P_1 \cup P_2$. WLoG suppose $x \in P_1$. Then the $u-x$ section of $P_1$ together with $x-v$ section of $Q$ forms a $u-v$ path. And the other $u-w$ path $P_2$ together with the edge $w-v$ forms a $u-v$ path which internally disjoint from the former. Thus there exists two internally disjoint paths between any two vertices $u,v$ of a $2$-connected graph.\\

	Suppose any two vertices $u,v$ of a graph has two internally disjoint paths. Then $G$ does not have any cut vertex. Therefore, $G$ is $2$-connected.
\end{proof}

\begin{theorem}
	A graph $G$ with at least three vertices is $2$-connected if and only if any two vertices of $G$ lie on a common cycle.
\end{theorem}
\begin{proof}
	Let $G$ be a $2$-connected graph with at least three vertices. Let $u,v$ be any two vertices of $G$. There there exists two internally disjoint paths $P,Q$ between $u$ and $v$. Then $P \cup Q$ is a cycle containing both $u$ and $v$.\\

	Suppose any two vertices $u,v$ of $G$ belongs to a common cycle. Then, there are two internally disjoint paths between any two pair of vertices in $G$. Therefore, by Whitney's theorem $G$ is $2$-connected.
\end{proof}

\begin{remark}
	A $2$-connected graph $G$ has at least two internally disjoint $u-v$ paths. However, given a $u-v$ path $P$ there may not exist another $u-v$ path $Q$ such that $P,Q$ are internally disjoint.
\end{remark}

\begin{exercise}
	\begin{enumerate}
		\item Show that a graph $G$ with at least three vertices is $2$-connected if and only if any vertex and any edge of $G$ lie on a common cycle of $G$
		\item Show that a graph $G$ with at least three vertices is $2$-connected if and only if any two edges of $G$ lie on a common cycle.
	\end{enumerate}
\end{exercise}
\begin{proof}[Solution]
	Let $G$ be a $2$-connected graph. Let $v$ be a vertex and $e = xy$ be an edge of $G$.
	yet to complete
\end{proof}

\begin{exercise}
	Prove that a graph is $2$-connected if and only if for every pair of disjoint connected subgraphs $G_1$ and $G_2$, there exist two internally disjoint paths $P_1$ and $P_2$ of $G$ between $G_1$ and $G_2$.
\end{exercise}
\begin{proof}[Solution]
\end{proof}

\begin{exercise}
	Prove that a graph $G$ with $n \ge 3$ is $2$-edge connected if and only if any two distinct vertices of $G$ are connected by at least two edge-disjoint paths in $G$.	
\end{exercise}
\begin{proof}[Solution]
\end{proof}

\begin{exercise}
	\begin{enumerate*}
		\item Disprove : $\kappa(G) = \kappa(L(G))$.
		\item Prove : $\lambda(G) = \kappa(L(G))$. Give an example of a graph $G$ for which $\lambda(G) < \kappa(L(G))$.
	\end{enumerate*}
\end{exercise}
\begin{proof}[Solution]
\end{proof}

\begin{theorem}
	In a $2$-connected graph $G$ any two longest cycles have at least two vertices in common.
\end{theorem}
\begin{proof}
\end{proof}

\begin{theorem}
	A connected simple graph $G$ is $3$-edge connected if and only if every edge of $G$ is the intersection of the edge sets of two cycles of $G$.
\end{theorem}
\begin{proof}
\end{proof}

\subsection{Blocks}
\begin{definition}
	A graph $G$ is \textbf{nonseparable} if it is nontrivial, connected and has no cut vertices. A \textbf{block} is a maximal nonseparable subgraph of $G$.\\

	If $G$ has no cut vertices, then $G$ itself is a block. A block having exactly one cut vertex is an \textbf{end block} of $G$.
\end{definition}

\begin{remark}
Let $G$ be a connected graph with $n \ge 3$. Then
\begin{enumerate}
	\item Each block of $G$ with at least three vertices is a $2$-connected subgraph of $G$.
	\item Each edge of $G$ belongs to exactly one block. Thus, $G$ is an edge disjoint union of its blocks.
	\item Any two blocks of $G$ have at most one vertex in common. Such a common vertex is a cut vertex of $G$.
	\item A vertex of $G$ belongs to exactly on block if it is not a cut vertex.
	\item A vertex of $G$ is a cut vertex if and only if it belongs to at least two blocks of $G$.
\end{enumerate}
\end{remark}

\begin{theorem}
	If $C$ is any cycle of a simple block $G$, then there exists a sequence of nonseparable subgraphs $C = B_0,B_1,\dots,B_r = G$ such that $B_{i+1}$ is an edge disjoint union of $B_i$ and a path $P_i$ where only vertices in common with $B_i$ and $P_i$ are the edge vertices of $P_i$, $0 \le i \le r-1$.
\end{theorem}
\begin{proof}
\end{proof}

\subsection{Cyclic Edge Connectivity}
\begin{definition}
	Let $G$ be a simple connected graph containing at least two disjoint cycles. Then the \textbf{cyclic edge connectivity} of $G$, $\lambda_c(G)$ is the minimum number of edges whose deletion gives a graph with two components each containing a cycle.
\end{definition}

\begin{exercise}
	Show that the cyclic edge connectivity of the Petersen graph is $5$.
\end{exercise}
\begin{proof}[Solution]
	do it yourself
\end{proof}

%\chapter{Trees}
\section{Tree}
\setcounter{subsection}{1}
\subsection{Properties}
\subsection{Centers and Centroids}
\subsection{Counting the number of spanning trees}
\subsection{Cayley's Formula}
\setcounter{subsection}{6}
\subsection{Applications}
\subsubsection{The connector problem}
\subsubsection{Kruskal's Algorithm}
\subsubsection{Prim's Algorithm}
\subsubsection{Shortest path problems}
\subsubsection{Dijkstra's Algorithm}

%\chapter{Matchings and Factors}
\setcounter{section}{5}
\pagebreak

{\Large Module 3}
%\chapter{Eulerian \& Hamiltonian Graphs}
\section{Eulerian \& Hamiltonian Graphs}
\setcounter{subsection}{1}
\subsection{Eulerian Graphs}
\begin{definition}
	An \textbf{Euler trail} is a spanning trail that contains all edges of $G$. An \textbf{Euler tour} is a closed Euler trail of $G$. Graph $G$ is \textbf{Eulerian} if it has an Euler tour.
\end{definition}

\begin{theorem}
	For a nontrivial connected graph $G$, the following are statements are equivalent
	\begin{enumerate}
		\item $G$ is Eulerian.
		\item The degree of each vertex of $G$ is an even positive integer.
		\item $G$ is an edge-disjoint union of cycles.
	\end{enumerate}
\end{theorem}
\begin{proof}
\end{proof}

\begin{theorem}
	A graph $G$ is Eulerian if and only if each edge $e$ of $G$ belongs to an odd number of cycles of $G$.
\end{theorem}
\begin{proof}
\end{proof}

\begin{corollary}
	A graph is Eulerian if and only if it has an odd number of cycle decompositions.
\end{corollary}
\begin{proof}
\end{proof}

\subsection{Hamiltonian Graphs}
\begin{theorem}
	If $G$ is Hamiltonian, then for every nonempty proper subset $S$ of $V$, $\omega(G-S) \le |S|$.
\end{theorem}
\begin{proof}
\end{proof}

\begin{theorem}[Ore]
	Let $G$ be a simple graph with $n \ge 3$ vertices. If, for every pair of nonadjacent vertices $u,v$ of $G$, $d(u) + d(v) \ge n$, then $G$ is Hamiltonian.
\end{theorem}
\begin{proof}
\end{proof}

\begin{corollary}[Dirac]
	If $G$ is a simple graph with $n \ge 3$ and $\delta \ge \frac{n}{2}$, then $G$ is Hamiltonian.
\end{corollary}
\begin{proof}
\end{proof}

\begin{theorem}
	Let $G$ be a simple graph of order $n \ge 3$. Then $G$ is Hamiltonian if and only if $G + uv$ is Hamiltonian for every pair of nonadjacent vertices $u$ and $v$ with $d(u) + d(v) \ge n$.
\end{theorem}
\begin{proof}
\end{proof}

\begin{definition}
	The \textbf{closure} of a graph $G$ is the maximal supergraph of $G$ obtained by recursively joining pairs of nonadjacent vertices whose degree sum is  at least $n$.
\end{definition}

\begin{theorem}
	The closure $cl(G)$ of a graph $G$ is well defined.
\end{theorem}
\begin{proof}
\end{proof}

\begin{theorem}
	If $cl(G)$ is Hamiltonian, then $G$ is Hamiltonian.
\end{theorem}
\begin{proof}
\end{proof}

\begin{corollary}
	If $cl(G)$ is complete, then $G$ is Hamiltonian.
\end{corollary}
\begin{proof}
\end{proof}

\begin{theorem}[Chv\'atal, Erd\"s]
	If, for a simple $2$-connected graph $G$, $\alpha \le \kappa$, then $G$ is Hamiltonian.
\end{theorem}
\begin{proof}
\end{proof}

\begin{definition}
	A graph $G$ is \textbf{Hamiltonian-connected} if any two vertices of $G$ are connected by a Hamiltonian path.
\end{definition}

 \begin{theorem}
	If $G$ is a simple graph with $n \ge 3$ vertices such that $d(u)+d(v) \ge n+1$ for every pair of nonadjacent vertices of $G$, then $G$ is Hamiltonian-connected.
\end{theorem}
\begin{proof}
\end{proof}

%\chapter{Graph Colorings}
\section{Graph Colorings}
\setcounter{subsection}{1}
\subsection{Vertex Colorings}
\begin{commentary}
	Usually we will admit a definition and then a few equivalent conditions. Here the author is suggesting two equivalent definitions for the chromatic number.
\end{commentary}

\begin{definition}
	The \textbf{chromatic number} $\chi(G)$ of a graph $G$ is the minimum number of indepedent subsets that partition the vertex set of $G$. Any such minimum partition is a \textbf{chromatic partition} of $V(G)$.
\end{definition}

\begin{definition}
	The \textbf{vertex coloring} is a map $f : V(G) \to \mathbb{N}$.\footnote{The author suggests $f : V(G) \to S$ where $S$ is the set of colors. I would strongly suggest against it as we are not interested in colors, but chromatic partition.}  And a vertex coloring is \textbf{proper} if adjacent vertices have different colors. The \textbf{chromatic number} of a graph $G$ is the minimum number of colors needed for a proper vertex coloring of $G$. A graph $G$ is $k$-chromatic if $\chi(G) = k$.
\end{definition}

\begin{definition}
	A $k$-coloring of a graph $G$ is a vertex coloring of $G$ that used at most $k$ colors.
\end{definition}

\begin{definition}
	A graph is $k$-colorable if $G$ has a proper vertex coloring that uses at most $k$ colors.
\end{definition}

\begin{theorem}
	If graph $G$ has $n$ vertices and independence number $\alpha$,
	$$ \frac{n}{\alpha} \le \chi \le n-\alpha+1 $$
\end{theorem}
\begin{proof}
\end{proof}

\begin{theorem}
	For any simple graph $G$
	$$ 2\sqrt{n} \le \chi + \chi^c \le n+1 \quad \text{and} \quad n \le \chi\chi^c \le \left(\frac{n+1}{2}\right)^2 $$
\end{theorem}
\begin{proof}
\end{proof}

\subsection{Critical Graphs}
\begin{definition}
	A graph $G$ is \textbf{critical} if for every proper subgraph $H$ of $G$, $\chi(H) < \chi(G)$. A graph $G$ is $k$-critical if it is $k$-chromatic and critical.
\end{definition}

\begin{remark}
	If $\chi(G) = 1$, then $G$ is either trivial or totally disconnected. If $\chi(G) = 2$, then $G$ is bipartite. If $\chi(G) = 3$, then $G$ has an odd cycle.\\

	Graph $G$ is $1$-critical if and only if $G \cong K_1$. Graph $G$ is $2$-critical if and only if $G \cong K_2$. Graph $G$ is $3$-critical if and only if $G \cong C_n$ where $n$ is odd.
\end{remark}

\begin{theorem}
	If $G$ is $k$-critical, then $\delta(G) \ge k-1$.
\end{theorem}
\begin{proof}
\end{proof}

\begin{corollary}
	For any graph $G$, $\chi(G) \le 1 + \Delta(G)$.
\end{corollary}
\begin{proof}
\end{proof}

\begin{theorem}
	In a critical graph $G$, no vertex cut is a clique.
\end{theorem}
\begin{proof}
\end{proof}

\begin{corollary}
	Every critical graph is a block.
\end{corollary}
\begin{proof}
\end{proof}

\subsubsection{Brook's Theorem}
\begin{theorem}
	If a connected graph $G$ is neither an odd cycle nor a complete graph, then $\chi(G) \le \Delta(G)$.
\end{theorem}
\begin{proof}
\end{proof}
\pagebreak

{\Large Module 4}
%\chapter{Planarity}
\section{Planarity}
%\subsection{Introduction}
\setcounter{subsection}{1}
\subsection{Planar and Nonplanar Graphs}
\begin{definition}
	A graph $G$ is \textbf{planar} if there exists a drawing of $G$ in the plane in which no two edges intersect in a point other than a vertex of $G$, where each edge is a Jordan arc. The drawing of $G$ is the \textbf{plane representation} of $G$.
	A Plane embedding of planar graph $G$ is a \textbf{plane graph} $G$.
\end{definition}

\begin{example}
	$K_4$ is a Planar Graph. All trees, cycles and wheels are planar. Petersen graph is nonplanar.
\end{example}

\begin{definition}
	Define the relation $\sim$ such that $A \sim B$ if there exists a Jordan arc from $A$ to $B$. Clearly, it is an equivalence relation on any subset of $\mathbb{R}^2$.
	The planar representation of a graph $G$ (if exists) divides the rest of the plane $\pi$ into equivalent classes under the relation $\sim$. These equivalence classes are the \textbf{faces} of the planar graph.
\end{definition}

\begin{remark}
\begin{itemize}
	\item A connected graph is a tree if and only if it has only one face.
	\item Any plane graph has exactly one unbounded face.
\end{itemize}
\end{remark}

\begin{theorem}
	A graph is planar if and only if it is embeddable on a sphere.
\end{theorem}
\begin{proof}
\end{proof}

\begin{theorem}
\begin{enumerate}
	\item Let $G$ be a plane graph and $f$ be a face of $G$. Then there exists a plane embedding of $G$ in which $f$ is the exterior face.
	\item Let $G$ be a planar graph. Then $G$ can be embedded on a plane such that any specific vertex(edge) belongs to the exterior of the plane graph.
\end{enumerate}
\end{theorem}
\begin{proof}
\end{proof}

\subsection{Euler Formula and its consequences}
\begin{theorem}
	For a connected, plane graph $G$, $n - m + f = 2$, where $n,m,f$ are number of vertices, edges and faces of $G$ respectively.
\end{theorem}
\begin{proof}
\end{proof}

\begin{corollary}
	All plane embeddings of a given planar graph have the same number of faces.
\end{corollary}
\begin{proof}
\end{proof}

\begin{corollary}
	If $G$ is a simple, planar graph with at least three vertices, then $m \le 3n-6$.
\end{corollary}
\begin{proof}
\end{proof}

\begin{example}
	The complement of a simple, planar graph with $11$ vertices is nonplanar.
\end{example}

\begin{corollary}
	For a simple, planar graph $G$, $\delta(G) \le 5$.
\end{corollary}
\begin{proof}
\end{proof}

\begin{definition}
	The girth of a graph $G$ is the length of the shortest cycle in $G$.
\end{definition}

\begin{theorem}
	If the girth $k$ of a connected plane graph is at least $3$, then $$m \le \frac{k(n-2)}{k-2}$$
\end{theorem}
\begin{proof}
\end{proof}

\begin{corollary}
	The Petersen graph is nonplanar.
\end{corollary}
\begin{proof}
\end{proof}

\begin{exercise}
	Show that every simple bipartite cubic planar graph contains a $C_4$.
\end{exercise}
\begin{proof}
\end{proof}

\begin{definition}
	A nonplanar graph $G$ is planar-vertex-critical if $G-v$ if planar for every vertex $v$ of $G$.
\end{definition}

\begin{exercise}
	Prove that every planar-vertex-critical graph is $2$-connected.
\end{exercise}
\begin{proof}
\end{proof}

\begin{exercise}
	Let $G$ be a simple, planar, cubic graph with eight faces. Determine $n(G)$. Draw two such graphs that are nonisomorphic.
\end{exercise}
\begin{proof}
\end{proof}

\begin{exercise}
	Prove that If $G$ is a simple, connected, planar, bipartite graph, then $m \le 2n-4$ where $n \ge 3$.
\end{exercise}
\begin{proof}
\end{proof}

\begin{exercise}
	Prove that if $G$ is a simple, planar graph with at least four vertices, then there exists at least four vertices with degree at most $5$.
\end{exercise}
\begin{proof}
\end{proof}

\begin{exercise}
	If $G$ is a nonplanar graph, show that it has either five vertices of degree at least $4$ or six vertices of degree at at least $3$.
\end{exercise}
\begin{proof}
\end{proof}

\begin{exercise}
	A simple planar graph with minimum degree at least five contains at least 12 vertices. Draw simple planar graph with $12$ vertices and minimum degree $5$.
\end{exercise}
\begin{proof}
\end{proof}

\begin{exercise}
	Show that there is no $6$-connected planar graph.
\end{exercise}
\begin{proof}
\end{proof}

\begin{exercise}
	Let $G$ be a plane graph of order $n$ and size $m$ in which every face is bounded a $k$-cycle. Show that $m = \frac{k(n-2)}{k-2}$.
\end{exercise}
\begin{proof}
\end{proof}

\begin{definition}
	A graph $G$ is \textbf{maximal planar} if $G$ is planar, but for any pair of nonadjacent vertices $u,v$ of $G$, $G+uv$ is nonplanar.
\end{definition}

\begin{definition}
	A plane triangulation is plane graph in which each of its faces are bounded by a triangle.
\end{definition}

\begin{remark}
	Show that a plane embedding of a simple, maximal planar graph is a plane triangulation.
\end{remark}

\begin{exercise}
	Embed a $3$-cube $Q_3$ in a maximal planar graph having the same vertex set as $Q_3$. Count the number of new edges added.
\end{exercise}
\begin{proof}
\end{proof}

\begin{exercise}
	Prove that for a simple, maximal planar matrix on $n \ge 3$ vertices, $m = 3n-6$.
\end{exercise}
\begin{proof}
\end{proof}

\begin{exercise}
	For any simple planar graph, $m \le 3n - 6$.
\end{exercise}
\begin{proof}
\end{proof}

\begin{exercise}
	Show that every triangulation of order $n \ge 4$ is $3$ connected.
\end{exercise}
\begin{proof}
\end{proof}

\begin{exercise}
	Let $G$ be a maximal planar graph with $n \ge 4$. Let $n_i$ be the number of vertices of degree $i$ in $G$. Prove that $$ 3n_3 + 2n_4 + n_5 = 12 + n_7 + 2n_8 + 3n_9 + 4n_{10} + \dots $$
\end{exercise}
\begin{proof}
\end{proof}

\begin{exercise}
	Generalise Euler Forumula for disconnected plane graphs.
\end{exercise}
\begin{proof}
\end{proof}

\subsection{$K_5$ and $K_{3,3}$ are Nonplanar Graphs}
\begin{theorem}
	$K_5$ is nonplanar.
\end{theorem}
\begin{proof}
\end{proof}

\begin{theorem}
	$K_{3,3}$ is nonplanar.
\end{theorem}
\begin{proof}
\end{proof}

\begin{exercise}
	Find the maximum number of edges in a planar complete tripartite graph with each part of size at least $2$.
\end{exercise}
\begin{proof}
\end{proof}

\begin{remark} $K_5, K_{3,3}$ have some common features
\begin{enumerate}
	\item both are regular.
	\item both give planar graph when a vertex(or an edge) is deleted.
	\item both give planar graph on contraction of an edge.
	\item both are smallest nonplanar graphs.
\end{enumerate}
\end{remark}

%\subsection{Dual of a Planar Graph}
%\subsection{The Four-Colour Theorem and the Heawood Five-Colour Theorem}
%\subsection{Kuratowski's Theorem}
%\subsection{Hamiltonian Plane Graphs}
%\subsection{Tait Colouring}

%\chapter{Triangulated Graphs}
%\chapter{Domination in Graphs}
%\chapter{Spectrum of Graphs}

\setcounter{section}{10}
\section{Spectral Properties of Graphs}
%\subsection{Introduction}

\setcounter{subsection}{1}
\subsection{The Spectrum of a Graph}
\begin{definition}
	The set of eigenvalues of the adjacency matrix of a graph $G$ is the \textbf{spectrum} of $G$. The spectrum is denoted by $Sp(G)$.
\end{definition}

\subsection{Spectrum of the Complete Graph $K_n$}
\subsection{The Spectrum of the Cycle $C_n$}
\subsubsection{Coefficients of the Characteristic Polynomial}
%\subsection{The Spectra of Regular Graphs}
%\subsection{Spectrum of the Complete Bipartite Graph $K_{p,q}$}
%\subsection{The Determinant of the Adjacency Matrix of a Graph}
%\subsection{Spectra of Product Graphs}
%\subsection{Cayley Graphs}
%\subsection{Strongly Regular Graphs}
%\subsection{Ramanujan Graphs}
%\subsection{The Energy of a Graph}
%\subsection{Energy of Mycielskian of a Regular Graph}
