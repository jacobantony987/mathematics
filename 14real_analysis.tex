%Text Books: \cite{apostol}, \cite{rudin}
%Module 1: Functions of bounded variation and rectifiable curves
%Introduction, properties of monotonic functions, functions of bounded variation, total variation, additive property of total variation, total variation on $(a,x)$ as a functions of $x$, functions of bounded variation expressed as the difference of increasing functions, continuous functions of bounded variation, curves and paths, rectifiable path and arc length, additive and continuity properties of arc length, equivalence of paths, change of parameter.
%(Chapter 6, Section: 6.1 - 6.12. of \cite{apostol}) (20 hours.)
%Module 2: The Riemann-Stieltjes Integral
%Definition and existence of the integral, properties of the integral, integration and differentiation, integration of vector valued functions.
%(Chapter 6 - Section 6.1 to 6.25 of \cite{rudin}) (20 hours.)
%Module 3: Sequence and Series of Functions
%Discussion of main problem, Uniform convergence, Uniform convergence and Continuity, Uniform convergence and Integration, Uniform convergence and Differentiation.
%(Chapter 7 Section. 7.1 to 7.18 of \cite{rudin}) (25 hours.)
%Module 4: Weierstrass Approximation \& Some Special Functions
%Equicontinuous families of functions, the Stone - Weierstrass theorem, Power series, the exponential and logarithmic functions, the trigonometric functions, the algebraic completeness of complex field.
%(Chapter 7 – Sections 7.19 to 7.27, Chapter 8 - Section 8.1 to 8.8 of \cite{rudin}) (25 hours.)

%Module 1 - \cite{apostol} 6
%Module 2 - \cite{rudin} 6
%Module 3 - \cite{rudin} 7a
%Module 4 - \cite{rudin} 7b

%Module 1
%\chapter{Functions of Bounded Variation \& Rectifiable Curves}
\section{Functions of Bounded Variation \& Rectifiable Curves}
\subsection{Properties of Monotone Functions}
\begin{theorem}
	Let $f$ be an increasing function defined on closed interval $[a,b]$.
	And let $x_0=a < x_1 < x_2 < \dots <x_{n-1} < x_n = b$.
	Then,
	\[ \sum_{k = 1}^{n-1} [f(x_k+)-f(x_k-)] \le f(b) - f(a) \]
\end{theorem}
\begin{proof}
\end{proof}

\begin{theorem}
	Let $f$ be a monotonic function defined on closed interval $[a,b]$.
	Then the set of discontinuities of $f$ is countable.
\end{theorem}
\begin{proof}
	Without loss of generality, let $f$ be an increasing function on $[a,b]$.
	Let $S_m$ be the set of all points on $[a,b]$ at which the jump exceeds $\frac{1}{m}$.\\

	We know that, the sum of jumps of an increasing function is bounded above by $f(b)-f(a)$.	
	Thus, cardinality of $S_m$ given by,
	\[ |S_m| < m[f(b)-f(a)] \]
	is finite for any positive integer $m$.\\

	If $f$ is discontinuous at a pont $x \in [a,b]$, then there exists some integer $m'$ such that $0 < \frac{1}{m'} < x $ and $x \in S_{m'}$.
	\[ \text{Number of discontinuities } = \left| \bigcup_{m=1}^\infty S_m \right| \le \sum_{m=1}^\infty |S_m| \text{ is countable.}\]
	since countable sum of finite values is countable.
	Therefore, the number of discontinuities of $f$ is countable.
\end{proof}

\subsection{Function of Bounded Variation}
\begin{definition}[partition]
	Let $[a,b]$ be a compact interval.
	Let $x_0 = a$, $x_0 < x_1 < x_2 < \dots < x_n$ and $x_n = b$.
	Then $P = \{ x_0,x_1,\dots,x_n \}$ is a \textbf{partition} of $[a,b]$.
	And $(x_{k-1},x_k)$ is the \textbf{$k$th subinterval} of the partition.
\end{definition}

\begin{definition}[bounded variation]
	Let $f$ be a function defined on closed interval $[a,b]$.
	If there exists a positive real-number $M$ such that
	\[ \sum_{k=1}^n |\Delta f_k| = \sum_{k=1}^n |f(x_k) - f(x_{k-1})| \le M \]
	for any partition $P$ on $[a,b]$.
	Then $f$ is a function of \textbf{bounded variation}, where $(x_{k-1},x_k)$ is the $k$th subinterval of the partition.\\

\begin{commentary}
	In other words, a function is of bounded variation on $[a,b]$ if the sum of variations is bounded for any (finite) partition of $[a,b]$.
\end{commentary}
\end{definition}

\begin{theorem}
	Let $f$ be a monotonic function on $[a,b]$.
	Then $f$ is of bounded variation on $[a,b]$.
\end{theorem}
\begin{proof}
	Without loss of generality, let $f$ be an increasing funciton.
	Then for any partition $P = \{x_0,x_1,\dots,x_n\}$ of $[a,b]$, we have
	\[ \sum_{k=1}^n \left[ f(x_k)-f(x_{k-1}) \right] \le f(b)-f(a) = M \]
	Therefore, $f$ is of bounded variation on $[a,b]$.
\end{proof}

\begin{theorem}
	Let $f$ be a continuous function on $[a,b]$ and its derivative $f'$ exists and $f'$ is bounded in $(a,b)$.
	Then $f$ is of bounded variation.	
\end{theorem}
\begin{proof}
	Let $f$ be a continuous function with derivative $f'$ on $(a,b)$.
	Since $f$ is continuous and $f'$ exists, from intermediate value theorem we have
	\[ \Delta f_k = f(x_k) - f(x_{k-1}) = f'(t_k) [x_k-x_{k-1}] \text{ where } t_k \in (x_{k-1},x_k) \]
	Since $f'$ is bounded, $f'(t_k) \le M'$ for any $t_k \in (a,b)$.
	Thus,
	\[ \sum_{k=1}^n |\Delta f_k| = \sum_{k=1}^n |f'(t_k)| (x_k-x_{k-1}) \le M'\sum_{k=1}^n x_k-x_{k-1} = M'(b-a) = M \]
	Therefore, $f$ is of bounded variation.
\end{proof}

\begin{theorem}
	Let $f$ be a function on $[a,b]$.
	If $f$ is of bounded variation, then $f$ is bounded.
\end{theorem}
\begin{proof}
	Let $x \in (a,b)$.
	Consider the partition $P = \{ a,x,b \}$.
	Since $f$ is of bounded variation, there exists a positive real-number $M$ such that
	\[ \sum_{k=1}^2 |\Delta f_k| = |f(x)-f(a)| + |f(b)-f(x)| \le M \]
	Clearly, $|f(x)-f(a)| \le M$, since $|f(b)-f(x)| > 0$.
	We have,
	\[ |f(x)| = |f(x)-f(a)+f(a)| \le |f(x)-f(a)|+|f(a)| \le M+|f(a)| \]
	Suppose $x = a$, then $|f(x)| = |f(a)|$.\\
	Suppose $x = b$, then $|f(x)| = |f(b)| = M' + |f(a)|$ where $M' = |f(b)|-|f(a)|$.\\

	Therefore, the function $f$ is bounded on $[a,b]$.
\end{proof}

\subsection{Total Variation}
\begin{definition}[total variation]
	Let $f$ be a function of bounded variation on $[a,b]$.
	Let $\Sigma (P)$ be the sum of variations with respect to the partition $P$ of $[a,b]$.
	Then the \textbf{total variation} of the function $f$ on $[a,b]$ is given by,
	\[ V_f(a,b) = V_f = \sup \{ \Sigma (P) : P \in \mathscr{P}[a,b] \} \]
\end{definition}
\textbf{Note 1} : $V_f$ is finite, since $f$ is of boundned variation on $[a,b]$.\\

\textbf{Note 2} : $V_f \ge 0$, since $|\Delta f_k| \ge 0$ for any subinterval of $[a,b]$.\\


\textbf{Note 3} : $V_f = 0$ if only if $f$ is a constant function on $[a,b]$. (Why ?)

\begin{theorem}
	Let $f,g$ be functions of bounded variation on $[a,b]$.
	Then their sum $f+g$, difference $f-g$, and product $fg$ are of bounded variation.
	Also, 
	\[ V_{f \pm g} \le V_f + V_g \quad \text{ and } \quad V_{fg} \le AV_f + BV_g \]
	where $\displaystyle A = \sup \left\{ |g(x)| : x \in [a,b] \right\}$ and $\displaystyle B = \sup \left\{ |f(x)| : x \in [a,b] \right\}$.
\end{theorem}
\begin{proof}
	Let $f,g$ be functions of bounded variation on $[a,b]$.
	Then $f,g$ are bounded and $\sup |f(x)|$ and $\sup |g(x)|$ exists.\\

	\textbf{Step 1 : $V_{f \pm g} \le V_f + V_g$}\\
	We have,
	\[ |(f+g)(x_k) - (f+g)(x_{k-1})| \le | f(x_k) - f(x_{k-1})| + |g(x_k) - g(x_{k-1})| \]
	Then
	\begin{align*}
	V_{f+g}  & = \sup_{P \in \mathscr{P}} \sum_{k=1}^n |(f+g)(x_k) - (f+g)(x_{k-1})| \\
	& \le \sup_{P \in \mathscr{P}} \sum_{k=1}^n |f(x_k) - f(x_{k-1})| + \sup_{P \in \mathscr{P}} \sum_{k=1}^n |g(x_k)-g(x_{k-1})| \\
	& = V_f + V_g 
\end{align*}
	Similarly, we have $V_{f-g} \le V_f + V_g$ since
	\[ |(f-g)(x_k) - (f-g)(x_{k-1})| \le |f(x_k)-f(x_{k-1})| + |g(x_k)-g(x_{k-1})| \]

	\textbf{Step 2 : $V_{fg} \le AV_f + BV_g$}\\
	We have,
	\begin{align*}
		|fg(x_k)-fg(x_{k-1})| & = |f(x_k)g(x_k) - f(x_{k-1}g(x_{k-1})| \\
		& = |f(x_k)g(x_k) - f(x_{k-1})g(x_k) + f(x_{k-1}g(x_k) - f(x_{k-1}g(x_{k-1}) | \\
		& \le |g(x_k)|\ |f(x_k)-f(x_{k-1})| + |f(x_{k-1}|\ |g(x_k) - g(x_{k-1})| \\
		& \le A |f(x_k)-f(x_{k-1})| + B |g(x_k)-g(x_{k-1})|
	\end{align*}
	where $A = \sup \{ |g(x)| : x \in [a,b] \}$ and $B = \sup \{ |f(x)| : x \in [a,b] \}$.\\

	Therefore,
	\begin{align*}
		V_{fg} & \le A \sup_{P \in \mathscr{P}} \left\{ \sum_{k=1}^n |f(x_k)-f(x_{k-1}| \right\} + B \sup_{P \in \mathscr{P}} \left\{ \sum_{k=1}^n |g(x_k)-g(x_{k-1})| \right\} \\
		& = AV_f + BV_g 
	\end{align*}
\end{proof}

\begin{definition}[bounded away from zero]
	A function $f$ is bounded away from zero on $[a,b]$ if there exists a real-number $m$ such that $0 < m \le f(x)$, $\forall x \in [a,b]$.
\end{definition}

\begin{commentary}
	Let $f$ be a function of bounded variation.
	Then $\frac{1}{f}$ is of bounded variation if and only if $f$ is bounded away from zero.(Why ?)
\end{commentary}

\begin{theorem}
	Let $f$ be a function of bounded variation on $[a,b]$ and $f$ is bounded away from zero.
	Then, $g = \frac{1}{f}$ is a function of bounded variation on $[a,b]$ and $V_g \le \frac{V_f}{m^2}$ whenever $0 < m \le |f(x)|$.
\end{theorem}
\begin{proof}
	Suppose $f$ is of bounded variation on $[a,b]$ and $f$ is bounded away from zero.
	That is, there exists positive real-number $m$ such that $0 < m \le |f(x)|$, $\forall x \in [a,b]$.
	Then, $\frac{1}{|f(x)|} \le \frac{1}{m}$, $\forall x \in [a,b]$.\\

	Define $g = \frac{1}{f}$.
	Then, we have
	\[ |\Delta g_k| = \left| \frac{1}{f(x_k)} - \frac{1}{f(x_{k-1})} \right| = \frac{|f(x_k)-f(x_{k-1})|}{|f(x_k)|\ |f(x_{k-1}|} \le \frac{|f(x_k)-f(x_{k-1})|}{m^2} \]
	The total variation of $g$ on $[a,b]$ is given by,
	\begin{align*}
		V_g & = \sup_{P \in \mathscr{P}} \left\{ \sum_{k=1}^n |g(x_k)-g(x_{k-1})| \right\} \\
		& \le \frac{1}{m^2} \sup_{P \in \mathscr{P}} \left\{ \sum_{k=1}^n |f(x_k)-f(x_{k-1})| \right\} \\
		& \le \frac{V_f}{m^2} \text{ where } 0 < m \le |f(x)|,\ \forall x \in [a,b]
	\end{align*}
\end{proof}
\begin{commentary}
	In other words, if $h,f$ are functions of bounded variation of $[a,b]$.
	And $f$ is bounded away from zero, then $g = \frac{1}{f}$ and $\frac{h}{f} = hg$ is a function of bounded variation and $V_{\frac{h}{f}} = V_{hg} \le AV_h + BV_f$.\\

	Now, we have analyzed sum, difference, product and quotient of functions of bounded variation.
	And we are not surprised about preservation of bounded variation under function composition. (Why ?)
\end{commentary}
\subsection{Additive Property of Total Variation}
\begin{theorem}[additive property]
	Let $f$ be a function of bounded variation on $[a,b]$.
	Let $c \in (a,b)$.
	Then $f$ is of bounded variation on both $[a,c]$ and $[c,b]$.
	And, $V_f(a,b) = V_f(a,c)+V_f(c,b)$.
\end{theorem}
\begin{proof}
	Let $f$ be a function of bounded variation on $[a,b]$.
	Let $P_1,P_2$ be partitions of $[a,c]$ and $[c,b]$ respectively.
	Then $P_0 = P_1 \cup P_2$ is a partition of $[a,b]$.
	Thus,
	\[ \sum (P_1) + \sum (P_2) = \sum (P_0) \le V_f(a,b) \]
	Taking supremums on the left side, we get
	\[ V_f(a,c) + V_f(c,b) \le V_f(a,b) \]	
	Clearly, $V_f(a,c) \le V_f(a,b)$ and $V_f(c,b) \le V_f(a,b)$.
	Therefore, function $f$ is of bounded variation on both $[a,c]$ and $[c,b]$.\\

	Let $P$ be a paritition of $[a,b]$.
	Then $P_0 = P\cup\{c\}$ is refinement of $P$.
	Now, we have two partitions $P_1, P_2$ of $[a,c]$ and $[c,b]$ such that $P_0 = P_1 \cup P_2$.
	Thus,
	\[ \sum (P) \le \sum (P_0) = \sum (P_1) + \sum (P_2) \le V_f(a,c) + V_f(c,b) \]
	Taking supremum on the left side, we get
	\[ V_f(a,b) \le V_f(a,c) + V_f(c,b) \]
	Therefore,
	\[ V_f(a,b) = V_f(a,c) + V_f(c,b),\quad \forall c \in (a,b) \]
\end{proof}

\subsection{Total Variation on $[a,x]$ as a function of $x$}
\begin{commentary}
	By additive property, we have existence of $V_f(a,x)$ for every $x \in (a,b]$.
	Assigning $V(x) = V_f(a,x)$, we have a well-defined function on $(a,b]$.
\end{commentary}

\begin{theorem}
	Let $f$ be a function of bounded variation on $[a,b]$.
	Define $V : [a,b] \to \mathbb{R}$ given by,
	\[ V(x) = \begin{cases} V_f(a,x) & x \in (a,b] \\ 0 & x = a \end{cases} \]
	Then,
	\begin{enumerate}
		\item $V$ is an increasing function on $[a,b]$.
		\item $V-f$ is an increasing function of $[a,b]$.
	\end{enumerate}
\end{theorem}
\begin{proof}
	Suppose $x = a$.
	Then $V(x) = 0$ and $V(y) = V_f(a,y) \ge 0 = V(x)$.\\
	Suppose $x \ne a$.
	Then, $a < x < y \le b$.
	By additive property of total variation, we have $V(y) = V_f(a,y) = V_f(a,x)+V_f(x,y) = V(x) + V_f(x,y)$.
	Since $V_f(x,y) \ge 0$, we have $V(x) \le V(y)$.
	Therefore, $V$ is an increasing function on $[a,b]$.\\

	Define $D : [a,b] \to \mathbb{R}$ given by $D(x)  = (V-f)(x) = V(x) - f(x)$.
	Suppose $a \le x < y \le b$.
	Then, $D(y)-D(x) = V(y)-f(y)-V(x)+f(x) = [V(y)-V(x)] - [f(y)-f(x)]$.
	We have, $V(y) = V_f(a,y) = V_f(a,x)+V_f(x,y)$.
	Thus, $V(y)-V(x) = V_f(x,y)$.
	Also we have, $f$ is of bounded variation on $[x,y]$.
	Consider the trivial partition $P = \{ x,y\}$ of $[x,y]$.
	Then, we have $f(y) - f(x) = \sum (P) \le V_f(x,y)$.
	Thus, $D(y)-D(x) = V_f(x,y) - [f(y)-f(x)] \ge 0$.
	Therefore, $D = V-f$ is an increasing function on $[a,b]$.
\end{proof}
%\chapter{The Riemann Stieltjes Integral}
%\chapter{The Riemann Stieltjes Integral}
%\chapter{Sequence \& Series of Functions}
%\chapter{Weierstrass Approximation \& Some Special Functions}
