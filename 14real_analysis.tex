%Text Books: \cite{apostol}, \cite{rudin}
%Module 1: Functions of bounded variation and rectifiable curves
%Introduction, properties of monotonic functions, functions of bounded variation, total variation, additive property of total variation, total variation on $(a,x)$ as a functions of $x$, functions of bounded variation expressed as the difference of increasing functions, continuous functions of bounded variation, curves and paths, rectifiable path and arc length, additive and continuity properties of arc length, equivalence of paths, change of parameter.
%(Chapter 6, Section: 6.1 - 6.12. of \cite{apostol}) (20 hours.)
%Module 2: The Riemann-Stieltjes Integral
%Definition and existence of the integral, properties of the integral, integration and differentiation, integration of vector valued functions.
%(Chapter 6 - Section 6.1 to 6.25 of \cite{rudin}) (20 hours.)
%Module 3: Sequence and Series of Functions
%Discussion of main problem, Uniform convergence, Uniform convergence and Continuity, Uniform convergence and Integration, Uniform convergence and Differentiation.
%(Chapter 7 Section. 7.1 to 7.18 of \cite{rudin}) (25 hours.)
%Module 4: Weierstrass Approximation \& Some Special Functions
%Equicontinuous families of functions, the Stone - Weierstrass theorem, Power series, the exponential and logarithmic functions, the trigonometric functions, the algebraic completeness of complex field.
%(Chapter 7 – Sections 7.19 to 7.27, Chapter 8 - Section 8.1 to 8.8 of \cite{rudin}) (25 hours.)

%Module 1 - \cite{apostol} 6
%Module 2 - \cite{rudin} 6
%Module 3 - \cite{rudin} 7a
%Module 4 - \cite{rudin} 7b

%The commands to Riemann upper/lower integrals are defined by Leo Liu in tex-stack-exchange.
%Leo Liu - https://tex.stackexchange.com/a/44245 
\def\upint{\mathchoice%
    {\mkern13mu\overline{\vphantom{\intop}\mkern7mu}\mkern-20mu}%
    {\mkern7mu\overline{\vphantom{\intop}\mkern7mu}\mkern-14mu}%
    {\mkern7mu\overline{\vphantom{\intop}\mkern7mu}\mkern-14mu}%
    {\mkern7mu\overline{\vphantom{\intop}\mkern7mu}\mkern-14mu}%
  \int}
\def\lowint{\mkern3mu\underline{\vphantom{\intop}\mkern7mu}\mkern-10mu\int}

%Module 1
%\chapter{Functions of Bounded Variation \& Rectifiable Curves}
\section{Functions of Bounded Variation \& Rectifiable Curves}
%\subsection{Introduction}
\setcounter{subsection}{1}
\subsection{Properties of Monotone Functions}
\begin{theorem}
	Let $f$ be an increasing function defined on closed interval $[a,b]$.
	And let $x_0=a < x_1 < x_2 < \dots <x_{n-1} < x_n = b$.
	Then,
	\[ \sum_{k = 1}^{n-1} [f(x_k+)-f(x_k-)] \le f(b) - f(a) \]
\end{theorem}
\begin{proof}
	Let $f$ be an increasing function on $[a,b]$.
	Let $\{x_0=a,\ x_1, \dots,\ x_n=b\}$ be a partition of $[a,b]$.
	Let $y_k \in (x_{k-1},x_k),\ \forall k$.
	Then, $f(y_k) \le f(x_k+)$ and $f(x_k-) \le f(y_{k-1})$.
	Therefore,
	\[ \sum_{k=1}^{n-1} [f(x_k+) - f(x_k-)] \le \sum_{k=1}^{n-1} [f(y_k) - f(y_{k-1})] \le f(b) - f(a) \]
\end{proof}
\begin{commentary}
	In other words, for monotonic functions the sum of jumps is bounded.
\end{commentary}

\begin{theorem}
	Let $f$ be a monotonic function defined on closed interval $[a,b]$.
	Then the set of discontinuities of $f$ is countable.
\end{theorem}
\begin{proof}
	Without loss of generality, let $f$ be an increasing function on $[a,b]$.
	Let $S_m$ be the set of all points on $[a,b]$ at which the jump exceeds $\frac{1}{m}$.\\

	We know that, the sum of jumps of an increasing function is bounded above by $f(b)-f(a)$.	
	Thus, cardinality of $S_m$ given by,
	\[ |S_m| < m[f(b)-f(a)] \]
	is finite for any positive integer $m$.\\

	If $f$ is discontinuous at a pont $x \in [a,b]$, then there exists some integer $m'$ such that $0 < \frac{1}{m'} < x $ and $x \in S_{m'}$.
	\[ \text{Number of discontinuities } = \left| \bigcup_{m=1}^\infty S_m \right| \le \sum_{m=1}^\infty |S_m| \text{ is countable.}\]
	since countable sum of finite values is countable.
	Therefore, the number of discontinuities of $f$ is countable.
\end{proof}

\subsection{Function of Bounded Variation}
\begin{definition}[partition]
	Let $[a,b]$ be a compact interval.
	Let $x_0 = a$, $x_0 < x_1 < x_2 < \dots < x_n$ and $x_n = b$.
	Then $P = \{ x_0,x_1,\dots,x_n \}$ is a \textbf{partition} of $[a,b]$.
	And $(x_{k-1},x_k)$ is the \textbf{$k$th subinterval} of the partition.
\end{definition}

\begin{definition}[bounded variation]
	Let $f$ be a function defined on closed interval $[a,b]$.
	If there exists a positive real-number $M$ such that
	\[ \sum_{k=1}^n |\Delta f_k| = \sum_{k=1}^n |f(x_k) - f(x_{k-1})| \le M \]
	for any partition $P$ on $[a,b]$.
	Then $f$ is a function of \textbf{bounded variation}, where $(x_{k-1},x_k)$ is the $k$th subinterval of the partition.\\

\begin{commentary}
	In other words, a function is of bounded variation on $[a,b]$ if the sum of variations is bounded for any (finite) partition of $[a,b]$.
\end{commentary}
\end{definition}

\begin{theorem}
	Let $f$ be a monotonic function on $[a,b]$.
	Then $f$ is of bounded variation on $[a,b]$.
\end{theorem}
\begin{proof}
	Without loss of generality, let $f$ be an increasing funciton.
	Then for any partition $P = \{x_0,x_1,\dots,x_n\}$ of $[a,b]$, we have
	\[ \sum_{k=1}^n \left[ f(x_k)-f(x_{k-1}) \right] \le f(b)-f(a) = M \]
	Therefore, $f$ is of bounded variation on $[a,b]$.
\end{proof}

\begin{theorem}
	Let $f$ be a continuous function on $[a,b]$ and its derivative $f'$ exists and $f'$ is bounded in $(a,b)$.
	Then $f$ is of bounded variation.	
\end{theorem}
\begin{proof}
	Let $f$ be a continuous function with derivative $f'$ on $(a,b)$.
	Since $f$ is continuous and $f'$ exists, from intermediate value theorem we have
	\[ \Delta f_k = f(x_k) - f(x_{k-1}) = f'(t_k) [x_k-x_{k-1}] \text{ where } t_k \in (x_{k-1},x_k) \]
	Since $f'$ is bounded, $f'(t_k) \le M'$ for any $t_k \in (a,b)$.
	Thus,
	\[ \sum_{k=1}^n |\Delta f_k| = \sum_{k=1}^n |f'(t_k)| (x_k-x_{k-1}) \le M'\sum_{k=1}^n x_k-x_{k-1} = M'(b-a) = M \]
	Therefore, $f$ is of bounded variation.
\end{proof}

\begin{theorem}
	Let $f$ be a function on $[a,b]$.
	If $f$ is of bounded variation, then $f$ is bounded.
\end{theorem}
\begin{proof}
	Let $x \in (a,b)$.
	Consider the partition $P = \{ a,x,b \}$.
	Since $f$ is of bounded variation, there exists a positive real-number $M$ such that
	\[ \sum_{k=1}^2 |\Delta f_k| = |f(x)-f(a)| + |f(b)-f(x)| \le M \]
	Clearly, $|f(x)-f(a)| \le M$, since $|f(b)-f(x)| > 0$.
	We have,
	\[ |f(x)| = |f(x)-f(a)+f(a)| \le |f(x)-f(a)|+|f(a)| \le M+|f(a)| \]
	Suppose $x = a$, then $|f(x)| = |f(a)|$.\\
	Suppose $x = b$, then $|f(x)| = |f(b)| = M' + |f(a)|$ where $M' = |f(b)|-|f(a)|$.\\

	Therefore, the function $f$ is bounded on $[a,b]$.
\end{proof}

\subsection{Total Variation}
\begin{definition}[total variation]
	Let $f$ be a function of bounded variation on $[a,b]$.
	Let $\Sigma (P)$ be the sum of variations with respect to the partition $P$ of $[a,b]$.
	Then the \textbf{total variation} of the function $f$ on $[a,b]$ is given by,
	\[ V_f(a,b) = V_f = \sup \{ \Sigma (P) : P \in \mathscr{P}[a,b] \} \]
\end{definition}
\textbf{Note 1} : $V_f$ is finite, since $f$ is of boundned variation on $[a,b]$.\\

\textbf{Note 2} : $V_f \ge 0$, since $|\Delta f_k| \ge 0$ for any subinterval of $[a,b]$.\\


\textbf{Note 3} : $V_f = 0$ if only if $f$ is a constant function on $[a,b]$. (Why ?)

\begin{theorem}
	Let $f,g$ be functions of bounded variation on $[a,b]$.
	Then their sum $f+g$, difference $f-g$, and product $fg$ are of bounded variation.
	Also, 
	\[ V_{f \pm g} \le V_f + V_g \quad \text{ and } \quad V_{fg} \le AV_f + BV_g \]
	where $\displaystyle A = \sup \left\{ |g(x)| : x \in [a,b] \right\}$ and $\displaystyle B = \sup \left\{ |f(x)| : x \in [a,b] \right\}$.
\end{theorem}
\begin{proof}
	Let $f,g$ be functions of bounded variation on $[a,b]$.
	Then $f,g$ are bounded and $\sup |f(x)|$ and $\sup |g(x)|$ exists.\\

	\textbf{Step 1 : $V_{f \pm g} \le V_f + V_g$}\\
	We have,
	\[ |(f+g)(x_k) - (f+g)(x_{k-1})| \le | f(x_k) - f(x_{k-1})| + |g(x_k) - g(x_{k-1})| \]
	Then
	\begin{align*}
	V_{f+g}  & = \sup_{P \in \mathscr{P}} \sum_{k=1}^n |(f+g)(x_k) - (f+g)(x_{k-1})| \\
	& \le \sup_{P \in \mathscr{P}} \sum_{k=1}^n |f(x_k) - f(x_{k-1})| + \sup_{P \in \mathscr{P}} \sum_{k=1}^n |g(x_k)-g(x_{k-1})| \\
	& = V_f + V_g 
\end{align*}
	Similarly, we have $V_{f-g} \le V_f + V_g$ since
	\[ |(f-g)(x_k) - (f-g)(x_{k-1})| \le |f(x_k)-f(x_{k-1})| + |g(x_k)-g(x_{k-1})| \]

	\textbf{Step 2 : $V_{fg} \le AV_f + BV_g$}\\
	We have,
	\begin{align*}
		|fg(x_k)-fg(x_{k-1})| & = |f(x_k)g(x_k) - f(x_{k-1}g(x_{k-1})| \\
		& = |f(x_k)g(x_k) - f(x_{k-1})g(x_k) + f(x_{k-1}g(x_k) - f(x_{k-1}g(x_{k-1}) | \\
		& \le |g(x_k)|\ |f(x_k)-f(x_{k-1})| + |f(x_{k-1}|\ |g(x_k) - g(x_{k-1})| \\
		& \le A |f(x_k)-f(x_{k-1})| + B |g(x_k)-g(x_{k-1})|
	\end{align*}
	where $A = \sup \{ |g(x)| : x \in [a,b] \}$ and $B = \sup \{ |f(x)| : x \in [a,b] \}$.\\

	Therefore,
	\begin{align*}
		V_{fg} & \le A \sup_{P \in \mathscr{P}} \left\{ \sum_{k=1}^n |f(x_k)-f(x_{k-1}| \right\} + B \sup_{P \in \mathscr{P}} \left\{ \sum_{k=1}^n |g(x_k)-g(x_{k-1})| \right\} \\
		& = AV_f + BV_g 
	\end{align*}
\end{proof}

\begin{definition}[bounded away from zero]
	A function $f$ is bounded away from zero on $[a,b]$ if there exists a real-number $m$ such that $0 < m \le f(x)$, $\forall x \in [a,b]$.
\end{definition}

\begin{commentary}
	Let $f$ be a function of bounded variation.
	Then $\frac{1}{f}$ is of bounded variation if and only if $f$ is bounded away from zero.(Why ?)
\end{commentary}

\begin{theorem}
	Let $f$ be a function of bounded variation on $[a,b]$ and $f$ is bounded away from zero.
	Then, $g = \frac{1}{f}$ is a function of bounded variation on $[a,b]$ and $V_g \le \frac{V_f}{m^2}$ whenever $0 < m \le |f(x)|$.
\end{theorem}
\begin{proof}
	Suppose $f$ is of bounded variation on $[a,b]$ and $f$ is bounded away from zero.
	That is, there exists positive real-number $m$ such that $0 < m \le |f(x)|$, $\forall x \in [a,b]$.
	Then, $\frac{1}{|f(x)|} \le \frac{1}{m}$, $\forall x \in [a,b]$.\\

	Define $g = \frac{1}{f}$.
	Then, we have
	\[ |\Delta g_k| = \left| \frac{1}{f(x_k)} - \frac{1}{f(x_{k-1})} \right| = \frac{|f(x_k)-f(x_{k-1})|}{|f(x_k)|\ |f(x_{k-1}|} \le \frac{|f(x_k)-f(x_{k-1})|}{m^2} \]
	The total variation of $g$ on $[a,b]$ is given by,
	\begin{align*}
		V_g & = \sup_{P \in \mathscr{P}} \left\{ \sum_{k=1}^n |g(x_k)-g(x_{k-1})| \right\} \\
		& \le \frac{1}{m^2} \sup_{P \in \mathscr{P}} \left\{ \sum_{k=1}^n |f(x_k)-f(x_{k-1})| \right\} \\
		& \le \frac{V_f}{m^2} \text{ where } 0 < m \le |f(x)|,\ \forall x \in [a,b]
	\end{align*}
\end{proof}
\begin{commentary}
	In other words, if $h,f$ are functions of bounded variation of $[a,b]$.
	And $f$ is bounded away from zero, then $g = \frac{1}{f}$ and $\frac{h}{f} = hg$ is a function of bounded variation and $V_{\frac{h}{f}} = V_{hg} \le AV_h + BV_f$.\\

	Now, we have analyzed sum, difference, product and quotient of functions of bounded variation.
	And we are not surprised about preservation of bounded variation under function composition. (Why ?)
\end{commentary}
\subsection{Additive Property of Total Variation}
\begin{theorem}[additive property]
	Let $f$ be a function of bounded variation on $[a,b]$.
	Let $c \in (a,b)$.
	Then $f$ is of bounded variation on both $[a,c]$ and $[c,b]$.
	And, $V_f(a,b) = V_f(a,c)+V_f(c,b)$.
\end{theorem}
\begin{proof}
	Let $f$ be a function of bounded variation on $[a,b]$.
	Let $P_1,P_2$ be partitions of $[a,c]$ and $[c,b]$ respectively.
	Then $P_0 = P_1 \cup P_2$ is a partition of $[a,b]$.
	Thus,
	\[ \sum (P_1) + \sum (P_2) = \sum (P_0) \le V_f(a,b) \]
	Taking supremums on the left side, we get
	\[ V_f(a,c) + V_f(c,b) \le V_f(a,b) \]	
	Clearly, $V_f(a,c) \le V_f(a,b)$ and $V_f(c,b) \le V_f(a,b)$.
	Therefore, function $f$ is of bounded variation on both $[a,c]$ and $[c,b]$.\\

	Let $P$ be a paritition of $[a,b]$.
	Then $P_0 = P\cup\{c\}$ is refinement of $P$.
	Now, we have two partitions $P_1, P_2$ of $[a,c]$ and $[c,b]$ such that $P_0 = P_1 \cup P_2$.
	Thus,
	\[ \sum (P) \le \sum (P_0) = \sum (P_1) + \sum (P_2) \le V_f(a,c) + V_f(c,b) \]
	Taking supremum on the left side, we get
	\[ V_f(a,b) \le V_f(a,c) + V_f(c,b) \]
	Therefore,
	\[ V_f(a,b) = V_f(a,c) + V_f(c,b),\quad \forall c \in (a,b) \]
\end{proof}

\subsection{Total Variation on $[a,x]$ as a function of $x$}
\begin{commentary}
	By additive property, we have existence of $V_f(a,x)$ for every $x \in (a,b]$.
	Assigning $V(x) = V_f(a,x)$, we have a well-defined function on $(a,b]$.
\end{commentary}

\begin{theorem}
	Let $f$ be a function of bounded variation on $[a,b]$.
	Define $V : [a,b] \to \mathbb{R}$ given by,
	\[ V(x) = \begin{cases} V_f(a,x) & x \in (a,b] \\ 0 & x = a \end{cases} \]
	Then,
	\begin{enumerate}
		\item $V$ is an increasing function on $[a,b]$.
		\item $V-f$ is an increasing function of $[a,b]$.
	\end{enumerate}
\end{theorem}
\begin{proof}
	Suppose $x = a$.
	Then $V(x) = 0$ and $V(y) = V_f(a,y) \ge 0 = V(x)$.\\
	Suppose $x \ne a$.
	Then, $a < x < y \le b$.
	By additive property of total variation, we have $V(y) = V_f(a,y) = V_f(a,x)+V_f(x,y) = V(x) + V_f(x,y)$.
	Since $V_f(x,y) \ge 0$, we have $V(x) \le V(y)$.
	Therefore, $V$ is an increasing function on $[a,b]$.\\

	Define $D : [a,b] \to \mathbb{R}$ given by $D(x)  = (V-f)(x) = V(x) - f(x)$.
	Suppose $a \le x < y \le b$.
	Then, $D(y)-D(x) = V(y)-f(y)-V(x)+f(x) = [V(y)-V(x)] - [f(y)-f(x)]$.
	We have, $V(y) = V_f(a,y) = V_f(a,x)+V_f(x,y)$.
	Thus, $V(y)-V(x) = V_f(x,y)$.
	Also we have, $f$ is of bounded variation on $[x,y]$.
	Consider the trivial partition $P = \{ x,y\}$ of $[x,y]$.
	Then, we have $f(y) - f(x) = \sum (P) \le V_f(x,y)$.
	Thus, $D(y)-D(x) = V_f(x,y) - [f(y)-f(x)] \ge 0$.
	Therefore, $D = V-f$ is an increasing function on $[a,b]$.
\end{proof}

\subsection{Function of bounded variation expressed as the difference of increasing functions}
\begin{theorem}
	Let $f$ be a function on $[a,b]$.
	Function $f$ is of bounded variation on $[a,b]$ if and only if $f$ can be expressed as difference of two increasing functions.
\end{theorem}
\begin{proof}
	Let $f$ be a function of bounded variation, then $f = V-D$ where total variation $V$ and $D = V-f$ are both increasing.\\

	Let $f$ be function on $[a,b]$.
	Let $f = V-D$ where $V,D$ are increasing functions. 
	Then $V,D$ are of bounded variation, since monotonic functions on $[a,b]$ are of bounded variation.
	Also, we have $V-D$ is of bounded variation, since for any two functions of bounded variation their difference is also of bounded variation.
\end{proof}

\subsection{Continuous functions of bounded variation}
\begin{theorem}
	Let $f$ be a function of bounded variation on $[a,b]$. %Why it has to be of bounded variation ? For the existence of $V$ ?
	Let $V$ be the total variation function of $f$ defined on $[a,b]$.
	$f$ is continuous at a point if and only if $V$ is continuous at that point.
\end{theorem}
\begin{proof}
	Let $f$ be a function of bounded variation on $[a,b]$.
	Let $V$ be the total variation of $f$.
	Let $x,y \in [a,b]$, such that $x<y$.
	We have $V$ is an increasing function on $[a,b]$.
	And $f$ is difference of two increasing functions on $[a,b]$.
	Thus, $f(x+),\ f(x-),\ V(x+),\ V(x-)$ exists for any $x \in (a,b)$.
	It remains to prove that $V,f$ are continuous at $x \in (a,b)$.\\

	\textbf{Part 1 : $V$ continuous $\implies$ $f$ continuous}\\
	Suppose $x \ne a$, then $a<x<y\le b$.
	Let $P$ be any partition on $[a,x]$.
	Then there exists a partition $P'$ on $[a,y]$ such that $P \subset P'$.
	Consider, $P' = P\cup\{y\}$.
	Then, $V(y) > V(x)$ and $V(y)-V(x) \ge |f(y)-f(x)|$.
	Thus,
	\[ 0 \le |f(y)-f(x)| \le V(y)-V(x) \]
	The inequality is true for any $y>x$.
	Therefore,
	\[ 0 \le |f(x+)-f(x)| \le V(x+)-V(x) \text{ as } y \to x+ \]
	Similarly, let $a<z<x$.
	Then,
	\[ 0 \le |f(x)-f(x-)| \le V(x)-V(x-) \text{ as } z \to x- \]
	Clearly, if $V$ is continuous at $x$ then $f$ is continuous $x$.\\

	\textbf{Part 2 : $f$ continuous $\implies$ $V$ continuous}\\
	Suppose $f$ is continuous at $c \in (a,b)$.
	Let $\varepsilon > 0$.
	We have,
	\[ V_f(c,b) = \sup_{P \in \mathscr{P}} \sum (P) \]
	Thus, there exists a partition $P_1 \in \mathscr{P}[c,b]$ such that $V_f(c,b) - \frac{\varepsilon}{2} < \sum (P_1)$.
	Since $f$ is continuous at $c$, there exists $x_1 \in (c,b)$ such that $|f(x_1)-f(c)| < \frac{\varepsilon}{2}$.
	Then,
	\[ V_f(c,b) - \frac{\varepsilon}{2} < \frac{\varepsilon}{2} + V_f(x_1,b) \]
	We have,
	\begin{align*}
		V(x_1)-V(c) & = V_f(a,x_1) - V_f(a,c) = V_f(c,x_1) \\
		& = V_f(c,b) - V_f(x_1,b) < \varepsilon
	\end{align*}
	Clearly, $V(c+h) \to V(c)$ as $h \to 0$.
	That is, $V(c+) = V(c),\ \forall c \in [a,b)$.
	Similarly we have,
	\[ V_f(a,c) = \sup_{P \in \mathscr{P}} \sum (P) \]
	And there exists $P_2 \in \mathscr{P}[a,c]$ such that
	\[ V_f(a,c) - \frac{\varepsilon}{2} < \sum (P_2) \]
	And there exists $x_2 \in (a,c)$ such that $|f(c)-f(x_2)| < \frac{\varepsilon}{2}$, since $f$ is continuous at $c$.
	Therefore,
	\[ V(c)-V(x_2) = V_f(a,c) - V_f(a,x_2) = V(x_2,c) < \varepsilon \]
	Thus, $V(c-) = V(c),\ \forall c \in (a,b]$.
	Therefore, $V$ is continuous at $c$.
\end{proof}
\begin{theorem}
	Let $f$ be a continuous function on $[a,b]$.
	Function $f$ is of bounded variation on $[a,b]$ if and only if $f$ can be expressed as difference of two increasing continuous functions.
\end{theorem}
\begin{proof}
	Let $f$ be a continuous function on $[a,b]$.
	Then total variation $V$ is a continuous increasing function on $[a,b]$.
	Clearly, $D = V-f$ is also a continuous, increasing function on $[a,b]$.
	Therefore, $f = V - D$ where $V,D$ are continuous, increasing functions.\\

	Let $V,D$ be continuous, increasing functions on $[a,b]$.
	Then $f = V-D$ is also a continuous function on $[a,b]$.
	We have, $V,D$ are increasing functions, therefore both $V,D$ are of bounded variation and their difference $f$ is also of bounded variation.
\end{proof}

\begin{commentary}
	Suppose $f$ is of bounded variation on $[a,b]$.
	Let $id : [a,b] \to [a,b]$ where $id(x) = x$.
	Then $V+id$ is a strictly increasing function and $D = V+id-f$ is also strictly increasing.
	Thus, any function of bounded variation on $[a,b]$ can be characterised as difference of two strictly increasing continuous functions on $[a,b]$.
\end{commentary}
\subsection{Curves and Paths}
\begin{definition}[path]
	Let $f : [a,b] \to \mathbb{R}^n$ be a continuous, vector-valued function.
	Then $f$ is a path in $\mathbb{R}^n$.
	And $f$ is a motion if $[a,b]$ is a time interval.
\end{definition}
\subsection{Rectifiable paths and Arc length}
\begin{definition}[rectifiable path]
	Let $f : [a,b] \to \mathbb{R}^n$ be a path in $\mathbb{R}^n$.
	Let $P = \{ t_0,t_1,\dots,t_m \}$ be a partition of $[a,b]$.
	\[ \Lambda_f (P) = \sum_{k=1}^m \| f(t_k) - f(t_{k-1}) \| = \sum_{k=1}^m \| \Delta f_k \| \]
	If $\Lambda_f (P)$ is bounded for any partition $P \in \mathscr{P}[a,b]$, then path $f$ is \textbf{rectifiable}.
	If $\Lambda_f (P)$ is unbounded, then $f$ is \textbf{nonrectifiable}.\\
\end{definition}
\begin{definition}[arc length]
	Let $f : [a,b] \to \mathbb{R}^n$ be a rectifiable path.
	Then \textbf{arc length} of path $f$ is given by,
	\[ \Lambda_f(a,b) = \sup_{P \in \mathscr{P}} \Lambda_f (P) \]
\end{definition}

\begin{theorem}
	A path $f : [a,b] \to \mathbb{R}^n$ is rectifiable if and only if each component $f_k$ of $f$ is of bounded variation on $[a,b]$.
	Let $V_k(a,b)$ be the total variation of $f_k$ on $[a,b]$.
	Then,
	\[ V_k(a,b) \le \Lambda_f(a,b) \le V_1(a,b) + V_2(a,b) + \dots + V_n(a,b) \]
\end{theorem}
\begin{proof}
	Let $x_j \in \mathbb{R}^n$ for $j = 1,2,\dots,m$.
	Then,
	\[ |x_r| \le \sqrt{\sum_{j=1}^n |x_j|^2} \le \sum_{j=1}^n |x_j| \text{ since } \sum_{j=1}^n x_j^2 \le \left(\sum_{j=1}^n x_j\right)^2 \]
	Let $f : [a,b] \to \mathbb{R}^n$ be a path in $\mathbb{R}^n$.
	Then $f = (f_1,f_2,\dots,f_n)$ where $f_k$'s are components of the path $f$.
	Let $P = \{ t_0,t_1,\dots,t_m \}$ be a partition of $[a,b]$.
	Now $f_r(t_j)-f_r(t_{j-1}) \in \mathbb{R}^n$ for each subinterval of $[a,b]$ and each component of $f$.
	Thus for each subinterval $(t_j,t_{j-1})$ we have,
	\[ |f_r(t_j)-f_r(t_{j-1})| \le \| f(t_j)-f(t_{j-1}) \| \le \sum_{j=1}^n |f_k(t_j)-f_k(t_{j-1})| \] 
	Adding inequalities for every subinterval of the partition, we get
	\[ \sum_{k=1}^m |f_k(t_j)-f_k(t_{j-1})| \le \sum_{k=1}^m \| f(t_j)-f(t_{j-1}) \| \le \sum_{k=1}^m \sum_{j=1}^n |f_k(t_j)-f_k(t_{j-1})| \] 
	Therefore,
	\[ V_k(a,b) \le \Lambda_f(a,b) \le \sum_{k=1}^n V_k(a,b) \]
	Suppose $f$ is a rectifiable path.
	Then for each $k$, $f_k$ is of bounded variation since $V_k(a,b) \le \Lambda_f(a,b)$ is bounded.
	Suppose $f_k$'s are of bounded variation.
	Then $f$ is a rectifiable path, since $\Lambda_f(a,b) \le V_1(a,b)+V_2(a,b)+\dots+V_n(a,b)$ is bounded.
\end{proof}
\subsection{Additive and Continuity Properties of Arc length}
\begin{theorem}[additive]
	Let $f : [a,b] \to \mathbb{R}^n$ be a rectifiable path.	
	Let $c \in (a,b)$.
	Then,
	\[ \Lambda_f(a,b) = \Lambda_f(a,c) + \Lambda_f(c,b) \]
\end{theorem}
\begin{proof}
	Let $P$ be a partition of $[a,b]$.
	Then $P' = P \cup \{c\}$ is a refinement of $P$ such that $P' = P_1 \cup P_2$ where $P_1,P_2$ are partition of $[a,c]$ and $[c,b]$ respectively.
	We have,
	\[ \Lambda_f(P) \le \Lambda_f(P') = \Lambda_f(P_1) + \Lambda_f(P_2) \]
	This inequality if true for any partition of $[a,b]$.
	Thus,
	\[ \Lambda_f(a,b) \le \Lambda_f(a,c) + \Lambda_f(c,b) \]
	Let $P_1,P_2$ be partition of $[a,c]$ and $[c,b]$ respectively.
	Then,
	\[ \Lambda_f(P_1) + \Lambda_f(P_2) \le \Lambda_f(P) \le \Lambda_f(a,b) \]
	This inequality if true for any paritions on $[a,c]$ and $[c,b]$.
	Thus,
	\[ \Lambda_f(a,c) + \Lambda_f(c,b) \le \Lambda_f(a,b) \]
\end{proof}
\begin{theorem}[continuity]
	Let $f : [a,b] \to \mathbb{R}^n$ be  rectifiable path.
	Let function $s : [a,b] \to \mathbb{R}$ defined by
	\[ s(x) = \begin{cases} 0 & x = a \\ \Lambda_f(a,x) & x \in (a,b] \end{cases} \]
	Then,
	\begin{enumerate}
		\item function $s$ is continuous and increasing on $[a,b]$.
		\item if there is no subinterval of $[a,b]$ in which $f$ is constant, then $s$ is strictly increasing.
	\end{enumerate}
\end{theorem}
\begin{proof}
	Let $a \le x < y \le b$.
	Then, $s(y)-s(x) = \Lambda_f(x,y) = \Lambda(a,y) - \Lambda(a,x) \ge 0$.
	Therefore, $s$ in increasing.\\
	
	Suppose $f$ is not constant in any subinterval $[x,y]$ of $[a,b]$.
	Suppose $s$ is not strictly increasing.
	Then, there exists $x,y \in (a,b)$ such that $x < y$ and $s(y)-s(x) = 0$.
	Thus,
	\[ \Lambda_f(a,y) - \Lambda(a,x) = \Lambda_f(x,y) = 0 \]
	Thus, $V_k(x,y) = 0,\ \forall k$ which is a contradition since $f$ is not constant in $[x,y]$.
	Therefore, $s$ is strictly increasing.
\end{proof}
\subsection{Equivalence of path, Change of parameter}
\begin{definition}[change of parameter]
	Let $f:[a,b] \to \mathbb{R}^n$ be a path.
	Let $g : [c,d] \to \mathbb{R}^n$ be another path.
	Then $f,g$ are \textbf{equivalent} if there exists a continuous, real-valued function, $u : [c,d] \to [a,b]$ such that $g = f \circ u$.
	That is, $g(t) = f(u(t)),\ \forall t \in [c,d]$.
	In other words, $f,g$ are different parametric representations of a common graph.\\

	Function $u$ defines a change of parameter.
	If $u$ is strictly increasing, then $f,g$ are in the same direction.
	And $u$ is an orientation preserving change of parameter.
	If $u$ is strictly decreasing, then $f,g$ are in opposite directions.
	And $u$ is an orientation reversing change of parameter.
\end{definition}

\begin{theorem}[change of parameter]
	Let $f:[a,b] \to \mathbb{R}^n$ and $g : [a,b] \to \mathbb{R}^n$ be two paths.
	Let $f,g$ be both injective functions.
	Then $f$ and $g$ are equivalent if they have the same graph.
\end{theorem}
\begin{proof}
	Let $f : [a,b] \to \mathbb{R}^n$ and $g : [c,d] \to \mathbb{R}^n$ be continuous, injective, vector-valued functions.
	Suppose $f,g$ are equivalent paths, then $f,g$ have the same graph.\\

	Suppose $f,g$ have the same graph.
	Since $f$ is injective and continuous on its domain $[a,b]$, function $f^{-1}$ exists and is continuous on its graph.
	\[ \text{Define } u:[c,d] \to [a,b],\ u(t) = f^{-1}(g(t)) \]
	Then $u$ is continuous and $g(t) = f(u(t))$.
	Suppose $u$ is not a strictly monotonic function. 
	Since $u$ is continuous, there exists $t_1,t_2 \in [c,d]$ such that $u(t_1) = u(t_2)$.
	Then $f(u(t_1)) = f(u(t_2)) \implies g(t_1) = g(t_2)$ which is a contradiction since $g$ is injective on $[c,d]$.
\end{proof}

%Module 2
%\chapter{The Riemann Stieltjes Integral} %Rudin chapter 6$
\pagebreak
\section{The Riemann-Stieltjes Integral}
\begin{definition}[unti step]
	The unit step function $I : \mathbb{R} \to \mathbb{R}$ is defined by
	\[ I(x) = \begin{cases} 0 & x \le 0 \\ 1 & x > 0 \end{cases} \] 
\end{definition}
\begin{definition}[Riemann Integral]
	Let $f$ be a bounded real function defined on $[a,b]$.
	Let $P = \{x_0,x_1,\dots,x_n\}$ be a partition of $[a,b]$.
	Let $M_k = \sup \{ f(x) : x \in [x_{k-1},x_k] \}$ and $m_k = \inf \{ f(x) : x \in [x_{k-1},x_k]$.\\

	Then Riemann upper sum of function $f$ with respect to parition $P$,
	\[ U(P,f) = \sum_P M_k \Delta x_k = \sum_{k=1}^n M_k (x_k-x_{k-1}) \]
	And Riemann lower sum of $f$ with respect to $P$,
	\[ L(P,f) = \sum_{k=1}^n m_k (x_k-x_{k-1}) \]
	Now, Riemann upper integral of $f$ over $[a,b]$,
	\[ \upint_a^b f\ dx = inf \{ U(P,f) : P \in \mathscr{P}[a,b] \} \]
	And, Riemann lower integral of $f$ over $[a,b]$,
	\[ \lowint_a^b f\ dx = sup \{ L(P,f) : P \in \mathscr{P}[a,b] \} \]
	A function $f$ is Riemann integrable over $[a,b]$ if Riemann lower and upper integrals of $f$ over $[a,b]$ are the same.
	Then Riemann integral of $f$ over $[a,b]$,
	\[ \int_a^b f\ dx = \upint_a^b f\ dx = \lowint_a^b f\ dx \]
\end{definition}
\begin{definition}[Riemann-Stieltjes Integral]
	Let $f$ be a bounded function on $[a,b]$.
	Let $\alpha$ be an increasing function on $[a,b]$.
	Let $P = \{ x_0,x_1,\dots,x_n\}$ be a partition of $[a,b]$.
	Then, the Riemann-Stieltjes upper sum of $f$ with respect to partition $P$ and increasing function $\alpha$,
	\[ U(P,f,\alpha) = \sum_{k=1}^n M_k \Delta \alpha_k \]
	where $M_k = \sup \{ f(x) : x \in [x_{k-1},x_k] \}$ and $\Delta \alpha_k = \alpha(x_k) - \alpha(x_{k-1})$.
	Similarly, Riemann-Stieltjes lower sum,
	\[ L(P,f,\alpha) = \sum_{k=1}^n m_k \Delta \alpha_k \]
	where $m_k = \inf \{ f(x) : x \in [x_{k-1},x_k] \}$.
	And function $f$ is Riemann-Stieltjes integrable if Riemann Stieltjes upper and lower integrals are the same.
	\[ \int_a^b f\ d\alpha = \upint_a^b f\ d\alpha = \lowint_a^b f\ d\alpha \]
	where $\displaystyle \upint_a^b f\ d\alpha = \upint_a^b f\ d\alpha(x) = \inf \left\{ U(P,f,\alpha) : P \in \mathscr{P}[a,b] \right\}$ and\\
	$\displaystyle \lowint_a^b f\ d\alpha = \sup \left\{ L(P,f,\alpha) : P \in \mathscr{P}[a,b] \right\}$.\\

	We write $f \in \mathscr{R}(\alpha)$ on $[a,b]$, which means that a bounded, real function $f$ is Riemann-Stieltjes integrable on $[a,b]$ with respect to the increasing function $\alpha$.
\end{definition}

\textbf{Note : } function $\alpha : [a,b] \to \mathbb{R}$ is monotonic (increasing), but not necessarily continuous.\\

\textbf{Remark : } With $\alpha = id$ identity function, Riemann-Stieltjes integral is Riemann integral itself.
	That is, Riemann integral is a special case of Riemann-Stieltjes integral.

\begin{theorem}
	Let $P^\ast$ be a refinement of a parition $P$ of $[a,b]$.
	Then, $L(P,f,\alpha) \le L(P^\ast,f,\alpha)$ and $U(P^\ast,f,\alpha) \le U(P,f,\alpha)$.
\end{theorem}
\begin{proof}
\end{proof}

\begin{theorem}
	Let $f$ be a bounded, real function on $[a,b]$ and $\alpha$ increasing function on $[a,b]$.
	Then,
	\[ \lowint_a^b f\ d\alpha \le \upint_a^b f\ d\alpha \]
\end{theorem}
\begin{proof}
\end{proof}

\begin{theorem}[criteria for integrability]
	Let $f$ be  a bounded, real function on $[a,b]$.
	Let $\alpha$ be an increasing function on $[a,b]$.
	Then, $f$ is Riemann-Stieltjes integrable on $[a,b]$ with repesct to $\alpha$ if and only if for every $\varepsilon > 0$ there exists a partition $P$ of $[a,b]$ such that $U(P,f,\alpha) - L(P,f,\alpha) < \varepsilon$.
\end{theorem}
\begin{commentary}
	In other words, $f \in \mathscr{R}(\alpha)$ on $[a,b]$ if and only if 
	\[ \forall \varepsilon > 0, \exists P \in \mathscr{P}[a,b] \text{ such that }U(P,f,\alpha) - L(P,f,\alpha) < \varepsilon \]
\end{commentary}
\begin{proof}
\end{proof}

\begin{theorem}
	Suppose $\varepsilon > 0$ and $U(P,f,\alpha) - L(P,f,\alpha) < \varepsilon$ for some partition $P$ of $[a,b]$.
	\begin{enumerate}
		\item The inequaility is true for any refinement of $P$.
		\item Let $s_i,t_i \in [x_{i-1},x_i]$ for each subinterval of the partition $P$.
			\[ \sum_{i=1}^n \left| f_(s_i)-f(t_i) \right|\ \Delta\alpha_i < \varepsilon \]
		\item If $f \in \mathscr{R}(\alpha)$ and $t_i \in [x_{i-1},x_i]$ for each subinterval of the partition $P$, then
			\[ \left| \sum_{i=1}^n f(t_i) \Delta \alpha_i - \int_a^b f \ d\alpha \right| < \varepsilon \]
	\end{enumerate}
\end{theorem}
\begin{proof}
\end{proof}
\begin{theorem}
	If $f$ is continuous on $[a,b]$, then $f \in \mathscr{R}(\alpha)$ on $[a,b]$.
\end{theorem}
\begin{proof}
\end{proof}

\begin{theorem}
	If $f$ is monotonic on $[a,b]$ and $\alpha$ is continuous on $[a,b]$, then $f \in \mathscr{R}(\alpha)$ on $[a,b]$.
\end{theorem}
\begin{proof}
\end{proof}

\begin{theorem}
	If $f$ bounded on $[a,b]$, $f$ has only finitely many points of discontinuities on $[a,b]$ and $\alpha$ is continuous at every point at which $f$ is discontinuous.
	Then $f \in \mathscr{R}(\alpha)$.
\end{theorem}
\begin{proof}
\end{proof}

\begin{theorem}
	Suppose $f \in \mathscr{R}(\alpha)$, $m \le f \le M$ on $[a,b]$, $\phi$ is continuous on $[m,M]$ and $h(x) = \phi(f(x))$.
	Then $h \in \mathscr{R}(\alpha)$ on $[a,b]$.
\end{theorem}
\begin{proof}
\end{proof}

\subsection{Properties of the Riemann-Stieltjes Integral}
\begin{theorem}
	Let $f,f_1,f_2$ be bounded real functions on $[a,b]$.
	Let $\alpha,\alpha_1,\alpha_2$ be increasing functions on $[a,b]$.
	\begin{enumerate}
		\item If $f_1,f_2,f \in \mathscr{R}(\alpha)$ on $[a,b]$, then $f_1+f_2 \in \mathscr{R}(\alpha)$ on $[a,b]$.
			And,
			\[ \int_a^b \left( f_1 + f_2 \right)\ d\alpha = \int_a^b f_1\ d\alpha + \int_a^b f_2\ d\alpha \]
			If $c \in \mathbb{R}$, then $cf \in \mathscr{R}(\alpha)$ on $[a,b]$.
			And,
			\[ \int_a^b cf\ d\alpha = c\int_a^b f\ d\alpha \]
		\item If $f_1(x) \le f_2(x)$ on $[a,b]$, then
			\[ \int_a^b f_1\ d\alpha \le \int_a^b f_2\ d\alpha \]
		\item If $c \in (a,b)$, then $f \in \mathscr{R}(\alpha)$ on $[a,c]$ and $[c,b]$, then
			\[ \int_a^c f\ d\alpha + \int_c^b f\ d\alpha = \int_a^b f\ d\alpha \]
		\item If $|f(x)| \le M$ on $[a,b]$, then 
			\[ \left| \int_a^b f\ d\alpha \right| \le M[\alpha(b)-\alpha(a)] \]
		\item If $f \in \mathscr{R}(\alpha_1)$ and $f \in \mathscr{R}(\alpha_2)$ on $[a,b]$, then $f \in \mathscr{R}(\alpha_1+\alpha_2)$.
			And,
			\[ \int_a^b f\ d(\alpha_1+\alpha_2) = \int_a^b f\ d\alpha_1 + \int_a^b f\ d\alpha_2 \]
			If $f \in \mathscr{R}(\alpha)$ on $[a,b]$, and $c \in \mathbb{R}$, then $f \in \mathscr{R}(c\alpha)$ on $[a,b]$.
			And,
			\[ \int_a^b f\ d(c\alpha) = c\int_a^b f\ d\alpha \]
	\end{enumerate}
\end{theorem}
\begin{proof}
\end{proof}

\begin{theorem}
	If $f,g \in \mathscr{R}(\alpha)$ on $[a,b]$, then
	\begin{enumerate}
		\item $fg \in \mathscr{R}(\alpha)$ on $[a,b]$
		\item $|f| \in \mathscr{R}(\alpha)$ on $[a,b]$.
			And,
			\[ \left| \int_a^b f\ d\alpha \right| \le \int_a^b |f|\ d\alpha \]
	\end{enumerate}
\end{theorem}
\begin{proof}
\end{proof}

\begin{theorem}
\end{theorem}
\begin{proof}
\end{proof}
%\chapter{Sequence \& Series of Functions}
%\chapter{Weierstrass Approximation \& Some Special Functions}
